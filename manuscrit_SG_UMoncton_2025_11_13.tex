 %Dans cette section, on défini le format du document et toute autre fonction qui seront utile pour le document
 \documentclass[superscriptaddress,12pt]{article}
 \usepackage[letterpaper,top=2.5cm,bottom=2.5cm,left=4cm,right=2.5cm]{geometry}

 %\documentclass[aps,pra,notitlepage,superscriptaddress,10pt]{revtex4-2}
 %\documentclass[aps,pra,notitlepage,superscriptaddress,twocolumn,10pt]{revtex4-2}
 \usepackage[utf8]{inputenc} % to insert the character directly by copy paste or as ^+i typed on your keyboard
 \usepackage[T1]{fontenc} % to lower the accent
 \usepackage{graphicx}		%Permet d'ajouter des figures
 \graphicspath{ {./figure/}}
 \usepackage{bm} 			%Permet d'utiliser des charactères gras dans les équations
 \usepackage{amsmath}		%Défini plusieurs fonctions utiles pour les équations
 \usepackage{physics}
 %\usepackage[pdfauthor={Deny Hamel},pdftitle={Title}]{hyperref} %Cette fonction permet de créer des liens dans le document PDF généré. TRÈS utile pour des documents électroniques.
 \usepackage{icomma}			%Enlève l'espace après la virgule dans les nombres 
 \usepackage{setspace}
 \usepackage{multirow}
 \usepackage{booktabs}
 \usepackage{titlesec}
 %\usepackage{nomencl}
 \usepackage{csvsimple}
 \usepackage{longtable}
 \usepackage{array}
% Custom format for nomenclature
\usepackage{etoolbox} % To patch commands
\usepackage{amsmath}
\usepackage[thinc]{esdiff}
\usepackage{setspace}
\usepackage[frak=esstix]{mathalpha}
\usepackage{dsfont}
\usepackage[normalem]{ulem}
\onehalfspacing
% Patch \nomentry to add the chapter appearance
%\renewcommand{\nomlabel}[1]{#1 \dotfill}

% This will define the chapter where each symbol appears
%\newcommand{\nomenclchapter}[1]{\nomenclature[Z]{\hspace{2em}\textit{First appears in Chapter #1}}{}}

% \makenomenclature
%\renewcommand{\nomname}{Liste des symboles}
%\renewcommand{\nomname}{}

\makeatletter
 \newcommand{\mathleft}{\@fleqntrue\@mathmargin0pt}
 \newcommand{\mathcenter}{\@fleqnfalse}
 \makeatother

 %\nomenclature{$x$}{Position along the x-axis}
\nomenclchapter{1}  % First appears in Chapter 1

\nomenclature{$\alpha$}{Angle of incidence}
\nomenclchapter{2}  % First appears in Chapter 2

\nomenclature{$\psi$}{Wavefunction}
\nomenclchapter{3}  % First appears in Chapter 3

\nomenclature{$\xi$}{description}
 %\newcommand{\nomenclchapter}[1]{\addcontentsline{toc}{section}{Chapitre #1}\markboth{Chapitre #1}{}}

 
 \usepackage{gensymb}
 \usepackage{titlesec}
 \titleformat{\section}
   {\normalfont\fontsize{12}{15}\bfseries}{\thesection}{1em}{}
 \titleformat{\subsection}
   {\normalfont\fontsize{12}{15}\bfseries}{\thesubsection}{1em}{}
 \titleformat{\subsubsection}
   {\normalfont\fontsize{12}{15}\bfseries}{\thesubsubsection}{1em}{}
 \usepackage{siunitx}
 
 \usepackage{caption}
 
 
 \usepackage{xcolor}

 
 \newcommand\tab[1][0.83cm]{\hspace*{#1}}
 


 \usepackage[style=numeric, backend=bibtex, sorting=none]{biblatex}
 \addbibresource{./reference_these.bib}
 
 
 \usepackage{parskip}
 \setlength{\parindent}{4em}
 \setlength{\parskip}{1em}
 \usepackage[ letterpaper, top=2.5cm ,  bottom=2.5cm , left=4cm , right =2.5cm , ]{geometry}
 %\pagestyle{myheadings}
 
 \usepackage[french]{babel}
 \usepackage{hyperref}
 \hypersetup{
     colorlinks,
     citecolor=black,
     filecolor=black,
     linkcolor=black,
     urlcolor=black
 }
 %J'ai changé la numérotation des sections pour avoir une numérotation avec des chiffres arabes. Voir https://tex.stackexchange.com/questions/3177/how-to-change-the-numbering-of-part-chapter-section-to-alphabetical-r
 \renewcommand\thesection{\arabic{section}}
 \renewcommand\thesubsection{\thesection.\arabic{subsection}} 
 \renewcommand\thesubsubsection{\thesubsection.\arabic{subsubsection}} 
 
 
 %\renewcommand\thetable{\arabic{table}} %Si on veut aussi une numérotation arabe pour les tableaux.
 
 
 %%Cette partie corrige un problème causé par les définitions ici-haut, qui faisait qu'au lieu de citer par exemple la section 2.1, latex donnait 2 2.1 (ce qui ferait du sens avec le sections romaine de PRA, i.e. II A). Voir : https://tex.stackexchange.com/questions/104486/section-reference-shows-section-then-subsection-number
 \makeatletter
 \def\p@subsection{}
 \makeatother
 \makeatletter
 \def\p@subsubsection{}
 \makeatother
 
 
 %J'ai modifié le format un peu pour que les tables, figures, etc. soient en français
 \def\andname{et} 			%Remplace le "and" dans la liste d'auteur avec un "et"
 \def\tablename{Tableau}	
 \def\figurename{Figure}
 
 
 %La typographie pour la plupart des opérateur mathématique est déjà défini par latex. Or, il y a quelques cas ou il n'y a pas de consensus, comme par exemple exemple pour l'opérateur différentiel d. Dans ces cas, il suffit de définir de nouveaux opérateurs avec la typographie désirée.
 \def\D{\mathrm{d}}
 \def\e{\mathrm{e}}
 \def\i{\mathrm{i}}
 
 
 \usepackage{scalerel}
 
 
 \usepackage{fancyhdr}
 \pagestyle{fancy}
 \fancyhf{}
 \fancyhead[R]{\thepage}
 \renewcommand{\headrulewidth}{0pt}
 
 
 \usepackage{gensymb}
 
 
 
 
 \makeatletter
 \newcommand*{\rom}[1]{\expandafter\@slowromancap\romannumeral #1@}
 \makeatother
 
 
 %%%%%%%%%%%%Le document commence ici%%%%%%%%%%%%%%%%%%%%-----------------------------------------------------------------------------
 \usepackage{titlesec}

 %\makenomenclature

\begin{document}
    \pagenumbering{roman}

    \thispagestyle{empty}
    \begin{titlepage}
    \addcontentsline{toc}{section}{\protect\numberline{}Page titre}%
       \begin{center}
           \vspace*{1cm}


           \textbf{Caractérisation directe d'un état de polarisation à l'aide des mesures faibles temporelles}


           \vspace{5cm}


            Thèse présentée à la faculté des sciences de l'université de Moncton \\
            pour l'obtention du grade de \\
            maitrise ès sciences et spécialisation physique (M. Sc.)

           \vspace{5cm}


           \textbf{Shane Gervais} \\
           A00198792


           \vfill

           Département de physique et d'astronomie \\
           Université de Moncton\\
           19 Novembre 2025
           %\textcolor{red}{DATE}

           \vspace{0.8cm}

       \end{center}
    \end{titlepage}




    \section*{Composition du jury}
    \setcounter{page}{2}

    \addcontentsline{toc}{section}{\protect\numberline{}Composition du jury}%


    \vspace{1.5cm}


    Président du jury : Alexandre Melanson \\
    \tab \tab Professeur, \\
    \tab \tab Université de Moncton


    \vspace{3cm}


    Examinateur interne : Normand Beaudoin \\
    \tab \tab Professeur, \\
    \tab \tab Université de Moncton


    \vspace{3cm}


    Examinateur externe : Guillaume Thekkadath \\
    \tab \tab Agent de recherche, \\
    \tab \tab Conseil national de recherches Canada


    \vspace{3cm}


    Directeur de thèse : Lambert Giner \\
    \tab \tab Professeur, \\
    \tab \tab Université de Moncton


    \pagebreak


    \section*{Remerciements}

    \addcontentsline{toc}{section}{\protect\numberline{}Remerciements}%
    \begin{onehalfspace}
    Ce parcours a connu des hauts et des bas, mais je peux dire avec fierté que j'en suis ressorti avec une meilleure compréhension non seulement de la physique quantique elle-même, mais aussi de moi-même en tant que personne. Ayant un handicap (dysphasie), j'ai plus de difficulté que les autres à comprendre des sujets complexes, à communiquer mes idées et encore plus à rédiger une thèse, surtout en français. Cependant, j'ai trouvé la force, la confiance et l'assurance nécessaires pendant mes études à l'Université de Moncton, où j'ai rencontré divers professeurs qui m'ont inspirée et poussée à dépasser mes limites pour surmonter l'adversité. Pour moi, ce diplôme n'était pas seulement un projet que j'avais choisi, mais quelque chose sur lequel je peux revenir et qui me permet de continuer à progresser pour m'améliorer en tant que scientifique et en tant que personne. 

    \noindent Je tiens tout d'abord à remercier mon directeur de thèse, le Dr Lambert Giner, pour sa grande patience et sa collaboration tout au long de ce projet. Lorsque les choses semblaient ne mener nulle part, que j'étais perdue ou simplement bloquée dans une impasse, il a su me remettre sur la voie de l'objectif réel pour lequel nous travaillons. Je lui suis reconnaissant de m'avoir aidé au laboratoire lorsque le projet semblait dans l'impasse et de m'avoir guidé dans mes compétences de présentation et de rédaction pour une communication scientifique efficace. Bien que j'aie encore beaucoup à apprendre, ses conseils m'ont permis de découvrir mes forces et mes faiblesses, sur lesquelles je peux continuer à travailler. 

    \noindent Je tiens à remercier le Dr Normand Beaudoin de m'avoir inspiré tout au long de ce diplôme. Je n'ai rencontré aucun autre professeur comme lui, sa passion et son envie d'aller au fond des choses m'inspirent. Je me suis reconnue en lui, il m'a donné un aperçu de ce que je peux vraiment devenir en tant que scientifique et en tant que personne. Vouloir comprendre les fondements de l'univers dans lequel nous vivons me motive encore aujourd'hui à continuer, une tâche sans fin à laquelle je suis fière de participer. 

    \noindent Je tiens à remercier le Dr Guilleaume Thekkadeth d'avoir pris le temps de faire partie de ce jury. Ce fut un plaisir d'être évaluée par quelqu'un dont j'ai lu les articles pendant mes études, ce qui a rendu cette expérience vraiment inspirante. 

    \noindent Je tiens à remercier le Dr Alexandre Melanson d'avoir accepté d'être le président du jury et d'avoir cru en moi tout au long de ce projet.

    \noindent Je tiens également à remercier divers autres professeurs qui ont rendu cette expérience vraiment enrichissante pour moi, notamment le Dr Jean-François Bison pour sa positivité et la joie qu'il m'a apportée dans la poursuite de mes études, le Dr Alain Haché pour nos conversations sur le hockey et les conseils qu'il m'a prodigués, le Dr Serge Gauvin pour m'avoir encouragé à remettre en question et à sortir des sentiers battus, le Dr Deny Hamel pour m'avoir inspiré à aller de l'avant et à percer les mystères de l'optique quantique pendant ses cours, les techniciens de notre département : Pierre, pour avoir toujours été là quand j'avais besoin d'aide et pour m'avoir toujours fait rire pendant les pauses, et Julian, pour m'avoir inspiré à enseigner la physique de manière ludique et compréhensible, et pour notre passion commune pour le death metal. 

    \noindent Je tiens à remercier tout particulièrement les personnes suivantes qui m'ont également marqué personnellement tout au long de ce parcours.

    \noindent Ma grand-mère, pour avoir toujours été là pour moi et m'avoir transmis sa passion pour les sciences, en se souvenant des difficultés que j'ai rencontrées pendant mes jeunes années à l'école jusqu'à ce que j'en arrive là où je suis aujourd'hui.

    \noindent Ma mère et mon père, pour leurs conseils que je garderai toute ma vie et pour m'avoir motivé à aller de l'avant malgré les difficultés. Ils m'ont toujours rappelé de continuer à travailler dur et m'ont inspiré à donner le meilleur de moi-même.

    \noindent Ma sœur, pour être à la fois la personne la plus agaçante et la plus attentionnée au monde. Pour avoir compris mes sentiments lorsque j'avais besoin d'être écouté, pour tous les bons moments que nous avons passés ensemble tout au long de mon parcours et pour être la sœur la plus loufoque, la plus agaçante et la meilleure que je puisse souhaiter. 

    \noindent Ma petite amie, pour avoir été là dans les bons comme dans les mauvais moments. Quand tout semblait sombre, elle était là pour m'éclairer. Elle m'a écouté quand personne ne me comprenait, elle a cru en moi quand personne ne le faisait, elle m'a inspiré à être moi-même, elle m'a donné de la force et m'a aidé à tracer la voie pour mes projets futurs. 

    \noindent Mon ami Kastriot, pour toutes les questions difficiles qu'il m'a posées et qui m'ont donné envie d'en savoir toujours plus sur notre univers. 

    \noindent \textit{Je vous aime tous.} 

    \noindent Et toutes les autres personnes que j'ai rencontrées pendant mon master et qui m'ont apporté de la joie, de l'amitié et du courage : Koceila, Mathéo, Elisa, Marie Céline, Paul-Henry, Mathias, Shiva, Chris et les autres.

    \noindent Je vous remercie tous infiniment, car cette réussite est autant la mienne que la leur. Ce fut un moment mémorable pour moi et je chérirai toujours les souvenirs que j'ai créés ici.

\end{onehalfspace}
    \pagebreak

    \guillemetleft Au cours de la rédaction et de la révision de ma thèse, j'ai utilisé Antidote pour vérifier la grammaire, la
    ponctuation et le style de base, CoPilot et Antidote Reformulation exclusivement pour des suggestions de formulation, le
    formatage et la cohérence esthétique. Tout le contenu intellectuel, les arguments et les interprétations restent les miens. Ces outils avaient pour but de rationaliser les aspects mécaniques de la
    rédaction, me permettant ainsi de me concentrer sur le contenu scientifique et l'exactitude de la recherche. À chaque
    étape, je me suis assuré que toutes les idées fondamentales, les interprétations des données
    et les conclusions reposaient sur mon propre travail. \guillemetright

    \pagebreak

    \section*{Sommaire}

    \begin{doublespace}
    La caractérisation des états quantiques est essentielle au développement de la technologie quantique. Pour effectuer des calculs complexes, vérifier l’intégrité des messages codés en communication quantique, ou encore créer des systèmes de télécommunication quantique, il est essentiel de connaître l’état des qubits. L'étude sur la caractérisation de ces états quantiques est cruciale pour le développement des technologies quantiques et leur intégration à notre infrastructure existante. 

    \noindent Les méthodes de caractérisation traditionnelles, telles que la tomographie quantique, reconstruisent l’état du système quantique en effondrant l'état en effectuant des mesures projectives dans toutes les bases possibles, ce qui permet d’obtenir une description de l'état du système à l’aide de la matrice densitée. Cette méthode est une approche indirecte qui devient rapidement complexe avec des systèmes quantiques à dimension élevée. Elle ne convient pas aux systèmes nécessitant une caractérisation en temps réel, où l’évolution temporelle de l’état est cruciale, que ce soit pour la métrologie quantique ou pour la détection d’erreurs dans les ordinateurs quantiques pendant les calculs. 

    \noindent Une méthode alternative, comme les mesures faibles, présente un bon potentiel pour la caractérisation directe de l’état quantique, sans effondrement complet du système, mais, dans sa forme actuelle, elle n’est pas intégrable aux technologies quantiques. Nous proposons de réaliser des mesure faible temporels en exploitant le domaine temporel des photons, comme le pointeur, et en concevant un dispositif expérimental dans ce régime pour caractériser un état quantique facile à implanter dans un laboratoire d'optique commun. Cela permettrait une intégration directe dans notre infrastructure photonique existante et dans les technologies quantiques. 

    \noindent Cette technique permet de déterminer directement et en temps réel l’évolution d’un état quantique, en minimisant la perte d’information grâce à une intervention minimale sur le système. Dans cette étude, nous illustrons la théorie requise pour cette méthode de caractérisation et réalisons des expériences visant à mesurer la partie réelle et imaginaire de la valeur faible du système. Enfin, nous élaborant sur les applications potentielles pour ce type de caractérisation dans les technologies quantique, notamment améliorer les computations quantiques, la détection en métrologie quantique, la sécurisation des communications quantiques et les infrastructures photoniques, que ce soit dans les télécommunications quantiques. 
\end{doublespace}
    
    \addcontentsline{toc}{section}{\protect\numberline{}Sommaire}%

    \pagebreak


    \section*{Abstract}

    \begin{doublespace}

    For the development of new quantum technologies, such as
    quantum computers, quantum sensors for quantum metrology, or
    quantum telecommunications, it is essential to be able to
    characterize the quantum states of quantum systems. The 
    characterization of quantum states is an active area of research,
    as it allows for understanding, measuring and manipulating quantum systems.

    \noindent Traditional characterization methods, such as quantum
    tomography, reconstructs a quantum system's state by performing projective measurements
    on the system, followed by a reconstruction of the state using the density matrix.
    However, this approach is an indirect method that requires
    many projective measurements and does not allow for real-time 
    tracking of the quantum state evolution, as it requires the 
    use of a computational algorithm. Furthermore, it is limited 
    by the increasing complexity of high-dimensional quantum systems,
    making the reconstruction of the system's state increasingly 
    difficult and resource-intensive. Thus, it is not suitable 
    for systems requiring real-time characterization, where the 
    temporal evolution of a state is crucial, such as in quantum
    metrology or for error detection in quantum computers
    during computational processes.

    \noindent An alternative method, such as weak measurements,
    is a promising approach for the direct characterization of quantum states,
    without complete collapse of the system; however, in its current form,
    it is not integrable into quantum technologies. In this thesis, we propose to perform
    temporal weak measurements by exploiting the temporal domain of photons
    as our pointer variable, by designing an experimental device in this regime
    to characterize a quantum state that can be easy to implement in a common optics laboratory.

    \noindent In this thesis, we aim to explore the different degrees of freedom
    that would allow weak measurements to be compatible with
    future quantum technologies and our existing photonic infrastructures.
    This could enable direct real-time tracking of the temporal
    evolution of a quantum state while minimizing information
    loss through the weak measurement regime.

\end{doublespace}

    \addcontentsline{toc}{section}{\protect\numberline{}Abstract}%


    \pagebreak


    \singlespacing
    \tableofcontents


    \newpage


    \listoftables


    \addcontentsline{toc}{section}{\protect\numberline{}Liste des tableaux}%


    \pagebreak


    \listoffigures

    \addcontentsline{toc}{section}{\protect\numberline{}Liste des figures}%





    \pagebreak
    %\printnomenclature

    \section*{Liste des symboles}
    \addcontentsline{toc}{section}{Liste des symboles}

    \begin{longtable}{@{}lll@{}}
      \textbf{Symbole} & \textbf{Description} & \textbf{Chapitre} \\
      \midrule
      $\ket{0}$ & Qubit dans l'état 0 & 1 \\
      $\ket{1}$ & Qubit dans l'état 1 & 1 \\
      $\ket{\uparrow}$ & Spin vers le haut & 1 \\
      $\ket{\downarrow}$ & Spin vers le bas & 1 \\
      $\ket{H}$ & Polarisation horizontale & 1 \\
      $\ket{V}$ & Polarisation verticale & 1 \\
      $n$ & Nombre de qubits & 1 \\
      $\ket{dé}$ & État d'un dé & 1 \\
      $\ket{H}$ & Polarisation horizontale $\begin{pmatrix} 1 \\ 0 \end{pmatrix}$ & 2 \\
      $\ket{V}$ & Polarisation verticale $\begin{pmatrix} 0 \\ 1 \end{pmatrix}$ & 2 \\
      $\ket{D}$ & Polarisation diagonale $\frac{1}{\sqrt{2}} \begin{pmatrix} 1 \\ 1 \end{pmatrix}$ & 2 \\
      $\ket{A}$ & Polarisation anti-diagonale $\frac{1}{\sqrt{2}} \begin{pmatrix} 1 \\ -1 \end{pmatrix}$ & 2 \\
      $\ket{R}$ & Polarisation circulaire droite $\frac{1}{\sqrt{2}} \begin{pmatrix} 1 \\ i \end{pmatrix}$ & 2 \\
      $\ket{L}$ & Polarisation circulaire gauche $\frac{1}{\sqrt{2}} \begin{pmatrix} 1 \\ -i \end{pmatrix}$ & 2 \\
      $\ket{\psi}$ & État de polarisation & 2 \\
      $\rho$ & Matrice densité d'un système & 2 \\
      DOP & Degré de polarisation & 2 \\
      $\psi$ & Fonction d'onde d'un état quantique & 2 \\
      $S_{0,1,2,3}$ & Paramètres de Stokes & 2 \\
      $\sigma_{x,y,z}$ & Matrices de Pauli & 2 \\
      $\hat{S}$ & Opérateur arbitraire du système & 2 \\
      $\ket{\psi_i}$ & État initial arbitraire & 2 \\
      $\ket{\psi_f}$ & État final arbitraire & 2 \\
      W & Indice pour une variable de type \guillemetleft valeur faible \guillemetright & 2 \\
      $\expval{\hat{S}}_W$ & Valeur faible du système & 2 \\
      $\hat{p}$ & Observable du système & 2 \\
      $\hat{q}$ & Observable conjuguée de celle du système & 2 \\
      $\sigma$ & Écart-type de la distribution du pointeur & 2 \\
      $i$ & Nombre imaginaire $\sqrt{-1}$ & 2 \\
      $S$ & Base du système & 2 \\
      $P$ & Base du pointeur & 2 \\
      $g$ & Force de couplage entre le système et le pointeur & 2 \\
      $\ket{\bar{p}}$ & Position moyenne de l'observable du pointeur & 2 \\
      $\ket{s}$ & Valeur ou vecteur propre du système & 2 \\
      $\hbar$ & Constante de Planck réduite& 2 \\
      $\hat{U}$ & Opérateur d'interaction de von Neumann & 2 \\
      $\mathcal{O}$ & Ordre analytique d'une série & 2 \\
      $\ket{\varphi}$ & État de projection & 2 \\
      $\ket{\Psi}$ & État total incluant le système et le pointeur & 2 \\
      $\expval{\hat{\pi}}_W$ & Valeur faible de polarisation & 2 \\
      $a$ & Amplitude de probabilité pour l'état horizontal & 2 \\
      $b$ & Amplitude de probabilité pour l'état vertical & 2 \\
      $\ket{\xi}$ & Distribution du pointeur & 2 \\
      $\ket{t}$ & Domaine temporel du pointeur & 2 \\
      $\ket{\xi(t)}$ & Fonction temporelle du pointeur & 2 \\
      H & Horizontal & 2 \\
      $\hat{U}^H$ & Interaction de von Neumann appliquée à $H$ & 2 \\
      $\mathcal{H}$ & Hamiltonien quantique du système & 2 \\
      $\hat{\pi}$ & Observable de polarisation & 2 \\
      $\hat{E}$ & Opérateur d’énergie & 2 \\
      $t$ & Temps & 2 \\
      $\tau$ & Délai temporel & 2 \\
      $\ket{\varsigma}$ & État de projection de polarisation & 2 \\
      $\mu$ & Amplitude de probabilité de l’état horizontal projeté & 2 \\
      $\nu$ & Amplitude de probabilité de l’état vertical projeté & 2 \\
      $\expval{\hat{t}}$ & Position temporelle moyenne & 2 \\
      $\omega$ & Fréquence angulaire & 2 \\
      $f$ & Fréquence & 2 \\
      $\expval{\hat{\omega}}$ & Position fréquentielle moyenne & 2 \\
      $A$ & $\mu^* a$ & 2 \\
      $B$ & $\nu^* b$ & 2 \\
      $F(\omega)$ & Fonction du domaine fréquentiel du pointeur & 2 \\
      $c$ & Vitesse de la lumière & 3 \\
      PBS & Séparateur de faisceau polarisant (\guillemetleft Polarizing Beam Splitter \guillemetright) & 3 \\
      $a_0, b_0, c_0$ & Paramètres d’ajustement & 3 \\
      u.a. & Unités arbitraires & 3 \\
      MZ & Mach-Zehnder & 3 \\
      $L$ & Longueur de câble & 3 \\
      $x$ & Distance parcourue par le signal & 3 \\
      $T$ & Matrice de Jones pour une lame d’onde & 3 \\
      HWP & Lame demi-onde (\guillemetleft Half Wave Plate \guillemetright) & 3 \\
      QWP & Lame quart d’onde (\guillemetleft Quarter Wave Plate \guillemetright) & 3 \\
      $\theta$ & Orientation physique d’une lame demi-onde & 3 \\
      $\frac{\lambda}{2}$ & Lame demi-onde & 3 \\
      $\frac{\lambda}{4}$ & Lame quart d'onde & 3 \\
      $\phi$ & Orientation physique d’une lame quart d'onde & 3 \\
      $\theta$ & Angle de polarisation (2 fois l'angle d'une lame demi-onde) & 4 \\
      $\sigma_{\alpha}$ & Incertitude d'une composante $\alpha$ & 4 \\
      $\mathcal{V}$ & Visibilité des franges d'interférence & 4 \\
      $P$ & Puissance de l'état de polarisation à une fréquence donnée $f$ & 4 \\
      PSD & Densité spectrale de puissance & 4 \\
      $f_{cent.}$ & Fréquence du centroïde de l'état de polarisation & 4 \\
      $f_0$ & Fréquence incidente & A \\
      $\Delta f$ & Shift fréquentiel causé par l'effet Doppler & A \\
      $v_{s}$ & Vitesse de la source du signal & A \\
      $v_{r}$ & Vitesse du récepteur du signal & A \\
      $\Delta v$ & Différence de vitesse entre la $v_r$ et $v_s$ & A \\
      $f$ & Nouvelle fréquence observée & A 
    \end{longtable}


    %\addcontentsline{toc}{section}{\protect\numberline{}Liste des symboles}%
    %\nomenclature{$x$}{Position along the x-axis}
\nomenclchapter{1}  % First appears in Chapter 1

\nomenclature{$\alpha$}{Angle of incidence}
\nomenclchapter{2}  % First appears in Chapter 2

\nomenclature{$\psi$}{Wavefunction}
\nomenclchapter{3}  % First appears in Chapter 3

\nomenclature{$\xi$}{description}


    %\addcontentsline{toc}{section}{\protect\numberline{}Liste des symboles}%

    %\setlength{\parskip}{0pt}
    %\addcontentsline{toc}{section}{Liste des symboles}

    %\printnomenclature
    

    \setlength{\parskip}{1em}
    \pagebreak


    \onehalfspacing


    \pagenumbering{arabic}


    \thispagestyle{empty}
    \section{INTRODUCTION AUX PROCÉDURES DIRECTES POUR LES MESURES QUANTIQUES}
    %\subsection{Les notions de la mécanique quantique}
    %Cette première section 
est un rappel des utiles de base pour la mécanique 
quantique nécessaires pour la compréhension des mesures
quantiques et cette thèse. La section suit des livres de mécanique
quantique de base soit 
\cite[Griffiths]{Griffiths} et \cite[Peebles]{Peebles}.
    %\subsubsection{Notation}
    %Nous commencerons par la notation. En mécanique 
quantique, il est commun d'utiliser ce que l'on appelle 
la notation braket pour décrire les états quantiques. 
Un ket aura la structure suivante tout au long de cette 
thèse:

\begin{equation}
    \ket{nom^{condition A}_{coniditon B}}_{base}
\end{equation}

Un ket est utilisé pour représenter un vecteur. À 
l'intérieur du ket se trouve le nom du ket qui décrit ce 
qu'est le ket. Il peut y avoir des indices d'un ket 
montrés comme condition A et condition B. La condition 
A sera principalement utilisée pour décrire si un état 
est dans un état initial ou dans son état final. La 
condition B sera principalement utilisée pour décrire 
plus en détail l'état, tel qu'un état qui est un stade 
intermédiaire avec l'indice INT ou un état qui est passé 
par une condition spécifique telle qu'un type de fibre 
optique que nous pourrions écrire avec l'indice FIB. Voici
un exemple d'un ket qui represente un vecteur à $N$ dimension 
avec ${a_n}$ composantes complexe.

\begin{align}
    \ket{a} \to a = \begin{pmatrix}
        a_1\\
        a_2\\
        \vdots\\
        a_N
    \end{pmatrix}
\end{align}

Un bra est la transposée conjuguée d'un vecteur, par 
exemple le ket $\ket{a}$, sa transposée conjuguée est écrite en 
bra :

\begin{equation}
    \bra{a} \to \bar{a} = \begin{pmatrix}
        \bar{a}_1 & \bar{a}_2 \dots \bar{a}_N
    \end{pmatrix}
\end{equation}

La bar represente que ceci est le complex conjugué de $a$. 
Donc, si $a = \alpha + i\beta$, le complex conjugué s'écrit 
$\bar{a} = \alpha - i\beta$ dont $\alpha in \mathcal{R}$ et
$\beta \in \mathcal{I}$. $\mathcal{R}$ et $\mathcal{I}$
represente l'espace réel et imaginaire respectivement. Le produit scalaire de 
deux vecteurs s'écrirait comme suit :

\begin{equation}
    \bra{a}\ket{b} = \bar{a}_{1}b_1 + \bar{a}_{2}b_2 + \dots + \bar{a}_{N}b_N
\end{equation}

L'exemple précédente utilise un vecteur $\ket{b}$ avec 
les mêmes conditions que $\ket{a}$.
Une autre notation que nous verrons tout au long de cette 
thèse est celle d'une fonction représenter avec un ket, 
qui sera écrite comme suit:

\begin{equation}
    \ket{f(x)} \to f(x) = x^2
\end{equation}

Le produit scalare de deux fonction $f(x)$ et $g(x)$ de
$a$ à $b$ s'écrite comme suit:

\begin{equation}
    \bra{f(x)}\ket{g(x)} \equiv \int_{a}^{b} \bar{f(x)}g(x) dx
\end{equation}

Les observables et les opérateurs auront la notation 
suivante (nous reviendrons plus tard sur ces objets 
quantiques) :

\begin{align}
    \expval{\hat{nom}_{conditon A}} &\to observable\\
    \hat{nom}^{condition B} &\to op\acute{e}rateur
\end{align}

Un opérateur et une observable on un chapeau. 
La condition A d'un observable est pour décrire une condition
spécifique du observable et la condition B du opérateur est pour
d'écrire spécifiquement où l'opérateur s'applique. Plus sur les opérateurs et observables se suit.

    %\subsubsection{Description d'un état quantique, opérateur et observable}
    %En mécanique quantique, en comparaison de la 
mécanique classique, interprète le système 
comme une composante probabilistique dont 
l'état du système est en 
superposition avec tous ses résultats possibles.
Comme Schrödigner le décrit avec le fameux chat
de Schrödigner, un chat dans une boite avec un 
dispositif explosive sans observation le chat
peux d'être décrit en superposition soit mort et
en vie en même temps. C'est quand qu'on ouvre la
boite qu'on collapse ou réduit l'état à une de ses
valeurs possibles. Un état quantique complèt 
se décrit comme une combinaison linéaire 
de toutes ses valeurs propres possibles.

\begin{equation}
    \ket{\psi} = \sum_{i}^{N} c_i \ket{c_i}
\end{equation}

Soit un état quantique $\ket{\psi}$ avec $N$ 
dimension qui vie dans un espace 
d'Hilbert $\ket{\psi}\in\mathcal{H}$, $\ket{c_i}$ 
vecteur propres et 
$c_i$ valeur propre dont $\sum_{i}^{N} c_i = 1$. 
Un état quantique peut d'être en 
superposition avec un autre état quantique, 
soit l'état total de ce système pour l'expemple du chat de Schrödigner, figure \ref{fig:cat}:

\begin{equation}
    \ket{\psi} = \ket{Vie} + \ket{Mort}
\end{equation}

Soit $\ket{Vie}$ l'état pour le chat soit en vie et $\ket{Mort}$ 
l'état pour le chat soit mort après observer. 

\begin{figure}[h]
    \centering
    \includegraphics[width=1.0\textwidth]{shrodigner_cat.pdf}
    \caption{Démonstration visuelle de l'expérience de 
    pensée du chat de Schrödigner. a) Le chat est dans 
    une boîte sans élément destructeur. b) Nous plaçons 
    un dispositif explosif et fermons la boîte. Le chat 
    se trouve maintenant dans une superposition où il 
    est mort ou vivant jusqu'à ce qu'il soit observé. c) 
    L'état où le chat est toujours vivant lorsqu'il est 
    observé. d) L'état où le chat est mort lorsqu'il est 
    observé. Ceci illustre l'idée de superposition 
    d'un état quantique. }
    \label{fig:cat}
\end{figure}

L'information d'un état quantique 
est décrit par sa fonction d'onde soit $\psi(r,t)$. 
Cette fonction est une solution de l’équation de 
Schrödigner (soit pour une particle de masse $m$):

\begin{equation}
    i\hbar\diffp{\psi(r,t)}{t} = -\frac{\hbar^2}{2m}\grad^{2}{\psi(r,t)} + V(r)\psi(r,t)
\end{equation}

Une équation différentielle avec la solution 
suivante pour une particule:

\begin{equation}
    \psi(r, t) = e^{-\frac{iHt}{\hbar}}\psi(r)
\end{equation}

Soit $\hat{H}$ l'Hamiltonien du système 
$\hat{H} = \hat{K} + \hat{V}$,  $\hat{K}$ 
l'énergie cinétique d'une particule et $\hat{V}$ son énergie potentiel.
La fonction d'onde contient l'information de comment 
l'état quantique évolue temporel et spatialement. 
Notons le term suivant $\hat{U} \equiv e^{-\frac{i\hat{H}t}{\hbar}}$ 
un opérateur temporel qui décrit 
l'évolution de l'état quantique dans le domaine 
temporel. Remarque aussi que le chapeau « $\hat{}$ » sur l'Hamiltonien, 
énergie cinétique et potentiel. On décrit ces objects comme un opérateur aussi.
Ce dernier, est une fonction mathématique quand 
appliqué sur un état quantique, le résultat est 
soit un nouvel état quantique ou un résultat 
scolaire.

\begin{align}
    \hat{A}\ket{\psi} &= \ket{\phi} & \hat{A}\ket{\psi} &= a\ket{\psi} 
\end{align}

Soit un opérateur $\hat{A}$ qui applique sur 
un état $\ket{\psi}$ qui peux soit donné un nouveau
état $\ket{\phi}$ ou une valeur propre $a$ du 
opérateur appliqué.   
    %\subsubsection{Observable et opérateur}
    %\input{section/section_1/section_0_1/section_0_1_3.tex}
    %\subsubsection{Mesure quantique}
    %Enfin que la fonction d'onde soit la fonction contient
l'information du système quantique peux d'être déterminer 
expériemntalement. Mais comment une fonction d'onde, 
une fonction mathématique
qui se distribue dans l'espace peux d'être déterminer?
Rappel qu'on a mentionné que l'état quantique est
interprèter statistiquement. La règle de Born définie que
la distribution probabilistique pour soit la position 
d'une particule
d'un état quantique $\psi(r,t)$
entre des espaces $a$ et $b$ se trouve:

\begin{equation}
    Prob = \int_{a}^{b} |\psi(r,t)|^2 dr
\end{equation}

Et pour sa position temporel on intègre par $dt$. Pour la quantité
de mouvement d'une particule $p$ on prend la transformation de Fourier de sa
fonction d'onde et on intègre par la quantité de mouvement.

\begin{align}
    \psi(p,t) &= \frac{1}{\sqrt{2\pi\hbar}} \int_{-\infty}^{\infty} e^{-\frac{ipr}{\hbar}} \psi(r,t) dr\\
    Prob_p &= \int_{-\infty}^{\infty} |\psi(p, t)|^2 dp
\end{align}

L'interprétation statistique de la mécanique quantique 
entraîne une incertitude sur la position et la quantité de 
mouvement de l'état, ainsi que sur son temps et son 
énergie, décrite par le principe d'incertitude 
d'Heisenberg.

\begin{align}
    \Delta x \Delta p &\geq \frac{\hbar}{2}\\
    \Delta t \Delta E &\geq \frac{\hbar}{2}
\end{align}

Ce qui nous amène à la question de savoir comment 
nous pouvons obtenir des informations sur le système 
quantique, soit sa position ou d'autre observation réel
qu'on puisque mesurer. Cela se fait à l'aide d'observables. Les 
observables sont des valeurs d'espérance hermitienne 
observable d'une information sur un état quantique, 
telle que sa position ou sa quantité de mouvement, avec 
un résultat réel, qui s'écrit comme suit :

\begin{equation}
    \expval{\hat{Q}} = \bra{\psi}\hat{Q}\ket{\psi}
\end{equation}

Soit pour un observable $\hat{Q}$ dont sa valeur d'espérance
$\expval{\hat{Q}}$ se trouve par un produit de scalare du 
vecteur de la fonction d'onde $\psi$. Ensuite, 
comment mesurer expérimentalement ces observables ? 
Les sections suivantes décriront les mesures quantiques 
traditionnelles, puis une technique que nous explorerons 
dans le cadre de cette thèse, à savoir les mesures faibles. 
    %\subsection{Compréhension analogique des mesures quantiques}
    \subsection{Introduction}

\begin{doublespace}
    L’émergence des technologies quantiques, telles que l'informatique 
    quantique et les communications quantiques, possèdent le potentiel 
    de nous faire entrer dans une nouvelle révolution technologique. Ces 
    avancées technologiques permettent d’envisager des temps de calcul, 
    nettement plus rapides que ceux d'un ordinateur classique 
    \cite{Feynman1982}. Contrairement à un ordinateur classique qui 
    utilise des bits $0$ et $1$ pour effectuer ses calculs, un ordinateur 
    quantique peut exister simultanément dans plusieurs états. En se 
    reposant sur un état quantique, le qubit est capable de prendre 
    $\ket{0}$, $\ket{1}$, ou une combinaison linéaire complexe 
    $a\ket{0} + b\ket{1}$ (superposition) de ces deux nombres. Cela 
    signifie qu’un ordinateur quantique peut contenir simultanément 
    $2^n$ états possibles. De plus, il sera possible de crypter nos 
    communications grâce à une cryptographie ultrasécurisée pour des 
    infrastructures de télécommunication 
    \cite{quantumcomputing40years, QTintelecomindustry}. D’autres 
    domaines, tels que la métrologie et l'industrie médicale, bénéficient 
    également à ces technologies quantiques, dont son implémentation 
    permet d’atteindre une précision et une résolution accrues dans la 
    détection de champs gravitationnels ou magnétiques, ce qui s’avère 
    crucial pour la navigation \cite{photonicInformation,QTapplImpl}. En 
    imagerie médicale, ces avancées contribuent à améliorer 
    significativement les diagnostics \cite{QuantumMedicalNexus}.
    
    \noindent En effet, pour connaitre le résultat d'un qubit ou même le 
    message lors d'un protocole de communication quantique, le 
    développement des technologies quantiques repose sur la capacité à 
    mesurer et à caractériser les états quantiques avec une précision et 
    une fiabilité accrues \cite{DiVincenzo_2000,QTapplImpl}. Dans ce 
    contexte, le domaine de la photonique quantique occupe une place 
    centrale grâce aux propriétés étonnantes des photons. Les photons se 
    distinguent par leur faible décohérence, divers degrés de liberté 
    (polarisation, domaine positionnel, quantité de mouvement, temporel 
    et fréquentiel), et leur intégration naturelle dans les 
    infrastructures optiques existantes \cite{browne2017quantumopticsquantumtechnologies,photonicInformation}. 
    L’état de polarisation des photons, sous forme de qubit, possède des 
    caractéristiques idéales pour de nombreuses applications, notamment 
    la télécommunication quantique sécurisée, l’imagerie quantique à 
    haute résolution, la métrologie quantique et d’autres encore, comme 
    mentionné \cite{QTintelecomindustry,metrology,Wang_2019,QTapplImpl}.

    \noindent Par conséquent, il est crucial d’effectuer des mesures afin 
    de caractériser avec précision les états quantiques pour soutenir ces 
    diverses technologies. Cependant, effectuer des mesures quantiques 
    pose un défi en raison de la nature aléatoire de la théorie. 
    Contrairement à la mécanique classique, il est impossible de mesurer 
    simultanément deux variables complémentaires (ex: le moment d’arrivée 
    et la fréquence spectrale) avec une précision absolue à cause du 
    principe d’incertitude d’Heisenberg. Dans ce cadre, on décrit que 
    l’état du système comme se trouvant dans une superposition de tous 
    ses états possibles. Une mesure ou une interaction avec le système 
    quantique cause une perturbation qui fait s'effondrer l’état dans 
    l’un de ses états propres possibles. Une fois perturbé à une de ces 
    valeurs propres, l’état du système demeure inchangé. La figure 
    \ref{fig:dice} ci-dessous illustre ce concept. 

    \noindent Les techniques conventionnelles, comme la tomographie 
    quantique, permettent une reconstruction complète des états à l’aide 
    de plusieurs mesures projectives prédéterminées. Toutefois, elles 
    deviennent rapidement inadaptées aux systèmes de grande dimension en 
    raison de leur coût computationnel et expérimental exponentiel. Comme 
    ces techniques reposent sur un grand nombre de mesures de projection, 
    ceci les rend inadaptées à certaines applications nécessitant des 
    mesures en temps réel.

    \begin{figure}[!htbp]
        \centering
        \includegraphics[width=1.0\textwidth,page=1]{FIGURES.pdf} % Selects page 2
        \caption{Envisageons maintenant que nous lancions un dé. Avant de 
        lançer un dé, nous supposons qu’il se trouve dans un état de 
        superposition $\ket{\text{dé}} = c_1\ket{1} + c_2\ket{2} + c_3\ket{3} + c_4\ket{4} + c_5\ket{5} + c_6\ket{6}$, 
        où tous les états propres possibles $\{ \ket{1},\ket{2},\ket{3},\ket{4},\ket{5},\ket{6} \}$ 
        et leurs coefficients de probabilité $c_1, c_2, c_3, c_4, c_5, c_6$ 
        se trouvent dans notre main. Une fois lancés sur la table et en 
        observant leur résultat, nous disons que le système s’effondre vers 
        l’une de ses valeurs possibles, comme l’état $\ket{5}$, qui a une 
        probabilité de $\frac{1}{6}$ de se produire. Une fois qu’il s’est 
        effondré, les mesures ultérieures du système restent les mêmes. 
        L’objectif des mesures quantiques est de caractériser complètement 
        l’état du système quantique du dé. Comme nous sommes limités aux 
        mesures projectives, nous discuterons des techniques possibles, 
        telles que l’effondrement continu du système via la tomographie et 
        la reconstruction indirecte de l’état quantique, ainsi que notre 
        méthode alternative utilisant la mesure faible pour la caractérisation 
        directe de l’état quantique.}
        \label{fig:dice}
    \end{figure}
    

    \noindent Une approche alternative consiste à utiliser des mesures 
    faibles qui permettent d’extraire des informations sur un état 
    quantique directement sans entraîner son effondrement complet. Ce 
    dernier repose sur le modèle de mesure quantique de von Neumann 
    \cite{vonNeumann} et exploite un pointeur couplé au système, dont le 
    déplacement minimal appliqué sur le système est proportionnel à un 
    observable complexe nommé la \guillemetleft valeur faible \guillemetright. 
    Cette technique a été introduite par Aharonov, Albert et Vaidman 
    (AAV) \cite{Aharonov}. Bien que les mesures faibles aient été 
    largement étudiées en théorie, leur mise en œuvre expérimentale dans 
    le domaine temporel des photons est relativement peu explorée, en 
    particulier dans le contexte d’applications pratiques pour des 
    technologies quantiques.

\end{doublespace}



\subsection{Motivation de la thèse}

\begin{doublespace}
    
    Les recherches et applications sur les mesures faibles ont démontré 
    leur potentiel, notamment la théorie des mesures quantiques, 
    l’informatique quantique, la télécommunication optique, etc. 
    \cite{Lundeen_Resch,QED,OpticalNetworks,Lundeen_thesis,Brunner_2004}. 
    Elles peuvent même se révéler plus efficaces que les méthodes 
    traditionnelles en certains cas. Dans l'étude: \cite{WeakorStd}, ils 
    comparent les mesures faibles à l'interférométrie standard pour 
    mesurer de petits décalages de phase longitudinaux. La partie 
    imaginaire de la valeur faible contient l'information complexe du 
    système quantique, soit l'ellipticité d'un état de polarisation. 
    Cette méthode permet de mesurer directement cette information, ce 
    qui s’avère plus efficace que l'interférométrie standard. De plus, 
    d’autres études montrent que les mesures faibles peuvent surpasser 
    des méthodes traditionelles \cite{Magaña-Loaiza_2017,Jeff_outperform}. 

    \noindent Cette thèse se concentre sur la caractérisation d’un état 
    de polarisation dans un système photonique en s’appuyant sur les 
    travaux: \cite{Lundeen_Direct_Measurement,Lundeen_Bamber,Hairiri,Guilleaum}. 
    Ils ont démontré la faisabilité de cette approche en utilisant des 
    mesures faibles, en exploitant le mode spatial et/ou quantité de 
    mouvement des photons comme pointeur. Leur méthode repose sur 
    l'observation de ces variables complémentaires du pointeur (position 
    et quantité de mouvement) permettant d'extraire respectivement les 
    parties réelles et imaginaire de la valeur faible pour caractériser 
    des états de polarisation complètement et directement.
    
    \noindent Toutefois, ces méthodes présentent certaines limites, 
    en vue d'applications et d'intégration dans des technologies photoniques 
    quantiques exigeant une configuration en espace libre. Elles reposent 
    généralement sur l’utilisation de cristaux BBO (bêta-borate de baryum) de 
    taille spécifique pour implémenter l’interaction faible 
    \cite{Hairiri,Guilleaum,Guilleaum_thesis}. Cette exigence rend leur 
    flexibilité difficile pour s’adapter à divers systèmes intégrés.

    \noindent Comme les photons possèdent différents degrés de liberté, 
    nous proposons d’utiliser le mode temporel comme pointeur pour la 
    caractérisation de l’état de polarisation à l’aide de méthodes 
    interférométriques. Cette approche offre une implémentation plus 
    facile, puisqu’elle ne nécessite pas de composants optiques 
    spécifiques, tels qu’un cristal. Il ne faut que des miroirs pour 
    interagir avec le système de manière contrôlable. Cette flexibilité 
    rend possible l’intégration de mesures faibles temporelles dans 
    des technologies quantiques \cite{kaneda2018highefficiencysinglephotongenerationlargescale,Dai_2020}. 
    Elles peuvent également être facilement implantées dans nos systèmes 
    optiques existant, par exemple dans les télécommunications à fibre 
    optique \cite{OpticalNetworks}. 
    
    \noindent Notre objectif est d’évaluer directement la composante 
    réelle et imaginaire de la valeur faible provenant d'un pointeur 
    temporel, ce qui permet une description directe de l’état de 
    polarisation. Cette méthode utilise la polarisation comme base 
    quantique, car elle est facile à contrôler et à mettre en pratique en 
    laboratoire. Bien que certaines études aient déjà exploré des mesures 
    faibles dans ce régime, elles se sont principalement concentrées 
    à partir d’un délai fréquentiel \cite{JohnCHowellFrequencyCombs,Salazar,OpticalNetworks,Steinberg_prob_div}. 
    Cette thèse vise à améliorer la méthode de caractérisation directe 
    des états quantiques, contribuant ainsi à l’avancement des 
    technologies quantiques et à l’exploration de nouvelles opportunités 
    dans les domaines scientifique et industriel.

    \noindent Pour résumé, on propose une nouvelle méthode de mesure 
    faible qui exploite le mode temporel des photons pour caractériser 
    des états de polarisation. Cette méthode offre un cadre plus robuste 
    pour son implantation dans les technologies quantiques. Dans la suite 
    de cette thèse, nous explorerons notre compréhension des mesures 
    faibles, de leur réalisation expérimentale et de leurs applications 
    dans l’informatique quantique, la communication quantique et d’autres 
    domaines. Nous commencerons par une analyse des principes 
    fondamentaux des mesures quantiques. Ensuite, nous discuterons les approches 
    tomographiques et des mesures faibles en termes de leurs avantages et 
    limites. Nous présenterons ensuite une méthode innovante, 
    spécialement conçue pour les systèmes photoniques, qui permettra de 
    collecter le plus d'information possible sur l'état quantique tout en 
    limitant l’influence de la mesure (interaction) sur le système. Cette 
    méthode sera évaluée à travers des expériences en laboratoire. Ensuite, 
    nous discuterons des résultats et des implications pour les 
    technologies quantiques émergentes. 
\end{doublespace}
    
    %\subsection{La procédure des mesures directes via mesure faible}	
    %\begin{doublespace}
    
    Traditionnellement, la tomographie quantique 
    est utilisée pour reconstruire la fonction 
    d’onde d’un état quantique à partir d’un 
    ensemble de mesures. Ce processus consiste 
    à effectuer des mesures de projection sur un 
    multiples d’état quantique en utilisant des 
    bases orthogonales variées. Les résultats 
    sont ensuite traités par un algorithme 
    complexe qui reconstruit indirectement la 
    fonction d’onde. Dans le cadre de Paul 
    Kwiat et ses collaborateurs, ils ont 
    développé des protocoles de tomographie 
    d’état photoniques permettant de 
    caractériser précisément l'état de 
    polarisation sous forme d'un qubit \cite{Kwiat}. Pour 
    ce faire, ils mesurent l’état du photon 
    dans plusieurs bases, soit $\{\ket{H}, \ket{V}\}$, 
    $\{\ket{D}, \ket{A}\}$ et 
    $\{\ket{R}, \ket{L}\}$, (polarization horizontale, 
    vertical), (diagonal, anti-diagonal) et 
    (circulaire droit, gauche) respectivement. 
    À partir de ces mesures, il est possible de 
    reconstruire la matrice densité de l’état 
    quantique. La matrice densité est observable
    qui contient tout l'information de l'état quantique
    comme la fonction d'onde mais inclue des propriétés
    statistique que le système pourrais possèder. Elle
    prend la 
    forme $\hat{\rho} = \ket{\psi}\bra{\psi}$
    pour un état $\ket{\psi}$ et on peut 
    aisément vérifier sa pureté de l'état en 
    prenant la trace $Tr(\rho)=1$. Plus d'informations à ce 
    sujet sous peu. Pour approfondir nos 
    connaissance, considérons un exemple 
    plus concret. Supposons un photon préparé 
    dans l’état de polarisation arbitraire soit

    \begin{equation}
        \ket{\psi} = a\ket{H} + b\ket{V}
    \end{equation}

    avec $a$, $b \in \mathcal{C}$ et $|a|^2 + |b|^2 = 1$
    La matrice densité associée à cet état pur s’écrit

    \begin{equation}
        \hat{\rho} = \begin{pmatrix}
            |a|^2 & a\bar{b}\\
            \bar{a}b & |b|^2
        \end{pmatrix}
    \end{equation}

    Expérimentalement, le but est de déterminer les 
    coefficients de la matrice $a$ et $b$. Pour ce faire,
    il faut mesurer les probabilités de 
    détection dans différentes bases de 
    polarisation, soit en orientent un polariseur 
    ou avec un séparateur de faisceau polarisant, 
    selon les bases de notre état $\ket{\psi}$ soit
    $\{\ket{H}, \ket{V}\}$. La fraction de photons 
    détectés en sortie $\ket{H}$ correspond 
    alors à $|a|^2$ et en sortie $\ket{V}$ 
    correspond à $|b|^2$. Pour accéder aux 
    termes d’interférence, comme $a\bar{b}$, 
    il faut réaliser des mesures dans des bases 
    complémentaires, telles que $\{\ket{D}, \ket{A}\}$ 
    pour la polarisation diagonale et antidiagonale, 
    ou $\{\ket{R}, \ket{L}\}$ pour la 
    polarisation circulaire. De manière 
    pratique, on insère des lames quart-d’onde 
    et demi-d’onde pour transformer la base $\{\ket{H}, \ket{V}\}$
    vers l’une de ces bases, puis on redirige à 
    nouveau les photons vers un séparateur de 
    faisceau polarisant. Les différences 
    d’intensité observées dans ces diverses 
    configurations expérimentales permettent 
    de reconstruire les éléments de matrice 
    densité. Cette démonstration illustre la 
    puissance de la tomographie quantique, 
    tout en soulignant sa complexité et les 
    ressources expérimentales nécessaires à sa 
    mise en œuvre pour des états de dimension 
    plus élevée. Dans le formalisme de 
    Paul Kwiat, cette matrice densité peut 
    également être exprimée en termes de 
    paramètres de Stokes. Les paramètres de 
    Stokes décrivent complètement l’état de 
    polarisation, et ils sont liés aux 
    probabilités de détection dans 
    différentes bases de polarisation. 
    Les paramètres sont défini par:

    \begin{equation}
        S = \begin{pmatrix}
            S_0\\
            S_1\\
            S_2\\
            S_3
        \end{pmatrix}
        = \begin{pmatrix}
            P_{\ket{H}} + P_{\ket{V}}\\
            P_{\ket{H}} - P_{\ket{V}}\\
            P_{\ket{D}} - P_{\ket{A}}\\
            P_{\ket{R}} - P_{\ket{L}}
        \end{pmatrix}
    \end{equation}

    Concrètement, $S_0$ représente 
    l’intensité ou probabilité 
    totale du faisceau, $S_1$ représente 
    la différence d’intensité entre les 
    polarisations $\ket{H}$ et $\ket{V}$, $S_2$ 
    représente la différence d’intensité entre 
    les polarisations $\ket{D}$ et $\ket{A}$ et
    $S_3$ représente la différence d’intensité 
    entre les polarisations $\ket{R}$ et $\ket{L}$.
    En notant  les intensités (ou probabilités de mesure) 
    pour chacune de ces bases, on reconstruit la matrice
    densité par les probabilités trouvé avec:

    \begin{equation}
        \hat{\rho} = \frac{1}{2}\sum_{i=0}^{3} S_i \sigma_i
    \end{equation}

    Soit $\sigma_i$ sont les matrices de Pauli défini comme suit:

    \begin{equation}
        \hat{\sigma}_0 = \begin{pmatrix}
            1 & 0\\
            0 & 1
        \end{pmatrix}
        \hat{\sigma}_1 = \begin{pmatrix}
            1 & 0\\
            0 & -1
        \end{pmatrix}
        \hat{\sigma}_2 = \begin{pmatrix}
            0 & 1\\
            1 & 0
        \end{pmatrix}
        \hat{\sigma}_3 = \begin{pmatrix}
            0 & -i\\
            i & 0
        \end{pmatrix}
    \end{equation}


Pour un état pur, comme notre example, on a la 
propriété , ce qui se traduit par une cohérence 
quantique maximale. En revanche, un état mixte 
se décrit par une matrice densité statistique, 
somme pondérée de matrices densité pures:

\begin{equation}
    \hat{\rho}_{mixte} = \sum_{i}^{N} p_i\hat{\rho}_i
\end{equation}


avec $N$ états avec des probabilités $p_i$ pour 
chaque matrice densité $\hat{\rho}_i$ dont $\sum_{i}^{N} = 1$. 
Dans ce cas, $Tr(\hat{\rho}_{mixte}) < 1$.
Ainsi, que la distinction de la pureté de la matrice densité peut se définir à 
partir de la trace de la matrice densité, 
l’état pur maintient une cohérence parfaite, 
tandis qu’un état mixte est issu d’un mélange 
statistique d’états. Les protocoles de tomographie 
quantique proposés par Kwiat, 
permettent de déterminer empiriquement ces 
coefficients (à partir des paramètres de Stokes) 
expérimentalement, et peut donc reconstruire 
et caractériser la 
matrice densité complète d’un photon. Cependant, 
cette approche présente des 
inconvénients majeurs: elle est indirecte, 
complexe et exige un traitement algorithmique 
intensif, ce qui limite son applicabilité, 
notamment dans les systèmes dynamiques ou 
les environnements industriels.

    Une alternative intéressante réside dans les mesures 
    faibles, une méthode directe permettant d'accéder à la 
    fonction d'onde d'un système quantique. Introduite par 
    Aharonov, Albert et Vaidman (AAV) dans les années 1980, 
    cette approche repose sur une interaction contrôlée 
    entre un pointeur et un système quantique \cite{Aharonov}. 
    Contrairement aux mesures fortes, qui provoquent un 
    effondrement complet de la fonction d'onde et détruisent 
    la superposition quantique, les mesures faibles 
    préservent cette superposition en minimisant la 
    perturbation du système. Pour 
    illustrer, la différence entre mesures fortes et 
    faibles peut être représentée par une impulsion 
    gaussienne où les états de base sont séparés soit 
    fortement, soit faiblement. Une figure montrant une 
    telle impulsion permettrait de clarifier comment les 
    mesures faibles minimisent l'interaction tout en 
    extrayant des informations précises.
\end{doublespace}
    
\begin{figure}[!h!t!p!b!]
    \centering
    \includegraphics[width=0.9\textwidth]{force_de_mesure.pdf}
    \caption{Représentation visuelle de la 
    différence entre une mesure indirecte et directe sur 
    un système quantique. Supposons un état quantique 
    initialement $\ket{\psi_i}$ avec des états de base $\{ \ket{0};\ket{1} \}$ avec 
    une dispersion d'une distribution de probabilité de l'état $\sigma$ et soit $\delta$ 
    la force de séparation de l'interaction effectuée. a) L'état 
    quantique subit ce que nous appelons une mesure 
    « forte » où l'interaction avec l'état quantique 
    sépare les états de base plus que la distribution de 
    probabilité $\delta \gg \sigma$. Aucune information ne peut donc être 
    récupérée. b) L'état quantique subit une interaction 
    plus faible où ses états de base sont séparés de 
    façon très inférieure à l'écart de distribution des 
    probabilités $\delta \ll \sigma$. L'information réside alors dans le 
    chevauchement de ces états de base, qui peut être 
    récupéré à l'aide d'une mesure projective.
    }
    \label{fig:force_de_mesure}
\end{figure}

\begin{doublespace}
    Le modèle von Neumann des mesures quantiques fournit 
    le cadre théorique pour comprendre les mesures faibles. 
    Dans ce modèle, le système quantique et le pointeur 
    sont intriqués via un opérateur d’interaction, 
    permettant d’extraire des informations sur la fonction 
    d'onde. Dans le contexte des mesures faibles, la force 
    de l'interaction est choisie pour que le déplacement du 
    pointeur reste plus petit que la largeur de la 
    distribution des probabilités, ce qui permet de mesurer 
    directement les composantes réelles et imaginaires de 
    la valeur faible associée à un état quantique \cite{vonNeumann,Lundeen_Resch,Lundeen_Direct_Measurement}.
    La figure suivante démontre une répresentation du modèle de von Neumann
    utilisée pour les mesures faibles.

    \begin{figure}[!h!t!p!b!]
        \centering
        \includegraphics[width=0.9\textwidth]{fuel_gauge.png}
        \caption{place holder}
        \label{fig:force_de_mesure}
    \end{figure}
    
    Les mesures faibles sert à un oeuil de juda pour 
    le monde quantique \cite{Peephole}. Ça nous permettons de pertubé
    le système le plus faiblement possible pour obtenir de 
    l'information sur le système quantique. 
    Jeff Lundeen et ses collaborateurs ont joué un rôle 
    crucial dans l’avancement de cette technique. Leur 
    approche typique utilisait un cristal BBO mince pour 
    ajuster la force d’interaction, permettant de réaliser 
    des mesures faibles sur des impulsions lumineuses 
    polarisées. Ils ont notamment démontré que les mesures 
    faibles pouvaient servir à caractériser la matrice 
    de densité et à valider des prédictions fondamentales de 
    la mécanique quantique \cite{Lundeen_Bamber,Lundeen_Direct_Measurement,Lundeen_Resch,Lundeen_thesis}. 
    L’adoption des mesures faibles repose sur plusieurs 
    avantages clés : elles réduisent les perturbations 
    induites sur le système, préservent la cohérence 
    quantique et permettent une approche directe et 
    intuitive pour caractériser des états quantiques.
    
\end{doublespace}
    

    %Le monde a évolué très rapidement au cours du siècle 
dernier grâce aux progrès de la physique et de la technologie. 
Nous pouvons désormais envoyer des messages à partir d'un petit 
appareil qui tient dans notre poche, calculer des simulations qui 
prendraient des siècles à faire à la main, envoyer des humains dans 
l'espace, guérir des maladies et faire progresser notre civilisation 
dans son ensemble. Tout cela ne serait pas possible sans des 
mesures précises et des technologies de pointe pour nous aider 
à calculer nos résultats. Ce dernier, il s'agissait de jeter un peu de lumière 
sur les mesures générales en physique. Lorsque nous pensons aux 
mesures, ce qui nous vient généralement à l'esprit, ce sont des 
mesures classiques, telles que la mesure de la distance entre 
deux points. La technologie nous aide à effectuer ces mesures et 
à résoudre les problèmes qu'elles posent. Par exemple, nous sommes 
capables de calculer la trajectoire pour avoir envoyé quelqu'un sur la 
lune et le ramener. La précision requise pour ces calculs est 
considérable et nous disposions de la technologie pour nous aider 
à y parvenir à l'aide de simulations et de calculs. Cependant, la 
technologie dont nous disposons aujourd'hui est limitée, car elle 
ne peut résoudre qu'un problème d'une certaine complexité. Les 
ordinateurs, par exemple, ne peuvent pas résoudre les problèmes 
NP-complets, NP-difficiles ou les problèmes de nature quantique. 
Pour poursuivre cet exemple, Richard Feynman a proposé en 1981 
une machine de Turing universelle capable de résoudre ces types 
de problèmes dont la complexité est exponentielle. Il s'agit d'un 
ordinateur quantique \cite{Feynman1982,quantumcomputing40years}. David P. DiVincenzo finit 
par écrire un article décrivant les critères d'un ordinateur 
quantique, l'un d'entre eux étant bien sûr les mesures, et plus 
précisément les mesures quantiques \cite{DiVincenzo_2000}. Ce 
n'est qu'un exemple de la façon dont les améliorations des mesures 
quantiques pourraient contribuer à améliorer notre technologie et, 
par conséquent, notre société. Donc, cette thèse sert à discuter et démontrer l'importance sur les mesures quantiques pour des avancements sur les technologies quantiques. Cependant, commençons par discuter les notions de mesure quantique. Les mesures en mécanique quantique ont troublé les physiciens 
et les philosophes depuis la naissance de la théorie quantique 
elle-même. Soit Einstein, Podolsky et Rosen (EPR) 
se questionnaient sur la complétude de la théorie disant 
qu'il y a des valeurs «cachées» \cite{EPR}. Même John S. Bell (celui qui a démontré que la mécanique quantique est une théorie complète et non déterministe) a écrit un article contre les 
mesures quantiques en 1990 \cite{Bell_1990}. Certains considèrent les mesures en mécanique 
quantique comme un problème et ceux qui les pratiquent comme 
des instrumentalistes \cite{MeasurementProb, MeasurementProb2}. Andrew N. Jordans et Irfan A. Siddiqi décrit ce sujet dans 
leur livre, où ils abordent les implications historiques des 
mesures quantiques et certains points de vue philosophiques \cite{JordanBook}. 
Ils poursuivent en disant que les mesures sont un raccourci de la 
façon dont nous interagissons avec le monde. Nous pouvons 
obtenir de l'information sur notre monde seulement par des 
observations. Je suis d'accord avec cela, car si nous pensons 
aux sciences humaines par exemple, et plus particulièrement à 
l'être de soi-même, nous ne pouvons comprendre qu'une partie 
d'une personne par l'observation, mais pas la totalité de la 
personne. Tout comme nous pouvons estimer que le champ électrique d'un photon se propage comme une onde plane, mais 
nous ne pouvons qu'observer son intensité et dériver une approximation de son champ. La mécanique 
quantique est un peu comme ça aussi, nous pouvons obtenir 
de l'information du système seulement par des mesures, mais nous ne pouvons jamais 
savoir complètement toute l'information du système surtout 
avec une seule mesure. En comparaison avec la mécanique 
classique, la mécanique quantique est une théorie 
probabiliste. Chaque mesure qu'on fait sur l'état du 
système, elle se réduit à une de ces valeurs possibles. 
Classiquement, considérons un dé à six faces. Avant de lancer le dé et
obtenir un résultat, l’état du dé peut s’exprimer par une superposition de
toutes ses valeurs possibles (de $1$ à $6$). Après le lancer, le dé est tombé sur une
face et son état est parfaitement défini comme étant la valeur inscrite sur cette
face. Définissons l’état du dé avant le lancer comme étant la superposition
des états propres représentant chaque face du dé soit $\ket{1} , \ket{2} ... \ket{6}$ avec des
probabilités égales de $\frac{1}{\sqrt{6}}$ . Après le lancer, le 
dé se trouve à être dans l’une $6$
de ses états propres. 

\begin{figure}[h]
    \centering
    \includegraphics[width=1.0\textwidth]{dice.png}
    \caption{L’état du dé en superposition et puisque réduit à une de ses valeurs}
    \label{fig:dice}
\end{figure}

Donc, en mécanique quantique, l’état d’un système 
quantique est une superposition de tous ses vecteurs. 
Lors d’une mesure, l’état se réduit à une de ces valeurs 
propres. L'interprétation statistique introduit une sorte d'indétermination dans la mécanique quantique, car même si l'on sait tout ce que la théorie a à nous dire sur le système (soit sa fonction d'onde), on ne peut pas prédire avec certitude le résultat d'une simple expérience visant à mesurer, comme nous l'avons mentionné plus haut \cite{Griffiths, JordanBook}. La mécanique quantique n'a à offrir que des informations statistiques sur les résultats possibles. Cette indétermination a profondément troublé les physiciens et les philosophes, et il est naturel de se demander s'il s'agit d'un fait de la nature ou d'un défaut de la théorie \cite{EPR}. En raison de cette nature probabiliste 
de la mécanique quantique, certaine, soutiennent que 
l'action de mesurer un système quantique est 
conceptuellement problématique et remet en question 
notre compréhension de la réalité physique, comme nous avons mentionné en introduisant les mesures quantiques \cite{Bell_1990,Wigner,Wakita1960MeasurementIQ}. Un autre aspect des mesures en mécanique quantique est le principe d'incertitude d'Heisenberg. Par une analogie du livre de Griffiths, imaginons que vous et un ami teniez les deux extrémités d'une corde à sauter et que vous commenciez tous les deux à la secouer au hasard. Si l'on demande : quelle est la position de la corde ? D'une part, elle n'est pas n'importe où, mais elle est au moins localisée à une certaine distance de tant de mètres. On pourrait poser la même question à propos de sa vitesse ou sa quantité de mouvement. Si nous localisons encore plus sa position, la quantité de mouvement devient plus dispersée et vice versa. Ceci se dit pour toutes les ondes. Ce dernier c'est le concept du principe d'incertitude : on ne peut pas connaître avec précision la position et la quantité de mouvement d'une particule. Pour en revenir aux concepts statistiques de la mécanique quantique, lorsque qu'on disons que la corde à sauté est « dispersée » dans sa position ou de sa quantité de mouvement, cela fait référence au fait que les mesures effectuées sur des systèmes préparés à l'identique ne donnent pas des résultats identiques. Vous pouvez, si vous le souhaitez, construire un état tel que les mesures de position seront très proches les unes des autres, mais vous en paierez le prix quantité de mouvement \cite{Griffiths}.
Cependant, 
les mesures en mécanique 
quantique ne sont pas contradictoires ni un problème \cite{Peebles}. Elles 
suivent toujours les suivants: 1) qu'un système quantique 
est décrit par une fonction d'onde représentée par un 
vecteur ce qu'on appelle un état quantique. Cet état est 
linéaire et vit dans un espace Hilbert qui compromit tous 
ses états propres possibles. Tout ce qu'on veut savoir du 
système est décrit par cet état. 2) Chaque attribut du système qui peut être mesuré est associé avec un opérateur, ce qu'on appelle un observable. 3) L'état quantique est complet \cite{Bell1966}. 
Donc avec cela, nous voulons être capables d'effectuer des mesures en mécanique quantique pour des avancements sur nos connaissances et la technologie quantique. Une proposition pour mesurer la fonction d'onde d'un état quantique est ce qu'on appelle une tomographie quantique. Une tomographique quantique vise à reconstruire la fonction d'onde avec les résultats de chaque mesure et à l'aide d'un algorithme de reconstruction nous pouvons obtenir la fonction d'onde indirectement. Nous disons «indirecte» parce que c'est une reconstruction de la fonction d'onde et non une mesure directe de la fonction d'onde \cite{Peebles}. 
En photonique quantique, utilisant l'état de polarisation 
comme état quantique, on mesure chacun des paramètres de 
Stokes en termes de probabilité. Ensuite avec les 
matrices de Pauli, on reconstruit la matrice densité du 
système qui est une représentation du système quantique 
\cite{Kwiat}. Cependant, nous sommes intéressés à savoir 
si qu'on peut mesurer l'état directement en place de le 
reconstruire. Ceci nous donnera la même affaire, mais 
dans une méthode directe. Pour l'instant, la motivation est de pouvoir mesurer un état quantique sans utiliser une reconstruction complexe de l'état quantique. En fait, pourquoi ne pas mesurer directement l'état quantique? Cela nous amène au sujet de 
cette thèse, au moins une partie importante de celui-ci, les 
mesures faibles. La meilleure façon de décrire l'utilité des 
mesures faibles et leur intérêt pour notre travail est de les 
décrire comme une manière d'obtenir approximativement toutes 
les informations sur l'état original de la fonction d'onde 
que nous voulons mesurer. Considérez l'analogie suivante, 
décrite pour la première fois par Yuval Gefen \cite{Peephole} : vous êtes 
dans votre appartement et vous entendez un bruit venant du 
couloir. Mais chaque fois que vous ouvrez votre porte, le 
bruit disparaît et il n'y a rien à voir. Vous fermez la porte 
et le bruit revient. On dirait que des enfants s'amusent 
autour de la porte de votre appartement, mais à chaque fois 
que vous rouvrez la porte, ils arrêtent et se cachent.  En 
mécanique quantique, l'action d'ouvrir la porte est une 
interaction avec le système ou une mesure, et chaque fois que 
vous mesurez, vous effondrez la fonction d'onde et modifiez 
le résultat de ce qui se passe réellement, les enfants se 
comportent différemment, se cachent et ne font pas de bruit. 
Cependant, les mesures faibles sont comme une sorte de judas 
dans la porte qui permet d'observer ce que font les enfants, 
sans effondrer complètement la fonction d'onde \cite{Lundeen_Direct_Measurement}. Bien sûr, les 
enfants savent probablement que vous êtes légèrement conscient 
de ce qui se passe, et ils ne se comporteront donc pas de 
manière totalement naturelle, mais on peut au moins savoir 
ce qui se passe. Les mesures faibles sont donc censées être le
judas de la mécanique quantique \cite{Peephole}. Cependant, comment peut-on 
obtenir l'information sur la fonction d'onde après une mesure 
faible et comment peut-on le faire directement?
    
    %Pour obtenir de l'information sur l'état quantique 
d'un système par le biais des mesures faibles, il faut 
trouver et mesurer la valeur faible. La valeur faible est une valeur complexe 
dont il a été démontré qu'elle est proportionnelle à la 
fonction d'onde d'un état quantique. La valeur faible a 
été introduite pour la première fois par Aharanov, Albert 
et Vaidman (AAV) dans « How the Result of a Measurement 
of a Component of the Spin of Spin-1/2 Particle Can Turn 
Out to be 100 » \cite{Aharonov}. Dans cet article, AAV a constaté que la 
procédure de mesure habituelle de préparation d'un état quantique d'un système dans un 
état particulier, une intéraction entre les états de bases et une mesure 
projective effectuée sur l'état quantique peut parfois 
conduire à des résultats inhabituels. Ils ont découvert 
qu'en cas de faiblesse de la mesure, son résultat définit 
systématiquement une nouvelle valeur dans la fonction 
d'onde quantique, qu'ils ont appelée la valeur faible. 
La valeur faible apparaît et peut être mesurée dans une 
condition où la force de la mesure ou de l'interaction 
avec le système est plus faible que l'étendue (dispersion) de la 
distribution de probabilité d'un état. La force de 
l'interaction est décrite par la mesure dans laquelle 
l'acte de mesurer le système sépare les états de base 
de l'état quantique initial. Ensuite, en 
augmentant le nombre d'ensembles dans le système, nous 
pouvons effectuer une mesure (projective) et récupérer 
la valeur faible en obtenant directement la fonction 
d'onde quantique du système. Cependant, lorsque cette 
valeur est plus grande que l'étendue de la distribution 
de probabilité, cela ne donne aucune information sur le 
système quantique. Nous pouvons imaginer qu'une 
interaction faible sépare les états de base d'un état 
quantique de manière à ce qu'elle ne soit pas plus grande 
que l'étendue de la distribution de probabilité des 
deux états de base, comme le montre la figure \ref{fig:force_de_mesure}. 

\begin{figure}[p]
    \centering
    \includegraphics[width=1.0\textwidth]{force_de_mesure.pdf}
    \caption{Représentation visuelle de la 
    différence entre une mesure indirecte et directe sur 
    un système quantique. Supposons un état quantique 
    initialement $\ket{\psi_i}$ avec des états de base $\{ \ket{0};\ket{1} \}$ avec 
    une dispersion d'une distribution de probabilité de l'état $\sigma$ et soit $\delta$ 
    la force de séparation de l'interaction effectuée. a) L'état 
    quantique subit ce que nous appelons une mesure 
    « forte » où l'interaction avec l'état quantique 
    sépare les états de base plus que la distribution de 
    probabilité $\delta \gg \sigma$. Aucune information ne peut donc être 
    récupérée. b) L'état quantique subit une interaction 
    plus faible où ses états de base sont séparés de 
    façon très inférieure à l'écart de distribution des 
    probabilités $\delta \ll \sigma$. L'information réside alors dans le 
    chevauchement de ces états de base, qui peut être 
    récupéré à l'aide d'une mesure projective.
    }
    \label{fig:force_de_mesure}
\end{figure}

C'est 
ainsi qu'est née une nouvelle méthode de mesure des 
états quantiques, qu'il convient donc d'explorer. 
Depuis lors, de nombreuses contributions de mesures 
faibles ont été publiées dans plusieurs articles 
scientifiques qui ont simplifié les calculs et 
procédures de mesure, notamment dans le domaine des 
télécommunications \cite{OpticalNetworks}, sur effets de la lumière lente et 
rapide dans les cristaux photoniques biréfringents \cite{Brunner_2004}, le 
paradoxe d'Hardy \cite{HardyParadox,Aharonov_2002_Hardy} 
et autres \cite{QED,Btwn_pre_post}. Notre objectif est de 
discuter les mesures directes d'un état quantique par 
rapport à sa contrepartie indirecte en raison de sa 
simplicité et de ses applications. Cependant, jetons un 
coup d'œil sur quelques articles intéressants qui nous 
amèneront à la proposition finale de cette thèse. 

    %\subsubsection{Mesure direct de la fonction d'onde d'un état quantique}
    %Le 
premier est un article de Jeff Lundeen et al. qui 
présente les résultats expérimentaux d'une mesure directe 
d'un état quantique 
\cite{Lundeen_Direct_Measurement,Lundeen_thesis}. 
Ils commencent par mentionner les 
implications du principe d'incertitude et du fait que 
nous ne pouvons pas connaître à la fois la position et 
quantité de mouvement d'une particule. Ils décrivent 
qu'en utilisant la tomographie avec un ensemble de 
particules, ils peuvent déduire indirectement la fonction 
d'onde par reconstruction algorithmique, mais cela est 
bien sûr compliqué et long. C'est là que les mesures 
faibles entrent en jeu. Ils décrivent la méthode ou la 
procédure comme étant une contrepartie directe de la 
tomographie quantique et exempte des calculs et de séries 
de mesures compliquées et extensives. La mesure directe 
d'un état quantique qu'ils font est une procédure 
comprenant une préparation de l'état d'entrée, une interaction faible 
(mesure faible) et une mesure projective où le 
résultat final est une mesure des deux variables 
complémentaires du système qui apparaît 
proportionnellement au système quantique. Ils décrivent 
que la mesure directe est une réduction de la perturbation 
induite par la première mesure, soit la mesure faible. 
Ils mentionnent que la mesure faible est décrite 
comme une extension du modèle von Neumann des mesures 
quantiques (décrit pour la première 
fois par AAV)\cite{Aharonov,vonNeumann}. Le 
modèle implique un couplage de deux parties : l'état 
quantique du système et le système ancillaire qui est 
généralement appelé le « pointeur » en référence à 
l'existence d'un compteur et d'une aiguille d'un appareil 
de mesure. Prenons l'exemple de la figure \ref{fig:fuel}, 
qui représente la jauge de carburant d'une voiture. Le 
système commence dans une superposition d'états propres 
d'un observable désirée $\hat{A}$. Le système se couple avec une 
force de $g$ à la variable complémentaire conjuguée 
de $\hat{A}$, soit $\hat{P}$. 

\begin{figure}[h]
    \centering
    \includegraphics[width=1.0\textwidth]{fuel_gauge.png}
    \caption{Schéma d'une jauge de carburant utilisée 
    pour décrire le modèle de von Neumann pour les 
    mesures quantiques où un système et un appareil de 
    mesure sont couplés. Ce schéma est tiré de la thèse 
    de doctorat de Jeff Lundeen à l'université de 
    Toronto en 2006 \cite{Lundeen_thesis}. Je l'utilise ici pour faciliter la 
    description d'un pointeur qui peut être considéré 
    comme l'aiguille de la jauge de carburant. }
    \label{fig:fuel}
\end{figure}

Lorsque nous appuyons sur l'accélérateur, nous 
interagissons avec le système pendant un certain temps 
$t$, ce qui finit par effondrer le système à l'état propre 
correspondant et déplace l'aiguille d'autant. L'aiguille 
ou pointeur est donc un indicateur du résultat d'une 
mesure. Ils décrivent que lors d'une procédure directe 
via mesure faible, la force de couplage est réduite de 
sorte qu'il y a une perturbation minimale et, bien sûr, 
l'étape de post-sélection nous permet de mesurer 
directement la fonction d'onde d'un système quantique. 
Dans l'article, ils ont construit un appareil 
expérimental (démontré dans la figure 
\ref{fig:lundeen_exp}) pour tenter de mesurer 
la valeur faible d'un 
système photonique quantique, où les parties réelles et 
imaginaires de la valeur faible sont les observables de 
position et de quantité de mouvement en conséquence. Ils 
ont effectué une mesure directe de la fonction d'onde 
spatiale transversale d'un photon. 

\begin{figure}[h]
    \centering
    \includegraphics[width=1.0\textwidth]{lundeen_exp.png}
    \caption{Schéma de l'appareil expérimental 
    utilisé par Jeff Lundeen et al. pour 
    mesurer directement la fonction d'onde 
    d'un système quantique. Ils utilisent ici une fibre 
    optique monomode qui laisse passer des photons 
    approximativement gaussiens monomode. 
    Les photons passent à 
    travers un polariseur à microfils pour être 
    collimatés avec lentille achromatique. La 
    lentille est masquée par une ouverture rectangulaire. 
    Les photons passent ensuite à travers une demi-plaque 
    d'onde 
    qui détermine la force de la mesure faible via 
    son angle et utiliser comme pointeur pour les décalages 
    transversale (initiallement à $x=0$). Ensuite, 
    les photons passent à travers 
    une fente en post-sélectionnant ceux qui ont un 
    déplacement de quantité de mouvement de $p=0$. Ensuite, 
    les photons 
    sont collimatés avec une lentille et passent à 
    travers une plaque d'onde demi ou quart, puis entrent 
    dans un séparateur de faisceaux polarisant où chaque 
    bras est équipé d'un détecteur 
    \cite{Lundeen_Direct_Measurement}.}
    \label{fig:lundeen_exp}
\end{figure}

Ils ont produit un 
flux de photons de deux manières : soit en atténuant un 
faisceau laser, soit en générant des photons uniques par 
le biais d'une conversion paramétrique descendante 
spontanée «spontaneous parametric down-conversion» 
(SPDC). L'expérience peut être divisée en quatre étapes 
séquentielles : la préparation de la fonction d'onde 
transverse, mesure faible de la position transverse du 
photon, postsélection des photons dont le moment 
transverse est nul et lecture de la mesure faible 
résultante. Ils mesurent faiblement la position 
transversale du photon en la couplant à un degré de 
liberté interne du photon, soit sa polarisation. 
L'utilisation de la polarisation comme état quantique du 
système pour caractériser un système quantique a 
également été utilisée dans \cite{weak_photon_pol}. Cela leur a 
permis d'utiliser les angles de polarisation linéaire du 
photon comme pointeur. Dans ce cas, la réduction sur la 
force de la mesure correspond à la réduction de l'angle 
de polarisation linéaire des photons, ce qui est réalisé 
à l'aide d'une lame demi-onde. Ils utilisent ensuite une 
lentille à transformation de Fourier et une fente pour 
post-sélectionner (projecter) uniquement les photons 
ayant un décalage de quantité de mouvement de 0. Les 
résultats sont présentés dans les figures 
\ref{fig:lundeen_exp_1} et \ref{fig:lundeen_exp_2}.

\begin{figure}[hp]
    \centering
    \includegraphics[width=1.0\textwidth]{lundeen_exp_res_1.png}
    \caption{Mesures de la fonction d'onde d'un photon 
    unique. a) Partie réelle (carrés bleus) et 
    imaginaire (carrés rouges) de la valeur faible 
    liée aux déplacements de position et de quantité de 
    mouvement. b) En utilisant les données de a), ils 
    tracent la phase (carrés noirs ouverts) et le module 
    au carré de la fonction d'onde (cercles rouges 
    ouverts) \cite{Lundeen_Direct_Measurement}.}
    \label{fig:lundeen_exp_1}
\end{figure}

\begin{figure}[hp]
    \centering
    \includegraphics[width=1.0\textwidth]{lundeen_exp_res_2.png}
    \caption{Mesures des fonctions d'onde modifiées. 
    Dans ces résultats, ils ont testé leur capacité à 
    mesurer la fonction d'onde en modifiant la fonction 
    de probabilité en plaçant un atténuateur à 
    apodisation spatiale inverse après la fibre. a) 
    Densité de probabilité calculée de la fonction d'onde 
    à partir des données (cercles bleus pleins) ainsi 
    que le balayage du détecteur de la fonction de 
    probabilité (ligne continue). b) Toujours avec 
    l'œil de bœuf en place, ils ont modifié le profil 
    de phase de la fonction d'onde en créant une 
    discontinuité de phase où il y a 0 décalages 
    translationnels imposés avec un verre à mi-chemin à 
    travers la fonction d'onde. Dans ce graphique, nous 
    avons la partie réelle (carrés bleus pleins) et la 
    partie imaginaire (carrés rouges ouverts) \cite{Lundeen_Direct_Measurement}.}
    \label{fig:lundeen_exp_2}
\end{figure}

Ces résultats montrent qu'ils sont capables de 
caractériser la fonction d'onde de translation 
quantique de leur système en fonction des 
déphasages induit. Ils peuvent lire la partie 
réelle de la fonction d'onde à l'aide d'une 
lame demi-onde et la partie imaginaire à l'aide 
d'une lame quart-onde. Cette méthode permet de 
caractériser de manière très simpliste une 
fonction d'onde quantique. Ce qui nous amène à 
l'article suivant dont nous allons discuter. 
%\begin{figure}[h]
%    \centering
%    \includegraphics[width=1.0\textwidth]{lundeen_exp_res_3.png}
%    \caption{Modification de phase de la fonction d'onde. 
%    a) Ici, ils déplacent la fente transversalement à 
%    différentes positions et observent les changements 
%    de phase. b) Ici, ils introduisent un changement de 
%    phase quadratique en déplaçant les premières fentes 
%    à différentes positions \cite{Lundeen_Direct_Measurement}. }
%    \label{fig:lundeen_exp_3}
%\end{figure}
    
    %\subsubsection{Lecture simultanée des parties réelle et imaginaire de la valeur faible}
    %Nous discuterons ici d'un article de Hairiri et 
al. dans lequel ils caractérisent la fonction 
d'onde quantique par des mesures faibles, comme 
Jeff Lundeen et al. dans l'article précédent, 
mais en utilisant simultanément les déplacements 
de position et de quantité de mouvement comme 
parties réelles et imaginaire de la valeur 
faible \cite{Hairiri, Lundeen_Direct_Measurement}. Ils effectuent une procédure de mesure 
directe similaire avec une étape de 
préparation de l'état d'entrée, d'interaction faible et une mesure projective. Cependant, au lieu d'utiliser 
la fonction d'onde de translation comme état 
quantique à caractériser, ils utilisent un 
autre degré de liberté, la polarisation, dont 
les états de base peuvent être bien définis, 
tout comme le spin dans l'article d'AAV. Un 
autre aspect différent est l'utilisation d'un 
cristal biréfringent (borate de baryum (BBO)) 
pour introduire de petits décalages positionnels 
entre les états de base de la polarisation en 
tant qu'interaction faible sur le système. 

\begin{figure}[hp]
    \centering
    \includegraphics[width=1.0\textwidth]{hairiri.png}
    \caption{Appareil expérimental pour la 
    lecture simultanée de la partie réelle et 
    imaginaire de la valeur faible. L'état de 
    polarisation d'un laser HeNe est 
    préparé à l'aide d'un séparateur de 
    faisceau polarisant, d'une demi-plaque 
    d'onde et d'un quart de plaque d'onde. 
    L'état de polarisation subit ensuite une 
    interaction faible par un décalage 
    positionnel induit par le cristal BBO, 
    puis post-sélectionné à l'aide d'un autre 
    séparateur de faisceau polarisant. 
    Le système d'imagerie 4f est utilisé comme 
    une transformée de Fourier rapide, tout 
    comme l'article de 
    Jeff Lundeen et al. \cite{Lundeen_Direct_Measurement} pour les 
    mesures des déplacements de quantité de 
    mouvement. Le capteur d'image est ensuite 
    utilisé pour détecter les déplacements de 
    position dans les coordonnées 
    x et y \cite{Hairiri}.}
    \label{fig:hairiri}
\end{figure}

Pour lire les résultats de la mesure faible, ils 
déterminent le déplacement moyen du pointeur au 
long des coordonnées x et y à l'aide d'un 
capteur d'images. Leurs résultats sont présentés 
ci-dessous, où, en fonction du décalage de 
position et de quantité de mouvement mesuré 
induit sur le système, ils peuvent déterminer 
l'état de polarisation de l'état d'entrée. Ce 
résultat est particulièrement intéressant, 
car il est très simple à réaliser et prometteur 
pour les mesures faibles par rapport à la 
tomographie quantique. 

\begin{figure}[hp]
    \centering
    \includegraphics[width=1.0\textwidth]{haririri_res.png}
    \caption{Résultats expérimentaux de Hairiri 
    et al. a) partie réelle de la valeur faible 
    déterminée par les décalages de position 
    pour caractériser l'état d'entrée b) partie 
    réelle des amplitudes de l'état du système 
    $c_A$ et $c_D$ pour les états de polarisation 
    antidiagonale et diagonale comme états de 
    base respectivement $\{\ket{A}, \ket{D}\}$. c) partie imaginaire 
    de la valeur faible d) partie imaginaire 
    des amplitudes de l'état du système 
    \cite{Hairiri}.}
    \label{fig:hairiri_res}
\end{figure}

Dans un autre article de 
Guilleaum et al., certains membres du même 
groupe ont réalisé la même expérience, mais pour 
caractériser un état de polarisation mixte par 
le biais des mesures faibles 
\cite{Guilleaum,Guilleaum_thesis}. En termes 
d'applications technologiques, il n'y a pas eu 
beaucoup de travaux utilisant ces types 
d'appareils pour caractériser les états 
quantiques. Mais encore une fois, la lumière a 
plus de degrés de liberté que la position, la 
quantité de mouvement et sa phase. Nous nous 
demandons donc ce qu'il en est du domaine 
temporel de la lumière, si des travaux ont été 
menés à ce sujet et s'il existe des applications 
technologiques potentielles par rapport aux 
interactions faibles positionnelles?


    %\subsection{Motivation de la thèse}
    %\begin{doublespace}
    
    Les travaux récents sur les mesures faibles ont montré 
    leur potentiel dans divers domaines, soit dans 
    le paradoxe de Hardy, l'électrodynamique 
    quantique, télécommunication optique et ainsi 
    suite \cite{Aharonov_2002_Hardy,HardyParadox,Lundeen_thesis,QED,Brunner_2004,OpticalNetworks}. 
    Par exemple, 
    Brunner et al. ont exploré les réseaux de 
    télécommunication optique comme un cadre expérimental 
    pour des mesures faibles. Dans certains cas les mesures faibles peuvent 
    même surpasser des mesures traditionelles 
    \cite{Jeff_outperform,Magaña-Loaiza_2017,WeakorStd}.
    Magana-Loaiza et Lundeen ont 
    étendu ces recherches aux mesures spatiales et 
    quantité de mouvement. Cependant, les mesures faibles 
    positionnelles 
    souffrent des limitations importantes pour les 
    applications technologiques quantiques. Elles 
    nécessitent souvent l’utilisation de cristaux 
    BBO (barium borate (borate de barium))
    d’une taille définie pour ajuster l’interaction faible \cite{Hairiri,Guilleaum,Guilleaum_thesis}. 
    Cette contrainte complique leur adaptabilité à 
    différents systèmes. En revanche, les techniques 
    interférométriques permettent de contrôler directement 
    les délais temporels en ajustant simplement la position 
    des miroirs, supprimant ainsi la dépendance à des 
    cristaux précis. Cette flexibilité fait des mesures 
    faibles temporelles un choix idéal pour les systèmes 
    dynamiques ou industriels. Certain on travailler dans 
    des mesures faibles temporelle mais soit par 
    rapport d'un délai fréquentielle ou 
    pour des aspects théorique \cite{Salazar,OpticalNetworks,Steinberg_prob_div}.
    Cependant, cette thèse propose de surmonter les limitations 
    des mesures faibles positionnelles en développant 
    une approche temporelle utilisant un système photonique 
    quantique. En particulier, l’objectif est de mesurer 
    directement la partie réelle et imaginaire de la valeur 
    faible pour complètement caractériser un état de polarisation. 
    Cette approche exploite la polarisation comme base 
    quantique, car elle est facilement contrôlable et 
    réalisable en laboratoire. En développant une méthode plus efficace pour 
    caractériser les états quantiques directement, cette 
    thèse vise à contribuer à l'avancement des technologies 
    quantiques et à ouvrir de nouvelles possibilités dans 
    les domaines scientifique et industriel.
\end{doublespace}
    \pagebreak
    
    \thispagestyle{empty}
    \section{LES MESURES FAIBLES TEMPORELLES D'UN SYSTÈME QUANTIQUE}
    \begin{onehalfspace}

    Dans ce chapitre, nous nous concentrerons sur les
    fondements théoriques des mesures quantiques, en ce qui concerne la
    tomographie quantique et les mesures faibles. Nous commencerons
    avec une tomographie quantique pour des états de polarisation dans
    le cadre de \cite{Kwiat}, puis introduirons une méthode alternative,
    les mesures faibles, selon le formalisme proposé par Aharonov,
    Albert et Vaidman (AAV) \cite{Aharonov}, ainsi que dans le cadre
    expérimental développé dans \cite{Lundeen_Resch}. Nous 
    démontrerons ensuite comment la valeur faible est liée à la
    fonction d’onde d’un état quantique à partir d'une interaction
    faible avec un pointeur, comme présenté dans \cite{Lundeen_Bamber}.
    Enfin, nous allons proposer un cadre théorique pour les mesures
    faibles temporelles et les prédictions théoriques pour les valeurs
    moyennes des observables expérimentales du système qui pourraient
    être mesurées directement dans notre laboratoire.

\end{onehalfspace}

\subsection{\textcolor{red}{Les mesures quantiques: une procédure reconstructive à caractériser un état quantique}}

\begin{onehalfspace}
    Dans ce qui suit, nous allons présenter deux méthodes de caractérisation d'un état quantique: la tomographie quantique et les mesures faibles.
    Nous allons démontrer un exemple de tomographie quantique pour des états de polarisation, puis introduire le formalisme des mesures faibles et comment elles permettent d'accéder directement à la fonction d'onde d'un état quantique.
    Ainsi, nous allons présenter les fondements théoriques des mesures faibles et comment elles peuvent être utilisées pour caractériser un état quantique.
\end{onehalfspace}

\subsubsection{\textcolor{red}{La tomographie quantique}}

\begin{onehalfspace}
        
    Traditionnellement, la tomographie quantique est utilisée pour 
    reconstruire la fonction d’onde d’un état quantique à partir d’un 
    ensemble de mesures projectives. En photonique quantique, ce 
    processus consiste à effectuer des mesures de projection sur divers 
    états quantiques en utilisant des bases orthogonales sélectionnées, 
    soit $\{\ket{H}, \ket{V}\}$, $\{\ket{D}, \ket{A}\}$ et $\{\ket{R}, \ket{L}\}$, 
    (polarisation horizontale, verticale) (diagonale, anti-diagonale) et 
    (circulaire droite, gauche) respectivement. Ensuite, les résultats 
    obtenus sont analysés par un algorithme qui reconstruit implicitement la 
    matrice densité de l’état quantique. La matrice densité représente un 
    opérateur hermitien qui renferme toutes les informations sur l’état 
    quantique. Elle se présente sous la forme $\rho = \ket{\psi}\bra{\psi}$ 
    pour un état $\ket{\psi}$ et on peut facilement vérifier sa pureté en prenant 
    la trace de $\rho^2$, $Tr(\rho^2)=1$.

    \noindent Pour approfondir nos connaissances, considérons un exemple 
    arbitraire. Supposons un photon préparé dans l’état de polarisation 
    suivant:
    
    \begin{equation}
        \ket{\psi} = a\ket{H} + b\ket{V}\label{eq:initial_tomo_state}
    \end{equation}

    \noindent où $a$, $b \in \mathcal{C}$, $|a|^2 + |b|^2 = 1$ et dans la 
    base $\{ \ket{H}, \ket{V} \}$. La matrice densité de cet état, que 
    l'on retrouvera prochainement dans cet exemple, s'écrit comme suit:
    
    \begin{equation}
        \rho = \begin{pmatrix}
            |a|^2 & ab^*\\
            a^*b & |b|^2
        \end{pmatrix}
    \end{equation}
    
    \noindent L'objectif d'une tomographie quantique est de déterminer 
    les coefficients de la matrice. Pour ce faire, il faut effectuer 
    un ensemble de mesures projectives prédéterminées à l'avance 
    pour obtenir les probabilités ou intensités de 
    détection dans différentes bases de polarisation. La fraction de 
    photons détectés en sortie $\ket{H}$ correspond alors à $|a|^2$ et en 
    sortie $\ket{V}$ correspond à $|b|^2$. Pour accéder aux termes 
    d’interférence, comme $ab^*$, il faut réaliser des mesures dans des 
    bases complémentaires, telles que $\{\ket{D}, \ket{A}\}$ qui sont les 
    polarisations diagonale et antidiagonale, et/ou $\{\ket{R}, \ket{L}\}$ 
    qui sont les polarisations circulaires. Les variations d’intensité 
    observées dans ces différentes configurations de mesure projective 
    permettent de reconstruire les éléments de la matrice densité 
    à l'aide d'un algorithme de maximum de vraisemblance. 
    
    \noindent En photonique quantique, cette matrice densité peut 
    également être exprimée en termes des paramètres de Stokes 
    \cite{Kwiat}. Ces derniers décrivent complètement l’état de 
    polarisation et ils sont liés aux probabilités de détection dans 
    différentes bases de polarisation \cite{hecht2012optics}. Avec ces
    paramètres, nous pouvons reconstruire la matrice densité
    à partir des probabilités de détection dans les différentes bases et
    ainsi visualiser l'état de polarisation sur la sphère de Poincaré.
    Les paramètres de Stokes sont définis par:

    \begin{equation}
        S = \begin{pmatrix}
            S_0\\
            S_1\\
            S_2\\
            S_3
        \end{pmatrix}
        = \begin{pmatrix}
            P_{\ket{H}} + P_{\ket{V}}\\
            P_{\ket{H}} - P_{\ket{V}}\\
            P_{\ket{D}} - P_{\ket{A}}\\
            P_{\ket{R}} - P_{\ket{L}}
        \end{pmatrix}
    \end{equation}

    \noindent où $P_{\ket{H}}$ est la probabilité de détection pour l'état de
    polarisation horizontale $\ket{H}$ et $P_{\ket{V}}$ la probabilité de
    détection pour l'état de polarisation verticale $\ket{V}$. Ainsi, le
    paramètre de Stokes $S_0$ représente l’intensité ou probabilité
    totale du faisceau, $S_1$ représente la différence de probabilités
    entre les polarisations $\ket{H}$ et $\ket{V}$, $S_2$ représente la
    différence de probabilités entre les polarisations 
    $\ket{D} \equiv \frac{1}{\sqrt{2}}(\ket{H} + \ket{V})$ et $\ket{A} \equiv \frac{1}{\sqrt{2}}(\ket{H} - \ket{V})$ 
    et $S_3$ représente la différence de probabilités entre les polarisations 
    $\ket{R} \equiv \frac{1}{\sqrt{2}}(\ket{H}+i\ket{V})$ et $\ket{L} \equiv \frac{1}{\sqrt{2}}(\ket{H} - i\ket{V})$. 
    En notant les probabilités de mesure pour chacune de ces bases, on 
    reconstruit la matrice densité à partir de la relation suivante \cite{Kwiat}:
    
    \begin{equation}
        \rho = \frac{1}{2}\sum_{i=0}^{3} S_i \hat{\sigma}_i \label{eq:density_stokes}
    \end{equation}

    \noindent où les $\hat{\sigma}_i$ sont les matrices de Pauli définies 
    comme suit:
    
    \begin{equation}
        \hat{\sigma}_0 = \begin{pmatrix}
            1 & 0\\
            0 & 1
        \end{pmatrix},
        \hat{\sigma}_1 = \begin{pmatrix}
            1 & 0\\
            0 & -1
        \end{pmatrix},
        \hat{\sigma}_2 = \begin{pmatrix}
            0 & 1\\
            1 & 0
        \end{pmatrix},
        \hat{\sigma}_3 = \begin{pmatrix}
            0 & -i\\
            i & 0
        \end{pmatrix}
    \end{equation}

    \noindent En utilisant l'état arbitraire que nous avons mentionné
    à l'équation \ref{eq:initial_tomo_state},  nous
    trouvons la matrice densité avec les paramètres de Stokes. 
    Commençons par réécrire la matrice densité, selon l'équation \ref{eq:density_stokes}, comme suit:

    \begin{equation}
        \rho = \frac{1}{2}(S_0\hat{\sigma}_0 + S_1\hat{\sigma}_1 + S_2\hat{\sigma}_2 + S_3\hat{\sigma}_3)
    \end{equation}

    \noindent Ensuite, trouvons chacun des paramètres de Stokes.
    Les deux premiers paramètres $S_0$ et $S_1$ sont simples;
    trouvons les probabilités $P_{\ket{H}}$ et $P_{\ket{V}}$
    en projetant les différentes bases sur l'état de polarisation.

    \begin{align}
        P_{\ket{H}} &= |\bra{H}\ket{\psi}|^2 = (a\bra{H}\ket{H} + b\bra{H}\ket{V})(a^*\bra{H}\ket{H} + b^*\bra{H}\ket{V})\\
        &= |a|^2\\
        P_{\ket{V}} &= |\bra{V}\ket{\psi}|^2 = (a\bra{V}\ket{H} + b\bra{V}\ket{V})(a^*\bra{V}\ket{H} + b^*\bra{V}\ket{V})\\
        &= |b|^2
    \end{align}

    \noindent Les paramètres de stokes $S_0$ et $S_1$ sont donc les 
    suivants:

    \begin{align}
        S_0 &= P_{\ket{H}} + P_{\ket{V}} = |a|^2 + |b|^2 = 1\\
        S_1 &= P_{\ket{H}} - P_{\ket{V}} = |a|^2 - |b|^2
    \end{align}

    \noindent Pour les deux paramètres suivants $S_2$ et $S_3$, nous 
    devons exprimer les états projetés dans nos états de base 
    $\{\ket{H},\ket{V}\}$. 

    \begin{align}
        P_{\ket{D}} &= |\bra{D}\ket{\psi}|^2 = \Biggl[\biggl(\frac{1}{\sqrt{2}}(a\bra{H}\ket{H} + a\bra{V}\ket{H})\biggr)\\ 
        &+ \biggl(\frac{1}{\sqrt{2}}(b\bra{H}\ket{V} + b\bra{V}\ket{V})\biggr)\Biggr]\Biggl[ \biggl(\frac{1}{\sqrt{2}}(a^*\bra{H}\ket{H} + a^*\bra{V}\ket{H})\biggr)\\
        &+ \biggl(\frac{1}{\sqrt{2}}(b^*\bra{H}\ket{V} + b^*\bra{V}\ket{V})\biggr) \Biggr] = \frac{1}{2}(a + b)(a^* + b^*)\\
        &= \frac{1}{2}(|a|^2 + ab^* + a^*b + |b|^2)\\
    \end{align}
    
    \begin{align}  
        P_{\ket{A}} &= |\bra{A}\ket{\psi}|^2 = \Biggl[\biggl(\frac{1}{\sqrt{2}}(a\bra{H}\ket{H} - a\bra{V}\ket{H})\biggr)\\ 
        &+ \biggl(\frac{1}{\sqrt{2}}(b\bra{H}\ket{V} - b\bra{V}\ket{V})\biggr)\Biggr]\Biggl[ \biggl(\frac{1}{\sqrt{2}}(a^*\bra{H}\ket{H} - a^*\bra{V}\ket{H})\biggr)\\
        &+ \biggl(\frac{1}{\sqrt{2}}(b^*\bra{H}\ket{V} - b^*\bra{V}\ket{V})\biggr) \Biggr] = \frac{1}{2}(a - b)(a^* - b^*)\\
        &= \frac{1}{2}(|a|^2 - ab^* - a^*b + |b|^2)\\
        S_2 &= P_{\ket{D}} - P_{\ket{A}} = ab^* + a^*b = 2\mathcal{R}(ab^*)
    \end{align} 

    \noindent On répète la même technique pour $S_3$:

    \begin{align}
        P_{\ket{R}} &= |\bra{R}\ket{\psi}|^2 = \Biggl[\biggl(\frac{1}{\sqrt{2}}(a\bra{H}\ket{H} + ia\bra{V}\ket{H})\biggr)\\ 
        &+ \biggl(\frac{1}{\sqrt{2}}(b\bra{H}\ket{V} + ib\bra{V}\ket{V})\biggr)\Biggr]\Biggl[ \biggl(\frac{1}{\sqrt{2}}(a^*\bra{H}\ket{H} + ia^*\bra{V}\ket{H})\biggr)\\
        &+ \biggl(\frac{1}{\sqrt{2}}(b^*\bra{H}\ket{V} + ib^*\bra{V}\ket{V})\biggr) \Biggr] = \frac{1}{2}(a + ib)(a^* + ib^*)\\
        &= \frac{1}{2}(|a|^2 + iab^* + ia^*b - |b|^2)\\
        P_{\ket{L}} &= |\bra{L}\ket{\psi}|^2 = \Biggl[\biggl(\frac{1}{\sqrt{2}}(a\bra{H}\ket{H} - ia\bra{V}\ket{H})\biggr)\\ 
        &+ \biggl(\frac{1}{\sqrt{2}}(b\bra{H}\ket{V} - ib\bra{V}\ket{V})\biggr)\Biggr]\Biggl[ \biggl(\frac{1}{\sqrt{2}}(a^*\bra{H}\ket{H} - ia^*\bra{V}\ket{H})\biggr)\\
        &+ \biggl(\frac{1}{\sqrt{2}}(b^*\bra{H}\ket{V} - ib^*\bra{V}\ket{V})\biggr) \Biggr] = \frac{1}{2}(a - ib)(a^* - ib^*)\\
        &= \frac{1}{2}(|a|^2 - iab^* - ia^*b - |b|^2)\\
        S_3 &= P_{\ket{R}} - P_{\ket{L}} = i(ab^* + a^*b) = 2\mathcal{I}(ab^*)
    \end{align}

    \noindent Ensuite, écrivons nous résultats dans notre matrice 
    densité:
    

    \begin{align}
        \rho &= \frac{1}{2}(S_0\hat{\sigma}_0 + S_1\hat{\sigma}_1 + S_2\hat{\sigma}_2 + S_3\hat{\sigma}_3)\\
        &= \frac{1}{2}\begin{pmatrix}
            S_0 + S_1 & S_2 -S_3\\
            S_2 + S_3 & S_0 - S_1
        \end{pmatrix}\\
        &= \begin{pmatrix}
            |a|^2 & \mathcal{R}(ab^*) - i\mathcal{I}(ab^*)\\
            \mathcal{R}(a^*b) - i\mathcal{I}(a^*b) & |b|^2
        \end{pmatrix}\\
        &= \begin{pmatrix}
            |a|^2 & ab^*\\
            a^*b & |b|^2
        \end{pmatrix}
    \end{align}

    \noindent Nous avons maintenant reconstruit notre matrice densité 
    à partir d'un état de polarisation arbitraire en utilisant les 
    paramètres de Stokes. 
    
    \noindent Pour un état pur, comme notre exemple, on a la propriété 
    $Tr(\rho^2) = 1$, ce qui se traduit par une cohérence quantique 
    maximale. En revanche, un état mixte se caractérise par une matrice 
    densité statistique, qui est une somme pondérée d'états purs:

    \begin{equation}
        \rho_{mixte} = \sum_{i}^{N} p_i\ket{\psi_i}\bra{\psi_i}
    \end{equation}


    \noindent Nous avons $N$ états, chacun étant associé à une 
    probabilité $p_i$, de sorte que $\sum_{i}^{N} p_i = 1 $. Chaque état 
    $\ket{\psi_i}$ correspond à un état pur dont la matrice densité 
    mixte $\rho_{mixte}$ correspond à un mélange statistique de chacun de ces états. Dans ce contexte, $Tr(\rho^{2}_{mixte}) < 1 $. 
    Cela signifie que la pureté d’une matrice densité peut être mesurée 
    par sa trace. Un état pur possède une cohérence parfaite, tandis 
    qu’un état mixte résulte d’un mélange statistique d’états. Il est 
    aussi possible d'évaluer la pureté d’un état à l'aide des paramètres 
    de Stokes, avec la relation suivante (en supposant une normalisation 
    avec $S_0=1$): 

    \begin{equation}
        \sqrt{\sum_{i=1}^{3} S_i^2} = \text{DOP}
    \end{equation}

    \noindent où \text{DOP}, le dégré de polarisation, peut être soit $1$ 
    pour un état pur (entièrement polarisé) ou strictement inférieur à $1$ 
    pour un état mixte (partiellement polarisé). Il est possible de 
    visualiser l’état de polarisation sur la sphère de 
    Poincaré (figure \ref{fig:spherepoincarre}) qui offre
    une représentation tridimensionnelle où 
    chaque axe correspond à un paramètre de Stokes, excluant ainsi $S_0$. 
    Chaque point sur cette surface représente un état distinct d'une
    polarisation pure.

    \begin{figure}[!h!t!p!b!]
        \centering
        \includegraphics[width=1.0\textwidth]{poincare_sphere.png}
        \caption{La sphère Poincaré est une représentation 
        géométrique des états de polarisation. Un point sur la surface de la sphère
        représente un état de polarisation complètement polarisé, alors qu'un point
        situé à l'intérieur de la sphère représente un état de polarisation
        partiellement polarisé. Les états de polarisation $\ket{H}$, $\ket{V}$,
        $\ket{D}$ et $\ket{A}$ sont situés sur l'équateur de la sphère, tandis que les états
        $\ket{R}$ et $\ket{L}$ sont situés aux pôles nord et sud respectivement.
        Les axes sont orientés selon les paramètres de Stokes $S_1$, $S_2$ et $S_3$.}
        \label{fig:spherepoincarre}
    \end{figure}

    \noindent Enfin, les protocoles de tomographie quantique 
    permettent de déterminer empiriquement ces coefficients de la matrice 
    densité, à partir des paramètres de Stokes, et peuvent donc 
    reconstruire et caractériser entièrement la matrice densité 
    d’un état de polarisation. Toutefois, cette méthode présente un inconvénient 
    majeur : elle nécessite un grand nombre de mesures projectives pour
    obtenir une estimation précise de la matrice densité. En effet,
    la tomographie quantique est une méthode indirecte pour 
    caractériser un état quantique \cite{Lundeen_Direct_Measurement}
    car elle repose sur la reconstruction de la matrice densité à partir
    de mesures projectives dans différentes bases.
    De plus, elle ne permet pas d'accéder facilement à des
    éléments individuels de la matrice densité, car elle se repose sur une reconstruction globale
    \cite{Guilleaum}. Ces limitations rendent la tomographie peu adaptée
    aux applications nécessitant un accès direct ou simultané à certains 
    paramètres de la matrice densité, ou des mesures en temps réel
    \cite{TomoReview}.
    \textcolor{red}{ Étant donné que les photons ont un grand nombre de 
    dimensions à caractériser, telles que leurs différents états 
    de polarisation, cela nécessiterait un grand nombre de 
    mesures projectives pour qu'une tomographie quantique 
    puisse caractériser l'état du système. Cela rendrait cette 
    technique impraticable dans un contexte d'application. }
\end{onehalfspace}

\subsubsection{\textcolor{red}{Introduction aux mesures faibles}}

\begin{onehalfspace}
    
    Une alternative intéressante consiste à utiliser des mesures faibles, 
    une méthode permettant d’accéder à la fonction d’onde d’un système 
    quantique directement. AAV ont proposé cette méthode dans leur 
    article \guillemetleft \space How the result of a measurement of a component 
    of the spin of a spin-1/2 particle can turn out to be 100 
    \guillemetright \space (Comment le résultat de la mesure de la composante 
    spin d’une particule ayant un spin-1/2 peut devenir 100) en 1988 
    \cite{Aharonov}. Cette méthode s’inspire du modèle de von Neumann 
    \cite{vonNeumann}, dans lequel un système faiblement lié à un 
    \guillemetleft \space pointeur \guillemetright \space subit une interaction 
    (perturbation) faible. La mesure du résultat est représentée par un 
    déplacement du pointeur proportionnel à ce que l’on appelle la 
    \guillemetleft \space valeur faible \guillemetright. 

    \noindent Le modèle von Neumann des mesures quantiques sert de 
    fondement théorique pour comprendre les mesures faibles. Dans ce 
    modèle, le système quantique et ce qu’on appelle un \guillemetleft \space 
    pointeur \guillemetright \space (nommé en référence à l’aiguille d’un 
    instrument de mesure \cite{vonNeumann}) sont enchevêtrés (couplés) par un 
    opérateur d’interaction faible, permettant ainsi d’extraire des 
    informations sur la fonction d’onde. Le pointeur indique l'état de la 
    mesure résultante de l'appareil de mesure \cite{vonNeumann}. 

    \noindent \textcolor{red}{Considérons les scénarios à la figure \ref{fig:interaction} 
    dans lequel le système à caractériser
    c'est l'état de polarisation d'un photon, et le pointeur est une autre
    propriété du photon, comme sa position temporelle décrit par 
    une distribution gaussienne avec une largeur $\sigma$. 
    Initialement, avant la mesure, le système et le pointeur sont couplés avec une force
    d'interaction $\delta$, un paramètre d'interaction dans la même base du pointeur. Dans le cas a) de la figure \ref{fig:interaction},
    la force d'interaction est forte, décrit par $\delta \gg \sigma$, ce qui provoque un effondrement
    complet de la fonction d'onde du système et découple le système
    du pointeur. Cela signifie que le pointeur est déplacé de manière
    significative, permettant ainsi de lire une des valeurs possibles
    du système (état de polarisation) en effectuant une mesure projective.
    Pour caractériser toutes les valeurs possibles du système,
    il est nécessaire d'effectuer plusieurs mesures projectives
    pour reconstruire la fonction d'onde du système correspondant 
    à une tomographie quantique. En revanche,
    dans le cas b) de la figure \ref{fig:interaction},
    la force d'interaction est faible, décrite par $\delta \ll \sigma$,
    ce qui préserve la superposition
    quantique des états de base du système et le couplage entre
    le système et le pointeur. Dans ce régime de mesure faible,
    le pointeur est déplacé de manière minimale, permettant ainsi
    d'extraire des informations sur la fonction d'onde du système
    directement à partir du déplacement du pointeur, sans effondrement
    complet de la fonction d'onde du système. }

    \begin{figure}[!htbp]
        \centering
        \includegraphics[width=1.0\textwidth,page=3]{FIGURES.pdf} % Selects page 2
        \caption{Considérons qu'un pointeur possède une forme gaussienne 
        avec une position moyenne, représentée par la ligne pointillée, avec 
        une distribution de probabilité $\sigma$ dans l’état 
        $\ket{\psi_i} = a\ket{H} + b\ket{V}$ et un coefficient d’interaction 
        $\delta$ qui décrit la force de séparation des états. 
        a) Mesure forte : une interaction forte impliquerait un 
        effondrement complet de l’état séparant complètement les états de 
        base. Ici, cet effondrement est représenté par la 
        séparation des composantes H et V de l'état par un délai 
        plus grand que la largeur de l'impulsion $\delta \gg \sigma$. 
        Par conséquent, nous mesurerions l’un ou l’autre via une mesure projective. 
        b) Mesure faible : Elle consiste en une interaction faible avec 
        le système qui permet aux deux états de base de se chevaucher, de 
        sorte que, lors d’une mesure projective, nous obtenions en retour un 
        état qui comprend essentiellement l’état initial du système. Cela se 
        produit lorsque $\delta \ll \sigma$.}
        \label{fig:interaction}
    \end{figure}

    \noindent \textcolor{red}{En effectuant une mesure projective après une interaction
    faible, nous obtenons une distribution de probabilité du pointeur similairement
    à l'état initial du système, mais légèrement déplacé.
    Ce déplacement du pointeur est proportionnel à ce qu'on appelle la 
    \guillemetleft \space valeur faible \guillemetright \space de l'observable mesurée.
    Le déplacement peut être mesuré directement pour extraire
    des informations sur la fonction d'onde initiale du système quantique
    à partir de cette valeur. La valeur faible $\expval{\hat{S}}_W$,
    dont l'indice $W$ signifie \guillemetleft Weak \guillemetright \space 
    cet-à-dire faible en anglais pour la base des
    mesures faibles \cite{Lundeen_Resch}, est définie
    comme la valeur moyenne faible d’un observable $\hat{S}$ du système,
    mesuré dans un état d’entrée $\ket{\psi_i}$ et un état de mesure
    projective $\ket{\psi_f}$, qui est le résultat d’une mesure faible écrite comme le suivant:}


    \begin{equation}
        \expval{\hat{S}}_W = \frac{\bra{\psi_f}\hat{S}\ket{\psi_i}}{\bra{\psi_f}\ket{\psi_i}}        
    \end{equation}
        
    
    \noindent La valeur faible est une variable complexe composée d'une partie
    réelle et imaginaire \cite{Aharonov,Lundeen_Resch}. Ces composantes renferment des
    informations sur les observables du système, soit celle de la variable du pointeur $\hat{p}$ dans la partie réelle,
    et celle de la variable conjuguée du pointeur $\hat{q}$ dans la partie imaginaire,
    permettant une caractérisation complète \cite{Lundeen_Bamber}.
    Ainsi, la valeur faible peut être exprimée en fonction de ces 
    observables du pointeur comme suit :

    \begin{equation}
        \expval{\hat{S}}_W = \frac{1}{\delta}\Biggl( \expval{\hat{p}} - i4\sigma^2 \expval{\hat{q}} \Biggr)\label{eq:weakvalue_components}
    \end{equation}

    \noindent Les mesures faibles servent d’œil de Judas au monde 
    quantique \cite{Peephole}. Cela nous permet de perturber le système le 
    moins possible pour obtenir de l’information sur le système quantique. 
    L’adoption des mesures faibles repose sur plusieurs avantages clés : 
    elles réduisent les perturbations induites sur le système, préservent 
    la cohérence quantique et permettent une approche directe et 
    intuitive pour caractériser des états quantiques \cite{ApplicationWeak}.
    Nous allons maintenant explorer, dans la prochaine section, les fondements théoriques des 
    mesures faibles et comment elles permettent d’accéder directement à 
    la fonction d’onde d’un état quantique en dérivant la valeur faible
    à partir d’une interaction faible entre le système et le pointeur ainsi
    que les procédures théoriques pour obtenir les composantes réelles et
    imaginaires de la valeur faible.

\end{onehalfspace}

\subsubsection{\textcolor{red}{Fondamentaux théoriques des mesures faibles}}

\begin{onehalfspace}
    Tout d’abord, nous aborderons les principes théoriques sous-jacents 
    aux mesures faibles, en expliquant comment la valeur faible est liée à la 
    fonction d’onde de l’état quantique et peut être mesurée directement. 
    Ces mesures sont une étape clé dans la procédure décrite par l’AAV 
    \cite{Aharonov} et dans \cite{Lundeen_Resch}, qui se compose des 
    éléments suivants. Voici les étapes de la procédure à suivre:

    \begin{itemize}
        %\item Préparation de l’état initial
        \item Interaction faible, sujet de discussion dans la section présente.
        \item Postsélection, une mesure projective utilisant un état qui projette sur l'ensemble du système faiblement interagi.
    \end{itemize}

    \noindent Dans cette section, nous nous attarderons sur l’étape de 
    l’interaction faible, qui correspond à une perturbation faible de 
    l’état quantique. Comme cela a été mentionné précédemment, cette 
    procédure directe via des mesures faibles s’appuie sur le modèle de 
    von Neumann pour les mesures quantiques. Ce modèle implique un 
    système quantique composé de deux objets : le système à mesurer $S$ 
    et l’appareil de mesure (pointeur) $P$. Ces deux objets sont traités 
    comme des objets de la mécanique quantique couplés dans un système 
    total $T$ \cite{vonNeumann}. Le système total $T$ est 
    défini comme le produit tensoriel du système $S$ et du pointeur $P$. 
    Lorsqu'un état quantique est 
    soumis à une mesure, il se trouve initialement dans un état superposé 
    inconnu, noté $\ket{\psi}_S = \sum_{j}^{N}c_{j}\ket{s_j}_S$, dont les 
    composantes sont des combinaisons linéaires de vecteurs propres 
    $\ket{s_j}_S$, d'une observable du système $\hat{S}$, avec des valeurs 
    propres $s_j$ (avec coefficients complexes), et $N$ est le nombre 
    d'états de base du système, qui est couplé à un pointeur 
    initialement dans l’état $\ket{\bar{p} = 0}_P$ dans la base $P$ 
    où $\bar{p}$ est la position moyenne de la variable du pointeur 
    $p$, qui est initialement à zéro \cite{Hairiri,vonNeumann}. 
    L’état du système total $T$ est alors écrit comme suit :

    \begin{equation}
        \ket{\Psi^i}_T = \ket{\psi}_S \otimes \ket{\bar{p} = 0}_P
    \end{equation}

    \noindent L’interaction sur le système $S$ et le pointeur $P$ sont
    décrites par l’opérateur 
    d’interaction de von Neumann, qui est appliqué à l’état
    initial du système total $T$ \cite{vonNeumann}. Cet opérateur d’interaction est 
    responsable de la perturbation du système quantique et de la 
    translation du pointeur en fonction de la force d’interaction $\delta$ 
    (voir figure \ref{fig:interaction}), dont l'état final du pointeur est 
    $\ket{\bar{p} = \delta}_P$, appelée la valeur moyenne 
    faible du pointeur après une mesure faible. Par conséquent, l'état final se trouve $\ket{\Psi^f}_T$ 
    \cite{Lundeen_Bamber,Hairiri,vonNeumann}. La quantification de cette interaction 
    se définit à travers un opérateur d’interaction $\hat{U}$, 
    communément désigné sous le nom d’opérateur d’interaction de 
    von Neumann. Ce dernier est 
    exprimé comme suit :

    \begin{equation}
        \hat{U} \equiv exp\Bigl( -\frac{i}{\hbar}\int \mathcal{H} dt \Bigr)
    \end{equation}

    \noindent Il s'agit d'un opérateur d'évolution temporelle, où $dt$ 
    représente le temps d'interaction avec le système, $\hbar$ la 
    constante de Planck et $\mathcal{H}$ le hamiltonien du système total 
    $T$, qui est défini comme suit :

    \begin{equation}
        \mathcal{H} \equiv g(t)(\hat{S} \otimes \hat{q})
    \end{equation}

    \noindent \textcolor{red}{où $g$ est la constante de couplage, qui doit être réelle 
    pour que le hamiltonien soit hermitien, $\hat{S}$ l'opérateur de mesure 
    et $\hat{q}$ la variable conjuguée du pointeur}. Dans un régime de mesures faibles, où l’interaction est 
    plus faible que la distribution de probabilité du pointeur, 
    c’est-à-dire $\delta \ll \sigma$, le système mesuré est faiblement 
    couplé avec le pointeur, entraînant ainsi un effondrement minimal 
    préservant la superposition (voir figure \ref{fig:interaction}).
    La force de l'interaction est définie par 
    $\delta \equiv \frac{\int g dt}{\hbar} = \frac{gt}{\hbar}$ \cite{Lundeen_Resch,Lundeen_thesis}, 
    donc l'opérateur d'interaction de von Neumann est écrit comme suit :

    \begin{equation}
        \hat{U} = e^{-i\delta(\hat{S} \otimes \hat{q})}
    \end{equation}

    \noindent Alors, au cours de l'interaction, 
    l’opérateur d’interaction de von Neumann est appliqué à 
    l’état initial du système total $T$ comme ce suit: 

    \begin{align}
        \hat{U}\ket{\Psi^{i}}_T &= \hat{U}\Bigl[ \ket{\psi}_S \otimes \ket{\bar{p} = 0}_P \Bigr]\\
        &= e^{-i\delta(\hat{S} \otimes \hat{q})}\Bigl[ \ket{\psi}_S \otimes \ket{\bar{p} = 0}_P \Bigr]
    \end{align}

    \noindent où nous allons maintenant examiner une étude plus approfondie
    sur cette interaction en récrivant l’opérateur d’interaction de von Neumann
    sous la forme d'une série de Taylor: \textcolor{red}{FIXED}

    \begin{align}
        \hat{U}\ket{\Psi^{i}}_T &= \Bigl(\mathds{1} - i\delta(\hat{S} \otimes \hat{q}) + \mathcal{O}(\delta^2) \Bigr)\Bigl[ \ket{\psi}_S \otimes \ket{\bar{p} = 0}_P \Bigr]\\
        &=  \ket{\psi}_S \otimes \ket{\bar{p} = 0}_P - i\delta\hat{S}\ket{\psi}_S\otimes\hat{q}\ket{\bar{p} = 0}_P + \mathcal{O}(\delta^2)
    \end{align}

    \noindent où $\mathds{1}$ est l'opérateur d'identité et 
    $\mathcal{O}(\delta^2)$ correspond aux ordres plus 
    élevés de la série de Taylor, que nous négligeons. En suivant la 
    procédure de mesure faible, nous devons effectuer une postsélection 
    ultérieure sur le système avec l’état $\ket{\varphi}_S$, qui est dans 
    la même base $S$ que $\ket{\psi}_S$:\textcolor{red}{FIXED}

    \begin{align}
        \ket{\varphi}_S\bra{\varphi}_S\hat{U}\ket{\Psi^{i}}_T &= \ket{\varphi}_S\bra{\varphi}_S\ket{\psi}_S \otimes \ket{\bar{p} = 0}_P - i\delta\ket{\varphi}_S\bra{\varphi}_S\hat{S}\ket{\psi}_S\otimes\hat{q}\ket{\bar{p} = 0}_P
    \end{align}

    \noindent en normalisant l’état du système total avec le module de la 
    probabilité $\bra{\varphi}_S\ket{\psi}_S =\sqrt{Prob}$ sur chaque côté, où
    $Prob\equiv |\bra{\varphi}_S\ket{\psi}_S|^2$ est la probabilité de trouver l’état $\ket{\psi}_S$
    dans l’état $\ket{\varphi}_S$ \cite{Lundeen_Resch,Lundeen_thesis, Steinberg_prob_div},
    nous obtenons :

    \begin{align}
        \ket{\varphi}_S\frac{\bra{\varphi}_S\hat{U}\ket{\Psi^{i}}_T}{\bra{\varphi}_S\ket{\psi}_S} &= \ket{\varphi}_S\frac{\bra{\varphi}_S\ket{\psi}_S}{\bra{\varphi}_S\ket{\psi}_S} \otimes \ket{\bar{p} = 0}_P - i\delta\ket{\varphi}_S\frac{\bra{\varphi}_S\hat{S}\ket{\psi}_S}{\bra{\varphi}_S\ket{\psi}_S}\otimes\hat{q}\ket{\bar{p} = 0}_P\label{eq:pre-div}\\
        &= \ket{\varphi}_S \otimes \ket{\bar{p} = 0}_P - i\delta\ket{\varphi}_S\frac{\bra{\varphi}_S\hat{S}\ket{\psi}_S}{\bra{\varphi}_S\ket{\psi}_S}\otimes\hat{q}\ket{\bar{p} = 0}_P
    \end{align}

    \noindent Dans l'expression \ref{eq:pre-div}, nous avons divisé par
    $\bra{\varphi}_S\ket{\psi}_S$, dans le but de normaliser l'état du
    système total. Dans ce cas, le $\frac{\bra{\varphi}_S\ket{\psi}_S}{\bra{\varphi}_S\ket{\psi}_S}$
    sur le côté droit est annulé et nous déplaçons le $\frac{1}{\bra{\varphi}_S\ket{\psi}_S}$
    du côté gauche vers le côté droit. L'état final est maintenant le
    suivant :\textcolor{red}{FIXED}

    \begin{align}
        \ket{\Psi^f}_T &\equiv \ket{\varphi}_S\bra{\varphi}_S\hat{U}\ket{\Psi^i}_T\\
        &\simeq \bra{\varphi}_S\ket{\psi}_S \Bigl[ \ket{\bar{p} = 0}_P - i\delta\frac{\bra{\varphi}_S\hat{S}\ket{\psi}_S}{\bra{\varphi}_S\ket{\psi}_S} \hat{q}\ket{\bar{p}=0}_P \Bigr] \otimes \ket{\varphi}_S
    \end{align}

    \noindent Les crochets représentent l’état final du 
    pointeur, c'est-à-dire le déplacement du pointeur après l’interaction
    faible s'écrivant comme suit:
    %ce qui nous permet de calculer les parties réelles et 
    %imaginaires de $\hat{S}$. 

    \begin{equation}
        \ket{\bar{p} = \delta s_j}_P \equiv \ket{\bar{p} = 0}_P - i\delta\frac{\bra{\varphi}_S\hat{S}\ket{\psi}_S}{\bra{\varphi}_S\ket{\psi}_S} \hat{q}\ket{\bar{p}=0}_P
    \end{equation}

    \noindent Nous observons que la position finale du pointeur est 
    proportionnelle à ce qui suit :

    \begin{equation}
        \expval{\hat{S}}_W \equiv \frac{\bra{\varphi}_S\hat{S}\ket{\psi}_S}{\bra{\varphi}_S\ket{\psi}_S}
    \end{equation}

    \noindent Il s’agit de la valeur faible dérivée pour la première 
    fois par AAV, une valeur qui peut être complexe 
    c'est-à-dire composée d’une partie réelle et 
    d’une partie imaginaire. Cette valeur correspond au décalage de la 
    variable du pointeur $p$ et à son décalage par rapport à sa variable 
    conjuguée $q$. En d’autres termes, si une particule présente un 
    décalage dans sa position, il y aura également un décalage dans sa 
    quantité de mouvement. Par exemple, le temps d'arrivée d’un photon et 
    sa fréquence centrale varieront l’une par rapport à l’autre lors 
    d’une interaction. Si l’interaction est faible, il est possible de 
    mesurer ces valeurs décalées individuellement lors d’une procédure 
    directe via une mesure faible \cite{Hairiri,Lundeen_Resch,Lundeen_Direct_Measurement,Guilleaum}. 
    Pour conclure, écrivons l’état final avec cette valeur :\textcolor{red}{FIXED}

    \begin{align}
        \ket{\Psi^f}_T &= \bra{\varphi}_S\ket{\psi}_S \Bigl[ \ket{\bar{p} = 0}_P - i\delta \expval{\hat{S}}_W \hat{q}\ket{\bar{p}=0}_P \Bigr] \otimes \ket{\varphi}_S\\
        &= \bra{\varphi}_S\ket{\psi}_S\Bigl[ 1 - i\delta \expval{\hat{S}}_W \hat{q} \Bigr] \ket{\bar{p} = 0}_P \otimes \ket{\varphi}_S\\
        &= \bra{\varphi}_S\ket{\psi}_S e^{-i\delta \expval{\hat{S}}_W \hat{q}} \ket{\bar{p} = 0}_P \otimes \ket{\varphi}_S\\
        &= \ket{\psi}_S e^{-i\delta \expval{\hat{S}}_W \hat{q}}\ket{\bar{p}=0}_P
    \end{align}

    \noindent Nous observons que si nous avons une mesure faible parfaite, 
    en prenant la limite que $\delta \to 0 $, nous avons essentiellement 
    l’état initial $\ket{\Psi^f}_T \approx \ket{\Psi^i}_T$. 
    Néanmoins, dans le cas pratique, nous avons un état final qui est légèrement décalé par rapport à 
    l’état initial, ce qui nous permet de mesurer la valeur faible 
    $\expval{\hat{S}}_W$ et de caractériser l’état quantique du système à partir de la 
    lecture de la variable et la variable conjuguée du pointeur.

    \noindent Nous pouvons même mesurer et caractériser l'état 
    quantique directement sans aucune reconstruction algorithmique à 
    l'aide des composantes réelles et imaginaires de la valeur faible
    comme mentionné dans l'équation \ref{eq:weakvalue_components}. 
    Démontrons cela en mesurant les observables $\hat{p}$ et $\hat{q}$ 
    comme dans \cite{Lundeen_thesis,Lundeen_Resch}. Commençons
    avec $\expval{\hat{p}}$ :\textcolor{red}{FIXED}

    \begin{align}
        \bra{\bar{p} = \delta s_j}\hat{p}\ket{\bar{p}=\delta s_j} &= \bra{\bar{p}=0} \Bigl(e^{-i\delta \expval{\hat{S}}_W \hat{q}}\Bigr)^{\dagger} \hat{p} \Bigl(e^{-i\delta \expval{\hat{S}}_W \hat{q}}\Bigr) \ket{\bar{p}=0}\\
        &= \bra{\bar{p}=0} \Bigl(1 + i\delta \expval{\hat{S}}_W^{*} \hat{q}\Bigr) \hat{p} \Bigl(1 - i\delta \expval{\hat{S}}_W \hat{q}\Bigr) \ket{\bar{p}=0}\\
        &= \bra{\bar{p}=0} \hat{p} \ket{\bar{p}=0} + i\delta \expval{\hat{S}}_W^{*} \bra{\bar{p}=0} \hat{q}\hat{p} \ket{\bar{p}=0}\\
        &- i\delta \expval{\hat{S}}_W \bra{\bar{p}=0} \hat{p}\hat{q} \ket{\bar{p}=0} + \mathcal{O}(\delta^2)
    \end{align}

    \noindent En négligeant les termes d’ordre supérieur et sachant que  
    $\bra{\bar{p}=0}\hat{p}\ket{\bar{p}=0} = 0$, nous obtenons:

    \begin{align}
        \bra{\bar{p} = \delta s_j}\hat{p}\ket{\bar{p}=\delta s_j} &=  -i\delta \expval{\hat{S}}_W^{*} \bra{\bar{p}=0} \hat{q}\hat{p} \ket{\bar{p}=0} \\
        &+ i\delta \expval{\hat{S}}_W \bra{\bar{p}=0} \hat{p}\hat{q} \ket{\bar{p}=0}\\
        &= -i\delta \Bigl( \mathcal{R}\Bigl(\expval{\hat{S}}_W\Bigr) - i\mathcal{I}\Bigl(\expval{\hat{S}}_W\Bigr) \Bigr) \bra{\bar{p}=0} \hat{q}\hat{p} \ket{\bar{p}=0}\\
        &+ i\delta \Bigl( \mathcal{R}\Bigl(\expval{\hat{S}}_W\Bigr) + i\mathcal{I}\Bigl(\expval{\hat{S}}_W\Bigr) \Bigr) \bra{\bar{p}=0} \hat{p}\hat{q} \ket{\bar{p}=0}\\
        &= -\delta \mathcal{R}(\expval{\hat{S}}_W)(\bra{\bar{p} = 0}\hat{p}\hat{q} \ket{\bar{p} = 0} - \bra{\bar{p} = 0}\hat{q}\hat{p} \ket{\bar{p} = 0})\\
        &+ i\delta \mathcal{I}(\expval{\hat{S}}_W)(\bra{\bar{p} = 0}\hat{p}\hat{q} \ket{\bar{p} = 0} + \bra{\bar{p} = 0}\hat{q}\hat{p} \ket{\bar{p} = 0})
    \end{align} 

    \begin{align}
        &= -i\delta \mathcal{R}(\expval{\hat{S}}_W)(\bra{\bar{p} = 0}\hat{p}\hat{q} - \hat{q}\hat{p}\ket{\bar{p} = 0})\\
        &+ \delta \mathcal{I}(\expval{\hat{S}}_W)(\bra{\bar{p} = 0}\hat{p}\hat{q} + \hat{q}\hat{p}\ket{\bar{p} = 0})\\
        &= -i\delta \mathcal{R}(\expval{\hat{S}}_W) \bra{\bar{p} = 0}[\hat{p},\hat{q}]\ket{\bar{p} = 0} \\
        &+ \delta \mathcal{I}(\expval{\hat{S}}_W) \bra{\bar{p} = 0}\{\hat{p},\hat{q}\}\ket{\bar{p} = 0}
    \end{align}

    \noindent En utilisant les relations de
    commutation $[\hat{p},\hat{q}] = i$ et
    l'anticommutateur $\{\hat{p},\hat{q}\} = -i$, nous obtenons:
    
    \begin{align}
        \bra{\bar{p} = \delta s_j}\hat{p}\ket{\bar{p}=\delta s_j} &= \delta\mathcal{R}\Bigl(\expval{\hat{S}}_W\Bigr) = \expval{\hat{p}}
    \end{align}

    \noindent Ensuite, répétons les mêmes étapes pour $\expval{\hat{q}}$:

    \begin{align}
        \bra{\bar{p} = \delta s_j}\hat{q}\ket{\bar{p}=\delta s_j} &= \bra{\bar{p}=0} \Bigl(e^{-i\delta \expval{\hat{S}}_W \hat{q}}\Bigr)^{\dagger} \hat{q} \Bigl(e^{-i\delta \expval{\hat{S}}_W \hat{q}}\Bigr) \ket{\bar{p}=0}\\
        &= \bra{\bar{p}=0} \Bigl(1 + i\delta \expval{\hat{S}}_W^{*} \hat{q}\Bigr) \hat{q} \Bigl(1 - i\delta \expval{\hat{S}}_W \hat{q}\Bigr) \ket{\bar{p}=0}\\
        &= \bra{\bar{p}=0} \hat{q} \ket{\bar{p}=0} + i\delta \expval{\hat{S}}_W^{*} \bra{\bar{p}=0} \hat{q}\hat{q} \ket{\bar{p}=0}\\
        &- i\delta \expval{\hat{S}}_W \bra{\bar{p}=0} \hat{q}\hat{q} \ket{\bar{p}=0} + \mathcal{O}(\delta^2)
    \end{align}

    \noindent En négligeant les termes d’ordre supérieur et sachant que  
    $\bra{\bar{p}=0}\hat{q}\ket{\bar{p}=0} = 0$, nous obtenons:

    \begin{align}
        \bra{\bar{p} = \delta s_j}\hat{q}\ket{\bar{p}=\delta s_j} &=  i\delta \expval{\hat{S}}_W^{*} \bra{\bar{p}=0} \hat{q}\hat{q} \ket{\bar{p}=0} \\
        &- i\delta \expval{\hat{S}}_W \bra{\bar{p}=0} \hat{q}\hat{q} \ket{\bar{p}=0}\\
        &= i\delta \Bigl( \mathcal{R}\Bigl(\expval{\hat{S}}_W\Bigr) - i\mathcal{I}\Bigl(\expval{\hat{S}}_W\Bigr) \Bigr) \bra{\bar{p}=0} \hat{q}\hat{q} \ket{\bar{p}=0}\\
        &- i\delta \Bigl( \mathcal{R}\Bigl(\expval{\hat{S}}_W\Bigr) + i\mathcal{I}\Bigl(\expval{\hat{S}}_W\Bigr) \Bigr) \bra{\bar{p}=0} \hat{q}\hat{q} \ket{\bar{p}=0}\\
        &= i\delta \mathcal{R}\Bigl(\expval{\hat{S}}_W\Bigr)(\bra{\bar{p} = 0}\hat{q}^2 \ket{\bar{p} = 0} - \bra{\bar{p} = 0}\hat{q}^2 \ket{\bar{p} = 0})\\
        &- \delta \mathcal{I}\Bigl(\expval{\hat{S}}_W\Bigr)(\bra{\bar{p} = 0}\hat{q}^2 \ket{\bar{p} = 0} + \bra{\bar{p} = 0}\hat{q}^2 \ket{\bar{p} = 0})\\
        &= i\delta \mathcal{R}\Bigl(\expval{\hat{S}}_W\Bigr)(\bra{\bar{p}= 0}\hat{q}^2 - \hat{q}^2\ket{\bar{p}=0})\\
        &- \delta \mathcal{I}\Bigl(\expval{\hat{S}}_W\Bigr)(\bra{\bar{p}= 0}\hat{q}^2 + \hat{q}^2\ket{\bar{p}=0})
    \end{align}

    \noindent En utilisant la relation $\bra{\bar{p}=0}\hat{q}^2\ket{\bar{p}=0} = \frac{1}{8\sigma^2}$, nous obtenons:

    \begin{align}
        \bra{\bar{q} = \delta s_j}\hat{q}\ket{\bar{p}=\delta s_j} = \frac{\delta}{4\sigma^2}\mathcal{I}\Bigl(\expval{\hat{S}}_W\Bigr) = \expval{\hat{q}}
    \end{align}

    \noindent Ensemble, la valeur faible s'écrit:

    \begin{equation}
        \expval{\hat{S}}_W = \frac{1}{\delta}\Bigl( \expval{\hat{p}} - i4\sigma^2\expval{\hat{q}} \Bigr)
    \end{equation}

    \noindent En démontrant que la valeur faible est proportionnelle à 
    la fonction d’onde (l'état quantique), comme dans 
    \cite{Lundeen_Direct_Measurement,Lundeen_thesis,Lundeen_Resch}, ainsi on peut en déduire directement les 
    valeur moyenne des observables du système quantique, qu'ils sont en fait
    les composantes réelles et imaginaires de la valeur faible. 
    Nous pouvons caractériser l'état de polarisation du
    système en mesurant la valeur faible sans avoir besoin de
    reconstruire la matrice densité.
    Cette approche directe permet de surmonter les limitations de la
    tomographie quantique traditionnelle, qui nécessite un grand nombre
    de mesures projectives pour obtenir une estimation précise de la matrice densité.
    En conséquence, nous avons ouvert un tout nouveau domaine dans les
    mesures quantiques et une alternative à la tomographie quantique
    traditionnelle \cite{Lundeen_Direct_Measurement,Guilleaum}.

\end{onehalfspace}

\subsection{Proposition d'une procédure directe avec une mesure faible temporelle}

    
\begin{onehalfspace}
    Les mesures faibles temporelles exploitent les propriétés temporelles 
    et fréquentielles d’une impulsion lumineuse pour caractériser un état 
    quantique. Cette approche repose sur lequel que
    les délais temporels peuvent être directement liés aux composantes 
    réelles et imaginaires de la valeur faible. Dans les sections 
    suivantes et dans ce projet de thèse, nous nous concentrerons sur la 
    mesure de la valeur faible à partir d’une interaction faible 
    temporelle, correspondante à un délai temporel, introduit sur le système. 
    Nous décrivons la procédure proposée pour
    caractériser l'état de polarisation d'un système photonique 
    en effectuant des mesures faibles répétées sur un 
    grand ensemble d’impulsions identiquement préparées, en suivant les principes théoriques décrits précédemment,
    pour retrouver nos attentes théoriques sur la partie réelle et imaginaire de la valeur faible. 
    %Ces mesures nous permettront de déterminer la 
    %position temporelle moyenne d’une impulsion gaussienne, utilisée 
    %comme pointeur, ainsi que l’effet sur sa variable conjuguée, un 
    %décalage fréquentiel variant avec la valeur faible. Ce faisant, nous 
    %pourrons caractériser l’état de polarisation du système.
\end{onehalfspace}

\subsubsection{La partie réelle du système}
    
\begin{onehalfspace}
    Nous voulons caractériser l'état de polarisation avec une mesure 
    faible temporelle. Pour ce faire, il est nécessaire de calculer 
    les composantes de la valeur faible $\expval{\hat{\pi}}_W$ du 
    système attendue que nous allons implémenter. Pour y parvenir, nous devons 
    d’abord analyser chaque composante de cette valeur. Tout d’abord, 
    examinons la partie réelle en définissant les paramètres de 
    l’expérience potentielle que nous souhaitons éventuellement réaliser. 
    L’état de polarisation du système que nous souhaitons mesurer est 
    défini comme suit :

    \begin{equation}
        \ket{\psi} \equiv a\ket{H} + b\ket{V}
    \end{equation}

    \noindent où $a$ et $b$ sont les amplitudes de probabilité associées aux états de polarisation 
    $\ket{H}$ et $\ket{V}$, respectivement, avec 
    $|a|^2 + |b|^2 = 1$, ainsi que 
    $\bra{H}\ket{\psi} = a$ et $\bra{V}\ket{\psi} = b$.
    Le pointeur du système utilisé possède un profil temporel 
    gaussien pour un faisceau, caractérisé par une variable temporelle 
    $t$ et une largeur temporelle $\sigma$. Le profil temporel du pointeur
    s’écrit comme suit:
    
    \begin{equation}
        \ket{\xi(t)} = \bra{t}\ket{\xi} \equiv \frac{1}{(\sqrt{2\pi}\sigma)^{1/2}}e^{-\frac{t^2}{4\sigma^2}}
    \end{equation}

    \noindent Ensemble, l’état initial total s’écrit:

    \begin{equation}
        \ket{\Psi(t)^i} \equiv \ket{\psi} \otimes \ket{\xi(t)}
    \end{equation}


    \noindent Procédons à une faible interaction temporelle sur l’état 
    $\ket{H}$ avec l’opérateur de von Neumann $\hat{U}^H$, dont l’indice 
    $H$ indique que l'interaction est seulement appliquée sur la composante horizontale. L’opérateur 
    d’interaction de von Neumann peut être étudié sous la forme 
    $\hat{U} = exp(-\frac{i}{\hbar}\int \mathcal{\hat{H}}dt)$, où 
    $\mathcal{\hat{H}} \equiv g(t)(\hat{\pi}\otimes\hat{E})$, avec 
    $\hat{\pi}$ l'observable du système (dont c'est valeurs propres sont les états de polarisation que le système peut prendre), 
    $\hat{E}$ la variable conjuguée du pointeur et $g(t)$ la force de couplage. Nous voulons appliquer 
    l’opérateur sur un état de base unique, donc l’observable du système 
    $\hat{\pi}$ devient $\ket{J}\bra{J}$, où $J \equiv H,V$ 
    \cite{Lundeen_Bamber}. La variable conjuguée de 
    la variable du pointeur est l'énergie $\hat{E}$, puisque, en 
    mécanique quantique, le temps et l’énergie se 
    comportent réciproquement \cite{Peebles,Griffiths}. L’opérateur 
    d’énergie s’écrit ainsi :

    \begin{equation}
        \hat{E} = i\hbar\frac{\partial}{\partial t}
    \end{equation}

    \noindent L’interaction de von Neumann pour une interaction 
    temporelle sur la composante horizontale (i.e. $J=H$) s’écrit alors comme ceci :

    \begin{equation}
        \hat{U}^H = exp\Bigl(-i\tau\ket{H}\bra{H} \otimes \frac{\partial}{\partial t}\Bigr)
    \end{equation}

    \noindent où $\int g(t) dt = \tau$ est le temps d’interaction appliqué sur le système
    et se trouve dans le régime de 
    mesures faibles, c'est-à-dire que le temps d'interaction 
    est beaucoup plus court que 
    la largeur temporelle du pointeur $\tau \ll \sigma$ (voir figure \ref{fig:interaction}). 
    Nous appliquons cet opérateur sur la partie horizontale de l’état $\ket{H}$
    désignée par l'indice $H$ sur l'opérateur, ce qui entraine un décalage du 
    pointeur $exp\Bigl( -\tau \frac{\partial }{\partial t} \Bigr) \xi(t) = \xi(t-\tau)$. 
    L’état évolue ensuite de cette manière:

    \begin{align}
        \hat{U^H}\ket{\Psi(t)^i} &= \hat{U}^H \Bigl[ \ket{\psi} \otimes \ket{\xi(t)} \Bigr]\\
        &= \hat{U}^H \Bigl[ a\ket{H} \otimes \ket{\xi(t)} + b\ket{V} \otimes \ket{\xi(t)}\Bigr]\\
        &= a\ket{H} \otimes \hat{U}^H \ket{\xi(t)} + b\ket{V} \otimes \ket{\xi(t)}\\
        &= a\ket{H} \otimes \ket{\xi(t-\tau)} + b\ket{V} \otimes \ket{\xi(t)}
    \end{align}

    \noindent Comme nous l’avons mentionné précédemment, 
    l’interaction faible préserve la superposition des états de polarisation 
    et n'effondre pas l'état complètement en comparaison 
    d'une mesure forte (voir la figure \ref{fig:interaction}).
    Pour en extraire des informations sur l'état quantique, nous réalisons 
    une postsélection à 
    l’aide d'un état superposé $\ket{\varsigma} = \mu\ket{H} + \nu\ket{V}$. 
    Les paramètres $\nu$ et $\mu$ représentent les amplitudes de 
    probabilités respectives des états $\ket{H}$ et $\ket{V}$ de l'état 
    de projection et que $|\mu|^2 + |\nu|^2 = 1$. L'état s'écrit:

    \begin{align}
        \ket{\Psi(t)^f} &= \ket{\varsigma}\bra{\varsigma}\hat{U}^H\ket{\Psi(t)^i} \\
        &= \Bigl[ \bar{\mu}\bra{H} + \bar{\nu}\bra{V} \Bigr]\Bigl[ a\ket{H} \otimes \ket{\xi(t-\tau)} + b\ket{V} \otimes \ket{\xi(t)} \Bigr] \otimes \ket{\varsigma}\\
        &= \Bigl[\bar{\mu}a\ket{\xi(t-\tau)} + \bar{\nu}b\ket{\xi(t)}\Bigr] \otimes \ket{\varsigma}\\
        &= F(t)\otimes\ket{\varsigma}
    \end{align}

    \noindent où $F(t) \equiv A\ket{\xi(t-\tau)} + B\ket{\xi(t)}$, 
    $A \equiv a\bar{\mu}$ et $B \equiv b\bar{\nu}$. Trouvons la valeur 
    moyenne de la position du pointeur $\expval{\hat{t}}$:

    \begin{align}
        \expval{\hat{t}} &= \bra{\Psi(t)^f}\hat{t}\ket{\Psi(t)^f}\\
        &= \int_{-\infty}^{\infty} F(t)^* t F(t) dt
    \end{align}

    \noindent Ensuite, nous pouvons le normaliser en 
    divisant par $\frac{1}{\bra{\Psi(t)^f}\ket{\Psi(t)^f}}$ :

    \begin{align}
        \expval{\hat{t}^{norm}} &= \frac{\bra{\Psi(t)^f}\hat{t}\ket{\Psi(t)^f}}{\bra{\Psi(t)^f}\ket{\Psi(t)^f}} = \frac{\int_{-\infty}^{\infty} I(t)tdt}{\int_{-\infty}^{\infty} I(t)dt}\\
        &= \frac{\int_{-\infty}^{\infty} |A|^2\Xi(t - \tau)t + |B|^2\Xi(t)t + A\bar{B}\Xi(t, \tau)t + \bar{A}B\Xi(t, \tau)t dt}{\int_{-\infty}^{\infty} |A|^2\Xi(t - \tau) + |B|^2\Xi(t) + A\bar{B}\Xi(t, \tau) + \bar{A}B\Xi(t, \tau) dt}\label{eq:t_norm_interférence}
    \end{align}

    \noindent où $\Xi(t) \equiv \frac{1}{\sqrt{2\pi}\sigma}e^{-\frac{t^2}{2\sigma^2}}$ 
    et $\Xi(t, \tau) \equiv \frac{1}{\sqrt{2\pi}\sigma}e^{-\frac{t^2 + (t - \tau)^2}{4\sigma^2}}$. 
    Remarquons qu’en raison d’une interaction faible avec le système, il 
    existe une superposition entre le pointeur et les composantes de 
    polarisation horizontale et verticale. Les solutions de chaque 
    intégrale sont énumérées ci-dessous:

    \begin{align*}
        \int_{-\infty}^{\infty} \Xi(t - \tau)t dt &= \tau & \int_{-\infty}^{\infty} \Xi(t) dt &= 1\\
        \int_{-\infty}^{\infty} \Xi(t - \tau) dt &= 1 & \int_{-\infty}^{\infty} \Xi(t, \tau)t dt &= \frac{\tau}{2}e^{-\frac{t^2}{8\sigma^2}}\\
        \int_{-\infty}^{\infty} \Xi(t)t dt &= 0 & \int_{-\infty}^{\infty} \Xi(t, \tau) dt &= e^{-\frac{t^2}{8\sigma^2}}
    \end{align*}

    \noindent Ensuite, nous allons reprendre notre analyse de la 
    partie réelle de la valeur faible à partir de 
    l'expression \eqref{eq:t_norm_interférence} et avec 
    ces solutions dérivées, la partie réelle se trouve:

    \begin{align}
        \expval{\hat{t}^{norm}} = \tau\frac{|A|^2 + (A\bar{B} + \bar{A}B)e^{-\frac{\tau^2}{8\sigma^2}}}{|A|^2 + |B|^2 + (A\bar{B} + \bar{A}B)e^{-\frac{\tau^2}{8\sigma^2}}}
        \label{eq:expval_t_norm}
    \end{align}

   \noindent Comme nous sommes dans le régime des mesures faibles, 
   prenons la limite où $\tau \ll \sigma$:\textcolor{red}{FIXED}

    \begin{align}
        \lim_{\frac{\tau}{\sigma} \to 0} \expval{\hat{t}^{norm}} &= \tau\frac{|A|^2 + A\bar{B} + \bar{A}B}{|A|^2 + |B|^2 + A\bar{B} + \bar{A}B}\\
        &\equiv \tau \mathcal{R}(\expval{\hat{\pi}}_W)
    \end{align}

    \noindent Ce terme représente la position moyenne temporelle du 
    pointeur lors d’une mesure. Il s’agit de la partie réelle de la 
    valeur faible $\expval{\hat{\pi}}_W$.
\end{onehalfspace}

\subsubsection{La partie imaginaire du système}
    
\begin{onehalfspace}
    Comme nous l'avons déjà évoqué, un déplacement de la variable du 
    pointeur, tel qu'un décalage $\tau$ de sa position temporelle $t$, 
    devrait entraîner un déplacement de son spectre fréquentiel $\omega$, 
    car $\hat{E} = \hbar\hat{\omega}$ \cite{Peebles,Griffiths}. 
    Vérifions-le en calculant la partie imaginaire de la valeur faible 
    $\expval{\hat{\pi}}_W$. Tout d’abord, effectuons la transformation de 
    Fourier de la fonction temporelle $F(t)$ de l'état quantique 
    $\ket{\Psi(t)^f}$:

    \begin{align}
        F(\omega) &= \frac{1}{\sqrt{2\pi}}\int_{-\infty}^{\infty} F(t)e^{-i\omega t}dt\\
        &= \frac{\sqrt[4]{2}\sqrt{\sigma}}{\sqrt[4]{\pi}}(A + Be^{i\omega\tau})e^{-\omega^2 \sigma^2 - i\omega\tau}
    \end{align}

    \noindent Avec ce dernier, l’état quantique s’écrit :

    \begin{equation}
        \ket{\Psi(\omega)^f} = F(\omega) \otimes \ket{\varsigma}
    \end{equation}

    \noindent Ensuite, déterminons la valeur moyenne de la position 
    fréquentielle en suivant les mêmes étapes que celles 
    pour la partie réelle :
    
    \begin{align}
        \expval{\hat{\omega}} &= \bra{\Psi(\omega)^f}\hat{\omega}\ket{\Psi(\omega)^f}\\
        &= \int_{-\infty}^{\infty} F(\omega)^* \omega F(\omega) d\omega
    \end{align}

    \noindent Normalisons cette expression en divisant par 
    $\frac{1}{\bra{\Psi(\omega)^f}\ket{\Psi(\omega)^f}}$:

    \begin{align}
        \expval{\hat{\omega}^{norm}} &= \frac{\sqrt{2}\sigma}{\sqrt{\pi}}\frac{\int_{-\infty}^{\infty} |A|^2\omega e^{i\omega\tau} + |B|^2\omega e^{i\omega\tau} + A\bar{B}\omega + \bar{A}Be^{2i\omega\tau} \omega d\omega}{\int_{-\infty}^{\infty} |A|^2 e^{i\omega\tau} + |B|^2 e^{i\omega\tau} + A\bar{B} + \bar{A}Be^{2i\omega\tau}d\omega}e^{-2\omega^2 \sigma^2 -i\omega\tau}
    \end{align}

    \noindent En utilisant des méthodes d’intégration similaires, nous 
    arrivons à :

    \begin{equation}
        \expval{\hat{\omega}^{norm}} = \frac{i\tau}{4\sigma^2}\frac{(B\bar{A} - A\bar{B})e^{-\frac{\tau^2}{8\sigma^2}}}{|A|^2 + |B|^2 + \bar{A}B + A\bar{B}}
    \end{equation}

    \noindent Prenons encore la limite dont $\tau \ll \sigma$, qui 
    s’applique au domaine des mesures faibles:\textcolor{red}{FIXED}

    \begin{align}
        \lim_{\frac{\tau}{\sigma} \to 0}\expval{\hat{\omega}^{norm}} &= \frac{i\tau}{4\sigma^2}\frac{B\bar{A} - A\bar{B}}{|A|^2 + |B|^2 + \bar{A}B + A\bar{B}}\\
        &\equiv \frac{\tau}{4\sigma^2}\mathcal{I}(\expval{\hat{\pi}}_W)
        \label{eq:imaginary_part}
    \end{align}

    \noindent Ce terme correspond à la partie imaginaire de la valeur 
    faible attendue $\expval{\hat{\pi}}_W$.

\end{onehalfspace}

\subsection{Proposition expérimentale pour la caractérisation de la valeur faible}\label{sec:proposition_exp}
    
\begin{onehalfspace}
    Nous continuons avec notre système photonique quantique, en nous 
    appuyant sur nos résulats théoriques concernant la partie réelle et 
    imaginaire de la valeur faible. Pour un 
    état d’entrée quelconque:

    \begin{equation}
        \ket{\psi^{i}} = a\ket{H} + b\ket{V} 
    \end{equation}

     \noindent puisque $a$ et $b$ sont des amplitudes de probabilité 
     associées aux états de base $\ket{H}$ et $\ket{V}$ respectivement 
     (c'est-à-dire $a=\bra{H}\ket{\psi^{i}}$ et $b=\bra{V}\ket{\psi^{i}}$)
     et encore que $|a|^2 + |b|^2 = 1$, on peut, en pratique, calculer directement ces amplitudes de 
     probabilités en fonction des parties de la valeur faible mesurée. Cette 
     possibilité découle du fait que la valeur faible est proportionnelle 
     à l'état quantique, comme nous l'avons démontré. Pour caractériser 
     l'état de polarisation d'un système quantique, il s'agit de mesurer 
     faiblement $\hat{\pi}^J = \ket{J}\bra{J}$ soit $J= H,V$ \cite{Hairiri,Lundeen_Direct_Measurement,Lundeen_Bamber}, 
     puis de mesurer par projection sur un état intermédiaire tel que 
     $\ket{D} = \frac{1}{\sqrt{2}}(\ket{H}+\ket{V})$. Avec cette
     mesure, nous pouvons obtenir un ensemble restreint d’essais 
     dont la moyenne des résultats expérimentaux (tels que les 
     déplacements temporels ou fréquentiels du pointeur) sera 
     proportionnelle à la partie réelle ou imaginaire de la valeur 
     faible:
    \textcolor{red}{FIXED}
     \begin{equation}
        \expval{\hat{\pi}^{J}}_W = \frac{\bra{D}\hat{\pi}^{J}\ket{\psi^{i}}}{\bra{D}\ket{\psi^{i}}} = \sqrt{N}\bra{J}\ket{\psi^{i}}\label{eq:weak_value_J}
    \end{equation}

    \noindent \textcolor{red}{où $N$ est une constante de normalisation indépendante de 
    $J$. On peut écrire l’état quantique d'entrée, après avoir subit une mesure directe, 
    en fonction de la valeur faible mesurée \cite{Lundeen_Direct_Measurement}: }\textcolor{red}{FIXED}

    \begin{equation}
        \ket{\psi^{f}} = \frac{1}{\sqrt{N}}\Bigl(\expval{\hat{\pi}^{H}}_W\ket{H}+\expval{\hat{\pi}^{V}}_W\ket{V}\Bigr)
    \end{equation}

    \noindent \textcolor{red}{Puisque $N$ est une constante de normalisation telle que}\textcolor{red}{FIXED}
    
    \begin{align}
        N &= \Bigl|\expval{\hat{\pi}^{H}}_W\Bigr|^2 + \Bigl|\expval{\hat{\pi}^{V}}_W\Bigr|^2\\
          &= \Bigl|\expval{\hat{\pi}^{H}}_W\Bigr|^2 + \Bigl|1- \expval{\hat{\pi}^{H}}_W\Bigr|^2
    \end{align}

    \noindent \textcolor{red}{nous pouvons récrire l'état quantique sous la forme suivante :}\textcolor{red}{FIXED}

    \begin{equation}
        \ket{\psi^{f}} = \frac{1}{\sqrt{N}}\Bigl(\expval{\hat{\pi}^{H}}_W\ket{H}+ \Bigl(1- \expval{\hat{\pi}^{H}}_W\Bigr)\ket{V}\Bigr)
    \end{equation}

    \noindent \textcolor{red}{Ce dernier est l'état d'entrée que nous pouvons 
    reconstruire à partir de la valeur faible, où $a = \frac{\expval{\hat{\pi}^{H}}_W}{\sqrt{N}}$
    et $b = \frac{1-\expval{\hat{\pi}^{H}}_W}{\sqrt{N}}$. En d'autres termes, nous 
    pouvons reconstruire directement l'état d'entrée à partir de la valeur faible 
    mesurée sans ambiguïté. Certains fixent la phase globale \cite{Hairiri}, qui varierait selon l'état 
    d'entrée, nous supposons que $a$ est toujours réel; donc 
    l'ellipticité (ou la phase) se trouve dans $b$ et sera dépendante 
    de la partie imaginaire.}
    
    \noindent \textcolor{red}{En performant une mesure faible
    sur $\ket{\psi^{i}}$, décrit par l'opérateur $\hat{\pi}^{H} = \ket{H}\bra{H}$, 
    avec un délai $\tau$, nous pouvons reconstruire l'état à partir 
    des données expérimentales des deux 
    valeurs moyennes $\expval{\hat{t}}$ et $\expval{\hat{\omega}}$ et
    calculer directement les amplitudes de probabilité. 
    En utilisant les équations \eqref{eq:weak_value_J}, \eqref{eq:expval_t_norm}
    et \eqref{eq:imaginary_part}, nous obtenons les relations suivantes }:\textcolor{red}{FIXED}

    %\noindent pour le cas générale où $a,b\in\mathcal{C}$, nous avons :

    %\begin{align}
    %    \mathcal{R}(a) &= \mathcal{R}(\expval{\hat{\pi}^{H}}_W) = \frac{\expval{\hat{t}}}{\tau}\label{eq:real_part_a}\\
    %    \mathcal{I}(a) &= \mathcal{I}(\expval{\hat{\pi}^{H}}_W) = -\frac{4\sigma^2 \expval{\hat{\omega}}}{\tau}\label{eq:imaginary_part_a}\\
    %    \mathcal{R}(b) &= 1 - \mathcal{R}(\expval{\hat{\pi}^{H}}_W) = 1 - \frac{\expval{\hat{t}}}{\tau}\label{eq:real_part_b_general}\\
    %    \mathcal{I}(b) &= - \mathcal{I}(\expval{\hat{\pi}^{H}}_W) = \frac{4\sigma^2 \expval{\hat{\omega}}}{\tau}\label{eq:imaginary_part_b_general} 
    %\end{align}

    %\noindent et pour le cas où $a\in\mathcal{R}$ et $b\in\mathcal{I}$, nous avons :

    \begin{align}
        a &= \sqrt{\mathcal{R}(\expval{\hat{\pi}^{H}}_W)} = \sqrt{\frac{\expval{\hat{t}}}{\tau}}\label{eq:amplitude_a}\\
        |b| &= \sqrt{1 - |a|^2}\label{eq:amplitude_b}\\
        \mathcal{R}(b) &= 0\label{eq:real_part_b}\\
        \mathcal{I}(b) &= \frac{\mathcal{I}(\expval{\hat{\pi}^{H}}_W)}{\sqrt{\mathcal{R}(\expval{\hat{\pi}^{H}}_W)}} = \frac{4\sigma^2 \expval{\hat{\omega}}}{\sqrt{\tau\expval{\hat{t}}}}\label{eq:imaginary_part_b}
    \end{align}

    %Considérons l'équation de la valeur faible:

    %\begin{equation}
    %    \expval{\hat{\pi}}_W \equiv \hat{\pi}_W = \frac{\bra{\varsigma}\hat{\pi}\ket{\psi}}{\bra{\varsigma}\ket{\psi}}
    %\end{equation}

    %\noindent Le numérateur et dénominateur peuvre être écrit en termes de
    %l'amplitude de probabilité, soit \cite{OpticalNetworks}:

    %\begin{align}
    %    \bra{\varsigma}\hat{\pi}\ket{\psi} &= \bar{\mu}a - \bar{\nu}b\\
    %    \bra{\varsigma}\ket{\psi} &= \bar{\mu}a + \bar{\nu}b
    %\end{align}

    %\noindent Vue que nous avons $\bar{\mu} = \bar{\nu} = \frac{1}{\sqrt{2}}$ et 
    %$|a|^2 + |b|^2 = 1$, nous pouvons écrire la valeur faible
    %en termes de les amplitudes de probabilités:

    %\begin{equation}
    %    \mathcal{R}\Bigl( \expval{\hat{\pi}}_W \Bigr) = |a|^2 - |b|^2
    %\end{equation}

    %\noindent Nous pouvons donc écrire chaque amplitudes de probabilités
    %en termes de la partie réelle de la valeur faible, soit:

    %\begin{align}
    %    |a|^2 &= \frac{1}{2}\Bigl(1 + \mathcal{R}\Bigl( \expval{\hat{\pi}}_W \Bigr)\Bigr)\label{eq:amplitude_a}\\
    %    |b|^2 &= \frac{1}{2}\Bigl(1 - \mathcal{R}\Bigl( \expval{\hat{\pi}}_W \Bigr)\Bigr)\label{eq:amplitude_b}
    %\end{align} 

    %\noindent De même, pour la partie imaginaire de la valeur faible, 
    %nous avons :

    %\begin{equation}
    %    \mathcal{I}\Bigl( \expval{\hat{\pi}}_W \Bigr) = 2|a||b|\sin(\phi)
    %\end{equation}

    %Où $\phi$ est la phase de l'état d'entrée, qui est la différence 
    %de phase entre les deux états de base $\ket{H}$ et $\ket{V}$. 
    %En utilisant les équations \ref{eq:amplitude_a} et \ref{eq:amplitude_b}, 
    %nous pouvons écrire la partie imaginaire de la valeur faible en 
    %termes des amplitudes de probabilités similairement à la partie 
    %réelle:

    %\begin{align}
    %    |a|^2 &= \frac{1}{2}\Bigl(1 + \mathcal{I}\Bigl( \expval{\hat{\pi}}_W \Bigr)\Bigr)\label{eq:amplitude_a_im}\\
    %    |b|^2 &= \frac{1}{2}\Bigl(1 - \mathcal{I}\Bigl( \expval{\hat{\pi}}_W \Bigr)\Bigr)\label{eq:amplitude_b_im}
    %\end{align} 

    \noindent Donc, en variant l'état d'entrée avec un délai 
    $\tau$ fixed et connue, ainsi que la largeur temporelle et
    l'état de postsélection, nous pouvons mesurer le déplacement 
    temporel du pointeur et le décalage fréquentiel, qui sont 
    proportionnels à la partie réelle et imaginaire de la valeur faible, 
    respectivement. En utilisant les équations \ref{eq:amplitude_a} et 
    \ref{eq:amplitude_b}, nous pouvons reconstruire l'état d'entrée 
    $\ket{\psi^{i}}$ en fonction de la valeur faible mesurée de façon 
    directe, sans avoir besoin de reconstruire la matrice densité.
    
    %Il est 
    %crucial de souligner que le délai $\tau$ correspond au délai maximal 
    %que nous utilisons pour interagir avec le système. Ce dernier 
    %normalise les amplitudes de probabilité. Lorsque nous modifions les 
    %états d’entrée, le résultat de la mesure varie en fonction 
    %de la valeur faible mesurée
    
    %le délai $\tau$ devrait varier entre l’absence de 
    %délai et le délai maximal, c’est-à-dire entre les polarisations 
    %$\ket{V}$ et $\ket{H}$. 

\end{onehalfspace}

    
\begin{onehalfspace}
    \noindent Ce chapitre a posé les bases théoriques des mesures faibles 
    temporelles et leur potentiel pour les systèmes photoniques. En 
    s’appuyant sur des techniques innovantes et des travaux antérieurs, 
    cette thèse vise à démontrer l’utilité des mesures faibles 
    temporelles pour caractériser directement les états quantiques. Le 
    prochain chapitre abordera les aspects expérimentaux de la mise en 
    œuvre de ces méthodes.
\end{onehalfspace}
    %\subsection{Proposition d'une procédure directe avec une mesure faible temporelle}
    %Nous commencerons par exposer les fondements 
théoriques essentiels des mesures faibles et que la 
valeur faible est proportionnelle à la fonction d'onde 
quantique ainsi qu'elle peut être mesuré directement. 
Comme indiqué précédemment, les mesures faibles font 
partie de la procédure directe décrite par Jeff Lundeen 
et l'AAV, qui comprend les éléments 
suivants \cite{Lundeen_Bamber,Lundeen_Direct_Measurement,Aharonov}:

\begin{itemize}
    \item Préparation de l'état d'entrée
    \item Une interaction faible (ce dont nous discuterons dans cette section)
    \item Une mesure projective, généralement effectuée avec un état qui possède une quantité égale des deux états de base de l'état d'entrée.
\end{itemize}

L'étape sur laquelle nous nous concentrerons dans cette 
section est celle de la mesure faible, qui implique une 
faible réduction de la fonction d'onde quantique, plus 
sur ceci à suivre. Comme évoqué précédemment, les 
principes des mesures faibles s’appuient sur le modèle 
de Von Neumann pour les mesures quantiques. Ce modèle 
implique l'état quantique que l'on souhaite à mesurer $S$ 
et le pointeur (l'appareil de mesure) $P$, qui sont traités 
comme des objets de la mécanique quantique couplée dans un
système totale $T$ \cite{Hairiri,vonNeumann}. 
La plupart des mesures quantiques peuvent généralement 
être décrites par ce modèle. Le modèle de von Neumann 
décrit que lorsqu'un état quantique est mesuré,
initialement dans un état de superposition arbitraire
$\ket{\psi}_S = \sum_{j}^{N}c_{j}\ket{s_j}_S$, soit 
avec des états propres $\ket{s_j}_S$ en base $S$, 
valeurs propre $s_j$, coefficient d'amplitude de probabilité,
une observable $\hat{S}$ qui doit être 
mesurée et dimension $N$ . Elle subit une 
réduction à l'un de ses vecteurs propres associés 
avec sa valeur propre \cite{Griffiths}. 
La mesure est décrite par un opérateur d'interaction 
appelé l'opérateur d'interaction de von Neumann :

\begin{equation}
    \hat{U} \equiv exp\Bigl( -\frac{i\mathcal{H}t}{\hbar} \Bigr)
\end{equation}

Il s'agit d'un opérateur d'évolution temporelle soit 
$t$ le temps d'interaction sur le système, $\hbar$ 
la constante de Planck et $\mathcal{H}$ le hamiltonien du 
système $T$ décrit par :

\begin{equation}
    \mathcal{H} \equiv g(\hat{S} \otimes \hat{p})
\end{equation}

Soit $g$ la constante de couplage qui est supposé d'être 
réelle pour que le hamiltonien reste hermitien et 
$\hat{p}$ la variable pointeur conjuguée de 
l'observable mesurée. Cette réduction de l'état 
quantique est représentée par un déplacement de la 
position du pointeur soit initialement dans un état 
$\ket{\xi}_P = \ket{\bar{q} = 0}_P$ en base $P$
dont $\bar{q}$ la valeur centrale d'une variable $q$
avec une écart de la distribution de probabilité $\sigma$.
Ensemble, le pointeur et l'état mesuré sont 
couplés dans un état décrivant l'ensemble du système 
$T$, écrit initialement sous la forme :

\begin{equation}
    \ket{\Psi^i}_T = \ket{\psi}_S \otimes \ket{\bar{q}=0}_P    
\end{equation}

Après la mesure le pointeur se déplace en fonction
de la force de l'intéraction 
$\delta \equiv \frac{gt}{\hbar}$ et une valeur 
propres $s_j$ du observable $\hat{S}$, 
$\Delta q = \delta s_j$. Ce dernier s'écrit 
dans lequel que l'état du pointeur passe 
de sa position initiale 
$\ket{\bar{q} = 0}_P$ à 
$\ket{\bar{q} = \delta s_j}_P$. Ensemble l'état du système 
évolue dans la façon suivante:

\begin{align}
    \ket{\Psi^f}_T &= \hat{U} \Bigl[\ket{\psi}_S \otimes \ket{\bar{q} = 0}_P\Bigr]\\
    &= \sum_{j}^{N} c_{j}\ket{s_j}_S \ket{\bar{q} = \delta s_j}_P
\end{align}

L'état final est maintenant intriqué entre le 
système et le pointeur. Ensuite, pour mesurer et 
caractériser l'état d'entrée, 
on lit le $\hat{q}$ du 
pointeur pour la mesure de $\hat{S}$. 
Si le décalage du 
pointeur $\delta$ est plus grand que l'écart de la 
probabilité de la fonction d'onde $\sigma$ dont 
$\delta \gg \sigma$, le résultat 
de la mesure à un sans ambiguïté, détruisant la 
superposition et réduisant la fonction d'onde à 
un résultat $s_j$, laissant un 
seul état $\ket{s_j}_S$ et 
aucune information sur l'ensemble de la fonction 
d'onde ne peuvent être récupérés. Cependant, 
lorsque le décalage du pointeur est inférieur à 
l'écart de la probabilité de la fonction d'onde, 
$\delta \ll \sigma$, 
dans le régime des mesures faibles, le système 
mesuré n'est plus que très peu intriqué avec le 
pointeur. La mesure de $\hat{S}$ 
par la mesure du déplacement de $\hat{q}$ ne perturbe 
plus que très peu 
la fonction d'onde \cite{Hairiri,Lundeen_Resch}. 
Considérons ce qui suit:

\begin{equation}
    \hat{U}\ket{\Psi^{i}}_T = \hat{U}\Bigl[ \ket{\psi}_S \otimes \ket{\bar{q} = 0}_P \Bigr]
\end{equation}

Examinons une étude plus approfondie du système 
initial total qui subit une interaction de 
mesure. Réécrivons l'opérateur d'interaction de 
von Neumann sous la forme d'une série de Taylor.

\begin{align}
    \hat{U}\ket{\Psi^{i}}_T &= \hat{U}\Bigl[ \ket{\psi}_S \otimes \ket{\bar{q} = 0}_P \Bigr]\\
    &= e^{-i\delta(\hat{S} \otimes \hat{p})}\Bigl[ \ket{\psi}_S \otimes \ket{\bar{q} = 0}_P \Bigr]\\
    &= \Bigl( 1 - i\delta(\hat{S} \otimes \hat{p}) - ... \Bigr)\Bigl[ \ket{\psi}_S \otimes \ket{\bar{q} = 0}_P \Bigr]\\
    &=  \ket{\psi}_S \otimes \ket{\bar{q} = 0}_P - i\delta\hat{S}\ket{\psi}_S\otimes\hat{p}\ket{\bar{q} = 0}_P - ...
\end{align}

En suivant la procédure de mesure faible, nous 
projetterons une mesure projective ultérieure 
sur le système avec l'état $\ket{\varphi}_S$ qui a les mêmes 
états de base que $\ket{\psi}_S$. 

\begin{align}
    \ket{\varphi}_S\bra{\varphi}_S\hat{U}\ket{\Psi^{i}}_T &= \ket{\varphi}_S\bra{\varphi}_S\ket{\psi}_S \otimes \ket{\bar{q} = 0}_P - i\delta\ket{\varphi}_S\bra{\varphi}_S\hat{S}\ket{\psi}_S\otimes\hat{p}\ket{\bar{q} = 0}_P - ...
\end{align}

Renomarisons l'état du système total en 
divisant par le 
module de l'amplitude de 
probabilité de $\bra{\varphi}_S\ket{\psi}_S = \sqrt{Prob}$,
dont $Prob \equiv |\bra{\varphi}_S\ket{\psi}_S|^2$
\cite{Lundeen_Resch,Lundeen_thesis, Steinberg_prob_div}.

\begin{align}
    \ket{\varphi}_S\frac{\bra{\varphi}_S\hat{U}\ket{\Psi^{i}}_T}{\bra{\varphi}_S\ket{\psi}_S} &= \ket{\varphi}_S\frac{\bra{\varphi}_S\ket{\psi}_S}{\bra{\varphi}_S\ket{\psi}_S} \otimes \ket{\bar{q} = 0}_P - i\delta\ket{\varphi}_S\frac{\bra{\varphi}_S\hat{S}\ket{\psi}_S}{\bra{\varphi}_S\ket{\psi}_S}\otimes\hat{p}\ket{\bar{q} = 0}_P - ...\\
    &= \ket{\varphi}_S \otimes \ket{\bar{q} = 0}_P - i\delta\ket{\varphi}_S\frac{\bra{\varphi}_S\hat{S}\ket{\psi}_S}{\bra{\varphi}_S\ket{\psi}_S}\otimes\hat{p}\ket{\bar{q} = 0}_P - ...
\end{align}

Dans ce cas, le 
$\frac{\bra{\varphi}_S\ket{\psi}_S}{\bra{\varphi}_S\ket{\psi}_S}$ 
du coté droit est annulée et nous ramenons le 
$\frac{1}{\bra{\varphi}_S\ket{\psi}_S}$ du 
côté gauche au coté droit. 
L'état final est maintenant le suivant :

\begin{align}
    \ket{\Psi^f}_T &\equiv \ket{\varphi}_S\bra{\varphi}_S\hat{U}\ket{\psi}_S\\
    &\simeq \bra{\varphi}_S\ket{\psi}_S \Bigl[ \ket{\bar{q} = 0}_P - i\delta\frac{\bra{\varphi}_S\hat{S}\ket{\psi}_S}{\bra{\varphi}_S\ket{\psi}_S} \hat{p}\ket{\bar{q}=0}_P - ... \Bigr] \otimes \ket{\varphi}_S
\end{align}

Dans les parenthèses carrées, cela correspond 
à l'état final du pointeur avec lequel nous 
pouvons calculer les parties réelles et 
imaginaires de S. 

\begin{equation}
    \ket{\bar{q} = \delta s_j}_P \equiv \ket{\bar{q} = 0}_P - i\delta\frac{\bra{\varphi}_S\hat{S}\ket{\psi}_S}{\bra{\varphi}_S\ket{\psi}_S} \hat{p}\ket{\bar{q}=0}_P - ...
\end{equation}

Remarquez que la position 
finale du pointeur est proportionnel à ce qui 
suit :

\begin{equation}
    \expval{\hat{S}_W} \equiv \frac{\bra{\varphi}_S\hat{S}\ket{\psi}_S}{\bra{\varphi}_S\ket{\psi}_S}
\end{equation}

Il s'agit de la valeur faible dérivée pour la 
première fois par AAV, une valeur complexe avec 
une partie réelle et imaginaire correspond au 
décalage de la variable du pointeur $q$ et à son 
décalage par rapport à sa variable conjuguée $p$ 
respectivement. Autrement dit, s'il y a un 
décalage dans la position d'une particule, il 
y aura également un décalage dans sa quantité 
de mouvement, soit comme nous allons explorer, 
la position temporelle d'un photon et sa 
position de fréquence se déplaceront l'une par 
rapport à l'autre lors d'une interaction. Si 
l'interaction est faible, il est possible de 
mesurer ces valeurs décalées individuellement 
lors d'une expérience 
\cite{Hairiri,Aharonov,Lundeen_Direct_Measurement}.
Pour terminer, écrivons l'état final avec cette valeur.

\begin{align}
    \ket{\Psi^f}_T &= \bra{\varphi}_S\ket{\psi}_S \Bigl[ \ket{\bar{q} = 0}_P - i\delta \expval{\hat{S}_W} \hat{p}\ket{\bar{q}=0}_P - ... \Bigr] \otimes \ket{\varphi}_S\\
    &= \bra{\varphi}_S\ket{\psi}_S\Bigl[ 1 - i\delta \expval{S_W}\hat{p} - ... \Bigr] \ket{\bar{q} = 0} \otimes \ket{\varphi}_S\\
    &= \bra{\varphi}_S\ket{\psi}_S e^{-i\delta \expval{S_W}\hat{p}} \ket{\bar{q} = 0} \otimes \ket{\varphi}_S\\
    &= \ket{\psi}_S e^{-i\delta \expval{S_W}\hat{p}}\ket{\bar{q}=0}
\end{align}

C’est-à-dire, si nous avons une mesure faible parfaite dont 
$\delta \ll \sigma$, prenons la limite 
que $\delta \to 0$, nous avons essentiellement 
l'état initial. Nous pouvons même mesurer la fonction 
d'onde directement sans aucune reconstruction algorithmique
et obtenir les parties réelles et imaginaires de la fonction d'onde 
à l'aide de la valeur faible. Démontrons cela \cite{Lundeen_thesis,Lundeen_Resch}:

\begin{align}
    \bra{\bar{q} = \delta s_j}\hat{q}\ket{\bar{q}=\delta s_j} &= -i\delta \mathcal{R}\Bigl(\expval{\hat{S}_W}\Bigr)\bra{\bar{q}=\delta s_j}(\hat{q}\hat{p} - \hat{p}\hat{q})\ket{\bar{q}=\delta s_j}\\
    &+ \delta \mathcal{I}\Bigl(\expval{\hat{S}_W}\Bigr)\bra{\bar{q} = \delta s_j}(\hat{q}\hat{p} + \hat{p}\hat{q})\ket{\bar{q}=\delta s_j}\\
    &= \delta\mathcal{R}\Bigl(\expval{\hat{S}_W}\Bigr) = \expval{\hat{q}}
\end{align}

Ainsi pour la variable conjuguée:

\begin{align}
    \bra{\bar{q} = \delta s_j}\hat{p}\ket{\bar{q}=\delta s_j} &= -i\delta \mathcal{R}\Bigl(\expval{\hat{S}_W}\Bigr)\bra{\bar{q}=\delta s_j}(\hat{p}^2 - \hat{p}^2)\ket{\bar{q}=\delta s_j}\\
    &+ \delta \mathcal{I}\Bigl(\expval{\hat{S}_W}\Bigr)\bra{\bar{q} = \delta s_j}(\hat{p}^2 + \hat{p}^2)\ket{\bar{q}=\delta s_j}\\
    &= \frac{\delta}{4\sigma^2}\mathcal{I}\Bigl(\expval{\hat{S}_W}\Bigr)
\end{align}

Ensemble la valeur faible s'écrit:

\begin{equation}
    \expval{\hat{S}_W} = \frac{1}{\delta}\Bigl( \expval{\hat{q}} + i4\sigma^2\expval{\hat{p}} \Bigr)
\end{equation}

En démontrant que la valeur faible est 
proportionnelle à la fonction d'onde, comme l'a fait 
AAV et que c'est paramètres peuvent être 
retrouver directement, on a ouvert un tout 
nouveau domaine dans les mesures 
quantiques et une alternative à la tomographie 
quantique traditionnelle.
    %\subsection{Mesure faible temporelle d'un système photonique quantique}
    %Les mesures faibles temporelles exploitent les 
propriétés temporelles et fréquentielles d’une 
impulsion lumineuse pour caractériser un état 
quantique. L’approche repose sur l’hypothèse 
que les délais temporels peuvent être 
directement liés aux composantes réelles et 
imaginaires de la valeur faible. Pour les sections suivantes et ce projet 
de thèse, nous allons nous concentrer sur la mesure de la 
valeur faible à partir d'une interaction faible temporelle. 
Nous utiliserons un système photonique quantique dans 
lequel nous caractériserons l'état de polarisation d'un 
faisceau de photons via les délais temporels d'une mesure 
faible. Nous avons choisi un système photonique parce 
qu’il est facilement réalisable en laboratoire avec un 
laser pulsé. Il permettrait aussi de miniaturiser la 
force de l'interaction faible, grâce à une sorte de miroir 
(nous y reviendrons plus tard), et, surtout, les états 
de base pourraient être bien définis expérimentalement 
en utilisant les états de polarisation horizontaux et 
verticaux comme base. Le profil temporel des lasers 
pulsés peut être utilisé pour voir l'impulsion déplacer 
son temps d'arrivée lorsque nous tournons une plaque 
d'onde pour caractériser les différents états de 
polarisation.
    %\subsubsection{La partie réelle du système}
    %Nous voulons caractériser 
l'état de polarisation avec une mesure faible temporelle.
Pour réaliser ce dernier, il faut calculer qu'il faut s'attendre
à la valeur faible $\expval{\hat{\pi}_W}$. Nous allons calculer chaque
partie de cette valeur à la fois. Commençons avec la partie réelle
et par définir les paramètres de cette 
expérience potentielle que nous voulons eventuellement 
effectuer. L'état de polarisation de notre système que 
nous voulons mesurer est défini comme suit:

\begin{equation}
    \ket{\psi} \equiv a\ket{H} + b\ket{V}
\end{equation}

Soit $a$ et $b$ des paramètres de probabilité pour les bases
$\ket{H}$ et $\ket{V}$ respectivement et $|a|^2 + |b|^2 = 1$, ainsi que $\ket{H}$ et $\ket{V}$ 
correspond à les polarisation horizontaux et verticaux d'un photon.

\begin{equation}
    \ket{\xi(t)} = \bra{t}\ket{\xi} \equiv \frac{1}{(\sqrt{2\pi}\sigma)^{1/2}}e^{-\frac{t^2}{4\sigma^2}}
\end{equation}

Soit le pointeur du système le profile temporel d'un faisceau,
généralement gaussien est utilisé, avec position temporel $t$ (par rapport à un temps $t_0$) 
et $\sigma$ l'écart du profile temporel. L'état totale initial s'écrit:

\begin{equation}
    \ket{\Psi(t)^i} \equiv \ket{\psi} \otimes \ket{\xi(t)}
\end{equation}

Effectuons une interaction faible temporelle sur la partie horizontale de l'état 
$\ket{H}$ avec l'opérateur de von Neumann $\hat{U}^H$, 
l'exposant $H$ est pour indiqué 
que l'opérateur est appliqué sur la partie horizontale.

\begin{align}
    \hat{U^H}\ket{\Psi(t)^i} &= \hat{U}^H \Bigl[ \ket{\psi} \otimes \ket{\xi(t)} \Bigr]\\
    &= \hat{U}^H \Bigl[ a\ket{H} \otimes \ket{\xi(t)} + b\ket{V} \otimes \ket{\xi(t)}\Bigr]\\
    &= a\ket{H} \otimes \hat{U}^H \ket{\xi(t)} + b\ket{V} \otimes \ket{\xi(t)}\\
    &= a\ket{H} \otimes \ket{\xi(t-\tau)} + b\ket{V} \otimes \ket{\xi(t)}
\end{align}

L'interaction de von Neumann subit un délai temporel $\tau$ sur le pointeur couplé 
avec la partie horizontale. Ensuite Effectuons une mesure projective avec 
l'état $\ket{\varsigma} \equiv \mu\ket{H} + \nu\ket{V}$ soit $\nu$ et $\mu$ des paramètres
probabilité pour $\ket{H}$ et $\ket{V}$ respectivement et $|\mu|^2 + |\nu|^2 = 1$. 

\begin{align}
    \ket{\Psi(t)^f} &= \ket{\varsigma}\bra{\varsigma}\hat{U}^H\ket{\Psi(t)^i} = \Bigl[ \bar{\mu}\bra{H} + \bar{\nu}\bra{V} \Bigr]a\ket{H} \otimes \ket{\xi(t-\tau)} + b\ket{V} \otimes \ket{\xi(t)}\\
    &= \Bigl[\bar{\mu}a\ket{\xi(t-\tau)} + \bar{\nu}b\ket{\xi(t)}\Bigr] \otimes \ket{\varsigma}\\
    &= F(t)\otimes\ket{\varsigma}
\end{align}

Soit $F(t) \equiv A\ket{\xi(t-\tau)} + B\ket{\xi(t)}$, $A \equiv a\bar{\mu}$ et $B \equiv b\bar{\nu}$. 
Trouvons la valeur d'espérance de la position 
temporel $\expval{\hat{t}}$.

\begin{align}
    \expval{\hat{t}} &= \bra{\Psi(t)^f}\hat{t}\ket{\Psi(t)^f}\\
    &= \int_{-\infty}^{\infty} I(t)tdt
\end{align}

Soit $I(t) \equiv |F(t)|^2$, nous pouvons le normalisé avec $\frac{1}{\bra{\Psi(t)^f}\ket{\Psi(t)^f}}$:

\begin{align}
    \expval{\hat{t}^{norm}} &= \frac{\bra{\Psi(t)^f}\hat{t}\ket{\Psi(t)^f}}{\bra{\Psi(t)^f}\ket{\Psi(t)^f}} = \frac{\int_{-\infty}^{\infty} I(t)tdt}{\int_{-\infty}^{\infty} I(t)dt}\\
    &= \frac{\int_{-\infty}^{\infty} |A|^2\Xi(t - \tau)t + |B|^2\Xi(t)t + A\bar{B}\Xi(t, \tau)t + \bar{A}B\Xi(t, \tau)t dt}{\int_{-\infty}^{\infty} |A|^2\Xi(t - \tau) + |B|^2\Xi(t) + A\bar{B}\Xi(t, \tau) + \bar{A}B\Xi(t, \tau) dt}
\end{align}

Soit $\Xi(t) \equiv \frac{1}{\sqrt{2\pi}\sigma}e^{-\frac{t^2}{2\sigma^2}}$ et $\Xi(t, \tau) \equiv \frac{1}{\sqrt{2\pi}\sigma}e^{-\frac{2t^2 - 2t\tau + \tau^2}{4\sigma^2}}$. 
Notons que vue que nous effectuons une interaction faible sur
le système, il y a une superposition entre les pointeurs 
pour la partie de polarisation horizontale et verticale.
Les solutions de chaque intégrale sont énumérées 
ci-dessous et nous reprendrons notre développement de 
la partie réelle de la valeur faible.

\begin{align*}
    \int_{-\infty}^{\infty} \Xi(t - \tau)t dt &= \tau & \int_{-\infty}^{\infty} \Xi(t) dt &= 1\\
    \int_{-\infty}^{\infty} \Xi(t - \tau) dt &= 1 & \int_{-\infty}^{\infty} \Xi(t, \tau)t dt &= \frac{\tau}{2}e^{-\frac{t^2}{8\sigma^2}}\\
    \int_{-\infty}^{\infty} \Xi(t)t dt &= 0 & \int_{-\infty}^{\infty} \Xi(t, \tau) dt &= e^{-\frac{t^2}{8\sigma^2}}
\end{align*}

Donc avec ces solutions, la partie réelle se trouve:

\begin{align}
    \expval{\hat{t}^{norm}} = \tau\frac{|A|^2 + (A\bar{B} + \bar{A}B)e^{-\frac{\tau^2}{8\sigma^2}}}{|A|^2 + |B|^2 + (A\bar{B} + \bar{A}B)e^{-\frac{\tau^2}{8\sigma^2}}}
\end{align}

Vue que nous sommes dans le régime des mesures faibles, prenons
la limite $\tau \ll \sigma$:

\begin{align}
    \lim_{\frac{\tau}{\sigma} \to 0} \expval{\hat{t}^{norm}} &= \tau\frac{|A|^2 + A\bar{B} + \bar{A}B}{|A|^2 + |B|^2 + A\bar{B} + \bar{A}B}\\
    &\equiv \mathcal{R}(\expval{\hat{\pi}_W})
\end{align}

Ce dernier est la partie réelle de la valeur faible $\expval{\hat{\pi}_W}$.
    %\subsubsection{La partie imaginaire du système}
    %Comme nous l'avons déjà mentionné, 
un déplacement de la variable du pointeur, 
tel que sa position temporelle $t$ par rapport à 
un $t_0$, devrait entraîner un déplacement de son 
spectre de fréquence. Vérifions-le en calculant 
la partie imaginaire de la valeur faible $\expval{\hat{\pi}_W}$. 
Commençons par prendre la transformation de 
Fourier de la fonction temporel $F(t)$ de l'état quantique $\ket{\Psi(t)^f}$:

\begin{align}
    F(\omega) &= \frac{1}{\sqrt{2\pi}}\int_{-\infty}^{\infty} F(t)e^{-i\omega t}dt\\
    &= \frac{\sqrt[4]{2}\sqrt{\sigma}}{\sqrt[4]{\pi}}(A + Be^{i\omega\tau})e^{-\omega^2 \sigma^2 - i\omega\tau}
\end{align}

Avec ce dernier la fonction d'onde s'écrit:

\begin{equation}
    \ket{\Psi(\omega)^f} = F(\omega) \otimes \ket{\varsigma}
\end{equation}

Ensuite trouvons la valeur d'espérance pour la position fréquentielle avec des étapes similaires que la partie réelle:

\begin{align}
    \expval{\hat{\omega}} &= \bra{\Psi(\omega)^f}\hat{\omega}\ket{\Psi(\omega)^f}\\
    &= \int_{-\infty}^{\infty} S(\omega) d\omega
\end{align}

Soit $S(\omega) \equiv |F(\omega)|^2$ et normalisons cette valeur:

\begin{align}
    \expval{\hat{\omega}^{norm}} &= \frac{\sqrt{2}\sigma}{\sqrt{\pi}}\frac{\int_{-\infty}^{\infty} |A|^2\omega e^{i\omega\tau} + |B|^2\omega e^{i\omega\tau} + A\bar{B}\omega + \bar{A}Be^{2i\omega\tau} \omega d\omega}{\int_{-\infty}^{\infty} |A|^2 e^{i\omega\tau} + |B|^2 e^{i\omega\tau} + A\bar{B} + \bar{A}Be^{2i\omega\tau}d\omega}e^{-2\omega^2 \sigma^2 -i\omega\tau}
\end{align}

Avec des solutions d'intégrale similaires nous obtenons:

\begin{equation}
    \expval{\hat{\omega}^{norm}} = \frac{i\tau}{4\sigma^2}\frac{(B\bar{A} - A\bar{B})e^{-\frac{\tau^2}{8\sigma^2}}}{|A|^2 + |B|^2 + \bar{A}B + A\bar{B}}
\end{equation}

Prenons encore la limite dont $\tau \ll \sigma$ vue que nous sommes dans le 
régime des mesures faibles:

\begin{align}
    \lim_{\frac{\tau}{\sigma} \to 0}\expval{\hat{\omega}^{norm}} &= \frac{i\tau}{4\sigma^2}\frac{B\bar{A} - A\bar{B}}{|A|^2 + |B|^2 + \bar{A}B + A\bar{B}}\\
    &\equiv \mathcal{I}(\expval{\hat{\pi}_W})
\end{align}


    %\subsection{Proposition expérimentale pour la caractérisation de la valeur faible}
    %Nous continuons avec notre système photonique 
quantique en nous appuyant sur nos découvertes 
concernant la partie réelle et imaginaire de la 
valeur faible. Nous pouvons calculer que pour 
un état d'entré soit:

\begin{equation}
    \ket{\psi^{in}} = a\ket{H} + b\ket{V} 
\end{equation}

Puisque $a$ et $b$ sont des amplitudes de 
probabilité pour les états de base $\ket{H}$ et $\ket{V}$ 
respectivement soit $a=\bra{H}\ket{\psi^{in}}$ et
$b=\bra{V}\ket{\psi^{in}}$. Expérimentalement, 
les parties mesurées de la valeur faible 
peuvent être 
calculées directement en fonction de la façon 
dont l'observable change en fonction de l'état 
d'entrée. Cela est possible car la valeur faible 
est proportionnelle à l'état quantique, comme le 
montre la section 2.1. Pour les états de 
polarisation, il s'agit de mesurer faiblement 
$\expval{\hat{S}^J} = \ket{J}\bra{J}$ 
soit $J= H,V$, puis de mesurer par projection sur un état 
intermédiaire tel que 
$\ket{D} = \frac{1}{\sqrt{2}}(\ket{H}+\ket{V})$. 
Si l'on y parvient, 
on obtient un sous-ensemble d'essais dont le 
résultat moyen est la valeur faible.

\begin{equation}
    \expval{\hat{S}_{W}^{J}} = \frac{\bra{D}\hat{S}^{J}\ket{\psi^{in}}}{\bra{D}\ket{\psi^{in}}} = \sqrt{N}\bra{J}\ket{\psi^{in}}
\end{equation}

Où $N$ est une constante de normalisation 
indépendante de $J$. L'état quantique peut être 
écrit en relation avec la valeur faible écrite. 

\begin{equation}
    \ket{\psi^{in}} = \frac{1}{\sqrt{N}}\Bigl(\expval{\hat{S}^{H}_{W}}\ket{H}+\expval{\hat{S}^{V}_{W}}\ket{V}\Bigr)
\end{equation}

Nous pouvons supposer que 
$N = \Bigl|\expval{\hat{S}^{H}_{W}}\Bigr|^2 + \Bigl|\expval{\hat{S}^{V}_{W}}\Bigr|^2$ 
puisque $|a|^2 + |b|^2 = 1$ donc 
$N = \Bigl|\expval{\hat{S}^{H}_{W}}\Bigr|^2 + \Bigl|1- \expval{\hat{S}^{H}_{W}}\Bigr|^2$. 
Donc,

\begin{equation}
    \ket{\psi^{in}} = \frac{1}{\sqrt{N}}\Bigl(\expval{\hat{S}^{H}_{W}}\ket{H}+ \Bigl(1-\expval{\hat{S}^{H}_{W}}\Bigr)\ket{V}\Bigr)
\end{equation}

Pour fixer la phase globale qui varierait selon l'état 
d'entrée, nous supposons que $a$ est toujours réel.
Cependant $b$ sera dépendant sur la partie imaginaire.
Avec les données expérimentales des deux 
observables $\expval{\hat{t}}$ et $\expval{\hat{\omega}}$, nous pouvons calculer 
directement les amplitudes de probabilité à partir des données expérimentales.  

\begin{align}
    |a|^2 &= \frac{\expval{\hat{t}}}{\tau}\\
    |b|^2 &= 1 - |a|^2
\end{align}

En fonction de l'état d'entrée, la valeur faible 
varie. Il est important de noter que le délai $\tau$ 
est le délai maximal que nous utilisons pour 
interagir avec le système. Ce dernier, normalise
les amplitudes de probabilité. Lorsque nous 
changeons les états d'entrée, le délai $\tau$ 
devrait évoluer entre l'absence de délai et le 
délai maximal, c'est-à-dire entre les 
polarisations $\ket{V}$ et $\ket{H}$. 

    %\subsection{Conclusion sur la théorie}
    %Ce chapitre a établi les fondements théoriques 
des mesures faibles temporelles et leur 
pertinence pour les systèmes photoniques. En 
s’appuyant sur des techniques innovantes et des 
travaux précédents, cette thèse vise à démontrer 
l’utilité des mesures faibles temporelles pour 
caractériser directement les états quantiques. 
Le prochain chapitre présentera les aspects 
expérimentaux liés à la mise en œuvre de ces 
méthodes.
    \pagebreak

    \thispagestyle{empty}
    \section{MESURE EXPÉRIMENTALE DIRECTE D'UN ÉTAT DE POLARISATION EN UTILISANT UNE MESURE FAIBLE TEMPORELLE}\label{chap:3}
    Pour caractériser les états de polarisation à 
l’aide des mesures faibles temporelles, nous 
devons d'abord évaluer notre capacité à mesurer 
les délais temporels avec notre oscilloscope. 
Nous choisissons l’oscilloscope en raison de sa 
présence dans tous les laboratoires de physique, 
ce qui facilite son intégration dans la plupart 
des laboratoires, et de sa facilité 
d'utilisation pour obtenir une haute résolution 
temporelle.
    \subsection{Mesure de délais temporels ultra courts}
    \begin{doublespace}
    Nous vous invitons à évaluer notre capacité à mesurer des délais temporels avec précision. Pour ce faire, nous devons déterminer notre précision de la vitesse de la lumière via des délais temporels. Nous allons utiliser un laser pulsé ultra-court de type nanoseconde dans le cadre de nos expériences \cite{ThorlabsNPL64B}. Ce dispositif laser est capable de générer des impulsions allant de $5$ à $39$ $ns$. Nous avons opté pour une impulsion de $10$ $ns$ dans cette plage, car les intervalles de temps plus longs ont tendance à présenter une distribution temporelle similaire à celle d’une fonction porte. Nous cherchons une impulsion dans le domaine temporel qui ressemble à une fonction gaussienne, ce qui se produit lorsque les impulsions du laser sont plus courtes. Cette dernière est souvent utilisée dans les mesures de faibles \cite{OpticalNetworks, Lundeen_Direct_Measurement,Hairiri,Guilleaum,Brunner_2004} pour faciliter l’identification de la position maximale de l’impulsion, que nous identifierons comme correspondant à la position temporelle moyenne de l’impulsion. Le laser utilisé possède une longueur d’onde comprise entre $640 \pm 10$ $nm$, avec une énergie d’impulsion maximale de $2,0$ $nJ$. Sa puissance de pointe s’élève à $50$ $mW$ lorsque le taux de répétition et la largeur d’impulsion maximale sont utilisés. Toutefois, dans le cadre de notre protocole, nous fixons la fréquence de répétition à $1$ $MHz$, assurant ainsi une fréquence constante tout au long de l’expérience. Cela n’affectera pas l’expérience elle-même. Pour nos tests, nous utiliserons un miroir pour régler des intervalles de distances variables et analyserons les résultats avec notre oscilloscope \cite{TektronixTDS5000}. Voici un diagramme du dispositif d’expérimentation, visible à la figure \ref{fig:speed-of-light}.
\end{doublespace}

\begin{figure}[!hpbt]
    \centering
    \includegraphics[width=1.0\textwidth]{speed_of_light_exp.png}
    \caption{Représentation de notre dispositif expérimental pour évaluer la précision de nos mesures temporelles en mesurant la vitesse de la lumière. L’impulsion du laser est d’abord réglée en intensité par une lame demi-onde, puis dirigée vers un séparateur de faisceau polarisant (PBS) qui divise les états de polarisation horizontaux et verticaux de base de l’impulsion d’entrée en deux voies orthogonales. Celui qui est réfléchi par le PBS sera notre signal de référence pour déclencher l’oscilloscope décu. L’autre subit encore un autre PBS. Une des voies sera ensuite ignorée par un bloc. Nous définissons les états de polarisation comme suit : le faisceau réfléchi représente l’état de base de polarisation verticale $\ket{V}$ de l’état d’entrée $\ket{\psi}$, tandis que le faisceau transmis correspond à l’état de base horizontal $\ket{H}$. Nous orientons l’état horizontal vers un miroir, que nous réglerons en fonction des différentes distances à évaluer. Il est ensuite renvoyé 
    vers le PBS pour y être réfléchi. Ce procédé utilise une lame quart d’onde pour convertir l’état de polarisation $\ket{H}$ en un état $\ket{V}$. Il est ensuite détecté avec un photodétecteur rapide à base de Si \cite{ThorlabsDET025A} puis interprété par notre oscilloscope.}
    \label{fig:speed-of-light}
\end{figure}
    \subsubsection{Notes sur l'importances de l’acquisition des données	}
    \begin{doublespace}
    
    La façon dont nous acquérons nos données est importante, car nous devons nous assurer que nous utilisons la méthode la plus précise pour déterminer la position temporelle moyenne de l’état en vue d’une analyse ultérieure des mesures faibles. Pour ce faire, nous effectuons deux expériences afin de mesurer la vitesse de la lumière. 
    
    \noindent L’une consiste à mesurer la vitesse de la lumière à partir d’un miroir et à prendre des mesures à différentes distances ($2,52$, $5,48$, $10,10$, $20,19$, $30,29$, $40,38$, $50,48$ et $65,62$ $cm$).
    (mesuré à l’œil avec une règle). Le délai mesuré part d’une référence, appelée position $0$, située à $11,5$ $cm$ du PBS. La lumière doit parcourir une distance double de celle envoyée dans les deux sens à partir du séparateur de faisceau. Ensuite, analyser les données afin de trouver le délai obtenu pour chaque distance du miroir.
    
    \noindent L’autre expérience consiste en un principe semblable, soit la mesure de la vitesse de la lumière dans nos câbles BNC RG-58 \cite{ThorlabsBNC} à différentes longueurs. Cette expérience facilite nos mesures, car les câbles ont une longueur déterminée par le fabricant. Le délai est lié aux variations de longueur entre les différentes longueurs de câbles. Ces dernières sont de $17$, $27$, $52$, $100$ et $300$ $cm$. Les délais commencent par une mesure avec un câble plus court que le premier. Dans ce cas, le miroir ajustable est fixe.
    
    \noindent Une implication importante à considérer pour la précision de nos mesures est la façon dont l’oscilloscope acquiert des données pour le domaine temporel. Le signal est détecté par le photodétecteur rapide, puis il est introduit dans l’oscilloscope à l’aide d’un câble BNC. Il est ensuite déclenché par le front montant gauche de notre signal de référence. Le signal de référence que nous déclenchons sert d’origine temporelle pour l’oscilloscope. Nous mesurons ainsi la position temporelle de chaque distance par rapport à notre origine temporelle de référence, soit celle où le miroir se trouve à notre position $0$. De cette manière, nous isolons l’expérience pour observer uniquement ce qui se produit lorsque nous déplaçons le miroir de son emplacement initial vers des positions plus éloignées. 
    
    \noindent Les capacités de l’oscilloscope sont influencées par sa méthode d’acquisition du signal entré, grâce à ses fonctions d’acquisition. Ce réglage gère le nombre de signaux sinusoïdaux que nous pouvons définir dans le signal d’entrée, ce qui permet de créer une moyenne des ondes sinusoïdales acquises. Nous calculons la moyenne de plus de $10000$ formes d’onde pour obtenir un signal propre et éliminer le bruit de fond, ce qui nous permet d’obtenir une mesure plus précise de la position temporelle moyenne d’un signal. Nous avons aussi réalisé l’expérience dans l’obscurité pour réduire le bruit de fond, mais nous avons constaté que cette étape n’était pas nécessaire. Les résultats n’ont pas été radicalement différents, mais nous l’avons quand même fait. 
    
    \noindent L’échantillonnage est un aspect crucial de l’oscilloscope, qui détermine la manière dont l’instrument collecte ses données. Ces méthodes sont l’échantillonnage en temps réel, l’interpolation et l’équivalence temporelle. En mode d’échantillonnage en temps réel, l’oscilloscope numérise tous les points qu’il a acquis après un événement déclencheur. Ce mode d’acquisition est principalement utilisé pour les mesures ponctuelles ou les variations en temps réel du signal. Le mode d’interpolation interpole entre les points d’échantillonnage en créant des points qui aident à combler les lacunes. Il en résulte une ligne droite ou une onde sinusoïdale entre les points, ce qui donne lieu à une courbe plus lisse. Nous ne désirons pas procéder ainsi, car nous ne souhaitons pas surcharger le signal avec un maximum d’interpolations. Enfin, le mode d’échantillonnage par équivalence de temps permet d’augmenter le taux d’échantillonnage au-delà du taux d’échantillonnage maximum en temps réel. La figure \ref{fig:EQ-time} illustre son fonctionnement. Ainsi, il est possible d’obtenir le taux d’échantillonnage complet de l’oscilloscope, soit $500$ $GS/s$ (gigéchantillons par seconde), en utilisant ce mode. Sachez que, si le déclenchement n’est pas en mode externe et que votre état d’entrée se trouve dans un canal distinct de votre signal de référence, votre taux d’échantillonnage maximal sera désormais divisé par deux. La fréquence d’échantillonnage maximale est cruciale pour l’oscilloscope, car elle permet d’atteindre sa résolution temporelle maximale pour notre signal, qui est de $4 \pm 2$ $ps$. Cela garantit des mesures temporelles précises. Nous enregistrons ensuite les signaux d’onde de sortie sous forme de tableau CSV sur un ordinateur, ce qui permettra une analyse plus détaillée des données.
    
\end{doublespace}

\begin{figure}[hp]
    \centering
    \includegraphics[width=1.0\textwidth]{EQ _time_oscillo_fig.png}
    \caption{Diagramme illustrant le fonctionnement du mode d’acquisition du temps d’équivalence de l’oscilloscope \cite{TektronixTDS5000}. Cet appareil collecte un petit nombre d’échantillons au moment où l’événement de déclenchement se produit, ce qui lui permet d’obtenir le signal complet de notre impulsion. Le taux d’échantillonnage est supérieur à celui de son homologue en temps réel. L’oscilloscope fonctionne en mode équivalence de temps en effectuant un échantillonnage aléatoire, qui est déclenché par des événements aléatoires définis par l’horloge d’échantillonnage de l’instrument. Cette horloge fonctionne de manière asynchrone par rapport au signal d’entrée et au signal de déclenchement. Il enregistre ensuite un certain nombre d’échantillons d’acquisition. Après cela, l’oscilloscope combine plusieurs échantillons d’un signal répétitif en cours d’acquisition. Il régule ensuite la fréquence d’échantillonnage du signal d’entrée pour un enregistrement d’ondes régulières et complètes. }
    \label{fig:EQ-time}
\end{figure}
    \subsubsection{Analyse et résultats de l'expérience de la vitesse d'un signal électrique dans un câble BNC}
    \begin{doublespace}
    Maintenant, analysons nos données de l'expérience de la vitesse d’un signal à travers un câble BNC à partir des délais d’impulsion. Nous avons configuré l’oscilloscope en mode EQ-time (temps d'équivalence), avec une durée de $100$ $ns$, une longueur d'enregistrement de $10 000$ points et une résolution de $4$ $ps$. Plus tard, lors d’expériences de mesures faibles, nous constaterons que nous n’avons pas besoin d’une telle durée. Nous commençons par observer l'impulsion typique du laser dans la figure \ref{fig:typical-pulse}. 
\end{doublespace}

\begin{figure}[hp]
    \centering
    \includegraphics[width=1.0\textwidth]{typ_pulse.png}
    \caption{Profil temporel typique de notre 
    impulsion laser ultra-courte NPL64B \cite{ThorlabsNPL64B}, mesurée avec un 
    photodétecteur DET025A à base de Si \cite{ThorlabsDET025A} et acquise 
    à l'aide du mode d'acquisition temporelle EQ-time 
    de l'oscilloscope \cite{TektronixTDS5000}.}
    \label{fig:typical-pulse}
\end{figure}

\begin{doublespace}
    \noindent Observez que le profil temporel des impulsions n’est pas une fonction de Gauss, mais plutôt une fonction de porte. En allongeant la durée de l’impulsion du laser, l’impulsion ressemble de plus en plus à une fonction de port au fur et à mesure. La raison pour laquelle nous souhaitons une forme gaussienne est qu'elle est simplement plus fréquente dans les mesures faibles ayant un sens physique plus naturel et qu'il est plus facile de déterminer la valeur moyenne de la position temporelle de l'impulsion (le moment le plus probable pour détecter un photon de cette impulsion) que nous définissons comme le pic. Par conséquent, notre objectif est de calculer numériquement la dérivée des données de l’impulsion, ce qui nous permettra de localiser précisément le pic et de le comparer à d'autres impulsions.
    
    \noindent Notez que le profil temporel des impulsions n’est pas une fonction gaussienne, mais plutôt une fonction porte. En prolongeant la durée de l’impulsion du laser, l’impulsion tend vers une fonction porte de plus en plus. Nous désirons une distribution gaussienne, car elle se révèle plus fréquente dans les mesures à faible intensité, présentant ainsi une signification physique plus naturelle. De plus, il est plus facile de déterminer la valeur moyenne de la position temporelle de l'impulsion (le moment le plus probable pour détecter un photon de cette impulsion), que nous nommons le \guillemetleft pic \guillemetright. Par conséquent, notre objectif est de calculer numériquement la dérivée des données de l’impulsion. La méthode des différences finies est utilisée comme type de dérivé numérique. Elle s’avère suffisante pour cette expérience. Ce dernier nous permettra de localiser précisément le pic en prenant un ajustement de courbe avec un domaine temporel plus fini pour la localisation du pic. Cette méthode nous permettra aussi de choisir un point temporel pour le temps d’arrivée, qui n’est pas nécessairement situé entre deux points de données. On comparera ensuite chacun de ces temps d’arrivée, pour chacune des longueurs de câble.

    \noindent À partir de là, l’ajustement de la courbe devient un aspect vraiment important dans la localisation du temps d’arrivée de notre signal. Nous effectuons de nombreux calculs numériques pour constater que l’option la plus adéquate était d’utiliser un pas de temps de $1/10000$ de la résolution des oscilloscopes pour l’axe des temps sur lequel nous effectuions l’ajustement. Nous avons remarqué qu’en négligeant environ $40 \%$ des données d’amplitude initiales, l’ajustement était plus susceptible d’être optimal. Le reste du signal ne présentait pas d’intérêt. On prend alors la position temporelle de ce pic et on l’ajoute dans un dictionnaire des temps d’arrivée, puis on la compare à notre câble de référence, qui est considéré comme notre origine, donc une position temporelle $0$ ou sans délai. Le tableau suivant, table \ref{table:BNC-fits}, montre les différents ajustements que nous avons essayés ainsi que le temps moyen d'arrivée pour chaque distance en utilisant ces ajustements. Il montre également l'écart-type de la façon dont la position du temps d'arrivée change pour chaque fichier individuel de cette même longueur de câble et le coefficient de détermination $R^2$ qui a également été utilisé comme paramètre pour déterminer la qualité de l'ajustement ainsi que la visualisation de tous ces ajustements. Nous avons sélectionné $5$ échantillons distincts pour chaque longueur de câble. Nous avions initialement prévu d’en prélever plus de $5$, mais nous avons réalisé que c'était excessif et qu'il suffisait d'en prélever au moins $3$. 

    
\end{doublespace}


\begin{longtable}{p{2.0cm} p{1.5cm} p{3.0cm} p{2.0cm} p{2.5cm}}
    \caption{Résultats des temps d'arrivées et écart-type de différent ajustement de courbe pour l'expérience de vitesse dans les câbles BNC} \\
    \toprule
    \label{table:BNC-fits}
    Type de fit & Longueur du câble (mm) & Temps d'arrivée (ns) & Écart-type (ns) & Qualité du fit \\
    \midrule
    \endfirsthead
    
    \toprule
    Type de fit & Longueur du câble (mm) & Temps d'arrivée (ns) & Écart-type (ns) & Qualité du fit \\
    \midrule
    \endhead
    
    \midrule
    \multicolumn{5}{r}{{Continued on next page}} \\
    \midrule
    \endfoot
    
    \bottomrule
    \endlastfoot
    
    % Paste all your rows here like:
    poly2     & 0            & 7.44310          & 0.00068       & 0.24743 \\
    poly2     & 172          & 8.37596          & 0.00105       & 0.24363 \\
    poly2     & 270          & 8.86583          & 0.00182       & 0.23058 \\
    poly2     & 522          & 10.12025         & 0.00104       & 0.24149 \\
    poly2     & 1032         & 12.68037         & 0.00099       & 0.24337 \\
    poly2     & 3000         & 22.62379         & 0.00354       & 0.24339 \\
    poly3     & 0            & 7.46342          & 0.00222       & 0.23412 \\
    poly3     & 172          & 8.39546          & 0.00278       & 0.23143 \\
    poly3     & 270          & 8.88637          & 0.00227       & 0.21571 \\
    poly3     & 522          & 10.14113         & 0.00196       & 0.22738 \\
    poly3     & 1032         & 12.70091         & 0.00150       & 0.23016 \\
    poly3     & 3000         & 22.64460         & 0.00396       & 0.23005 \\
    poly4     & 0            & 7.45750          & 0.00114       & 0.21840 \\
    poly4     & 172          & 8.38888          & 0.00177       & 0.21391 \\
    poly4     & 270          & 8.88005          & 0.00136       & 0.19823 \\
    poly4     & 522          & 10.13472         & 0.00099       & 0.20934 \\
    poly4     & 1032         & 12.69463         & 0.00132       & 0.21404 \\
    poly4     & 3000         & 22.63830         & 0.00336       & 0.21629 \\
    poly5     & 0            & 7.46484          & 0.00131       & 0.21682 \\
    poly5     & 172          & 8.39631          & 0.00106       & 0.21231 \\
    poly5     & 270          & 8.88844          & 0.00185       & 0.19591 \\
    poly5     & 522          & 10.14197         & 0.00119       & 0.20773 \\
    poly5     & 1032         & 12.70253         & 0.00211       & 0.21218 \\
    poly5     & 3000         & 22.64039         & 0.00294       & 0.21483 \\
    poly6     & 0            & 7.46521          & 0.00202       & 0.21675 \\
    poly6     & 172          & 8.39659          & 0.00153       & 0.21216 \\
    poly6     & 270          & 8.88736          & 0.00307       & 0.19582 \\
    poly6     & 522          & 10.14084         & 0.00182       & 0.20762 \\
    poly6     & 1032         & 12.69214         & 0.00663       & 0.21204 \\
    poly6     & 3000         & 22.29098         & 0.18052       & 0.21469 \\
    poly7     & 0            & 7.33751          & 0.03120       & 0.21499 \\
    poly7     & 172          & 8.21566          & 0.13047       & 0.21160 \\
    poly7     & 270          & 8.52569          & 0.11581       & 0.19487 \\
    poly7     & 522          & 9.72629          & 0.12391       & 0.20682 \\
    poly7     & 1032         & 12.23442         & 0.00780       & 0.21036 \\
    poly7     & 3000         & 22.16566         & 0.01151       & 0.21381 \\
    poly8     & 0            & 6.99619          & 0.00997       & 0.21454 \\
    poly8     & 172          & 7.92408          & 0.01360       & 0.21110 \\
    poly8     & 270          & 8.41125          & 0.01215       & 0.19425 \\
    poly8     & 522          & 9.66966          & 0.01255       & 0.20656 \\
    poly8     & 1032         & 12.23447         & 0.00776       & 0.21055 \\
    poly8     & 3000         & 22.16570         & 0.01148       & 0.21527 \\
    poly9     & 0            & 6.99605          & 0.00983       & 0.21387 \\
    poly9     & 172          & 7.92413          & 0.01359       & 0.21270 \\
    poly9     & 270          & 8.41132          & 0.01220       & 0.19766 \\
    poly9     & 522          & 9.66971          & 0.01257       & 0.20799 \\
    poly9     & 1032         & 12.23498         & 0.00810       & 0.21169 \\
    poly9     & 3000         & 22.16569         & 0.01146       & 0.22500 \\
    fourier1  & 0            & 7.44586          & 0.00075       & 0.22821 \\
    fourier1  & 172          & 8.37913          & 0.00125       & 0.22118 \\
    fourier1  & 270          & 8.86896          & 0.00128       & 0.20785 \\
    fourier1  & 522          & 10.12332         & 0.00084       & 0.21928 \\
    fourier1  & 1032         & 12.68338         & 0.00115       & 0.22325 \\
    fourier1  & 3000         & 22.62676         & 0.00359       & 0.22515 \\
    fourier2  & 0            & 7.46383          & 0.00189       & 0.21728 \\
    fourier2  & 172          & 8.39629          & 0.00139       & 0.21263 \\
    fourier2  & 270          & 8.88716          & 0.00276       & 0.19641 \\
    fourier2  & 522          & 10.14204         & 0.00308       & 0.20802 \\
    fourier2  & 1032         & 12.70221         & 0.00217       & 0.21270 \\
    fourier2  & 3000         & 22.64465         & 0.00462       & 0.21526 \\
    fourier3  & 0            & 7.41817          & 0.06115       & 0.21482 \\
    fourier3  & 172          & 8.37827          & 0.00987       & 0.21003 \\
    fourier3  & 270          & 8.87882          & 0.01193       & 0.19474 \\
    fourier3  & 522          & 10.12619         & 0.01515       & 0.20527 \\
    fourier3  & 1032         & 12.68675         & 0.01349       & 0.21032 \\
    fourier3  & 3000         & 22.61673         & 0.00260       & 0.20832 \\
    fourier4  & 0            & 7.42123          & 0.00635       & 0.21111 \\
    fourier4  & 172          & 8.34673          & 0.00717       & 0.20424 \\
    fourier4  & 270          & 8.84979          & 0.00872       & 0.19021 \\
    fourier4  & 522          & 10.10293         & 0.00667       & 0.20169 \\
    fourier4  & 1032         & 12.66846         & 0.00449       & 0.20740 \\
    fourier4  & 3000         & 22.60242         & 0.00497       & 0.20594 \\
    fourier5  & 0            & 7.42623          & 0.00692       & 0.21058 \\
    fourier5  & 172          & 8.36756          & 0.05475       & 0.20214 \\
    fourier5  & 270          & 8.85968          & 0.01496       & 0.18814 \\
    fourier5  & 522          & 10.10500         & 0.00430       & 0.19736 \\
    fourier5  & 1032         & 12.67241         & 0.00725       & 0.20597 \\
    fourier5  & 3000         & 22.52103         & 0.19931       & 0.20521 \\
    fourier6  & 0            & 7.43023          & 0.01046       & 0.21021 \\
    fourier6  & 172          & 8.41189          & 0.06274       & 0.20157 \\
    fourier6  & 270          & 8.85976          & 0.01835       & 0.18777 \\
    fourier6  & 522          & 10.19384         & 0.19938       & 0.19661 \\
    fourier6  & 1032         & 12.76117         & 0.19404       & 0.20516 \\
    fourier6  & 3000         & 22.52226         & 0.19765       & 0.20480 \\
    gauss1    & 0            & 7.44584          & 0.00059       & 0.22856 \\
    gauss1    & 172          & 8.37897          & 0.00111       & 0.22152 \\
    gauss1    & 270          & 8.86887          & 0.00131       & 0.20805 \\
    gauss1    & 522          & 10.12319         & 0.00082       & 0.21959 \\
    gauss1    & 1032         & 12.68330         & 0.00116       & 0.22352 \\
    gauss1    & 3000         & 22.62690         & 0.00346       & 0.22570
\end{longtable}


\begin{doublespace}
    
    \noindent En plus de la vérification de l'ajustement présentant le moins d'écart entre les fichiers pour chaque longueur de câble possible, le tableau suivant, table \ref{table:speed-BNC-table}, présente les différents types d'ajustement, en testant également la précision de la vitesse des mesures du signal en fonction des ajustements trouvés pour le temps d'arrivée. La vitesse est déterminée en effectuant un ajustement linéaire sur les valeurs moyennes du temps d'arrivée, pour chaque longueur de câble. Nous comparons ensuite ces données à la valeur théorique, qui est d’environ $0,66 c$ \cite{arrl2019}. 

    \noindent Nous en concluons que, selon nos deux tableaux, le meilleur ajustement possible est un polynôme du troisième ordre, une série de Fourier du premier ou du deuxième ordre, ainsi qu’un ajustement gaussien. Nous avons choisi un ajustement gaussien comme type d’ajustement à utiliser, car il présente un bon équilibre des écarts-types à travers des différents fichiers, \ref{table:BNC-fits}. Cela est optimal pour obtenir la meilleure résolution possible pour les procédures directes via mesure faible, ainsi que la forme typique qu’on retrouve. Il présente également un bon pourcentage d’erreur par rapport à la théorie. Vous pouvez trouver les mêmes tableaux que nous avons créés pour mesurer la vitesse de la lumière sur différentes longueurs de miroir à l’annexe A, table \ref{table:AnnexeFits}. 
\end{doublespace}

\begin{longtable}{p{2.0cm} p{2.0cm} p{2.0cm} p{2.0cm} p{2.0cm}}
    \caption{Mesure de la vitesse du signal dans les câbles BNC pour différent ajustement de courbe} \\
    \toprule
    \label{table:speed-BNC-table}
    Type de fit & Vitesse mesurée (m/s) & Vitesse théorique (m/s) & Erreur (\%) & Qualité du fit \\
    \midrule
    \endfirsthead
    
    \toprule
    Type de fit & Vitesse mesurée (m/s) & Vitesse théorique (m/s) & Erreur (\%) & Qualité du fit \\
    \midrule
    \endhead
    
    \midrule
    \multicolumn{5}{r}{{Continued on next page}} \\
    \midrule
    \endfoot
    
    \bottomrule
    \endlastfoot
    
    % Paste all your rows here like:
    poly2     & 198125399     & 197863022           & 0.1326      & 0.0303       \\
    poly3     & 198116966     & 197863022           & 0.1283      & 0.0301       \\
    poly4     & 198117476     & 197863022           & 0.1286      & 0.0298       \\
    poly5     & 198191774     & 197863022           & 0.1662      & 0.0301       \\
    poly6     & 203054996     & 197863022           & 2.6240      & 0.0637       \\
    poly7     & 201780752     & 197863022           & 1.9800      & 0.1358       \\
    poly8     & 198220958     & 197863022           & 0.1809      & 0.0271       \\
    poly9     & 198220258     & 197863022           & 0.1805      & 0.0271       \\
    fourier1  & 198125807     & 197863022           & 0.1328      & 0.0304       \\
    fourier2  & 198124682     & 197863022           & 0.1322      & 0.0302       \\
    fourier3  & 198150017     & 197863022           & 0.1450      & 0.0454       \\
    fourier4  & 198105100     & 197863022           & 0.1223      & 0.0295       \\
    fourier5  & 199361343     & 197863022           & 0.7573      & 0.0386       \\
    fourier6  & 199614259     & 197863022           & 0.8851      & 0.0779       \\
    gauss1    & 198122675     & 197863022           & 0.1312      & 0.0304 
\end{longtable}

\begin{doublespace}
    \noindent Discutons nos données dans une façon plus visuelle. Sur la figure \ref{fig:BNC-pulse}, on voit chaque impulsion provenant de différentes longueurs de câble BNC. Cette expérience ne mesure pas seulement la vitesse de la lumière dans les câbles BNC, elle teste aussi nos paramètres d’ajustement, puisque nous avons utilisé les mêmes paramètres pour la vitesse de la lumière pour les miroirs ajustables ainsi que nos mesure faibles. 
\end{doublespace}


\begin{figure}[h]
    \centering
    \includegraphics[width=1.0\textwidth]{overlay_pulses_with_fits_BNC.png}
    \caption{Profil temporel de la dérivée des données d'impulsion dans l'expérience des câbles BNC RG-58 avec chacun de ses ajustements de courbe. }
    \label{fig:BNC-pulse}
\end{figure}


\begin{doublespace}
    \noindent Nous procédons à l’ajustement des données en leur appliquant une fonction gaussienne, qui s’écrit comme suit :
\end{doublespace}

\begin{equation}
    y(t) = a_0 e^{-(\frac{t-b_0}{c_0})^2}
\end{equation}

\begin{doublespace}
    \noindent Les paramètres d’ajustement $a_0$, $b_0$ et $c_0$ de la fonction avec variable $y$, qui représente l’amplitude, et $t$, qui correspond à la position temporelle des courbes, sont sélectionnés pour optimiser l’ajustement de nos données. Parmi ces données, $40\%$ des points de l’axe d’amplitude et de l’axe temporel sont ignorés. Ces paramètres correspondent le mieux à nos données. Il est difficile d’attribuer une valeur numérique pour évaluer la qualité du réglage de notre courbe, puisque celui-ci a principalement découlé d’une analyse visuelle. Nous avons néanmoins utilisé le coefficient de détermination ($R^2$) comme boussole, mais nous avons tenté d'éviter un ajustement excessif. Ce réglage nous permet maintenant d’identifier la position optimale, qui correspond à une position réelle observée dans nos données. Ensuite, nous comparons chaque position temporelle à celle des distances de référence pour obtenir les délais mesurés pour notre expérience. Ces délais sont tracés en fonction de la distance associée, et, par ajustement linéaire de la courbe, nous pouvons déterminer que la pente correspond à la vitesse du signal. 
\end{doublespace}

\begin{figure}[h]
    \centering
    \includegraphics[width=1.0\textwidth]{speed_of_light_BNC.png}
    \caption{Délais mesurés pour la longueur du câble BNC eux avec son ajustement de courbe.}
    \label{fig:BNC-res}
\end{figure}

\begin{doublespace}
    \noindent Notre résultat pour la vitesse du signal est $197 863 022$ $m/s$, ce qui représente une erreur en pourcentage de $0.1312\%$ correspondant que la vitesse du sginal dans un câble BNC possède un différentes de $0.66$ par rapport à $c$, table \ref{table:speed-BNC-table}, ce qui est réaliste et cohérent avec l'erreur observée lors de l'expérience précédente \cite{arrl2019,thorlabs2021}. En effet, nous avons donc démontré notre capacité à mesurer des délais très précis. C’est un élément essentiel pour pouvoir commencer à mesurer de petits délais entre des états polarisation d'entrée de changement dans le biais de mesures faibles.
\end{doublespace}

\begin{doublespace}
    \noindent Voici les résultats des données issues des impulsions de l’expérience du miroir déplacé, figure \ref{fig:speed-pulses}. Ces données dérivées montrent clairement une forme gaussienne avec un pic maximal nettement visible. Cela facilite grandement son identification de sa position temporelle. 
\end{doublespace}

\begin{figure}[!hptb]
    \centering
    \includegraphics[width=1.0\textwidth]{overlay_pulses_mirror.png}
    \caption{Profil temporel de la dérivée des données d'impulsion pour chacune des distances mesurées et ajustement de la courbe pour l'expérience de la vitesse de la lumière avec un miroir réglable.}
    \label{fig:speed-pulses}
\end{figure}

\begin{doublespace}
    \noindent Le résultat de notre expérience sur la vitesse de la lumière est de $296 991 901$ $m/s$ avec une marge d'erreur de $0.91$ $\%$ par rapport à la valeur actuelle, table \ref{table:AnnexeASpeed}. Cela correspond à une différence de $0,9998$ par rapport à la vitesse de la lumière $c$ dans l’air ($c_{air} = 0.9998c$) \cite{hecht2012optics}.
\end{doublespace}

\begin{figure}[!hptb]
    \centering
    \includegraphics[width=1.0\textwidth]{speed_of_light_mirror.png}
    \caption{Résultats des délais mesurés pendant l'expérience sur la vitesse de la lumière, ainsi que leur ajustement linéaire. Les barres d’erreur horizontales représentent l’incertitude de nos mesures de la distance, soit $\pm 0,5$ $mm$, qui est trop petite pour être visible sur le graphique. Les barres d'erreurs verticales correspondent à l'erreur de l'ajustement Fourier basé sur l'expérience de la vitesse de la lumière dans les câbles BNC $\pm 0,03\%$.}
    \label{fig:speed-res}
\end{figure}

\begin{doublespace}
    \noindent À partir des résultats de ces deux expériences, nous concluons que nous pouvons mesurer des délais temporels très courts, de l'ordre de 2 ps avec une variation inférieure à 1 ps entre les ensembles de données. Cette résolution est suffisante pour mettre en œuvre notre proposition de mesure du temps. Les sections suivantes seront consacrées à la caractérisation des états de polarisation par un petit décalage temporel entre les états de base. 
\end{doublespace}






    \subsubsection{Analyse et résultats de l'expérience de mesure de la vitesse de la lumière}
    \begin{doublespace}

    Le dispositif expérimental, figure \ref{fig:speed-of-light} 
    est conçu pour mesurer la vitesse de la 
    lumière dans les câbles BNC, en utilisant un miroir ajustable pour 
    créer un délai temporel. Le principe de l'expérience repose sur le 
    fait que le signal parcourt une distance connue dans un câble BNC, 
    et en mesurant le temps que met le signal à parcourir cette 
    distance, nous pouvons calculer sa vitesse. Peux être que utiliser
    pour mesurer la vitesse de la lumière. Ce dernier ce fait en
    déplaçant le miroir ajustable, ce qui modifie la distance parcourue
    par le signal. En mesurant le temps que met le signal à parcourir 
    cette distance, nous pouvons calculer la vitesse de la lumière dans notre
    milieu. Je vais maintenant présenter les résultats de cette expérience,
    ainsi que les données dérivées de l'impulsion mesurée.

    \noindent J'analyse les données de l'expérience en utilisant la
    méthode que nous avons acquise précédemment, en effectuant un ajustement de
    la courbe de la dérivée de l'impulsion mesurée. Cette approche
    permet de déterminer le temps d'arrivée des impulsions. Le tableau 
    \ref{table:fits_light} présente les résultats de l'expérience,
    qui présente les temps d’arrivée mesurés pour différentes distances 
    parcourues par les impulsions, ainsi que les écarts-types et la 
    qualité des ajustements de courbe. 


\begin{longtable}{p{2.0cm} p{1.5cm} p{3.0cm} p{2.0cm} p{2.5cm}}
    \caption{ Résultats des temps d’arrivée mesurés pour différentes parcout des impulsions et types d’ajustement de courbe, avec leurs écarts-types et qualités des adjustements pour l'expérience de la vitesse de la lumière} \\
    \toprule
    \label{table:fits_light}
    Type de fit & Parcours (cm) & Temps d'arrivée (ns) & Écart-type (ns) & Qualité du fit \\
    \midrule
    \endfirsthead
    
    \toprule
    Type de fit & Parcours (cm) & Temps d'arrivée (ns) & Écart-type (ns) & Qualité du fit \\
    \midrule
    \endhead
    
    \midrule
    \multicolumn{5}{r}{{\dots}} \\
    \midrule
    \endfoot
    
    \bottomrule
    \endlastfoot
    
    % Paste all your rows here like:
    poly2     & 0            & 4.71331          & 0.00540       & 0.65535 \\
poly2     & 5.08         & 4.89051          & 0.00260       & 0.67606 \\
poly2     & 10.16        & 5.06522          & 0.00269       & 0.62149 \\
poly2     & 20.32        & 5.39393          & 0.00350       & 0.68970 \\
poly2     & 40.64        & 6.07951          & 0.00302       & 0.61930 \\
poly2     & 60.96        & 6.76310          & 0.00531       & 0.66085 \\
poly2     & 81.28        & 7.44000          & 0.00568       & 0.62715 \\
poly2     & 101.6        & 8.11442          & 0.00226       & 0.66930 \\
poly2     & 121.92       & 8.79659          & 0.00352       & 0.65540 \\
\midrule
poly3     & 0            & 4.72475          & 0.01273       & 0.65280 \\
poly3     & 5.08         & 4.90564          & 0.00548       & 0.67392 \\
poly3     & 10.16        & 5.07506          & 0.00277       & 0.62064 \\
poly3     & 20.32        & 5.41906          & 0.00608       & 0.68383 \\
poly3     & 40.64        & 6.08960          & 0.00444       & 0.61804 \\
poly3     & 60.96        & 6.76516          & 0.01389       & 0.65857 \\
poly3     & 81.28        & 7.45457          & 0.01290       & 0.62294 \\
poly3     & 101.6        & 8.12238          & 0.01007       & 0.66779 \\
poly3     & 121.92       & 8.80146          & 0.01551       & 0.65261 \\
\midrule
poly4     & 0            & 4.72134          & 0.00366       & 0.64088 \\
poly4     & 5.08         & 4.89551          & 0.00187       & 0.66138 \\
poly4     & 10.16        & 5.07134          & 0.00393       & 0.60899 \\
poly4     & 20.32        & 5.31017          & 0.21437       & 0.67474 \\
poly4     & 40.64        & 6.08475          & 0.00196       & 0.60412 \\
poly4     & 60.96        & 6.76839          & 0.00235       & 0.64813 \\
poly4     & 81.28        & 7.44816          & 0.00308       & 0.61352 \\
poly4     & 101.6        & 8.11833          & 0.00449       & 0.65716 \\
poly4     & 121.92       & 8.80287          & 0.00408       & 0.64269 \\
\midrule
poly5     & 0            & 4.72634          & 0.00810       & 0.64049 \\
poly5     & 5.08         & 4.80277          & 0.20913       & 0.66133 \\
poly5     & 10.16        & 4.99172          & 0.18918       & 0.60878 \\
poly5     & 20.32        & 5.11453          & 0.26545       & 0.67455 \\
poly5     & 40.64        & 6.08785          & 0.00248       & 0.60394 \\
poly5     & 60.96        & 6.96124          & 0.26574       & 0.64786 \\
poly5     & 81.28        & 7.35435          & 0.21549       & 0.61320 \\
poly5     & 101.6        & 8.12024          & 0.00382       & 0.65707 \\
poly5     & 121.92       & 8.80513          & 0.00999       & 0.64219 \\
\midrule
poly6     & 0            & 4.72661          & 0.00817       & 0.63983 \\
poly6     & 5.08         & 4.97781          & 0.18094       & 0.66097 \\
poly6     & 10.16        & 5.07586          & 0.00696       & 0.60835 \\
poly6     & 20.32        & 5.40079          & 0.00798       & 0.67433 \\
poly6     & 40.64        & 6.08851          & 0.00310       & 0.60386 \\
poly6     & 60.96        & 6.86658          & 0.21361       & 0.64768 \\
poly6     & 81.28        & 7.35519          & 0.21417       & 0.61316 \\
poly6     & 101.6        & 8.12101          & 0.00479       & 0.65679 \\
poly6     & 121.92       & 8.80259          & 0.00943       & 0.64184 \\
\midrule
poly7     & 0            & 4.71966          & 0.01125       & 0.63930 \\
poly7     & 5.08         & 4.81158          & 0.19912       & 0.66074 \\
poly7     & 10.16        & 5.07004          & 0.00832       & 0.60817 \\
poly7     & 20.32        & 5.40111          & 0.00790       & 0.67401 \\
poly7     & 40.64        & 6.08673          & 0.00729       & 0.60372 \\
poly7     & 60.96        & 6.67275          & 0.21349       & 0.64746 \\
poly7     & 81.28        & 7.44627          & 0.00925       & 0.61295 \\
poly7     & 101.6        & 8.11835          & 0.00488       & 0.65661 \\
poly7     & 121.92       & 8.81123          & 0.30484       & 0.64079 \\
\midrule
poly8     & 0            & 4.72029          & 0.01048       & 0.63920 \\
poly8     & 5.08         & 4.81604          & 0.19913       & 0.66011 \\
poly8     & 10.16        & 5.07020          & 0.00792       & 0.60801 \\
poly8     & 20.32        & 5.40480          & 0.00838       & 0.67384 \\
poly8     & 40.64        & 6.08686          & 0.00920       & 0.60324 \\
poly8     & 60.96        & 6.67051          & 0.19889       & 0.64680 \\
poly8     & 81.28        & 7.43096          & 0.33318       & 0.61283 \\
poly8     & 101.6        & 7.92769          & 0.20935       & 0.65582 \\
poly8     & 121.92       & 8.42229          & 0.13732       & 0.64046 \\
\midrule
poly9     & 0            & 4.72486          & 0.33006       & 0.63897 \\
poly9     & 5.08         & 4.70720          & 0.22780       & 0.65972 \\
poly9     & 10.16        & 4.88903          & 0.22563       & 0.60786 \\
poly9     & 20.32        & 5.20094          & 0.25460       & 0.67365 \\
poly9     & 40.64        & 5.80773          & 0.23018       & 0.60284 \\
poly9     & 60.96        & 6.38899          & 0.10911       & 0.64666 \\
poly9     & 81.28        & 7.09352          & 0.15865       & 0.61282 \\
poly9     & 101.6        & 7.69569          & 0.08447       & 0.65868 \\
poly9     & 121.92       & 8.38099          & 0.05428       & 0.64209 \\
\midrule
fourier1  & 0            & 4.71572          & 0.00457       & 0.64231 \\
fourier1  & 5.08         & 4.89408          & 0.00229       & 0.66186 \\
fourier1  & 10.16        & 5.06706          & 0.00226       & 0.60996 \\
fourier1  & 20.32        & 5.39963          & 0.00321       & 0.67576 \\
fourier1  & 40.64        & 6.08169          & 0.00254       & 0.60503 \\
fourier1  & 60.96        & 6.76347          & 0.00355       & 0.64888 \\
fourier1  & 81.28        & 7.44291          & 0.00361       & 0.61461 \\
fourier1  & 101.6        & 8.11617          & 0.00191       & 0.65769 \\
fourier1  & 121.92       & 8.79741          & 0.00198       & 0.64380 \\
\midrule
fourier2  & 0            & 4.72877          & 0.01125       & 0.64064 \\
fourier2  & 5.08         & 4.89736          & 0.00518       & 0.66119 \\
fourier2  & 10.16        & 4.99303          & 0.19275       & 0.60877 \\
fourier2  & 20.32        & 5.40637          & 0.00155       & 0.67472 \\
fourier2  & 40.64        & 6.08968          & 0.00487       & 0.60397 \\
fourier2  & 60.96        & 6.77192          & 0.00455       & 0.64791 \\
fourier2  & 81.28        & 7.45179          & 0.00598       & 0.61331 \\
fourier2  & 101.6        & 7.93620          & 0.25209       & 0.65693 \\
fourier2  & 121.92       & 8.89100          & 0.19440       & 0.64262 \\
\midrule
fourier3  & 0            & 4.63993          & 0.18354       & 0.63971 \\
fourier3  & 5.08         & 4.89476          & 0.00496       & 0.66074 \\
fourier3  & 10.16        & 5.07082          & 0.01047       & 0.60822 \\
fourier3  & 20.32        & 5.30076          & 0.21511       & 0.67417 \\
fourier3  & 40.64        & 6.08743          & 0.00893       & 0.60372 \\
fourier3  & 60.96        & 6.76993          & 0.00735       & 0.64743 \\
fourier3  & 81.28        & 7.44726          & 0.01016       & 0.61299 \\
fourier3  & 101.6        & 8.12069          & 0.01187       & 0.65622 \\
fourier3  & 121.92       & 8.71801          & 0.19372       & 0.64159 \\
\midrule
fourier4  & 0            & 4.72830          & 0.01700       & 0.63849 \\
fourier4  & 5.08         & 4.82117          & 0.19003       & 0.65918 \\
fourier4  & 10.16        & 5.16231          & 0.16442       & 0.60734 \\
fourier4  & 20.32        & 5.32363          & 0.21445       & 0.67340 \\
fourier4  & 40.64        & 6.18942          & 0.19740       & 0.60294 \\
fourier4  & 60.96        & 6.85717          & 0.18283       & 0.64564 \\
fourier4  & 81.28        & 7.47014          & 0.00785       & 0.61182 \\
fourier4  & 101.6        & 8.01239          & 0.20774       & 0.65498 \\
fourier4  & 121.92       & 8.80669          & 0.04605       & 0.63934 \\
\midrule
fourier5  & 0            & 4.71827          & 0.03783       & 0.63602 \\
fourier5  & 5.08         & 4.81994          & 0.18129       & 0.65602 \\
fourier5  & 10.16        & 4.99705          & 0.21753       & 0.60598 \\
fourier5  & 20.32        & 5.30335          & 0.20536       & 0.67253 \\
fourier5  & 40.64        & 6.09325          & 0.01973       & 0.60150 \\
fourier5  & 60.96        & 6.60507          & 0.26801       & 0.64472 \\
fourier5  & 81.28        & 7.44084          & 0.06341       & 0.61077 \\
fourier5  & 101.6        & 7.96551          & 0.26103       & 0.65341 \\
fourier5  & 121.92       & 8.70882          & 0.22482       & 0.63844 \\
\midrule
fourier6  & 0            & 4.67487          & 0.21258       & 0.63513 \\
fourier6  & 5.08         & 4.87912          & 0.30044       & 0.65545 \\
fourier6  & 10.16        & 5.01425          & 0.15367       & 0.60510 \\
fourier6  & 20.32        & 5.40065          & 0.04759       & 0.67185 \\
fourier6  & 40.64        & 6.02640          & 0.20386       & 0.60054 \\
fourier6  & 60.96        & 6.82890          & 0.21281       & 0.64294 \\
fourier6  & 81.28        & 7.56540          & 0.18348       & 0.61024 \\
fourier6  & 101.6        & 8.01208          & 0.20554       & 0.65208 \\
fourier6  & 121.92       & 8.67386          & 0.20449       & 0.63782 \\
\midrule
gauss1    & 0            & 4.71424          & 0.00450       & 0.64553 \\
gauss1    & 5.08         & 4.89228          & 0.00232       & 0.66632 \\
gauss1    & 10.16        & 5.06612          & 0.00245       & 0.61264 \\
gauss1    & 20.32        & 5.39707          & 0.00293       & 0.67962 \\
gauss1    & 40.64        & 6.08066          & 0.00273       & 0.60835 \\
gauss1    & 60.96        & 6.76292          & 0.00380       & 0.65146 \\
gauss1    & 81.28        & 7.44163          & 0.00415       & 0.61718 \\
gauss1    & 101.6        & 8.11554          & 0.00192       & 0.66113 \\
gauss1    & 121.92       & 8.79674          & 0.00190       & 0.64651
\end{longtable}

\noindent Le tableau \ref{table:speed_light} 
présente les résultats de la mesure de la vitesse de la lumière pour 
différents ajustements de courbe. Les valeurs mesurées sont comparées
 à la vitesse théorique de la lumière dans le vide, qui est de 
 $299792458$ $m/s$. Les erreurs en pourcentage sont calculées par 
 rapport à cette valeur théorique, et la qualité du fit est 
 indiquée par le coefficient de détermination ($R^2$) pour chaque 
 ajustement.

\begin{longtable}{p{2.0cm} p{2.0cm} p{2.0cm} p{2.0cm} p{2.0cm}}
    \caption{Mesure de la vitesse de la lumière pour différent ajustement de courbe} \\
    \toprule
    \label{table:speed_light}
    Type de fit & Vitesse mesurée (m/s) & Vitesse théorique (m/s) & Erreur (\%) & Qualité du fit \\
    \midrule
    \endfirsthead
    
    \toprule
    Type de fit & Vitesse mesurée (m/s) & Vitesse théorique (m/s) & Erreur (\%) & Qualité du fit \\
    \midrule
    \endhead
    
    \midrule
    \multicolumn{5}{r}{{À suivre sur la prochaine page}} \\
    \midrule
    \endfoot
    
    \bottomrule
    \endlastfoot
    
    % Paste all your rows here like:
    poly2     & 298948041     & 299792458           & 0.2817      & 0.0048       \\
    poly3     & 299593785     & 299792458           & 0.0663      & 0.0061       \\
    poly4     & 297557352     & 299792458           & 0.7456      & 0.0361       \\
    poly5     & 291020596     & 299792458           & 2.9260      & 0.1405       \\
    poly6     & 302282720     & 299792458           & 0.8307      & 0.0605       \\
    poly7     & 297220542     & 299792458           & 0.8579      & 0.0440       \\
    poly8     & 320556240     & 299792458           & 6.9261      & 0.0943       \\
    poly9     & 326791560     & 299792458           & 9.0059      & 0.0690       \\
    fourier1  & 299104313     & 299792458           & 0.2295      & 0.0042       \\
    fourier2  & 299334235     & 299792458           & 0.1528      & 0.0874       \\
    fourier3  & 298537843     & 299792458           & 0.4185      & 0.0546       \\
    fourier4  & 300580478     & 299792458           & 0.2629      & 0.0943       \\
    fourier5  & 303509743     & 299792458           & 1.2400      & 0.0682       \\
    fourier6  & 301689685     & 299792458           & 0.6328      & 0.0907       \\
    gauss1    & 299025597     & 299792458           & 0.2558      & 0.0044     
\end{longtable}


    Je remarque que les ajustements de courbe sont de bonne qualité, avec des valeurs similaires
    que l'expérience précédente, ce qui indique que les données sont
    cohérentes et fiables. Les écarts-types sont également faibles, ce qui
    suggère que les mesures sont précises et répétables. Je procède ensuite
    aux résultats des données issues des impulsions de l’expérience du miroir déplacé, 
    figure \ref{fig:speed-pulses}.
    
    \begin{figure}[!hptb]
        \centering
        \includegraphics[width=1.0\textwidth]{fits_light.png}
        \caption{Profil temporel de la dérivée des données d'impulsion pour chacune des distances mesurées et ajustement de la courbe pour l'expérience de la vitesse de la lumière avec un miroir réglable.}
        \label{fig:speed-pulses}
    \end{figure}

    \noindent D'ici je place les résultats des délais obtenue dans l'expérience
    de la vitesse de la lumière et les comparant à le temps d'arrivée
    d'un impulsion où le miroir est fixe, ce qui nous permet
    de déterminer les délais pour la vitesse de la lumière. La figure
    \ref{fig:speed-res} présente le résultat de notre expérience.

    \begin{figure}[!hptb]
    \centering
    \includegraphics[width=1.0\textwidth]{speed_light.png}
    \caption{Résultats des délais mesurés pendant l'expérience sur la vitesse de la lumière, ainsi que leur ajustement linéaire. Les barres d’erreur horizontales représentent l’incertitude de nos mesures de la distance, soit $\pm 0,5$ $mm$, qui est trop petite pour être visible sur le graphique. Les barres d'erreurs verticales correspondent à l'erreur de l'ajustement Fourier basé sur l'expérience de la vitesse de la lumière dans les câbles BNC $\pm 0,03\%$.}
    \label{fig:speed-res}
    \end{figure}

    \noindent En utilisant les résultats de l'expérience, nous avons
    calculé la vitesse de la lumière dans les câbles BNC. La vitesse
    mesurée est de $299 025 597$ $m/s$ avec un écart-type de 
    $0.03\%$ par rapport à la valeur théorique de la vitesse de la
    lumière dans l'aire libre, qui est de $0.9998c$. À partir de ces 
    résultats de ces deux expériences, nous concluons que nous pouvons 
    mesurer des délais temporels très courts, de l'ordre de $4$ $ps$ avec 
    une variation inférieure à $2$ $ps$ entre les ensembles de données. 
    Les sections suivantes seront 
    consacrées à la caractérisation des états de polarisations en utilisant
    les méthodologies que nous avons développées dans les sections précédentes.
    
\end{doublespace}






    \subsection{Caractérisation de la partie réelle de la valeur faible}
    Pour caractériser la partie réelle de la valeur 
faible, nous introduisons une interaction faible 
entre les états de base de la 
polarisation $\ket{H}$ et $\ket{V}$ 
via un délai temporel. Ce délai doit être 
inférieur au profil temporel du 
laser $\sigma \ll \tau$. Aucun 
modèle ne décrit spécifiquement comment 
l'interaction devrait être faible, mais il 
doit y avoir un chevauchement évident entre 
les états de base. Nous supposons qu’au moins 
$90\%$ de chevauchement entre les états de 
base sont nécessaires pour être dans le régime 
des mesures faibles. Ensuite, pour mesurer 
l'état directement par des mesures faibles, 
nous devons effectuer une mesure projective qui
contient les deux états de base afin de pouvoir 
caractériser les états d'entrée de polarisation 
entre nos états de base. L’un de ces états 
correspond au délai maximal appliqué, tandis 
que l'autre correspond à l'absence de délai. 
Ici, le terme délai fait référence à un 
signal extérieur qui active l'oscilloscope, 
comme dans l'expérience sur la vitesse de la 
lumière. La différence est que nous postulons 
que la manière la plus simple de créer des 
écarts temporels entre les états de base est 
d'utiliser un type d'interféromètre de 
polarisation dont l'un des bras est légèrement 
décalé d'une quantité correspondant à notre 
délai maximum par rapport à l'autre bras non 
décalé. La section suivante décrit le 
dispositif expérimental que nous avons utilisé 
pour caractériser la partie réelle de la valeur 
faible. 
    \subsubsection{Montage et étapes de préparation}
    \begin{doublespace}
    Notre dispositif expérimental, représenté à la figure 
    \ref{fig:realexp}, est composé de notre laser pulsé de tout à 
    l’heure et suit une configuration similaire à celle de notre 
    montage précédent (figure \ref{fig:speed-of-light}). Cependant, 
    après le premier séparateur de faisceau, l’état de polarisation 
    d’entrée est préparé et ensuite soumis à une mesure faible en 
    introduisant un court délai avec une différence de parcours entre 
    les deux composantes orthogonales venant du deuxième séparateur de 
    faisceau. Finalement, une mesure projective est effectuée pour une 
    caractérisation complète et directe.
\end{doublespace}

\begin{figure}[!htpb]
    \centering
    \includegraphics[width=1.0\textwidth]{partiereelexp.png}
    \caption{Dispositif expérimental pour la partie réelle de la valeur 
    faible. Comme la dernière configuration, 
    l’impulsion du laser est réglée en intensité par une lame demi-onde, 
    puis dirigée vers un PBS qui divise les états de polarisation 
    horizontaux et verticaux de
    base de l’impulsion d’entrée en deux voies orthogonales. Celui qui 
    est réfléchi sera notre signal de référence (ou signal de déclenchement) 
    pour déclencher 
    l’oscilloscope et l’autre suit la procédure directe. 
    Ce dernier est préparé en combinant différentes lames, dont on change 
    pour chaque trajet de polarisation, et ensuite subit une mesure 
    faible. Lors de la mesure faible, le couplage faible temporel est 
    réalisé en introduisant 
    une différence de parcours entre la voie transmise et réfléchie du 
    deuxième PBS. Le faisceau réfléchi représente l’état
    de base de polarisation verticale $\ket{V}$ de l’état d’entrée 
    $\ket{\psi}$, tandis que le faisceau
    transmis correspond à l’état de base horizontal $\ket{H}$. Le miroir 
    de la voie de l’état vertical 
    est considéré sans délai et il se trouve à $10,5$ $cm$ du PBS. Le 
    miroir de la voie de l’état horizontal est positionné à la même 
    distance, mais avec le délai que nous ajouterons. Les deux voies 
    subissent une rotation pour revenir et se projettent sur une lame 
    demi-onde inclinée à $45$ degrés par rapport à un polariseur 
    initialement orienté à $0$ degré. 
    Cela crée une mesure de projection avec l’état $\ket{D}$, qui 
    retient le chevauchement des états de base. Ce dernier est ensuite 
    détecté avec un
    photodétecteur, puis interprété par notre oscilloscope. 
}
    \label{fig:realexp}
\end{figure}

\begin{doublespace}
    \noindent Une fois de plus, nous souhaitons déclencher le signal 
    de référence de manière externe, car nous voulons bénéficier de la 
    résolution temporelle maximale offerte par l’oscilloscope, qui 
    possède une fréquence d’échantillonnage de $500$ $GS/s$, pour 
    détecter les délais. Un séparateur de faisceau polarisant divise 
    le faisceau laser en deux voies : celui réfléchi sert de signal de 
    référence pour le déclenchement de l’oscilloscope, tandis que celui 
    transmit sera préparé dans divers trajets de polarisation sur la 
    sphère Poincarré et subira une mesure faible pour une caractérisation. 
    Les trajets de polarisation testés sont les suivants et sont obtenus 
    en changeant les lames d'onde lors de l'étape de préparation. Le 
    premiers consiste seulement à une lame demi-onde définie par 
    l’opérateur suivant:
\end{doublespace}

\begin{align}
    \hat{T}_{HWP}(\theta) = \begin{pmatrix}
        cos(2\theta) & sin(2\theta)\\
        sin(2\theta) & -cos(2\theta)
    \end{pmatrix}
\end{align}

\begin{doublespace}
    \noindent Où l'indice $HWP$ fait référence à une lame demi-onde 
    pour un angle $\theta$ (\guillemetleft halfwaveplat \guillemetright 
    en anglais). Ce premier trajet consiste à passer d’un état de base 
    à un autre sans polarisation circulaire, de 
    $H \to D \to V \to A \to \dots$, et ainsi de suite. Pour comprendre 
    en détail, l'état commence dans l’état horizontal 
    $\ket{H} \equiv \begin{pmatrix}
        1\\
        0
    \end{pmatrix}$ défini par la transmission d'un séparateur 
    de faisceau polarisant. Ensuite, l'état évolue de la façon 
    suivante en fonction de l'angle de la lame d'onde $\theta$:
\end{doublespace}

\begin{align}
    \hat{T}_{HWP}(\theta)\ket{H} &= \begin{pmatrix}
        cos(2\theta) & sin(2\theta)\\
        sin(2\theta) & -cos(2\theta)
    \end{pmatrix} \begin{pmatrix}
        1\\0
    \end{pmatrix}\\
    &= \begin{pmatrix}
        cos(2\theta)\\sin(2\theta)
    \end{pmatrix}
\end{align}

\begin{doublespace}
    \noindent Donc, l'état d'entrée préparé est soit:
\end{doublespace}

\begin{equation}
    \ket{\psi_{i}^{1}} = cos(2\theta)\ket{H} + sin(2\theta)\ket{V}
\end{equation}

\begin{doublespace}
    \noindent En fonction des paramètres de Stokes pour démontrer le 
    trajet sur la sphère Poincarré (figure \ref{fig:path3sphere}):
\end{doublespace}

\begin{equation}
    S = \begin{pmatrix}
        S_0 = |a|^2 + |b^2|\\
        S_1 = |a|^2 - |b|^2\\
        S_2 = 2\mathcal{R}(\bar{a}b)\\
        S_3 = 2\mathcal{I}(\bar{a}b)
    \end{pmatrix} = \begin{pmatrix}
        1\\
        cos^{2}(2\theta) - sin^{2}(2\theta) = cos(4\theta)\\
        2cos(2\theta)sin(2\theta) = sin(4\theta)\\
        0
    \end{pmatrix}
\end{equation}

\begin{figure}[!htpb]
    \centering
    \includegraphics[width=1.0\textwidth]{poincare_sphere_HDVAH.png}
    \caption{Schéma du trajet $\ket{H}\rightarrow\ket{D}\rightarrow\ket{V}\rightarrow\ket{A}\dots$
    utilisant seulement une lame demi-onde dans la préparation de l'état d'entrée}
    \label{fig:path3sphere}
\end{figure}

\begin{doublespace}
    \noindent Ce trajet est réalisé en tournant uniquement une lame 
    demi-onde. On tourne l'angle de la lame d'onde de $2,5$
    degrés pour chaque acquisition. Chaque degré $\theta^\prime$ 
    que nous tournons en réalité 
    équivaut à tourner de $5$ degrés sur un plan circulaire 
    $\theta^\prime \equiv 2\theta$ ou de $10$ degrés sur la sphère de 
    Poincarré. 

    \noindent Le trajet suivant consiste à passer d'un état de base à 
    un autre en passant par une polarisation circulaire 
    $\ket{H} \to \ket{R} \to \ket{V} \to \ket{L} \dots$. Cela se fait 
    avec une lame demi-onde tournant de la même manière que précédemment, 
    et une lame quart d'onde réglée à $0$ degré par rapport à 
    $\ket{H}$. L’opération de cette lame d’onde se définit par 
    l'opérateur suivant:
\end{doublespace}

\begin{align}
    \hat{T}_{QWP}(\phi) &= \begin{pmatrix}
        cos^{2}(\phi) + isin^{2}(\phi) & (1-i)cos(\phi)sin(\phi)\\
        (1-i)cos(\phi)sin(\phi) & sin^{2}(\phi) + icos^{2}(\phi)
    \end{pmatrix}\\
    \hat{T}_{QWP}(\phi = 0^{\degree}) &= \begin{pmatrix}
        1 & 0 \\
        0 & i
    \end{pmatrix}
\end{align}

\begin{doublespace}
    \noindent La forme de cet opérateur $\hat{T}_{QWP}(\phi)$, avec 
    l'angle $\phi$ pour la lame d'onde et l'indice QWP fait référence 
    à une lame quart d'onde 
    (\guillemetleft quarter waveplate \guillemetright en anglais), 
    permet de conserver $a \in \mathcal{R}$ et de laisser 
    $b\in \mathcal{C}$ contenir l'information complexe. Nous 
    procédons ainsi pour que la partie imaginaire de la valeur 
    faible soit principalement contenue dans $b$ pour des raisons de 
    simplicité. Avec cette opération, l'état évolue comme suit:
\end{doublespace}

\begin{align}
    \hat{T}_{QWP}(\phi = 0^{\degree})\hat{T}_{HWP}(\theta)\ket{H} &= 
    \begin{pmatrix} 
        1 & 0 \\
        0 & i
    \end{pmatrix} 
    \begin{pmatrix}
        cos(2\theta) & sin(2\theta)\\
        sin(2\theta) & -cos(2\theta)
    \end{pmatrix} \begin{pmatrix}
        1\\0
    \end{pmatrix}\\
    &= \begin{pmatrix}
        cos(2\theta)\\isin(2\theta)
    \end{pmatrix}
\end{align}

\begin{doublespace}
    \noindent Donc, l'état d'entrée est: 
\end{doublespace}

\begin{equation}
    \ket{\psi_{i}^{2}} = cos(2\theta)\ket{H} + isin(2\theta)\ket{V}
\end{equation}

\begin{doublespace}
    \noindent Avec les paramètres de Stokes pour démontrer sa 
    trajectoire (figure \ref{fig:path4sphere}):
\end{doublespace}

\begin{equation}
    S = \begin{pmatrix}
        S_0 = |a|^2 + |b^2|\\
        S_1 = |a|^2 - |b|^2\\
        S_2 = 2\mathcal{R}(\bar{a}b)\\
        S_3 = 2\mathcal{I}(\bar{a}b)
    \end{pmatrix} = \begin{pmatrix}
        1\\
        cos^{2}(2\theta) - sin^{2}(2\theta) = cos(4\theta)\\
        0\\
        2cos(2\theta)sin(2\theta) = sin(4\theta)
    \end{pmatrix}
\end{equation}

\begin{figure}[!htpb]
    \centering
    \includegraphics[width=1.0\textwidth]{poincare_sphere_HRVLH.png}
    \caption{Schéma du trajet 
    $\ket{H}\rightarrow\ket{R}\rightarrow\ket{V}\rightarrow\ket{L}\dots$
    utilisant seulement une lame demi-onde et une lame quart d'onde à 
    $0$ dégré dans la préparation de l'état d'entrée}
    \label{fig:path4sphere}
\end{figure}

\begin{doublespace}
    \noindent Le trajet final est un parcours captivant qui nous fait 
    passer constamment entre deux états de base, soit d'une polarisation 
    linéaire $\{\ket{D}, \ket{A}\}$ à une polarisation circulaire 
    $\{ \ket{R}, \ket{L}\}$. La trajectoire résultante est 
    $\ket{D} \to \ket{R} \to \ket{A} \ket{L} \dots$. Cette dernière est 
    obtenue en tournant une lame demi-onde avec une lame quart d'onde 
    réglée à $45$ degrés par rapport à l'état de base $\ket{H}$, dont 
    la lame quart d'onde à ce réglage est défini comme suit: 
\end{doublespace}

\begin{equation}
    \hat{T}_{QWP}(\phi = 45^{\degree}) = \frac{1}{2}\begin{pmatrix}
        1+i & 1-i \\
        1-i & 1+i
    \end{pmatrix}
\end{equation}

\begin{doublespace}
    \noindent Avec cette opération, l'état évolue comme suit:
\end{doublespace}

\begin{align}
    \hat{T}_{QWP}(\phi = 45^{\degree})\hat{T}_{HWP}(\theta)\ket{H} &= 
    \begin{pmatrix} 
        \frac{1+i}{2} & \frac{1-i}{2} \\
        \frac{1-i}{2} & \frac{1+i}{2}
    \end{pmatrix} 
    \begin{pmatrix}
        cos(2\theta) & sin(2\theta)\\
        sin(2\theta) & -cos(2\theta)
    \end{pmatrix} \begin{pmatrix}
        1\\0
    \end{pmatrix}\\
    &= \frac{1}{2}\begin{pmatrix}
        cos(2\theta) + sin(2\theta) + i(cos(2\theta) - sin(2\theta))\\cos(2\theta) + sin(2\theta) -i(cos(2\theta)-sin(2\theta))
    \end{pmatrix}
\end{align}

\begin{doublespace}
    \noindent Donc, l'état d'entrée est: 
\end{doublespace}

\begin{equation}
    \ket{\psi_{i}^{3}} = \frac{1}{2}\Bigl(((1-i)sin(2\theta) + (1+i)cos(2\theta))\ket{H} + ((1+i)sin(2\theta) + (1-i)cos(2\theta))\ket{V}\Bigr)
\end{equation}

\begin{doublespace}
    \noindent Avec les paramètres de Stokes pour démontrer sa 
    trajectoire (figure \ref{fig:path5sphere}):
\end{doublespace}

\begin{equation}
    S = \begin{pmatrix}
        S_0 = |a|^2 + |b^2|\\
        S_1 = |a|^2 - |b|^2\\
        S_2 = 2\mathcal{R}(\bar{a}b)\\
        S_3 = 2\mathcal{I}(\bar{a}b)
    \end{pmatrix} = \begin{pmatrix}
        1\\
        0\\
        sin(4\theta)\\
        sin^2(2\theta) - cos^2(2\theta)
    \end{pmatrix}
\end{equation}

\begin{figure}[!htpb]
    \centering
    \includegraphics[width=1.0\textwidth]{poincare_sphere_DRALD.png}
    \caption{Schéma du trajet $\ket{D}\rightarrow\ket{R}\rightarrow\ket{A}\rightarrow\ket{L}\dots$
    utilisant seulement une lame demi-onde et une lame quart d'onde à 
    $45$ dégrés dans la préparation de l'état d'entrée}
    \label{fig:path5sphere}
\end{figure}



    \subsubsection{Mesure faible temporelle}
    \begin{doublespace}
    Après avoir préparé l’état, nous interagissons faiblement avec le 
    système en introduisant un court délai temporel entre deux états 
    $\ket{H}$ et $\ket{V}$ de base. Pour ce faire, nous utilisons 
    un deuxième séparateur de faisceau polarisant dans l’étape d’interaction faible. Nous faisons 
    en sorte qu’un des bras soit légèrement plus long que 
    l’autre de $167$ $ps$ (voir figure \ref{fig:realexp}). 
    Chaque bras de l'interféromètre est équipé d’une lame quart 
    d’onde pour inverser l’état de polarisation afin que 
    la lumière dans le bras 
    réfléchi soit transmise et que celle dans le bras transmis 
    soit réfléchie, de sorte 
    qu’ils puissent recombiner sur le PBS. La partie transmise, que nous définissons 
    comme étant la partie horizontale de l’état de polarisation 
    $a\ket{H}$, subit une interaction faible en parcourant un trajet 
    plus long. Cela introduit un délai $\tau$ sur le pointeur couplé 
    avec cet état $a\ket{H} \otimes \ket{\xi(t-\tau)}$.  
    La partie réfléchie, c’est-à-dire la partie 
    verticale de l'état de polarisation $b\ket{V}$ reste inchangée 
    et est couplée avec l'état du pointeur
    $b\ket{V} \otimes \ket{\xi(t)}$.
    Le système est donc dans l'état suivant après l'interaction faible:

    \begin{equation}
        \hat{U}^{H}\ket{\psi_i} = a\ket{H} \otimes \ket{\xi(t-\tau)} + b\ket{V} \otimes \ket{\xi(t)}
    \end{equation}

    \noindent Pour avoir une interférence avec des poids égaux de 
    $\ket{H}$ et $\ket{V}$, nous avons opté pour une postsélection sur l'état 
    $\ket{D} = \frac{1}{\sqrt{2}}(\ket{H}+\ket{V})$, qui est réalisée à 
    l’aide d'une lame demi-onde à $45$ degrés et d'un polariseur à $90$ degrés
    dont le polariseur sert de référence pour une postsélection diagonale. 
    L'état final après postsélection est : 

    \begin{equation}
        \ket{\psi_f} \equiv \bra{D}\hat{U}^{H}\ket{\psi_i} = \frac{a}{\sqrt{2}}\ket{\xi(t-\tau)} + \frac{b}{\sqrt{2}}\ket{\xi(t)}
    \end{equation}

    \noindent Avant de poursuivre avec la caractérisation quantique de 
    la partie réelle de la valeur faible, nous effectuons une calibration
    en envoyant l'état $\ket{V}$ et $\ket{H}$ 
    qui correspondent respectivement au temps de référence $t_0$ par rapport 
    auxquels tous les délais seront calculés et le temps
    $t_0 + \tau$ correspondant au délai de référence et au délai introduit. 
    On s'assure que le délai $\tau$ est égal à celui implémenté dans 
    l'expérience de $167$ $ps$.
    Nous pouvons alors mesurer la partie réelle de la valeur faible
    en utilisant l'état de polarisation préparé et l'état de polarisation
    post-sélectionné.

    \noindent Nous caractérisons ensuite les trajets de polarisation pour mesurer 
    la partie réelle de la valeur faible, c’est-à-dire mesurer les 
    délais temporels associés aux impulsions des états de polarisation. 
    Pour rappeler, la partie réelle de la valeur faible est donnée par
    l'équation \ref{eq:expval_t_norm}, ce que nous allons donc calculer
    $\expval{\hat{t}}$ qui est lié à la partie réelle de la valeur faible
    pour chaque trajets de polarisation.
    
    \noindent Nous commençons par l'état de polarisation linéaire 
    $\ket{\psi_{i}^{1}} = cos(2\theta)\ket{H} + sin(2\theta)\ket{V}$:

    \begin{equation}
        \expval{\hat{t}} = \frac{\tau}{2}(cos^2(2\theta) + 2sin(2\theta)cos(2\theta))
        \label{eq:expval_t_1}
    \end{equation}

    \noindent Ensuite, pour l'état de polarisation circulaire 
    $\ket{\psi_{i}^{2}} = cos(2\theta)\ket{H} + isin(2\theta)\ket{V}$ nous avons:

    \begin{equation}
        \expval{\hat{t}} = \frac{\tau}{2}(cos^2(2\theta))
        \label{eq:expval_t_2}
    \end{equation}

    \noindent et pour l'état de polarisation de superposition (en référence qu'il y a toujours
    une superposition entre $\ket{H}$ et $\ket{V}$ pour ce trajet de polarisation) 
    $\ket{\psi_{i}^{3}} = \frac{1}{2}\Bigl(((1-i)sin(2\theta) + (1+i)cos(2\theta))\ket{H} + ((1+i)sin(2\theta) + (1-i)cos(2\theta))\ket{V}\Bigr)$
    nous avons:

    \begin{equation}
        \expval{\hat{t}} = \frac{\tau}{2}\Bigl(1+\frac{sin(4\theta)}{4}\Bigr)
        \label{eq:expval_t_3}
    \end{equation}

    \noindent Cette expérience a été optimisée pour qu’elle fonctionne 
    de manière autonome grâce à des supports de rotation motorisés 
    \cite{ThorlabsROT} pour les lames d'onde contrôlés par un code 
    \textsc{Python} et un oscilloscope interfacé à un ordinateur en utilisant
    la bibliothèque Kinesis \cite{pylablib_thorlabs_kinesis} pour faire 
    tourner ces supports et l’API de l'oscilloscope \cite{TektronixTDS5000,oscilloscope_coding} 
    pour enregistrer chaque fichier de chaque état d’entrée. Toutes les 
    données sont ensuite analysées avec \textsc{MATLAB}. Les résultats sont 
    présentés au chapitre 4. 
\end{doublespace}


    \subsection{Caractérisation de la partie imaginaire de la valeur faible}
    \begin{doublespace}
    Dans cette section, nous aborderons l'expérience que nous proposons pour mesurer la partie imaginaire de la valeur faible, ainsi que les résultats attendus. La partie imaginaire de la valeur faible contient l'information complexe de l'état quantique. Dans notre cas, c’est l’ellipticité de l'état de polarisation. Certaines expériences impliquant une interaction de fréquence faible ont réussi à mesurer la valeur faible \cite{Salazar}, mais, étant donné qu’on utilise un délai temporel, nous devons contourner ce problème. Des approches théoriques ont été développées à ce sujet, mais aucune n’a été appliquée en pratique \cite{OpticalNetworks, Time_delay_china}.
\end{doublespace}
    \subsubsection{Montage proposé}
    \begin{doublespace}
    %À partir de l'expérience de la partie réelle, notre interaction 
    %faible est réalisée en introduisant un décalage temporel faible entre 
    %les deux composantes de la polarisation du système. Ce décalage modifie la 
    %forme temporelle du paquet d’ondes (pointeur) de manière cohérente, 
    %sans perturber fortement l’état quantique. Ces effets se manifestent 
    %dans le domain conjugué du domaine temporelle c'est à direle 
    %spectre d’interférence sous la forme d’un déplacement de la 
    %position des peignes de fréquences du spectre. 
    %Comme le temps et la fréquence sont des 
    %quantités conjuguées au sens de Fourier, le principe d’incertitude 
    %d’Heisenberg se manifeste lors d’une interaction faible avec le 
    %système. 

    %\noindent Pour observer ce spectre, on peut soit utiliser un 
    %spectromètre ou un interféromètre avec une résolution suffisante. 
    %Cependant, ces deux méthodes ont leurs limites dans notre expérience. 
    %D’une part, un spectromètre offre une résolution suffisante, mais 
    %il est coûteux et peu pratique à intégrer dans une infrastructure 
    %photonique quantique intégrée. D’autre part, la faible longueur de 
    %cohérence du laser (entre $2$ et $8$ $\mu m$) limite 
    %l’interférométrie classique. Cela nécessiterait un système de 
    %translation piézoélectrique motorisé à haute précision. À la place, 
    %nous avons incorporé notre dispositif de mesure faible temporelle 
    %dans un interféromètre de Mach-Zehnder (MZ) afin de caractériser 
    %l’état de polarisation d’un système photonique de manière hétérodyne
    %(voir la figure \ref{fig:imagexp}).

    \noindent Pour observer ce spectre, on peut soit utiliser un 
    spectromètre. Cependant, cette méthode peut offrir une 
    bonne résolution, mais il reste coûteux et difficilement intégrable 
    dans une plateforme photonique intégrée. 
    %En ce qui concerne 
    %l'interférométrie, la faible longueur de cohérence du laser (entre $2$ et $8$ $\mu m$) 
    %rend délicate l’observation d’interférences stables sans un système 
    %de translation piézoélectrique haute précision. 
    Donc, nous avons choisi d’intégrer notre dispositif de mesure faible 
    temporelle (figure \ref{fig:realexp}) 
    dans un interféromètre de 
    Mach-Zehnder (MZ) ce qui permet d’extraire le décallage 
    dans le domaine fréquentiel afin de caractériser 
    des états de polarisation à partir des variations d’intensité 
    en sortie avec sont spectre de puissance (figure \ref{fig:imagexp}).

    \begin{figure}[!htpb]
        \centering
        \includegraphics[width=1.0\textwidth]{partieimagexp.png}
        \caption{Dispositif expérimental pour la partie imaginaire 
        de la valeur faible. Cette configuration utilise le 
        dispositif de la partie réelle dans un interféromètre de 
        Mach-Zedner. Un séparateur de faisceau non polarisant 
        (NPBS) est utilisé avant l'étape de préparation de l'état 
        pour diviser le faisceau en deux voies. L’une de ces voies 
        sert de référence pour interférer avec l’autre voie, qui a 
        subi une mesure directe. Une lame demi-onde a été placée sur 
        le trajet de cette référence afin de modifier son état de 
        polarisation pour permettre des interférences avec le dernier.}
        \label{fig:imagexp}
    \end{figure}

    \noindent Cette technique repose sur l’interférence entre deux 
    impulsions émises par un seul laser, l’une ayant subi une mesure 
    faible (signal expérimental) et l’autre servant de référence (signal de référence)
    (figure \ref{fig:imagexp}). L’interférence entre ces deux signaux 
    modifie le spectre de fréquence du signal mesuré. L’information 
    sur l’état de polarisation se retrouve dans les 
    modifications de ce spectre. Cependant 
    la résolution de l’interféromètre doit être suffisante pour 
    distinguer les variations de fréquence causées par l’interaction 
    faible. Car que notre laser a une faible longueur de cohérence 
    (entre $2$ et $8$ $\mu m$) et qu'il y a plusieurs modes 
    fréquentiels \cite{ThorlabsNPL64B}, c'est difficile d'obtenir 
    une résolution suffisante pour observer directement les franges 
    d'interférence et distinguer les variations de fréquence causées par 
    l’interaction faible. Pour contourner ce problème, nous
    allons observer les variations de fréquence avec l'enveloppe
    de ce spectre, c'est-à-dire le spectre de puissance, qui est
    obtenu en effectuant une transformation de Fourier rapide (FFT)
    sur les données temporelles de l’interféromètre.

    \noindent Nous commençons par régler l’interféromètre de MZ de 
    manière à ce qu’il y ait une visibilité maximale avec le bras 
    correspondant à l'état de polarisation vertical $\ket{V}$ 
    dans la partie mesure faible du 
    dispositif, correspondant à l’absence de délai. Le 
    spectre de visibilité obtenu lors de l’alignement avec ce dernier 
    est présenté à la figure \ref{fig:vis}.

    \begin{figure}[!htpb]
        \centering
        \includegraphics[width=1.0\textwidth]{visibility_2.png}
        \caption{Spectre de visibilité de notre laser. }
        \label{fig:vis}
    \end{figure}

    \noindent Nous avons décidé d’aligner l’interaction faible avec le 
    troisième pique du spectre de visibilité de l’interféromètre. Cela 
    signifie que nous plaçons l’état de polarisation horizontale 
    ($\ket{H}$) sur ce pic, un déplacement du mirroir d'environ 
    $1.2$ $cm$ (allé-retour), de sorte que l’effet du décalage temporel 
    introduit soit maximal à cet endroit. Cette configuration nous permet 
    ensuite d’obtenir une information sensible sur les états de 
    polarisation circulaire, en particulier à partir des variations 
    observées sur le deuxième pic du spectre. Cependant, le spectre de 
    visibilité seul ne possède pas la résolution nécessaire pour détecter 
    les petits décalages fréquentiels causés par l’interaction faible. 
    Nous revenons donc dans le chapitre suivant sur une analyse spectrale 
    plus poussée.

    \noindent Ensuite, on superpose les impulsions sous le même état de 
    polarisation. Comme mentionné précédemment, le polariseur est réglé 
    sur un état de polarisation verticale pendant notre mesure projective. 
    À l'aide d'une lame demi-onde, nous pouvons autoriser des 
    interférences avec le bras non faiblement mesuré dans le MZ et 
    l’utiliser comme interférence de référence pour identifier quels 
    pics sont causés par des interférences. 

    \noindent Les données sont prélevées en mode FFT (transformation de 
    Fourier rapide) de l’oscilloscope sur plusieurs impulsions pendant 
    $400$ $ns$. On les moyenne ensuite à partir de $10$ séries de mesures. 
    Comme nous ne pourrons pas caractériser la partie imaginaire de la 
    valeur faible avec autant d'états d'entrée, comme la partie réelle, 
    nous allons vérifier si nous pouvons mesurer l’écart maximal attendu 
    entre des polarisations linéaires et circulaires qui est attendu par 
    la relation: $\expval{\hat{\omega}} \propto \frac{\tau}{8\sigma^2}$. 
    Les variations du spectre de puissance entre les états de polarisation 
    linéaires et circulaires, où un état de polarisation circulaire 
    devrait entraîner le décalage fréquentiel maximal par rapport à un 
    état de polarisation linéaire, devraient suivre cette relation. 
    Cependant, nous avons rencontré des difficultés qui seront abordées 
    dans le chapitre suivant. 

\end{doublespace}
    \subsubsection{Résultats attendus théoriques de la partie imaginaire de la valeur faible}
    
\begin{doublespace}
    
    Nous allons maintenant écrire les résultats attendus pour 
    la partie imaginaire de la valeur faible, comme nous l'avons fait avec 
    la partie réelle dans la section précédente, pour chaque trajet
    de polarisation. La relation de la partie imaginaire de la valeur faible
    est donnée par l'équation \ref{eq:imaginary_part}. Voici les 
    trajets de polarisation correspondant aux
    états d’entrée que nous avons utilisés pour la
    partie imaginaire de la valeur faible. 
    
    \noindent Nous commençons par l'état de polarisation linéaire
    $\ket{\psi_{i}^{1}} = cos(2\theta)\ket{H} + sin(2\theta)\ket{V}$
    qui est préparé par la lame demi-onde à $\theta$ et plaçons
    ses amplitudes de probabilité dans l'équation
    \ref{eq:imaginary_part} avec une postsélection sur l'état de polarisation
    $\ket{D}$, nous obtenons: 

    \begin{equation}
        \expval{\hat{\omega}} = 0
    \end{equation}

    \noindent ce qui signifie que l'état de polarisation linéaire ne
    devrait pas produire de décalage fréquentiel. Cela est
    attendu, car l'état de polarisation linéaire ne possède pas de 
    différence de phase dans son transport, contrairement aux trajets 
    suivants. Ensuite, pour un état de polarisation elliptique
    $\ket{\psi_{i}^{2}} = e^{\frac{-i\pi}{4}}cos(2\theta)\ket{H} + ie^{\frac{-i\pi}{4}} sin(2\theta)\ket{V}$
    préparé par la lame demi-onde à $\theta$ et une lame quart d'onde 
    à $0$ degré, nous avons:

    \begin{equation}
        \expval{\hat{\omega}} = \frac{\tau}{8\sigma^2}sin(4\theta)\label{eq:expval_omega_circular}
    \end{equation}

    \noindent ce type d'état de polarisation devrait produire un
    décalage fréquentiel proportionnel à l'état d'entrée et à la
    durée de l'interaction faible $\tau$. Enfin, pour l'état de polarisation
    de superposition 
    $\ket{\psi_{i}^{3}} = \frac{1}{2}(((1-i)sin(2\theta)+(i+1)cos(2\theta))\ket{H}
    + ((1+i)sin(2\theta)+(i-1)cos(2\theta))\ket{V})$
    préparé par la lame demi-onde à $\theta$ et une lame quart d'onde à $45$ degrés,
    nous avons:

    \begin{equation}
        \expval{\hat{\omega}} = -\frac{\tau}{8\sigma^2}sin(4\theta)
    \end{equation}

    \noindent ce qui signifie que l'état de polarisation de superposition
    devrait également produire un décalage fréquentiel proportionnel à l'état d'entrée
    et à la durée de l'interaction faible $\tau$.

    \noindent Cela résume nos approches et attendus expérimentaux
    menées dans le cadre de ce projet de maîtrise. Le chapitre suivant
    présentera nos méthodes d'analyse, nos résultats et les implications
    de ce projet.
\end{doublespace}
    
    \pagebreak

    \thispagestyle{empty}
    \section{RÉSULTATS ET DISCUSSION}    
    \begin{doublespace}
    L'objectif de ce chapitre est de démontrer que nous avons
    effectivement mesuré la partie réelle de la valeur faible et 
    confirmer l'existence de la partie imaginaire de la valeur faible,
    ainsi que la possibilité d'effectuer une
    caractérisation complète de l'état quantique à l'aide des délais 
    temporels comme pointeur. Nous allons commencer 
    par présenter les résultats expérimentaux
    obtenus pour la partie réelle de la valeur faible et les
    évaluer à l'aide de nos attentes théoriques.
    Pour la partie imaginaire de la
    valeur faible, nous allons discuter de la façon dont nous avons
    mesuré les décalages fréquentiels démonstrant que cette
    partie de la valeur faible peux être mesurée
    expérimentalement dans le future. Dans cette
    thèse, nous confirmerons l'existence de la partie imaginaire
    de la valeur faible en mesurant les décalages fréquentiels
    induit par notre interaction faible dans les centaines de
    kilohertz. Nous allons également aborder les défis
    rencontrés lors de la mesure de la partie imaginaire de la
    valeur faible et comment nous avons surmonté certains de ces défis 
    pour observer des décalages fréquentiels.
\end{doublespace}
    %\subsection{Interprétation des données expérimentaux}
    \subsection{Analyse des résultats expérimentaux pour la partie réelle de la valeur faible}
    \begin{doublespace}
    Dans cette section, nous allons présenter les résultats expérimentaux
    obtenus pour la partie réelle de la valeur faible.
    Nous avons procédé à la mesure des délais temporels entre les
    états de polarisation d'entrée en utilisant la même ajustement
    paramétrique que celle effectuée dans les expériences de la vitesse
    de la lumière/signal électrique au chapitre précédent.
    Cependant, pour obtenir 
    des délais, il faut d'abord calibrer l'expérience. Pour ce faire, 
    nous mesurons le temps d'arrivé du signal expérimental sans introduire
    un délai temporel sur le dispositif expérimental ainsi que le temps 
    d'arrivé du signal avec le délai maximal
    introduit. 
    Pendant cette phase de calibration, notre dispositif 
    expérimental consistant en une seule lame 
    demi-onde à l'étape de préparation de l'état d'entrée (voir figure  
    \ref{fig:realexp}). Cette lame est ensuite tournée pour envoyer la 
    polarisation $\ket{V}$, qui est notre état d’entrée possédant l'absence 
    de délai $\tau = 0$. L'oscilloscope prend des données pour dix 
    acquisitions de données différentes pour cet état d’entrée,
    puis nous faisons tourner la lame demi-onde pour envoyer le délai
    maximal $\tau$, soit l’état de polarisation $\ket{H}$. Ces deux mesures 
    sont effectuées séparément, puis nous en calculons la moyenne à partir
    de dix acquisitions de données différentes. Cette calibration
    nous permet de déterminer les délais temporels pour chaque état de
    polarisation d'entrée. Après la calibration, nous pouvons exécuter chaque 
    trajet de polarisation mentionné dans le chapitre précédent. Pour 
    changer de trajet de polarisation, nous devons ajouter une lame 
    quart d’onde à la préparation de l’état d’entrée, en fonction du 
    trajet que nous caractérisons.
    
    %Comme nous l’avons 
    %vu au chapitre 3, notre méthode expérimentale consiste à préparer des 
    %états de polarisation selon les trajectoires de polarisation mesurées. 
    %Ils subiront ensuite une interaction faible, puis ils seront projetés 
    %par un état de polarisation diagonale. 

\end{doublespace}
    \subsubsection{La partie réelle}
    \begin{doublespace}
    %Cette dernière est ensuite tournée pour envoyer la 
    %polarisation $\ket{V}$ correspondant à notre position temporelle 
    %moyenne de référence, qui est notre état d’entrée possédant un délai 
    %de $0$. L’oscilloscope prend des données, puis notre code fait 
    %tourner la lame demi-onde pour envoyer le délai maximal, soit 
    %l’état de polarisation $\ket{H}$. 
    
    
    %L’oscilloscope est évidemment utilisé pour mesurer les temps 
    %d’équivalence et il calcule ensuite la moyenne de plus de $10000$ 
    %formes d’onde. 

    Nous avons optimisé notre expérience pour qu’elle soit 
    automatisée avec un code \textsc{Python} qui contrôle le moteur pas à pas
    la lame demi-onde à l'étape d'entrée 
    \cite{pylablib_thorlabs_kinesis}. Nous collectons des données tous les 
    $2,5$ dégrés. Pour chaque état d’entrée de polarisation, nous 
    prenons cinq fichiers distincts que nous moyennons, puis nous 
    comparons leur position temporelle moyenne avec le dossier de 
    calibration pour obtenir le délai associé à cet état.
    Voici les 
    résultats de chaque délai pour les différents états d’entrée pour 
    le trajet $\ket{H}\rightarrow\ket{D}\rightarrow\ket{V}\rightarrow\ket{A}$ 
    (figure \ref{fig:HDAV_delay_results}).
\end{doublespace}

\begin{figure}[!h!t!p!b]
    \centering
    \includegraphics[width=1.0\textwidth]{real_weak_value_measurement_3.pdf}
    \caption{Résultats expérimentaux pour la partie réelle de la valeur faible du trajet de polarisation $\ket{H}\rightarrow\ket{D}\rightarrow\ket{V}\rightarrow\ket{A}$. Les barres d’erreurs horizontales représentent la variance de la position temporelle de chaque fichier, et les barres d’erreurs verticales, l’erreur de la position de l’angle induite par notre stade motorisé. L’erreur est trop petite pour être prise en considération, alors nous l’avons négligée. }
    \label{fig:HDAV_delay_results}
\end{figure}

\begin{doublespace}
    \noindent Cette série de données a été collectée avec une interaction faible correspondant à un délai de $167$ $ps$, obtenu en allongeant le bras $\ket{H}$ de $2,5$ $cm$ par rapport au bras $\ket{V}$. En utilisant les mêmes méthodes d’ajustement que celles employées dans l’expérience sur la vitesse de la lumière, nous mesurons les délais de chaque état $\psi(\theta)$. Nous redéfinissons l’angle $\theta$ pour qu’il ait une forme similaire à un plan circulaire, et nous combinons les résultats pour la partie réelle de la valeur faible correspondant à nos valeurs théoriques calculées dans le chapitre précédent. En réalité, la courbe théorique de cette partie réelle est une valeur cosinus carrée en ignorant la partie $2sin(theta)cos(theta)$ de notre résultat théorique, car elle correspond mieux à nos résultats expérimentaux. Cette partie $2sin(theta)cos(theta)$ n’a pas de signification physique, car, lorsque nous passons de $\ket{V}$ à $\ket{H}$, nous devrions éviter tout délai négatif, ce que nous constatons effectivement et pour garder la symmétrie positive de la mesure. Une erreur additionnelle à noter est l’incohérence de nos modifications physiques dans notre expérience, que nous ne pouvons pas compenser numériquement. Par exemple, les vibrations de la table et les variations d’alignement du laser sont des facteurs que nous ne pouvons pas prendre en compte pendant les longues périodes d’acquisition de données pour chaque état d’entrée. Cette erreur se trouve dans le trajet à venir. Voici les résultats pour le trajet $\ket{H} \rightarrow \ket{R} \rightarrow \ket{V} \rightarrow \ket{L}$, figure \ref{fig:HRAL_delay_results}.
\end{doublespace}

\begin{figure}[!h!t!p!b]
    \centering
    \includegraphics[width=1.0\textwidth]{real_weak_value_measurement_4.pdf}
    \caption{Résultats expérimentaux pour la partie réelle de la valeur faible du trajet de polarisation $\ket{H} \rightarrow \ket{R} \rightarrow \ket{V} \rightarrow \ket{L}$}
    \label{fig:HRAL_delay_results}
\end{figure}

\begin{doublespace}
    \noindent Ces résultats sont similaires à ceux du trajet précédent, ce qui est logique, car les états de polarisation $\ket{R}$ et $\ket{L}$ possèdent toujours les mêmes états de base que $\ket{D}$ et $\ket{A}$ mais avec une composante imaginaire . On ne peut pas déterminer à partir de la partie réelle de la valeur faible si l’état initial est elliptique ou circulaire. Les informations sur le passage de phase ou d’ellipticité des états de polarisation se trouvent dans la partie imaginaire de la valeur faible. Ensuite, voici les résultats pour le trajet $\ket{D} \rightarrow \ket{R} \rightarrow \ket{A} \rightarrow \ket{L}$, figure \ref{fig:HDAV_delay_results}.
\end{doublespace}

\begin{figure}[!h!t!p!b]
    \centering
    \includegraphics[width=1.0\textwidth]{real_weak_value_measurement_5_2.pdf}
    \caption{Résultats expérimentaux pour la partie réelle de la valeur faible du trajet de polarisation $\ket{D} \rightarrow \ket{R} \rightarrow \ket{A} \rightarrow \ket{L}$}
    \label{fig:DRAL_delay_results}
\end{figure}

\begin{doublespace}
    \noindent Cela s’avère captivant, car cela démontre incontestablement que nous conservons toujours la même composante réelle de la faible valeur. Cette constatation met en évidence la symétrie quantique, qui correspond à la symétrie optique des états de polarisation D, A, L et R. Par conséquent, l’ensemble des données sur l’ellipticité entre ces polarisations est contenu dans la variable conjuguée du décalage temporel, c’est-à-dire le décalage de fréquence obtenu en mesurant la partie imaginaire de la valeur faible. En outre, à partir de ces données pour chaque trajet, nous pouvons directement calculer les amplitudes de probabilité de l’état quantique à partir de ces mesures de délai, comme nous l’avons mentionné dans la section 2.3. Voici $\ket{H} \to \ket{D} \to \ket{V} \to \ket{A}$, figure \ref{fig:HDVA_prob_amp}:
\end{doublespace}

\begin{figure}[!h!t!p!b]
    \centering
    \includegraphics[width=1.0\textwidth]{real_prof_amp_measurement_3.pdf}
    \caption{Résultats de l'amplitude de probabilité pour le trajet de polarisation $\ket{H} \rightarrow \ket{D} \rightarrow \ket{V} \rightarrow \ket{A}$. Les courbes théoriques provenant de nos calculs dans la section 2.3. Les barres d'erreur sont les incertitudes propagées à partir de notre partie réelle de la valeur faible via Monte-Carlo, car la propagation analytique fait exploser certaines incertitudes de façon irrégulières. }
    \label{fig:HDVA_prob_amp}
\end{figure}

\begin{doublespace}
    \noindent Ensuite $\ket{H}\rightarrow\ket{R}\rightarrow\ket{V}\rightarrow\ket{L}$, figure \ref{fig:HRVL_prob_amp}:
\end{doublespace}

\begin{figure}[!h!t!p!b]
    \centering
    \includegraphics[width=1.0\textwidth]{real_prof_amp_measurement_4.pdf}
    \caption{Résultats de l'amplitude de probabilité pour le trajet de polarisation $\ket{H} \rightarrow \ket{R} \rightarrow \ket{V} \rightarrow \ket{L}$}
    \label{fig:HRVL_prob_amp}
\end{figure}

\begin{doublespace}
    \noindent Finalement $\ket{D}\rightarrow\ket{R}\rightarrow\ket{A}\rightarrow\ket{L}$, figure \ref{fig:DRAL_prob_amp}:
\end{doublespace}

\begin{figure}[!h!t!p!b]
    \centering
    \includegraphics[width=1.0\textwidth]{real_prof_amp_measurement_5_2.pdf}
    \caption{Résultats de l'amplitude de probabilité pour le trajet de polarisation $\ket{D} \rightarrow \ket{R} \rightarrow \ket{A} \rightarrow \ket{L}$}
    \label{fig:DRAL_prob_amp}
\end{figure}
    \subsubsection{Discussion sur les résultats expérimentaux de la partie réelle}
    \begin{doublespace}
    Nous avons calculé les amplitudes de probabilité en appliquant les 
    principes fondamentaux étudiés dans notre chapitre consacré à la 
    théorie. Ces résultats correspondent à nos attentes théoriques 
    concernant la norme des amplitudes de probabilité réelles de chaque 
    trajet de polarisation. Cette concordance nous permet de conclure 
    que nous avons correctement mesuré la partie réelle de la valeur 
    faible et nous l'avons utilisée pour caractériser les amplitudes de 
    probabilité de nos états initiaux. En conclusion, nous sommes en mesure de caractériser les états de 
    polarisation d'entrée avec succès. Il y a cependant 
    l'information de la partie complexe de l'état quantique que nous devons mesurer. 
    La section suivante traitera de nos tentatives de mesure de la 
    partie imaginaire. 
\end{doublespace}
    \subsection{La partie imaginaire}
    \label{sec:imaginary_results}
\begin{doublespace}
    Notre méthode pour mesurer la partie imaginaire de la valeur faible 
    est assez simple. Nous faisons interférer dans un 
    interféromètre de MZ une impulsion faiblement 
    mesurée avec un signal d'interférence 
    correspondant au spectre du laser, voir la figure \ref{fig:imagexp}. 
    Ainsi, nous pouvons accéder 
    après interférences au spectre du signal qui aura été modifié 
    par la mesure faible. Cependant, 
    certaines implications comme 
    la faible résolution de la mesure et la faible longueur 
    de cohérence du laser rendent difficile la 
    mesure de la partie imaginaire de la valeur faible.
\end{doublespace}


\begin{doublespace}

     \noindent À partir de l'équation \ref{eq:imaginary_part}, le 
     déplacement fréquentiel maximal attendu pour la partie imaginaire de la valeur faible est donné par:
    $\expval{\hat{\omega}}_{max} = \frac{\tau}{8\sigma^2}$. 
    Cela signifie que, pour une impulsion avec une durée de 
    $10$ $ns$, et un délai de $167$ $ps$ appliqué sur le système, on 
    s'attend à un déplacement dans le spectre de fréquence de seulement 
    $3,98$ kHz (avec $\omega \equiv 2\pi f$). Cette valeur est 
    extrêmement petite par rapport à la fréquence de notre laser, qui se 
    situe dans l'ordre $THz$ avec plusieurs modes \cite{ThorlabsNPL64B}. 
    %Les mesures interférométriques 
    %régulières effectuées dans un laboratoire ne présentent qu’une 
    %résolution de l'ordre $MHz$, ce qui nécessite des bras 
    %d'interféromètres assez longs. 
    Donc, pour améliorer l'effet par l'interaction faible,
    il faut réduire la durée de l'impulsion. 
    Nous allons choisir une longueur d'impulsion de $4$ $ns$ (le minimum 
    que notre laser peut effectuer) pour un attendue de $16,61$ $kHz$.
    
    \subsubsection{Vérification de la partie imaginaire}
    
    Nous avons commencer par si nous pouvions observer
    des déplacements fréquentiels dans la DSP
    causés par la mesure faible. Pour ce faire, nous avons utilisé
    les états de polarisation $\ket{H}$, $\ket{V}$, $\ket{R}$,
    $\ket{A}$, $\ket{L}$ et $\ket{D}$ comme états d'entrée.
    Pour trouver des déplacements induits par la 
    mesure faible, nous calculons la densité spectrale du spectre de 
    puissance (DSP) sur un ensemble d'impulsions du signal expérimentale.  
    La fonction \textit{pspectrum} de \textsc{MATLAB} effectue une transformation de Fourier à haute résolution sur 
    plusieurs impulsions pour obtenir la DSP. La figure \ref{fig:spectre} 
    montre le spectre de puissance total.

    %Voici ce spectre que nous avons obtenu à partir de nos données 
    %expérimentales (figure \ref{fig:spectre}).

    \begin{figure}[!htpb]
        \centering
        \includegraphics[width=1.0\textwidth]{PSD copy.png}
        \caption{DSP totale résultant d'une mesure faible temporelle. Ici, les 
        lignes verticales représentent la position des pics de puissance 
        dans le spectre. La fréquence centrale est de 
        $703.12$ $MHz$ pour une durée spectrale de $1.4062$ $GHz$ avec une 
        résolution temporelle de $4$ $ps$.
        La figure \ref{fig:spectre1054} montre un zoom 
        sur le pic de puissance à $87,9$ $MHz$.}
        \label{fig:spectre}
    \end{figure}

    \begin{figure}[!htpb]
        \centering
        \includegraphics[width=1.0\textwidth]{zoom_at_87MHz.png}
        \caption{Zoom sur le pic de puissance à $87,9$ $MHz$. On peut 
        observer la présence de plusieurs pics de puissance, avec leur 
        ajustements de paramétriques polynomiales du quatrième ordre pour 
        determiner chaque position fréquentielle des états d'entrée.}
        \label{fig:spectre1054}
    \end{figure}

    \noindent Nous analysons ensuite le spectre pour repérer les pics
    avec une amplitude supérieure à $1\%$ de la puissance maximale. Pour 
    chacun de ces pics, nous effectuons un ajustement de paramétrique
    polynomiale du quatrième ordre, comme dans nos dernières analyses 
    pour la mesure de la vitesse d'un signal électrique et de la lumière
    dans la section 3.1, pour estimer précisément la position fréquentielle du pic. En 
    comparant chacune de ces fréquences à celles de l’état $\ket{V}$, 
    nous obtenons le tableau suivant pour le pic à la fréquence $87,9$ $MHz$:

    \begin{longtable}{p{3.0cm} p{3.0cm} p{3.0cm}}
    \caption{Tableau récapitulatif des pics de puissance dans le spectre de
    puissance, avec les états d'entrée, les fréquences mesurées et les
    déplacements fréquentiels. Les fréquences sont mesurées en
    mégahertz (MHz) et les déplacements fréquentiels en kilohertz (kHz) 
    par rapport à l'état $\ket{R}$.} \\
    \toprule
    \label{table:imaginare-table}
    État d'entrée & Fréquence (MHz) & Déplacement fréquentiel (kHz)\\
    \midrule
    \endfirsthead
    
    \toprule
    État d'entrée & Fréquence (MHz) & Déplacement fréquentiel (kHz)\\
    \midrule
    \endhead
    
    \midrule
    \multicolumn{3}{r}{{\dots}} \\
    \midrule
    \endfoot
    
    \bottomrule
    \endlastfoot
    
    $\ket{H}$ & 87.907 &  +1.000 \\
    $\ket{V}$ & 87.915 &  +9.000 \\
    $\ket{R}$ & 87.906 &   0.000 \\
    $\ket{A}$ & 87.910 &  +4.000 \\
    $\ket{L}$ & 87.902 &  -4.000 \\
    $\ket{D}$ & 87.916 & +10.000 \\

    %\midrule


    \end{longtable}

    \noindent Bien que la différence prévue soit de $16,61$ $kHz$ (du minimum au maximum), ce qui 
    n’a pas été strictement observé, l’analyse de Fourier révèle 
    effectivement un déplacement de certains pics dans le spectre, 
    mesuré dans l'ordre de kilohertz. Cependant, ces déplacements ne 
    semblent pas être cohérents ni suivre une tendance correspondant à 
    notre théorie, ainsi que ceux des autres pics dans la figure \ref{fig:spectre}.
    Nous pouvons attribuer ce problème à un manque de stabilité
    du laser, ce qui rend difficile la mesure de la partie 
    imaginaire de la valeur faible.

    \subsubsection{Tentatives de mesurer la partie imaginaire}

    \noindent Considérons une prise de données sur plusieurs
    état d'entrée, comme l'on fait dans notre expérience de la partie
    réelle de la valeur faible. Cette fois ici, nous avons
    pris la mesure de la DSP pour chaque état d'entrée et considéré
    comment le centroïde de chaque pic se déplace en fonction de l'état
    d'entrée d'écrit par:

    \begin{equation}
        f_{cent.} = \frac{\sum_{i=1}^{n} f_i \cdot P_i}{\sum_{i=1}^{n} P_i}
        \label{eq:centroid}
    \end{equation}
    \noindent où $f_{cent.}$ est la fréquence du centroïde, $n$ est le
    nombre de pics dans le spectre, $f_i$ est la fréquence de chaque pic 
    et $P_i$ est la puissance
    de chaque pic. Nous avons ensuite calculé la différence entre le
    centroïde de chaque état d'entrée et celui de l'état $\ket{L}$.
    Le centroïde du spectre est une mesure de la position moyenne des pics
    de puissance dans le spectre, ce qui nous permet de mesurer les
    déplacements fréquentiels causés par la mesure faible, 
    cette méthode prend inspiration de: \cite{WeakorStd,OpticalNetworks}.
    Nous évaluons la partie imaginaire de la valeur faible de les 
    trajets de polarisation discutés dans le chapitre 3. Voici les
    résultats que nous avons obtenus pour les déplacements fréquentiels
    pour le trajet de polarisation $\ket{H} \to \ket{R} \to \ket{V} \to \ket{L} \to \ket{H}$
    présenté dans la figure \ref{fig:imaginary_part_HRVLH}.

    \begin{figure}[!htpb]
        \centering
        \includegraphics[width=1.0\textwidth]{imag_weak_value_path_4 copy 2.png}
        \caption{Déplacement fréquentiel mesuré pour les trajets de polarisation 
        $\ket{H} \to \ket{R} \to \ket{V} \to \ket{L} \to \ket{H}$, 
        avec un délai de $167$ $ps$ et une durée d'impulsion de $4$ $ns$. 
        Les barres d'erreur verticale représentent l'incertitude de la mesure,
        calculée à partir de la variations des fréquences
        mesurées pour chaque état d'entrée. Les déplacements sont mesurés en
        kilohertz (kHz) par rapport à l'état $\ket{V}$. La courbe théorique
        n'est pas représentée car les déplacements ne correspondent pas
        avec nos attentes théoriques.}
        \label{fig:imaginary_part_HRVLH}
    \end{figure}

    \noindent Ces données on été obtenues en prenant les résultats de la DSP
    pour chaque état d'entrée, en calculant le centroïde de chaque pic
    en utilisant l'équation \ref{eq:centroid} et le comparer à 
    fréquence du centroïde de l'état $\ket{V}$ pour
    calculer la partie imaginaire de la valeur faible. Nous avons 
    observé que les déplacements fréquentiels sont positives et c'est 
    dû à la nature de l'analyse. Ces résultats nous indiquent que nous sommes entrains 
    d'observer la valeur absolue de la partie imaginaire de la valeur faible,
    représentée par avec la courbe théorique dans la figure \ref{fig:imaginary_part_HRVLH}.
    Cependant, les déplacements ne correspondent pas à nos attentes théoriques,
    car la courbe théorique est bien sinusoïdale, mais les déplacements
    mesurés suivent une tendance de $sin(1.75\theta)$ ce qui est différent
    à celui attendu dans l'équation \ref{eq:expval_omega_circular}.
    Néanmoins, ces déplacements dans la DSP indiquent que la
    partie imaginaire de la valeur faible est présente et elle suit
    une tendance sinusoïdale et permet l'état soit caractérisable à partir de
    ces résultats. Nous pouvons effectivement
    calculé les amplitudes de probabilité (voir figure \ref{fig:prob_amp_imaginary_part_HRVLH})
    pour chaque état d'entrée en utilisant les relations suivantes:

    \begin{align}
        |a| &= \sqrt{\mathcal{I}(\expval{\hat{\pi}_W})} \label{eq:amplitude_a_im}\\
        |b| &= \sqrt{1-|a|^2} \label{eq:amplitude_b_im}
    \end{align}

    \begin{figure}[!htpb]
        \centering
        \includegraphics[width=1.0\textwidth]{imaginary_probability_path_4.png}
        \caption{
        La partie imaginaire des amplitudes de probabilité mesurées pour le trajet de polarisation
        $\ket{H} \to \ket{R} \to \ket{V} \to \ket{L} \to \ket{H}$.
        Les barres d'erreur sont propagées à partir de l'incertitude
        de la mesure des déplacements fréquentiels de la partie imaginaire de ce trajet de polarisation.
        }
        \label{fig:prob_amp_imaginary_part_HRVLH}
    \end{figure}

    \noindent Ces résultats démontrent que la partie imaginaire de la valeur faible
    est effectivement mesurable, mais que les déplacements ne correspondent pas
    avec nos attentes théoriques. Cependant, nous avons observé des
    déplacements dans le spectre de fréquence, ce qui indique que la
    partie imaginaire de la valeur faible est présente et elle suit
    une tendance sinusoïdale. À ce moment présent, les résultats présentés
    sont nos meilleures
    résultats pour la partie imaginaire de la valeur faible et que les autres
    tentatives de mesure n'ont pas été concluantes.
    
    \subsubsection{Implications pour la partie imaginaire}

    Les résultats des décalages fréquentiels mesurés présentent un 
    comportement imprévisible, signe d’une instabilité dans notre 
    source laser. De plus, la longueur des bras de l’interféromètre 
    de MZ rend l’appareil sensible aux vibrations et aux 
    fluctuations environnementales: la moindre fluctuation de 
    phase dans un bras décale le profil du spectre de puissance, 
    aggravant ainsi l’instabilité observée. Ces facteurs combinés 
    peuvent affecter la précision des mesures.
    On entend par effet fréquentiel un déplacement dans le spectre 
    de fréquence causé par la mesure faible, déplacement qui se 
    manifeste dans la DSP. Si nos données révèlent bien un décalage 
    du centroïde spectral, nous n’avons pas pu mesurer la partie 
    imaginaire de la valeur faible: les déplacements restent trop 
    faibles et incohérents pour coïncider avec les prédictions 
    théoriques.
    Pour assurer une meilleure cohérence, il est nécessaire de 
    stabiliser ou de modifier la source laser. Des études futures 
    examineront l’intégration d’une source impulsionnelle 
    cohérente; dans ces conditions, la partie imaginaire devrait 
    devenir mesurable, permettant une caractérisation complète.
    

\end{doublespace}



    \pagebreak

    \thispagestyle{empty}
    \section{CONCLUSION}
    \begin{doublespace}

    %Nous allons conclure cette thèse en résumant ces
    %résultats expérimentaux obtenus et en discutant de l'importance 
    %sur les applications potentielles des mesures faibles temporelles
    %et des perspectives futures pour la mesure de la partie
    %imaginaire de la valeur faible.
    Nous allons conclure cette thèse avec des mots finales sur
    notre projet de recherche et des travaux futurs que nous pouvons 
    envisager pour la mesure de la partie imaginaire de la valeur faible.

    \subsection{Conclusion sur la thèse}
    
    Nous avons maintenant démontré que nous pouvons 
    caractériser complètement la partie réelle de la valeur faible, 
    tout en montrant la faisabilité de la mesure de la partie imaginaire
    de la valeur faible. Utiliser le domaine temporel comme 
    pointeur nous permet de caractériser la polarisation d'un système 
    photonique quantique, ouvrant ainsi la voie à des applications 
    intégrables dans les technologies quantiques. Grâce à notre 
    méthodologie interférométrique des mesures faibles, il est possible 
    d’envisager des applications dans des technologies telles que les 
    systèmes de télécommunication photonique quantique 
    \cite{OpticalNetworks,Brunner_2004}, l’informatique quantique 
    \cite{quantuminfoweak}, la cryptographie quantique 
    \cite{troupe2017quantumcryptographyweakmeasurements}, la métrologie 
    quantique et l'avancement sur nos mesures quantiques précises
    \cite{intlundeen}. 
    
    \subsection{Applications et projets futurs}

    Cette approche pour la caractérisation des états
    quantiques ouvre la voie à de nombreuses applications dans les
    technologies quantiques. En intégrant notre dispositif experimental
    dans les systèmes de télécommunication photonique quantique, il
    devient possible de caractériser les états de polarisation
    des photons de manière efficace et précise. 
    La mesure faible permet une intéraction minimale avec le
    système quantique, dont quand on postsélectionne avec un état connu,
    nous pouvons caractériser directement l'état quantique du message
    transmis en observant les déplacements temporelles et fréquentiels
    des photons à l'aide de notre pointeur couplé \cite{troupe2017quantumcryptographyweakmeasurements}. De plus,
    en utilisant cette méthode pour caractériser les états quantiques
    dans les ordinateurs quantiques, il devient possible de détecter et
    de corriger les erreurs de manière plus efficace en suivant
    l'évolution temporelle des états quantiques, ce qui est crucial
    pour le développement de systèmes quantiques robustes et fiables.
    En outre, en intégrant cette approche dans les systèmes de
    métrologie quantique, il devient possible de mesurer des
    grandeurs physiques avec une précision sans précédent, ouvrant la
    voie à de nouvelles découvertes.

    \noindent Nos futurs travaux consisteront à caractériser des 
    polarisations élliptiques, à l’aide des décalages fréquentiels 
    provoqués par notre décalage temporel du pointeur. Pour ce faire, 
    nous pourrions concevoir un système photonique sophistiqué capable 
    de mesurer de petits décalages fréquentiels. Nous pouvons envisager 
    une expérience où nous remplacerions la source laser pulsé par
    une source qui possède une longueur de cohérence plus longue qui
    nous permettrait de facilité la mesure de la partie imaginaire de
    la valeur faible. Nous pourrions utiliser un laser HeNe et un 
    modulateur acousto-optique pour générer des impulsions de
    lumière cohérente. Comme les impulsions proviennent d'un
    laser ayant une grande longueur de cohérence, le spectre
    d'interférence sera plus clair et nous permettra de mesurer
    la variation des décalages fréquentiels du signal en fonction
    des états de polarisation d'entrée et, ainsi, de mesurer
    avec succès la partie imaginaire de la valeur faible
    dans un système photonique quantique.

    \noindent En conclusion, nous avons démontré que les mesures
    faibles temporelles peuvent être utilisées pour la caractérisation
    des états quantiques et que cette approche est intégrable dans les
    technologies quantiques \cite{OpticalNetworks,Brunner_2004,quantuminfoweak,troupe2017quantumcryptographyweakmeasurements,intlundeen}. 
    Nous pouvons enfin conclure cette thèse et 
    laissent la porte ouverte à des
    futures perspectives en matière de mesures quantiques pour des
    applications technologiques.

\end{doublespace}
    \pagebreak

    \printbibliography	

    \pagebreak

    \thispagestyle{empty}

    \pagebreak

    \appendix
    \section*{ANNEXE A: Validation de la précision fréquencielle par l'effet Doppler}
    \addcontentsline{toc}{section}{\protect\numberline{} ANNEXE A}
    \begin{doublespace}
    Pour l’illustrer, considérons le son d’une sirène:
    à l’approche, le bruit semble plus aigu, alors qu'au départ, 
    le bruit semble grave. Cet effet est 
    causé par l’effet Doppler, qui fait varier 
    la fréquence selon la vitesse de la source 
    en relation de l'observateur ou récepteur. 
    Ce principe s’applique à toutes les ondes. 
    Pour les ondes électromagnétiques, on parle 
    de décalage vers le rouge lorsque la 
    fréquence diminue et de décalage vers le 
    bleu quand elle augmente. On observe souvent 
    ce phénomène dans des phénomènes 
    astronomiques, comme quand la lumière 
    traverse un espace-temps courbé ou compressé.  
    L'effet Doppler est décrit par les relations 
    physiques suivantes: 
\end{doublespace}

\begin{equation}
    \Delta f = \frac{\Delta v}{c}f_0
\end{equation}

\begin{doublespace}
    \noindent Le changement de fréquence est directement 
    proportionnel à la différence de vitesse 
    $\Delta v \equiv -(v_r - v_s)$ entre la 
    vitesse de la source de l'onde $v_s$ et la vitesse du 
    récepteur ou observateur $v_r$. Soit $c$ la vitesse de la 
    lumière et $f_0$ la fréquence initiale de l'onde. Donc, 
    la fréquence mesurée de l'observateur sera:

    \begin{equation}
        f = \Bigl( 1 + \frac{\Delta v}{c} \Bigr)f_0
    \end{equation}

    \noindent Nous savons qu’en mesurant un changement 
    spectral, le laser a souvent une fréquence 
    initiale dans les ordres des $THz$. Donc, il 
    semble difficile de mesurer les décalages 
    fréquentiels, si les décalages sont dans les 
    ordres des $kHz$ ou même dans les $Hz$. Ceci 
    arrive lors d’une mesure à faible temporel. 
    Le décalage fréquentiel d’une mesure faible 
    temporel se décrit par la partie imaginaire 
    de la valeur faible:

    \begin{equation}
        \expval{\hat{\omega}} \equiv \frac{\tau}{8\sigma^2}sin(2\theta)
    \end{equation}

    \noindent Cela signifie que pour une impulsion possédant une 
    taille $\sigma$ de $4$ à $10$ $ns$ du profil temporel, le 
    délai $\tau$ utilisé pour une mesure faible est 
    souvent d’environ $10 \%$ de sa taille, donc 
    au moins $0,4$ à $1$ $ns$. Si on veut les 
    meilleurs résultats possibles pour une 
    mesure faible, il faut un délai plus petit 
    que cela. Le délai effectué pour les résultats 
    obtenus pour la partie réelle de la valeur 
    faible à date est dans l’ordre des 
    picosecondes, soit $167$ à $210$ $ps$. Donc, ces 
    derniers nous donnent des décalages 
    fréquentiels de $3$ $kHz$ pour une taille de 
    $10$ $ns$ et une mesure faible de $167$ $ps$ avec 
    une résolution de $5$ degrés dans les parties 
    les plus petites du sinusoïdal $sin(2\theta)$ 
    (Le $2\theta$ représente une lame demi-onde dont $\theta^{\prime} \equiv 2\theta$ 
    représente l'angle actuel de la lame d'onde). 
    Toutefois, il serait intéressant de 
    quantifier ces décalages fréquentiels, comme 
    le ferait un radar Doppler. Nous mesurons 
    les décalages fréquentiels par rapport à un 
    délai temporel induit par notre mesure 
    faible. Cependant, nous devons savoir si 
    nous sommes capables de mesurer des 
    décalages fréquentiels aussi petits. 
    L'expérience suivante sert à évaluer notre 
    capacité à quantifier des décalages 
    fréquentiels. Nous allons tenter de mesurer 
    le décalage fréquentiel dans un 
    interféromètre induit par l'effet Doppler. 
    Cette expérience repose sur la technique 
    d’interférométrie hétérodyne, où un laser 
    émet à une fréquence de départ, soit $f_0$. 
    Le miroir bouge à une vitesse $v$; ensuite, 
    nous interférons avec la nouvelle fréquence 
    et celle initiale pour mesurer le 
    décalage.
\end{doublespace}

\begin{doublespace}
    \noindent L’expérience consiste en un laser pulsé dont son signal subit une 
    séparation de ses bases de polarisation par 
    une séparatrice de faisceau polarisant (PBS). 
    L’une d’elles, soit celui horizontal, subit 
    un effet Doppler par un miroir de transition 
    motorisé à la place d'un miroir stationaire (figure \ref{fig:imagexp}) tandis que l’autre sert à notre 
    fréquence de référence (fréquence initiale 
    (la source)). Les deux bras sont ensuite 
    recombinés avec un autre PBS et analysés à l’aide d’un oscilloscope. 
    L’oscilloscope utilise un mode d’acquisition 
    de données à haute résolution, ce qui entraîne 
    une diminution de la réponse, qui passe de 
    $1$ en mode continu à $0,63$
    (fréquence égale à la moitié de la fréquence 
    d’échantillonnage) dans son mode. Ensuite, nous appliquons 
    le mode MATH avec l’option spectrale 
    magnitude pour observer le spectre du signal. 
    Ce dernier applique une transformation de 
    Fourier rapide sur le signal pour passer du 
    profil temporel au profil spectral. Lors de 
    l'expérience, nous déplacons le miroir de 
    transition à l'aide d'un code Python à une 
    vitesse de $0,27$ $cm/s$ ($30$ \% de 
    la vitesse maximale du miroir motorisé) sur une distance de 
    $0,585$ $cm$. Calculons théoriquement que sera 
    la valeur dont la fréquence initiale de $467.33$ $THz$ sera 
    décalée. 
\end{doublespace}

\begin{align}
    \Delta f &= \frac{2\Delta v}{c}f_0\\
    &= \frac{2*(0.0027 m/s)}{299792458 m/s}(467.33*10^{12} (1/s))\\
    &= 8.418kHz
\end{align}

\begin{doublespace}
    \noindent Le facteur $2$ est utilisé pour compenser le 
    trajet aller-retour du faisceau. Sur le 
    graphique obtenu à l’aide de l’oscilloscope, 
    on devrait voir apparaitre un pic à $8.4$ $kHz$ 
    causé par l’effet Doppler. Dès que notre 
    miroir de transition a atteint sa vitesse 
    maximale, l’oscilloscope capture une mesure 
    du signal. 
    
    \noindent Voici le résultat obtenu au laboratoire
    lors de l’expérience (figure \ref{fig:doppler_res}).
\end{doublespace}


\begin{figure}[hp]
    \centering
    \includegraphics[width=1.0\textwidth]{figure/entire_spec.png}
    \caption{Ensemble du spectre obtenu de l'oscilloscope pendant l'effet Doppler}
    \label{fig:doppler_res}
\end{figure}
\begin{doublespace}
    
    \noindent La figure \ref{fig:doppler_res} illustre l’ensemble des données 
    recueillies en laboratoire lors de la 
    détermination de l’effet Doppler. Comme nous 
    l’avons souligné auparavant, un pic devrait 
    apparaitre autour de $\sim$ $8$ $kHz$. Notre laser 
    fonctionnant à un taux de répétition de 
    $1$ $MHz$. Prenons ce premier pic de fréquence 
    comme point de repère. Voici un agrandissement 
    de cette zone dans la figure \ref{fig:doppler_res_zoom}. 
\end{doublespace}

\begin{figure}[h]
    \centering
    \includegraphics[width=1.0\textwidth]{figure/zoom_doppler_res_scatter.png}
    \caption{Agrandissement du spectre à 1MHz de la figure 
    \ref{fig:doppler_res}}
    \label{fig:doppler_res_zoom}
\end{figure}

\begin{doublespace}
    
    \noindent D'après notre analyse de donnée, nous 
    déterminons que le décalage fréquentiel causé 
    par l’effet Doppler est estimé à $8,7535$ $kHz$. À noter 
    que les pics avant et après le pic central 
    étaient absents avant de bouger le miroir de 
    transition. Ces résultats sont très 
    encourageants, car ils suggèrent qu’il est 
    théoriquement possible de quantifier le 
    décalage fréquentiel causé par une mesure 
    temporelle.

\end{doublespace}


    \pagebreak
    \thispagestyle{empty}

\end{document}