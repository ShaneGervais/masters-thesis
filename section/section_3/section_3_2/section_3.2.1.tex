\begin{doublespace}
    Notre dispositif expérimental, figure \ref{fig:realexp}, est composé de notre laser pulsé de tout à l'heure entrant dans une demi-plaque d'onde principalement utilisée pour réguler l’intensité. En effet, l’entrée externe de l'oscilloscope a besoin d'une intensité assez élevée pour se déclencher sur le signal de référence. 
\end{doublespace}

\begin{figure}[!htpb]
    \centering
    \includegraphics[width=1.0\textwidth]{partiereelexp.png}
    \caption{Dispositif expérimental pour la partie réelle de la valeur faible}
    \label{fig:realexp}
\end{figure}

\begin{doublespace}
    \noindent Une fois de plus, nous souhaitons déclencher le signal de référence de manière externe, car nous voulons bénéficier de la résolution temporelle maximale offerte par l’oscilloscope, qui possède une fréquence d’échantillonnage de $500$ $GS/s$, pour détecter les délais. Un séparateur de faisceau polarisant divise le faisceau laser en deux voies : celui réfléchi sert de signal de référence pour le déclenchement de l’oscilloscope, tandis que celui transmit sera préparé dans divers trajets de polarisation sur la sphère Poincarré et subira une mesure faible pour une caractérisation. Les trajets de polarisation testés sont les suivants et sont obtenus en changeant les plaques d'onde lors de l'étape de préparation. La premiers consiste seulement d'une lame demi-onde définie par l’opérateur suivant:
\end{doublespace}

\begin{align}
    \hat{T}_{HWP}(\theta) = \begin{pmatrix}
        cos(2\theta) & sin(2\theta)\\
        sin(2\theta) & -cos(2\theta)
    \end{pmatrix}
\end{align}

\begin{doublespace}
    \noindent Où l'indice $HWP$ fait référence à une demi-plaque d'onde pour un angle $\theta$ (\guillemetleft halfwaveplat \guillemetright en anglais). Ce premier trajet consiste à passer d’un état de base à un autre sans polarisation circulaire, de $H \to D \to V \to A \to \dots$, et ainsi de suite. Pour comprendre en détail, l'état commence dans l’état horizontal $\ket{H} \equiv \begin{pmatrix}
        1\\
        0
    \end{pmatrix}$ défini par la transmission d'un séparateur de faisceau polarisant. Ensuite, l'état évolue dans cette façon suivante en fonction de l'angle de la plaque d'onde $\theta$:
\end{doublespace}

\begin{align}
    \hat{T}_{HWP}(\theta)\ket{H} &= \begin{pmatrix}
        cos(2\theta) & sin(2\theta)\\
        sin(2\theta) & -cos(2\theta)
    \end{pmatrix} \begin{pmatrix}
        1\\0
    \end{pmatrix}\\
    &= \begin{pmatrix}
        cos(2\theta)\\sin(2\theta)
    \end{pmatrix}
\end{align}

\begin{doublespace}
    \noindent Donc, l'état d'entrée préparé est soit:
\end{doublespace}

\begin{equation}
    \ket{\psi_{i}^{1}} = cos(2\theta)\ket{H} + sin(2\theta)\ket{V}
\end{equation}

\begin{doublespace}
    \noindent En fonction des paramètres de Stokes pour démontrer le trajet sur la sphère Poincarré, figure \ref{fig:path3sphere}:
\end{doublespace}

\begin{equation}
    S = \begin{pmatrix}
        S_0 = |a|^2 + |b^2|\\
        S_1 = |a|^2 - |b|^2\\
        S_2 = 2\mathcal{R}(\bar{a}b)\\
        S_3 = 2\mathcal{I}(\bar{a}b)
    \end{pmatrix} = \begin{pmatrix}
        1\\
        cos^{2}(2\theta) - sin^{2}(2\theta) = cos(4\theta)\\
        2cos(2\theta)sin(2\theta) = sin(4\theta)\\
        0
    \end{pmatrix}
\end{equation}

\begin{figure}[!htpb]
    \centering
    \includegraphics[width=1.0\textwidth]{poincare_sphere_HDVAH.png}
    \caption{Schéma du trajet $\ket{H}\rightarrow\ket{D}\rightarrow\ket{V}\rightarrow\ket{A}\dots$
    utilisant seulement une lame demi-onde dans la préparation de l'état d'entrée}
    \label{fig:path3sphere}
\end{figure}

\begin{doublespace}
    \noindent Ce trajet est réalisé en tournant uniquement une lame demi-onde. On tourne l'angle de la plaque d'onde par pas de $2,5$ degrés. Chaque degré $\theta^\prime$ que nous tournons en réalité équivaut à tourner de $5$ degrés sur un plan circulaire $\theta^\prime \equiv 2\theta$ ou de $10$ degrés sur la sphère de Poincarré. 

    \noindent Le trajet suivant consiste à passer d'un état de base à un autre en passant par une polarisation circulaire $\ket{H} \to \ket{R} \to \ket{V} \to \ket{L} \dots$. Cela se fait avec une demi-plaque d'onde tournant de la même manière que précédemment, et un quart de plaque d'onde réglée à $0$ degré par rapport à $\ket{H}$. L’opération de cette plaque d’onde se définit par l'opérateur suivant:
\end{doublespace}

\begin{align}
    \hat{T}_{QWP}(\phi) &= \begin{pmatrix}
        cos^{2}(\phi) + isin^{2}(\phi) & (1-i)cos(\phi)sin(\phi)\\
        (1-i)cos(\phi)sin(\phi) & sin^{2}(\phi) + icos^{2}(\phi)
    \end{pmatrix}\\
    \hat{T}_{QWP}(\phi = 0^{\degree}) &= \begin{pmatrix}
        1 & 0 \\
        0 & i
    \end{pmatrix}
\end{align}

\begin{doublespace}
    \noindent La forme de cet opérateur $\hat{T}_{QWP}(\phi)$, avec l'angle $\phi$ pour la plaque d'onde et l'indice QWP signifiant quart de plaque d'onde ou \guillemetleft lame quart d'onde \guillemetright (\guillemetleft quarter waveplate \guillemetright en anglais), permet de conserver $a \in \mathcal{R}$ et de laisser $b\in \mathcal{C}$ contenir l'information complexe. Nous procédons ainsi pour que la partie imaginaire de la valeur faible soit principalement contenue dans $b$ pour des raisons de simplicité. Avec cette opération l'état évolue comme suit:
\end{doublespace}

\begin{align}
    \hat{T}_{QWP}(\phi = 0^{\degree})\hat{T}_{HWP}(\theta)\ket{H} &= 
    \begin{pmatrix} 
        1 & 0 \\
        0 & i
    \end{pmatrix} 
    \begin{pmatrix}
        cos(2\theta) & sin(2\theta)\\
        sin(2\theta) & -cos(2\theta)
    \end{pmatrix} \begin{pmatrix}
        1\\0
    \end{pmatrix}\\
    &= \begin{pmatrix}
        cos(2\theta)\\isin(2\theta)
    \end{pmatrix}
\end{align}

\begin{doublespace}
    \noindent Donc, l'état d'entrée est: 
\end{doublespace}

\begin{equation}
    \ket{\psi_{i}^{2}} = cos(2\theta)\ket{H} + isin(2\theta)\ket{V}
\end{equation}

\begin{doublespace}
    \noindent Avec les paramètres de Stokes pour démontrer sa trajectoire, figure \ref{fig:path4sphere}:
\end{doublespace}

\begin{equation}
    S = \begin{pmatrix}
        S_0 = |a|^2 + |b^2|\\
        S_1 = |a|^2 - |b|^2\\
        S_2 = 2\mathcal{R}(\bar{a}b)\\
        S_3 = 2\mathcal{I}(\bar{a}b)
    \end{pmatrix} = \begin{pmatrix}
        1\\
        cos^{2}(2\theta) - sin^{2}(2\theta) = cos(4\theta)\\
        0\\
        2cos(2\theta)sin(2\theta) = sin(4\theta)
    \end{pmatrix}
\end{equation}

\begin{figure}[!htpb]
    \centering
    \includegraphics[width=1.0\textwidth]{poincare_sphere_HRVLH.png}
    \caption{Schéma du trajet $\ket{H}\rightarrow\ket{R}\rightarrow\ket{V}\rightarrow\ket{L}\dots$
    utilisant seulement une lame demi-onde et une lame quart d'onde à 0 dégré dans la préparation de l'état d'entrée}
    \label{fig:path4sphere}
\end{figure}

\begin{doublespace}
    \noindent Le trajet final est un parcours captivant qui nous fait passer constamment entre deux états de base, soit d'une polarisation linéaire $\{\ket{D}, \ket{A}\}$ à une polarisation circulaire $\{ \ket{R}, \ket{L}\}$. La trajectoire résultante est $\ket{D} \to \ket{R} \to \ket{A} \ket{L} \dots$. Cette dernière est obtenue en tournant une demi-plaque d'onde avec un quart de plaque d'onde réglée à $45$ degrés par rapport à l'état de base $\ket{H}$, dont le quart de plaque d'onde à ce réglage est défini comme suit: 
\end{doublespace}

\begin{equation}
    \hat{T}_{QWP}(\phi = 45^{\degree}) = \frac{1}{2}\begin{pmatrix}
        1+i & 1-i \\
        1-i & 1+i
    \end{pmatrix}
\end{equation}

\begin{doublespace}
    \noindent Avec cette opération l'état évolue comme suit:
\end{doublespace}

\begin{align}
    \hat{T}_{QWP}(\phi = 45^{\degree})\hat{T}_{HWP}(\theta)\ket{H} &= 
    \begin{pmatrix} 
        \frac{1+i}{2} & \frac{1-i}{2} \\
        \frac{1-i}{2} & \frac{1+i}{2}
    \end{pmatrix} 
    \begin{pmatrix}
        cos(2\theta) & sin(2\theta)\\
        sin(2\theta) & -cos(2\theta)
    \end{pmatrix} \begin{pmatrix}
        1\\0
    \end{pmatrix}\\
    &= \frac{1}{2}\begin{pmatrix}
        cos(2\theta) + sin(2\theta) + i(cos(2\theta) - sin(2\theta))\\cos(2\theta) + sin(2\theta) -i(cos(2\theta)-sin(2\theta))
    \end{pmatrix}
\end{align}

\begin{doublespace}
    \noindent Donc, l'état d'entrée est: 
\end{doublespace}

\begin{equation}
    \ket{\psi_{i}^{3}} = \frac{1}{2}\Bigl(((1-i)sin(2\theta) + (1+i)cos(2\theta))\ket{H} + ((1+i)sin(2\theta) + (1-i)cos(2\theta))\ket{V}\Bigr)
\end{equation}

\begin{doublespace}
    \noindent Avec les paramètres de Stokes pour démontrer sa trajectoire, figure \ref{fig:path5sphere}:
\end{doublespace}

\begin{equation}
    S = \begin{pmatrix}
        S_0 = |a|^2 + |b^2|\\
        S_1 = |a|^2 - |b|^2\\
        S_2 = 2\mathcal{R}(\bar{a}b)\\
        S_3 = 2\mathcal{I}(\bar{a}b)
    \end{pmatrix} = \begin{pmatrix}
        1\\
        0\\
        sin(4\theta)\\
        sin^2(2\theta) - cos^2(2\theta)
    \end{pmatrix}
\end{equation}

\begin{figure}[!htpb]
    \centering
    \includegraphics[width=1.0\textwidth]{poincare_sphere_DRALD.png}
    \caption{Schéma du trajet $\ket{D}\rightarrow\ket{R}\rightarrow\ket{A}\rightarrow\ket{L}\dots$
    utilisant seulement une lame demi-onde et une lame quart d'onde à 45 dégrés dans la préparation de l'état d'entrée}
    \label{fig:path5sphere}
\end{figure}

\subsubsection{Mesure faible temporelle}

\begin{doublespace}
    Après avoir préparé l’état, nous interagissons faiblement avec le système en introduisant un petit délai temporel entre les deux états de base. Pour ce faire, nous utilisons un deuxième séparateur de faisceau polarisant dans l’étape d’interaction faible. Nous faisons en sorte qu’un des bras parcourt un trajet légèrement plus long que l’autre. Chaque bras du séparateur de faisceaux est équipé d’un quart de plaque d’onde pour inverser l’état de base pour que le bras réfléchi soit transmis et que celui transmis soit réfléchi, afin qu’ils puissent se chevaucher. Il y a donc un changement d’état de base à considérer dans notre théorie, mais celui-ci n’affecte pas radicalement le résultat. La partie transmise, que nous définissons comme étant la partie horizontale de l’état de polarisation $a\ket{H}$, subit une interaction faible en parcourant un trajet plus long. Cela introduit un délai $\tau$ sur le pointeur couplé avec cet état $a\ket{H} \otimes \ket{\xi(t-\tau)}$. Celui-ci se transforme de $90$ degrés en traversant une plaque d’onde quart d’onde et en revenant. Il produira un état de polarisation $\ket{V}$ qui sera réfléchi. Le bras initialement réfléchi, c’est-à-dire la partie verticale de l'état de polarisation $b\ket{V}$, subit également un changement, mais son pointeur demeure pareil $b\ket{V} \otimes \ket{\xi(t)}$: il est transformé de $90$ degrés pour pouvoir être transmis à travers du séparateur de faisceau de polarisation, puis mis en forme avec le nouvel état de base $\ket{V}$, légèrement retardé en tant qu'état de base $\ket{H}$. 
    \begin{equation}
        \hat{U}^{H}\ket{\psi_i} = a\ket{H} \otimes \ket{\xi(t-\tau)} + b\ket{V} \otimes \ket{\xi(t)}
    \end{equation}

    \noindent L’impulsion superposée est ensuite mesurée de manière projective par un état de polarisation qui contient les deux états de base. Pour des raisons de simplicité, nous avons opté pour une mesure projective avec l'état $\ket{D} = \frac{1}{\sqrt{2}}(\ket{H}+\ket{V})$, qui est réalisée à l’aide d'une demi-plaque d'onde et d'un polariseur. Le polariseur sert de référence. Il est réglé pour être polarisé verticalement, et la plaque demi-onde est réglée à $45$ degrés par rapport à ce polariseur, ce qui donne un état de polarisation $\ket{D}$ qui est projeté sur notre état à faible interaction. 

    \begin{equation}
        \ket{\psi_f} \equiv \bra{D}\hat{U}^{H}\ket{\psi_i} = \frac{a}{\sqrt{2}}\ket{\xi(t-\tau)} + \frac{b}{\sqrt{2}}\ket{\xi(t)}
    \end{equation}

    \noindent Avant de poursuivre avec une caractérisation quantique de la partie réelle de la valeur faible, nous effectuons une séance de calibration en envoyant l'état $\ket{H}$ et $\ket{V}$ et en notant leurs positions temporelles, ainsi qu’en assurant le délai entre les deux, qui correspond au délai que nous avons orienté le bras de la partie horizontale dans l'état de l'interaction faible. 

    \noindent Nous caractérisons ensuite le trajet de polarisation par la partie réelle de la valeur faible, c’est-à-dire le délai entre les positions temporelles trouvées pour chaque état d’entrée lorsque nous tournons la plaque d’onde en comparaison avec notre calibration. 

    \begin{align}
        \expval{\hat{t}} &= \bra{\psi_f}\hat{t}\ket{\psi_f}\\
        \mathcal{R}\Bigl(\expval{\hat{\pi}_W}\Bigr) &\equiv \frac{\expval{\hat{t}}}{\tau} \propto \ket{\psi_i}
    \end{align}

    \noindent Comme nous avons calculé dans le chapitre 2, voici les parties réelles de la valeur faible pour chaque état d'entrée que nous allons investiguer. Soit pour $\ket{\psi_{i}^{1}} = cos(2\theta)\ket{H} + sin(2\theta)\ket{V}$:

    \begin{equation}
        \expval{\hat{t}} = \frac{\tau}{2}(cos^2(2\theta) + 2sin(2\theta)cos(2\theta))
    \end{equation}

    \noindent Ainsi pour $\ket{\psi_{i}^{2}} = a\ket{H} + b\ket{V}$:
    
    \begin{equation}
        \expval{\hat{t}} = \frac{\tau}{2}(cos^2(2\theta))
    \end{equation}

    \noindent et $\ket{\psi_{i}^{3}} = \frac{1}{2}\Bigl(((1-i)sin(2\theta) + (1+i)cos(2\theta))\ket{H} + ((1+i)sin(2\theta) + (1-i)cos(2\theta))\ket{V}\Bigr)$:

    \begin{equation}
        \expval{\hat{t}} = \frac{\tau}{2}\Bigl(1+\frac{sin(4\theta)}{4}\Bigr)
    \end{equation}

    \noindent Cette expérience a été optimisée pour qu’elle fonctionne de manière autonome grâce à des supports de rotation motorisés \cite{ThorlabsROT} contrôlés par un code Python. Ce dernier utilise la bibliothèque Kinesis \cite{pylablib_thorlabs_kinesis} pour faire tourner ces supports et l’API de l'oscilloscope \cite{TektronixTDS5000} pour enregistrer chaque fichier de chaque état d’entrée. Toutes les données sont ensuite analysées numériquement. Les résultats sont présentés au chapitre 4. La différenciation entre un trajet de polarisation linéaire ou circulaire se trouve dans la partie imaginaire de la valeur faible. Cette distinction est proposée dans la section suivante.
\end{doublespace}

