\begin{doublespace}
    Notre dispositif expérimental, représenté à la figure 
    \ref{fig:realexp}, est composé de notre laser pulsé de tout à 
    l’heure et suit une configuration similaire à celle de notre 
    montage précédent (figure \ref{fig:speed-of-light}). Cependant, 
    après le premier séparateur de faisceau, l’état de polarisation 
    d’entrée est préparé et ensuite soumis à une mesure faible en 
    introduisant un court délai avec une différence de parcours entre 
    les deux composantes orthogonales venant du deuxième séparateur de 
    faisceau. Finalement, une mesure projective est effectuée pour une 
    caractérisation complète et directe.
\end{doublespace}

\begin{figure}[!htpb]
    \centering
    \includegraphics[width=1.0\textwidth]{partiereelexp.png}
    \caption{Dispositif expérimental pour la partie réelle de la valeur 
    faible. Comme la dernière configuration, 
    l’impulsion du laser est réglée en intensité par une lame demi-onde, 
    puis dirigée vers un PBS qui divise les états de polarisation 
    horizontaux et verticaux de
    base de l’impulsion d’entrée en deux voies orthogonales. Celui qui 
    est réfléchi sera notre signal de référence (ou signal de déclenchement) 
    pour déclencher 
    l’oscilloscope et l’autre suit la procédure directe. 
    Ce dernier est préparé en combinant différentes lames, dont on change 
    pour chaque trajet de polarisation, et ensuite subit une mesure 
    faible. Lors de la mesure faible, le couplage faible temporel est 
    réalisé en introduisant 
    une différence de parcours entre la voie transmise et réfléchie du 
    deuxième PBS. Le faisceau réfléchi représente l’état
    de base de polarisation verticale $\ket{V}$ de l’état d’entrée 
    $\ket{\psi}$, tandis que le faisceau
    transmis correspond à l’état de base horizontal $\ket{H}$. Le miroir 
    de la voie de l’état vertical 
    est considéré sans délai et il se trouve à $10,5$ $cm$ du PBS. Le 
    miroir de la voie de l’état horizontal est positionné à la même 
    distance, mais avec le délai que nous ajouterons. Les deux voies 
    subissent une rotation pour revenir et se projettent sur une lame 
    demi-onde inclinée à $45$ degrés par rapport à un polariseur 
    initialement orienté à $0$ degré. 
    Cela crée une mesure de projection avec l’état $\ket{D}$, qui 
    retient le chevauchement des états de base. Ce dernier est ensuite 
    détecté avec un
    photodétecteur, puis interprété par notre oscilloscope. 
}
    \label{fig:realexp}
\end{figure}

\begin{doublespace}
    \noindent Une fois de plus, nous souhaitons déclencher le signal 
    de référence de manière externe, car nous voulons bénéficier de la 
    résolution temporelle maximale offerte par l’oscilloscope, qui 
    possède une fréquence d’échantillonnage de $500$ $GS/s$, pour 
    détecter les délais. Un séparateur de faisceau polarisant divise 
    le faisceau laser en deux voies : celui réfléchi sert de signal de 
    référence pour le déclenchement de l’oscilloscope, tandis que celui 
    transmit sera préparé dans divers trajets de polarisation sur la 
    sphère Poincarré et subira une mesure faible pour une caractérisation. 
    Les trajets de polarisation testés sont les suivants et sont obtenus 
    en changeant les lames d'onde lors de l'étape de préparation. Le 
    premiers consiste seulement à une lame demi-onde définie par 
    l’opérateur suivant:
\end{doublespace}

\begin{align}
    \hat{T}_{HWP}(\theta) = \begin{pmatrix}
        cos(2\theta) & sin(2\theta)\\
        sin(2\theta) & -cos(2\theta)
    \end{pmatrix}
\end{align}

\begin{doublespace}
    \noindent Où l'indice $HWP$ fait référence à une lame demi-onde 
    pour un angle $\theta$ (\guillemetleft halfwaveplat \guillemetright 
    en anglais). Ce premier trajet consiste à passer d’un état de base 
    à un autre sans polarisation circulaire, de 
    $H \to D \to V \to A \to \dots$, et ainsi de suite. Pour comprendre 
    en détail, l'état commence dans l’état horizontal 
    $\ket{H} \equiv \begin{pmatrix}
        1\\
        0
    \end{pmatrix}$ défini par la transmission d'un séparateur 
    de faisceau polarisant. Ensuite, l'état évolue de la façon 
    suivante en fonction de l'angle de la lame d'onde $\theta$:
\end{doublespace}

\begin{align}
    \hat{T}_{HWP}(\theta)\ket{H} &= \begin{pmatrix}
        cos(2\theta) & sin(2\theta)\\
        sin(2\theta) & -cos(2\theta)
    \end{pmatrix} \begin{pmatrix}
        1\\0
    \end{pmatrix}\\
    &= \begin{pmatrix}
        cos(2\theta)\\sin(2\theta)
    \end{pmatrix}
\end{align}

\begin{doublespace}
    \noindent Donc, l'état d'entrée préparé est soit:
\end{doublespace}

\begin{equation}
    \ket{\psi_{i}^{1}} = cos(2\theta)\ket{H} + sin(2\theta)\ket{V}
\end{equation}

\begin{doublespace}
    \noindent En fonction des paramètres de Stokes pour démontrer le 
    trajet sur la sphère Poincarré (figure \ref{fig:path3sphere}):
\end{doublespace}

\begin{equation}
    S = \begin{pmatrix}
        S_0 = |a|^2 + |b^2|\\
        S_1 = |a|^2 - |b|^2\\
        S_2 = 2\mathcal{R}(\bar{a}b)\\
        S_3 = 2\mathcal{I}(\bar{a}b)
    \end{pmatrix} = \begin{pmatrix}
        1\\
        cos^{2}(2\theta) - sin^{2}(2\theta) = cos(4\theta)\\
        2cos(2\theta)sin(2\theta) = sin(4\theta)\\
        0
    \end{pmatrix}
\end{equation}

\begin{figure}[!htpb]
    \centering
    \includegraphics[width=1.0\textwidth]{poincare_sphere_HDVAH.png}
    \caption{Schéma du trajet $\ket{H}\rightarrow\ket{D}\rightarrow\ket{V}\rightarrow\ket{A}\dots$
    utilisant seulement une lame demi-onde dans la préparation de l'état d'entrée}
    \label{fig:path3sphere}
\end{figure}

\begin{doublespace}
    \noindent Ce trajet est réalisé en tournant uniquement une lame 
    demi-onde. On tourne l'angle de la lame d'onde de $2,5$
    degrés pour chaque acquisition. Chaque degré $\theta^\prime$ 
    que nous tournons en réalité 
    équivaut à tourner de $5$ degrés sur un plan circulaire 
    $\theta^\prime \equiv 2\theta$ ou de $10$ degrés sur la sphère de 
    Poincarré. 

    \noindent Le trajet suivant consiste à passer d'un état de base à 
    un autre en passant par une polarisation circulaire 
    $\ket{H} \to \ket{R} \to \ket{V} \to \ket{L} \dots$. Cela se fait 
    avec une lame demi-onde tournant de la même manière que précédemment, 
    et une lame quart d'onde réglée à $0$ degré par rapport à 
    $\ket{H}$. L’opération de cette lame d’onde se définit par 
    l'opérateur suivant:
\end{doublespace}

\begin{align}
    \hat{T}_{QWP}(\phi) &= \begin{pmatrix}
        cos^{2}(\phi) + isin^{2}(\phi) & (1-i)cos(\phi)sin(\phi)\\
        (1-i)cos(\phi)sin(\phi) & sin^{2}(\phi) + icos^{2}(\phi)
    \end{pmatrix}\\
    \hat{T}_{QWP}(\phi = 0^{\degree}) &= \begin{pmatrix}
        1 & 0 \\
        0 & i
    \end{pmatrix}
\end{align}

\begin{doublespace}
    \noindent La forme de cet opérateur $\hat{T}_{QWP}(\phi)$, avec 
    l'angle $\phi$ pour la lame d'onde et l'indice QWP fait référence 
    à une lame quart d'onde 
    (\guillemetleft quarter waveplate \guillemetright en anglais), 
    permet de conserver $a \in \mathcal{R}$ et de laisser 
    $b\in \mathcal{C}$ contenir l'information complexe. Nous 
    procédons ainsi pour que la partie imaginaire de la valeur 
    faible soit principalement contenue dans $b$ pour des raisons de 
    simplicité. Avec cette opération, l'état évolue comme suit:
\end{doublespace}

\begin{align}
    \hat{T}_{QWP}(\phi = 0^{\degree})\hat{T}_{HWP}(\theta)\ket{H} &= 
    \begin{pmatrix} 
        1 & 0 \\
        0 & i
    \end{pmatrix} 
    \begin{pmatrix}
        cos(2\theta) & sin(2\theta)\\
        sin(2\theta) & -cos(2\theta)
    \end{pmatrix} \begin{pmatrix}
        1\\0
    \end{pmatrix}\\
    &= \begin{pmatrix}
        cos(2\theta)\\isin(2\theta)
    \end{pmatrix}
\end{align}

\begin{doublespace}
    \noindent Donc, l'état d'entrée est: 
\end{doublespace}

\begin{equation}
    \ket{\psi_{i}^{2}} = cos(2\theta)\ket{H} + isin(2\theta)\ket{V}
\end{equation}

\begin{doublespace}
    \noindent Avec les paramètres de Stokes pour démontrer sa 
    trajectoire (figure \ref{fig:path4sphere}):
\end{doublespace}

\begin{equation}
    S = \begin{pmatrix}
        S_0 = |a|^2 + |b^2|\\
        S_1 = |a|^2 - |b|^2\\
        S_2 = 2\mathcal{R}(\bar{a}b)\\
        S_3 = 2\mathcal{I}(\bar{a}b)
    \end{pmatrix} = \begin{pmatrix}
        1\\
        cos^{2}(2\theta) - sin^{2}(2\theta) = cos(4\theta)\\
        0\\
        2cos(2\theta)sin(2\theta) = sin(4\theta)
    \end{pmatrix}
\end{equation}

\begin{figure}[!htpb]
    \centering
    \includegraphics[width=1.0\textwidth]{poincare_sphere_HRVLH.png}
    \caption{Schéma du trajet 
    $\ket{H}\rightarrow\ket{R}\rightarrow\ket{V}\rightarrow\ket{L}\dots$
    utilisant seulement une lame demi-onde et une lame quart d'onde à 
    $0$ dégré dans la préparation de l'état d'entrée}
    \label{fig:path4sphere}
\end{figure}

\begin{doublespace}
    \noindent Le trajet final est un parcours captivant qui nous fait 
    passer constamment entre deux états de base, soit d'une polarisation 
    linéaire $\{\ket{D}, \ket{A}\}$ à une polarisation circulaire 
    $\{ \ket{R}, \ket{L}\}$. La trajectoire résultante est 
    $\ket{D} \to \ket{R} \to \ket{A} \ket{L} \dots$. Cette dernière est 
    obtenue en tournant une lame demi-onde avec une lame quart d'onde 
    réglée à $45$ degrés par rapport à l'état de base $\ket{H}$, dont 
    la lame quart d'onde à ce réglage est défini comme suit: 
\end{doublespace}

\begin{equation}
    \hat{T}_{QWP}(\phi = 45^{\degree}) = \frac{1}{2}\begin{pmatrix}
        1+i & 1-i \\
        1-i & 1+i
    \end{pmatrix}
\end{equation}

\begin{doublespace}
    \noindent Avec cette opération, l'état évolue comme suit:
\end{doublespace}

\begin{align}
    \hat{T}_{QWP}(\phi = 45^{\degree})\hat{T}_{HWP}(\theta)\ket{H} &= 
    \begin{pmatrix} 
        \frac{1+i}{2} & \frac{1-i}{2} \\
        \frac{1-i}{2} & \frac{1+i}{2}
    \end{pmatrix} 
    \begin{pmatrix}
        cos(2\theta) & sin(2\theta)\\
        sin(2\theta) & -cos(2\theta)
    \end{pmatrix} \begin{pmatrix}
        1\\0
    \end{pmatrix}\\
    &= \frac{1}{2}\begin{pmatrix}
        cos(2\theta) + sin(2\theta) + i(cos(2\theta) - sin(2\theta))\\cos(2\theta) + sin(2\theta) -i(cos(2\theta)-sin(2\theta))
    \end{pmatrix}
\end{align}

\begin{doublespace}
    \noindent Donc, l'état d'entrée est: 
\end{doublespace}

\begin{equation}
    \ket{\psi_{i}^{3}} = \frac{1}{2}\Bigl(((1-i)sin(2\theta) + (1+i)cos(2\theta))\ket{H} + ((1+i)sin(2\theta) + (1-i)cos(2\theta))\ket{V}\Bigr)
\end{equation}

\begin{doublespace}
    \noindent Avec les paramètres de Stokes pour démontrer sa 
    trajectoire (figure \ref{fig:path5sphere}):
\end{doublespace}

\begin{equation}
    S = \begin{pmatrix}
        S_0 = |a|^2 + |b^2|\\
        S_1 = |a|^2 - |b|^2\\
        S_2 = 2\mathcal{R}(\bar{a}b)\\
        S_3 = 2\mathcal{I}(\bar{a}b)
    \end{pmatrix} = \begin{pmatrix}
        1\\
        0\\
        sin(4\theta)\\
        sin^2(2\theta) - cos^2(2\theta)
    \end{pmatrix}
\end{equation}

\begin{figure}[!htpb]
    \centering
    \includegraphics[width=1.0\textwidth]{poincare_sphere_DRALD.png}
    \caption{Schéma du trajet $\ket{D}\rightarrow\ket{R}\rightarrow\ket{A}\rightarrow\ket{L}\dots$
    utilisant seulement une lame demi-onde et une lame quart d'onde à 
    $45$ dégrés dans la préparation de l'état d'entrée}
    \label{fig:path5sphere}
\end{figure}


