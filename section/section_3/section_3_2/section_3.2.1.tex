Notre appareil expérimental est constitué de 
notre laser pulsé de tout à l'heure entrant dans 
une demi-plaque d'onde qui est principalement 
utilisée pour le contrôle de l'intensité puisque 
l'entrée externe de l'oscilloscope a besoin 
d'une intensité assez importante pour se 
déclencher sur le signal de référence. 

\begin{figure}[h]
    \centering
    \includegraphics[width=1.0\textwidth]{partiereelexp.png}
    \caption{Dispositif expérimental pour la partie réel de la valeur faible}
    \label{fig:realexp}
\end{figure}

Encore 
une fois, nous voulons déclencher le signal de 
référence de manière externe puisque nous 
voulons utiliser le taux d'échantillonnage de 
500GS/s de l'oscilloscope pour une résolution 
temporelle maximale pour la détection des 
retards. Le faisceau passe à travers un 
séparateur de faisceau non polarisant qui divise 
le faisceau laser en deux voies, l'une étant 
utilisée comme signal de référence pour le 
déclenchement de l'oscilloscope et l'autre 
subissant une mesure faible. L'état est alors 
préparé pour que différents chemins de 
polarisation sur la sphère de Poincarré soient 
caractérisés par le retard temporel. Les 
chemins de polarisation testés sont les 
suivants et sont obtenus en changeant les 
plaques d'onde lors de l'étape de préparation. 
Le premier chemin consiste à passer d'un état 
de base à un autre sans polarisation circulaire, 
de H à D à V à A. Note que les figures des trajets sont 
dans la cadres du paramètre de Stokes définit par:

\begin{equation}
    S \equiv \begin{pmatrix}
        S_0\\
        S_1\\
        S_2\\
        S_3
    \end{pmatrix} \equiv \begin{pmatrix}
        \bra{\psi}\ket{\psi}\\
        \bra{\psi}\sigma_z\ket{\psi}\\
        \bra{\psi}\sigma_x\ket{\psi}\\
        \bra{\psi}\sigma_y\ket{\psi}
    \end{pmatrix}
\end{equation}

Soit les matrices de Pauli $\sigma_z = \ket{H}\bra{H} - \ket{V}\bra{V}$, 
$\sigma_x = \ket{H}\bra{V} + \ket{V}\bra{H}$ et $\sigma_y = i(\ket{V}\bra{H} - \ket{H}\bra{V})$.
Pour un état de polarisation $\ket{\psi}$ défini par 
$\ket{\psi} \equiv a\ket{H} + b\ket{V}$ avec les amplitudes
de probabilité $a$ et $b$, nous pouvons écrire
les paramètres de Stokes avec ces paramètres comme ceci:

\begin{equation}
    S = \begin{pmatrix}
        S_0 = |a|^2 + |b^2|\\
        S_1 = |a|^2 - |b|^2\\
        S_2 = 2\mathcal{R}(\bar{a}b)\\
        S_3 = 2\mathcal{I}(\bar{a}b)
    \end{pmatrix}
\end{equation}

Où $\bar{}$ comporte au complexe conjugué de la variable. 
Les paramètres de Stokes sont utilisé pour démontré comment
l'état de polarization change en fonction de l'angle de telle composantes
optique utilisé pour les différents chemins.

\begin{figure}[h]
    \centering
    \includegraphics[width=0.8\textwidth]{path3_sphere.png}
    \caption{Schéma du trajet $\ket{H}\rightarrow\ket{D}\rightarrow\ket{V}\rightarrow\ket{A}\dots$
    utilisant seulement une lame demi-onde dans la préparation de l'état d'entrée}
    \label{fig:path3sphere}
\end{figure}

Ce chemin est réalisé en 
tournant uniquement une lame demi-onde par pas 
de 2,5 degrés. Chaque degré que nous tournons 
en réalité équivaut à tourner de 5 degrés sur 
un plan circulaire ou de 10 degrés sur la sphère 
de Poincarré. Le chemin suivant consiste à 
passer d'un état de base à un autre, mais en 
passant par une polarisation circulaire. Cela se 
fait avec une demi-plaque d'onde tournant de la 
même manière que précédemment, mais avec un 
quart de plaque d'onde réglé à 0 degré par 
rapport à l'état de base H. La trajectoire 
résultante est H à R à V à L lorsque nous 
tournons la demi-plaque d'onde. 

\begin{figure}[h]
    \centering
    \includegraphics[width=0.8\textwidth]{path4_2_sphere.png}
    \caption{Schéma du trajet $\ket{H}\rightarrow\ket{R}\rightarrow\ket{V}\rightarrow\ket{L}\dots$
    utilisant seulement une lame demi-onde et une lame quart d'onde à 0 dégrée dans la préparation de l'état d'entrée}
    \label{fig:path4sphere}
\end{figure}

Ces deux 
trajectoires devraient suivre une fonction 
cosinusoïdale, comme cela a été théorisé. Le 
chemin suivant est un chemin intéressant où 
nous avons toujours les deux états de base mais 
où nous passons d'une polarisation linéaire à 
une polarisation circulaire. La trajectoire 
résultante est D à R à A à L. 

\begin{figure}[h]
    \centering
    \includegraphics[width=0.8\textwidth]{path5_2_sphere.png}
    \caption{Schéma du trajet $\ket{D}\rightarrow\ket{R}\rightarrow\ket{A}\rightarrow\ket{L}\dots$
    utilisant seulement une lame demi-onde et une lame quart d'onde à 45 dégrée dans la préparation de l'état d'entrée}
    \label{fig:path5sphere}
\end{figure}

Cette dernière est 
obtenue en tournant une demi-plaque d'onde avec 
un quart de plaque d'onde réglé à 45 degrés par 
rapport à l'état de base H. Le résultat doit 
avoir une partie réelle et doit donc toujours 
être égal à une amplitude de probabilité de 
1ésqrt(2). 



Une fois l'état préparé, nous 
interagissons faiblement avec le système en 
introduisant un petit délai temporel entre les 
deux états de base en divisant les états de base 
à l'aide d'un séparateur de faisceau polarisant 
et en faisant en sorte que l'un des bras 
parcourt un chemin légèrement plus long. Chaque 
bras du séparateur de faisceaux est équipé d'un 
quart de plaque d'onde pour inverser l'état de 
base afin qu'ils puissent se chevaucher. Il y a 
donc un changement d'état de base à prendre en 
compte dans notre théorie, mais ce n'est qu'un 
changement mineur qui ne modifie pas 
radicalement le résultat. La partie H est 
faiblement interférée et se retourne à 45 degrés 
en passant une fois par la plaque d'onde quart 
d'onde et en revenant, ce qui donne un état de 
polarisation V, de sorte qu'elle peut être 
réfléchie à notre étape de postsélection. Il est 
donc bon de noter que l'interaction faible est 
maintenant sur l'état de base V. L'état de base 
V d'origine suit une histoire similaire : il 
est retourné de 90 degrés pour pouvoir être 
transmis à travers le séparateur de faisceau de 
polarisation, puis mis en forme avec le nouvel 
état de base V légèrement retardé en tant 
qu'état de base H. L'impulsion superposée est 
ensuite mesurée de manière projective via un 
état de polarisation qui contient les deux états 
de base. Pour des raisons de simplicité, nous 
avons choisi de faire la postsélection avec 
l'état D, ce qui se fait par l'intermédiaire 
d'une demi-plaque d'onde et d'un polariseur. Le 
polariseur est utilisé comme référence, il est 
réglé pour être polarisé verticalement et la 
plaque demi-onde est réglée à 45 degrés par 
rapport à ce polariseur, ce qui donne un état de 
polarisation D qui est projeté sur notre état à 
faible interaction. Nous caractérisons ensuite 
le chemin de polarisation en sauvegardant chaque 
fichier csv pour chaque degré d'état d'entrée 
lorsque nous tournons la plaque d'onde. 
L'expérience a été optimisée pour fonctionner 
automatiquement à l'aide de supports de rotation 
motorisés fournis par thorlabs et contrôlés par 
un code python qui fait tourner ces supports via 
la bibliothèque Kinesis et utilise l'API des 
oscilloscopes pour enregistrer chaque fichier 
csv à chaque état d'entrée. Les données 
complètes sont ensuite caractérisées pour la 
partie réelle de la valeur faible, dont les 
résultats sont présentés au chapitre 4. La 
différenciation entre un chemin de polarisation 
linéaire et circulaire se fait dans la partie 
imaginaire de la valeur faible. C'est ce qui est 
proposé dans la section suivante.
