\begin{doublespace}
    Après avoir préparé l’état, nous interagissons faiblement avec le 
    système en introduisant un court délai temporel entre deux états 
    $\ket{H}$ et $\ket{V}$ de base. Pour ce faire, nous utilisons 
    un deuxième séparateur de faisceau polarisant dans l’étape d’interaction faible. Nous faisons 
    en sorte qu’un des bras soit légèrement plus long que 
    l’autre de $167$ $ps$ (voir figure \ref{fig:realexp}). 
    Chaque bras de l'interféromètre est équipé d’une lame quart 
    d’onde pour inverser l’état de polarisation afin que 
    la lumière dans le bras 
    réfléchi soit transmise et que celle dans le bras transmis 
    soit réfléchie, de sorte 
    qu’ils puissent recombiner sur le PBS. La partie transmise, que nous définissons 
    comme étant la partie horizontale de l’état de polarisation 
    $a\ket{H}$, subit une interaction faible en parcourant un trajet 
    plus long. Cela introduit un délai $\tau$ sur le pointeur couplé 
    avec cet état $a\ket{H} \otimes \ket{\xi(t-\tau)}$.  
    La partie réfléchie, c’est-à-dire la partie 
    verticale de l'état de polarisation $b\ket{V}$ reste inchangée 
    et est couplée avec l'état du pointeur
    $b\ket{V} \otimes \ket{\xi(t)}$.
    Le système est donc dans l'état suivant après l'interaction faible:

    \begin{equation}
        \hat{U}^{H}\ket{\psi_i} = a\ket{H} \otimes \ket{\xi(t-\tau)} + b\ket{V} \otimes \ket{\xi(t)}
    \end{equation}

    \noindent Pour avoir une interférence avec des poids égaux de 
    $\ket{H}$ et $\ket{V}$, nous avons opté pour une postsélection sur l'état 
    $\ket{D} = \frac{1}{\sqrt{2}}(\ket{H}+\ket{V})$, qui est réalisée à 
    l’aide d'une lame demi-onde à $45$ degrés et d'un polariseur à $90$ degrés
    dont le polariseur sert de référence pour une postsélection diagonale. 
    L'état final après postsélection est : 

    \begin{equation}
        \ket{\psi_f} \equiv \bra{D}\hat{U}^{H}\ket{\psi_i} = \frac{a}{\sqrt{2}}\ket{\xi(t-\tau)} + \frac{b}{\sqrt{2}}\ket{\xi(t)}
    \end{equation}

    \noindent Avant de poursuivre avec la caractérisation quantique de 
    la partie réelle de la valeur faible, nous effectuons une calibration
    en envoyant l'état $\ket{V}$ et $\ket{H}$ 
    qui correspondent respectivement au temps de référence $t_0$ par rapport 
    auxquels tous les délais seront calculés et le temps
    $t_0 + \tau$ correspondant au délai de référence et au délai introduit. 
    On s'assure que le délai $\tau$ est égal à celui implémenté dans 
    l'expérience de $167$ $ps$.
    Nous pouvons alors mesurer la partie réelle de la valeur faible
    en utilisant l'état de polarisation préparé et l'état de polarisation
    post-sélectionné.

    \noindent Nous caractérisons ensuite les trajets de polarisation pour mesurer 
    la partie réelle de la valeur faible, c’est-à-dire mesurer les 
    délais temporels associés aux impulsions des états de polarisation. 
    Pour rappeler, la partie réelle de la valeur faible est donnée par
    l'équation \ref{eq:expval_t_norm}, ce que nous allons donc calculer
    $\expval{\hat{t}}$ qui est lié à la partie réelle de la valeur faible
    pour chaque trajets de polarisation.
    
    \noindent Nous commençons par l'état de polarisation linéaire 
    $\ket{\psi_{i}^{1}} = cos(2\theta)\ket{H} + sin(2\theta)\ket{V}$
    qui est préparé par la lame demi-onde à $\theta$, nous obtenons:

    \begin{equation}
        \expval{\hat{t}} = \frac{\tau}{2}(cos^2(2\theta) + 2sin(2\theta)cos(2\theta))
        \label{eq:expval_t_1}
    \end{equation}

    \noindent Ensuite, pour l'état de polarisation circulaire 
    $\ket{\psi_{i}^{2}} = cos(2\theta)\ket{H} + isin(2\theta)\ket{V}$
    préparé par la lame demi-onde à $\theta$ et une lame quart d'onde 
    à $0$ degré, nous avons:

    \begin{equation}
        \expval{\hat{t}} = \frac{\tau}{2}(cos^2(2\theta))
        \label{eq:expval_t_2}
    \end{equation}

    \noindent et pour l'état de polarisation de superposition (en référence qu'il y a toujours
    une superposition entre $\ket{H}$ et $\ket{V}$ pour ce trajet de polarisation) 
    $\ket{\psi_{i}^{3}} = \frac{1}{2}\Bigl(((1-i)sin(2\theta) + (1+i)cos(2\theta))\ket{H} + ((1+i)sin(2\theta) + (1-i)cos(2\theta))\ket{V}\Bigr)$
    préparé par la lame demi-onde à $2\theta$ et une lame quart d'onde à $45$ degrés,
    nous avons:

    \begin{equation}
        \expval{\hat{t}} = \frac{\tau}{2}\Bigl(1+\frac{sin(4\theta)}{4}\Bigr)
        \label{eq:expval_t_3}
    \end{equation}

    \noindent Pour automatiser l'expérience, nous utilisons la 
    bibliothèque Kinesis pour contrôler les supports de rotation motorisés \cite{pylablib_thorlabs_kinesis,ThorlabsROT} 
    des lames d'onde et pilotons l’acquisition via l’API de l’oscilloscope (interface VISA) \cite{oscilloscope_coding,TektronixTDS5000}. 
    Nous envoyons un signal via \textsc{Python} à Kinesis pour faire 
    tourner la lame demi-onde d'entrée (voir figure \ref{fig:realexp}) 
    de $90$ degrés par $2.5$ degrés à chaque itération. 
    Ensuite, un signal est envoyé à l'oscilloscope pour acquérir 
    les données de l'impulsion et les enregistrer automatiquement, $5$
    fichiers sont créés pour chaque état d'entrée. Ce processus 
    est répété pour chaque état d'entrée, permettant 
    une collecte de données efficace et reproductible.
\end{doublespace}

