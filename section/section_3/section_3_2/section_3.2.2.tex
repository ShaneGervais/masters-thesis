\begin{doublespace}
    Après avoir préparé l’état, nous interagissons faiblement avec le 
    système en introduisant un court délai temporel entre les deux états 
    de base. Pour ce faire, nous utilisons un deuxième séparateur de 
    faisceau polarisant dans l’étape d’interaction faible. Nous faisons 
    en sorte qu’un des bras soit légèrement plus long que 
    l’autre. Chaque bras de l'interféromètre est équipé d’une lame quart 
    d’onde pour inverser l’état de polarisation afin que 
    la lumière dans le bras 
    réfléchie soit transmise et que celle dans le bras transmise 
    soit réfléchie, de sorte 
    qu’ils puissent recombiner sur le PBS. La partie transmise, que nous définissons 
    comme étant la partie horizontale de l’état de polarisation 
    $a\ket{H}$, subit une interaction faible en parcourant un trajet 
    plus long. Cela introduit un délai $\tau$ sur le pointeur couplé 
    avec cet état $a\ket{H} \otimes \ket{\xi(t-\tau)}$.  
    La partie réfléchie, c’est-à-dire la partie 
    verticale de l'état de polarisation $b\ket{V}$ restent inchangée 
    et est couplée avec l'état du pointeur
    $b\ket{V} \otimes \ket{\xi(t)}$:

    \begin{equation}
        \hat{U}^{H}\ket{\psi_i} = a\ket{H} \otimes \ket{\xi(t-\tau)} + b\ket{V} \otimes \ket{\xi(t)}
    \end{equation}

    \noindent Pour avoir une interférence avec des poids égaux de 
    $\ket{H}$ et $\ket{V}$, nous avons opté pour une postsélection sur l'état 
    $\ket{D} = \frac{1}{\sqrt{2}}(\ket{H}+\ket{V})$, qui est réalisée à 
    l’aide d'une lame demi-onde à $45$ dégré et d'un polariseur à $90$ dégré
    dont le polariseur sert de référence pour une postsélection diagonale. 
    L'état final après postsélection est : 

    \begin{equation}
        \ket{\psi_f} \equiv \bra{D}\hat{U}^{H}\ket{\psi_i} = \frac{a}{\sqrt{2}}\ket{\xi(t-\tau)} + \frac{b}{\sqrt{2}}\ket{\xi(t)}
    \end{equation}

    \noindent Avant de poursuivre avec la caractérisation quantique de 
    la partie réelle de la valeur faible, nous effectuons une séance de 
    calibration en envoyant l'état $\ket{V}$ et $\ket{H}$ 
    qui corresponde respectivement au temps de référence $t_0$ par rapport 
    auxquels tous les délais seront calculé et le temps de référence 
    $t_0 + \tau$ correspondant au délai de référence et du délai introduit. 
    On s'assure que le délai $\tau$ est égal à celui implémenté dans 
    l'expérience. 
    
    \noindent Nous caractérisons ensuite les trajets de polarisation pour mesurer 
    la partie réelle de la valeur faible, c’est-à-dire mesurer les 
    délais temporels associés aux impulsions des états de polarisation. 
    Pour rappeler, la partie réelle de la valeur faible est donnée par
    la relation suivante:

    \begin{align}
        \expval{\hat{t}} &= \bra{\psi_f}\hat{t}\ket{\psi_f}\\
        \mathcal{R}\Bigl(\expval{\hat{\pi}_W}\Bigr) &\equiv \frac{\expval{\hat{t}}}{\tau} \propto \ket{\psi_i}
    \end{align}

    \noindent Voici les parties réelles de la valeur faible pour chaque 
    trajets que nous allons investiguer. Soit pour 
    $\ket{\psi_{i}^{1}} = cos(2\theta)\ket{H} + sin(2\theta)\ket{V}$:

    \begin{equation}
        \expval{\hat{t}} = \frac{\tau}{2}(cos^2(2\theta) + 2sin(2\theta)cos(2\theta))
    \end{equation}

    \noindent Ainsi pour $\ket{\psi_{i}^{2}} = a\ket{H} + b\ket{V}$:
    
    \begin{equation}
        \expval{\hat{t}} = \frac{\tau}{2}(cos^2(2\theta))
    \end{equation}

    \noindent et $\ket{\psi_{i}^{3}} = \frac{1}{2}\Bigl(((1-i)sin(2\theta) + (1+i)cos(2\theta))\ket{H} + ((1+i)sin(2\theta) + (1-i)cos(2\theta))\ket{V}\Bigr)$:

    \begin{equation}
        \expval{\hat{t}} = \frac{\tau}{2}\Bigl(1+\frac{sin(4\theta)}{4}\Bigr)
    \end{equation}

    \noindent Cette expérience a été optimisée pour qu’elle fonctionne 
    de manière autonome grâce à des supports de rotation motorisés 
    \cite{ThorlabsROT} pour les lames d'onde contrôlés par un code 
    \textsc{Python} et un oscilloscope interfaçé à un ordinateur. Ce dernier utilise 
    la bibliothèque Kinesis \cite{pylablib_thorlabs_kinesis} est utilisé pour faire 
    tourner ces supports et l’API de l'oscilloscope \cite{TektronixTDS5000} 
    pour enregistrer chaque fichier de chaque état d’entrée. Toutes les 
    données sont ensuite analysées avec \textsc{MATLAB}. Les résultats sont 
    présentés au chapitre 4. 
\end{doublespace}

