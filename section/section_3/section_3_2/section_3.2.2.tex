\begin{doublespace}
    Après avoir préparé l’état, nous interagissons faiblement avec le 
    système en introduisant un petit délai temporel entre les deux états 
    de base. Pour ce faire, nous utilisons un deuxième séparateur de 
    faisceau polarisant dans l’étape d’interaction faible. Nous faisons 
    en sorte qu’un des bras parcourt un trajet légèrement plus long que 
    l’autre. Chaque bras du séparateur de faisceaux est équipé d’un 
    quart de lame d’onde pour inverser l’état de base afin que le bras 
    réfléchi soit transmis et que celui transmis soit réfléchi, de sorte 
    qu’ils puissent se chevaucher. Il y a donc un changement d’état de 
    base à considérer dans notre théorie, mais celui-ci n’affecte pas 
    radicalement le résultat. La partie transmise, que nous définissons 
    comme étant la partie horizontale de l’état de polarisation 
    $a\ket{H}$, subit une interaction faible en parcourant un trajet 
    plus long. Cela introduit un délai $\tau$ sur le pointeur couplé 
    avec cet état $a\ket{H} \otimes \ket{\xi(t-\tau)}$. Celui-ci se 
    transforme de $90$ degrés en traversant une lame d’onde quart d’onde. 
    En revenant, il produira un état de polarisation $\ket{V}$ qui 
    sera réfléchi. Le bras initialement réfléchi, c’est-à-dire la partie 
    verticale de l'état de polarisation $b\ket{V}$, subit également un 
    changement, mais son pointeur demeure pareil 
    $b\ket{V} \otimes \ket{\xi(t)}$: il est transformé de $90$ degrés 
    pour pouvoir être transmis à travers du séparateur de faisceau de 
    polarisation, puis mis en forme avec le nouvel état de base 
    $\ket{V}$, légèrement retardé en tant qu'état de base $\ket{H}$. 

    \begin{equation}
        \hat{U}^{H}\ket{\psi_i} = a\ket{H} \otimes \ket{\xi(t-\tau)} + b\ket{V} \otimes \ket{\xi(t)}
    \end{equation}

    \noindent L’impulsion superposée est ensuite mesurée de manière 
    projective par un état de polarisation qui contient les deux états 
    de base. Pour des raisons de simplicité, nous avons opté pour une 
    mesure projective avec l'état 
    $\ket{D} = \frac{1}{\sqrt{2}}(\ket{H}+\ket{V})$, qui est réalisée à 
    l’aide d'une lame demi-onde et d'un polariseur. Le polariseur sert 
    de référence. Il est réglé pour être polarisé verticalement et la 
    lame demi-onde est réglée à $45$ degrés par rapport à ce polariseur, 
    ce qui donne un état de polarisation $\ket{D}$ qui est projeté sur 
    notre état à faible interaction. 

    \begin{equation}
        \ket{\psi_f} \equiv \bra{D}\hat{U}^{H}\ket{\psi_i} = \frac{a}{\sqrt{2}}\ket{\xi(t-\tau)} + \frac{b}{\sqrt{2}}\ket{\xi(t)}
    \end{equation}

    \noindent Avant de poursuivre avec une caractérisation quantique de 
    la partie réelle de la valeur faible, nous effectuons une séance de 
    calibration en envoyant l'état $\ket{H}$ et $\ket{V}$ et en notant 
    leurs positions temporelles, ainsi qu’en assurant le délai entre les 
    deux, qui correspond au délai que nous avons orienté (le bras de la 
    partie horizontale dans l'état de l'interaction faible). 

    \noindent Nous caractérisons ensuite le trajet de polarisation par 
    la partie réelle de la valeur faible, c’est-à-dire le délai entre 
    les positions temporelles trouvées pour chaque état d’entrée lorsque 
    nous tournons la lame d’onde en comparaison avec notre calibration. 
    Comme rappel, nous avons calculé que dans le chapitre 2, la 
    partie réelle de la valeur faible est donnée par le suivant:

    \begin{align}
        \expval{\hat{t}} &= \bra{\psi_f}\hat{t}\ket{\psi_f}\\
        \mathcal{R}\Bigl(\expval{\hat{\pi}_W}\Bigr) &\equiv \frac{\expval{\hat{t}}}{\tau} \propto \ket{\psi_i}
    \end{align}

    \noindent Voici les parties réelles de la valeur faible pour chaque 
    état d'entrée que nous allons investiguer. Soit pour 
    $\ket{\psi_{i}^{1}} = cos(2\theta)\ket{H} + sin(2\theta)\ket{V}$:

    \begin{equation}
        \expval{\hat{t}} = \frac{\tau}{2}(cos^2(2\theta) + 2sin(2\theta)cos(2\theta))
    \end{equation}

    \noindent Ainsi pour $\ket{\psi_{i}^{2}} = a\ket{H} + b\ket{V}$:
    
    \begin{equation}
        \expval{\hat{t}} = \frac{\tau}{2}(cos^2(2\theta))
    \end{equation}

    \noindent et $\ket{\psi_{i}^{3}} = \frac{1}{2}\Bigl(((1-i)sin(2\theta) + (1+i)cos(2\theta))\ket{H} + ((1+i)sin(2\theta) + (1-i)cos(2\theta))\ket{V}\Bigr)$:

    \begin{equation}
        \expval{\hat{t}} = \frac{\tau}{2}\Bigl(1+\frac{sin(4\theta)}{4}\Bigr)
    \end{equation}

    \noindent Cette expérience a été optimisée pour qu’elle fonctionne 
    de manière autonome grâce à des supports de rotation motorisés 
    \cite{ThorlabsROT} contrôlés par un code Python. Ce dernier utilise 
    la bibliothèque Kinesis \cite{pylablib_thorlabs_kinesis} pour faire 
    tourner ces supports et l’API de l'oscilloscope \cite{TektronixTDS5000} 
    pour enregistrer chaque fichier de chaque état d’entrée. Toutes les 
    données sont ensuite analysées numériquement. Les résultats sont 
    présentés au chapitre 4. La différenciation entre un trajet de 
    polarisation linéaire ou circulaire se trouve dans la partie 
    imaginaire de la valeur faible. Cette distinction est proposée dans 
    la section suivante.
\end{doublespace}

