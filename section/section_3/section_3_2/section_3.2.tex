\begin{doublespace}
    Pour caractériser la partie réelle de la valeur faible, nous 
    introduisons une interaction faible entre les états de base de 
    la polarisation $\ket{H}$ et $\ket{V}$ via un délai temporel. Ce 
    délai doit être inférieur à la largeur à mi-hauteur de l'impulsion du laser 
    $\sigma \ll \tau$ \cite{Aharonov,Lundeen_Resch}. On suppose 
    que l'exponentielle dans l'équation \ref{eq:expval_t_norm} est
    négligeable, ce qui signifie que l'interaction est faible. 
    En effet, si l'interaction est faible, la valeur de l'exponentielle
    est proche de $1$, ce qui signifie que l'interaction n'a pas
    d'effet significatif sur l'état quantique. En revanche, si l'interaction
    est forte, la valeur de l'exponentielle peut être significativement
    différente de $1$, ce qui signifie que l'interaction a un effet
    significatif sur l'état quantique. Dans le régime des mesures
    faibles, l'état de polarisation est perturbé de manière
    minimale, ce qui permet d'avoir un chevauchement évident entre les états
    $\ket{H}$ et $\ket{V}$, ce qui permet
    d'extraire significativement l'état de polarisation de l'état
    d'entrée (voir la figure \ref{fig:interaction}). Pour l'interaction 
    faible, nous introduisons un délai temporel
    suffisamment court entre les deux états de polarisation $\ket{H}$ et $\ket{V}$
    avec un interféromètre de polarisation. Nous supposons que, pour être dans
    le régime des mesures faibles, au moins $90 \%$ du chevauchement
    entre les états de base est suffisant. Pour ce faire, l'un de ses
    bras est légèrement plus long que l'autre. Ensuite, pour mesurer
    l'état directement, nous devons déterminer le délai
    moyen que l'état a subit en traversant l'interféromètre de
    polarisation. Pour ce faire, nous effectuons une postsélection qui 
    est une mesure projective permettant de faire interférer les 
    composantes $\ket{H}$ et $\ket{V}$ de la polarisation. Si l'état n'a subit aucun 
    délai, cela correspond à $\ket{V}$. Si l'état est maximalement retardé, 
    cela correspond à $\ket{H}$. Et, pour les autres états, nous comparons le 
    délai à un étalon afin de pouvoir associer délai à son état de 
    polarisation correspondant. La section suivante décrit le dispositif 
    expérimental que nous avons utilisé pour caractériser la partie 
    réelle de la valeur faible. 
\end{doublespace}