Pour caractériser la partie réelle de la valeur 
faible, nous introduisons une interaction faible 
entre les états de base de la 
polarisation $\ket{H}$ et $\ket{V}$ 
via un délai temporel. Ce délai doit être 
inférieur au profil temporel du 
laser $\sigma \ll \tau$. Aucun 
modèle ne décrit spécifiquement comment 
l'interaction devrait être faible, mais il 
doit y avoir un chevauchement évident entre 
les états de base. Nous supposons qu’au moins 
$90\%$ de chevauchement entre les états de 
base sont nécessaires pour être dans le régime 
des mesures faibles. Ensuite, pour mesurer 
l'état directement par des mesures faibles, 
nous devons effectuer une mesure projective qui
contient les deux états de base afin de pouvoir 
caractériser les états d'entrée de polarisation 
entre nos états de base. L’un de ces états 
correspond au délai maximal appliqué, tandis 
que l'autre correspond à l'absence de délai. 
Ici, le terme délai fait référence à un 
signal extérieur qui active l'oscilloscope, 
comme dans l'expérience sur la vitesse de la 
lumière. La différence est que nous postulons 
que la manière la plus simple de créer des 
écarts temporels entre les états de base est 
d'utiliser un type d'interféromètre de 
polarisation dont l'un des bras est légèrement 
décalé d'une quantité correspondant à notre 
délai maximum par rapport à l'autre bras non 
décalé. La section suivante décrit le 
dispositif expérimental que nous avons utilisé 
pour caractériser la partie réelle de la valeur 
faible. 