\begin{doublespace}
    À partir de l'expérience de la partie réelle, notre interaction faible est réalisée en introduisant un décalage temporel faible entre les deux composantes polarisées du système. Ce décalage modifie la forme temporelle du paquet d’ondes (pointeur) de manière cohérente, sans perturber fortement l’état quantique et proportionnel à la valeur faible. Ces effets se manifestent dans le spectre d’interférence sous la forme d’un décalage de la position d’un des peignes de fréquences, causé par l’interaction faible. Comme le temps et la fréquence sont des quantités conjuguées au sens de Fourier, le principe d’incertitude d’Heisenberg se manifeste lors d’une interaction faible avec le système. 

    \noindent Pour observer ce spectre, on peut soit utiliser un interféromètre avec une résolution suffisante, soit un spectromètre. Cependant, ces deux méthodes ont leurs limites dans notre expérience. D’une part, un spectromètre offre une résolution suffisante, mais il est coûteux et peu pratique à intégrer dans une infrastructure photonique quantique intégrée. D’autre part, la faible longueur de cohérence du laser (entre $2$ et $8$ $\mu m$) limite l’interférométrie classique. Cela nécessiterait un système de translation piézoélectrique motorisé à haute précision, également onéreux.

    \noindent À la place, nous avons incorporé notre dispositif de mesure faible temporelle dans un interféromètre de Mach-Zehnder (MZ) afin de caractériser l’état de polarisation d’un système photonique. Voir la figure \ref{fig:imagexp}.

    \begin{figure}[!htpb]
        \centering
        \includegraphics[width=1.0\textwidth]{partieimagexp.png}
        \caption{Dispositif expérimental pour la partie imaginaire de la valeur faible}
        \label{fig:imagexp}
    \end{figure}

    \noindent Cette technique repose sur l’interférence entre deux impulsions émises par un seul laser, l’une ayant subi une mesure faible et l’autre servant de référence. Cette interférence génère une fréquence de battement, dont le spectre de puissance est modulé par l’interaction faible. L’information sur l’état de polarisation se retrouve dans la valeur faible.

    \noindent Nous commençons par régler l’interféromètre de MZ de manière à ce qu’il y ait une visibilité maximale avec le bras vertical (l’état $\ket{V}$) dans la partie mesure faible du dispositif, correspondant à l’absence de délai (interaction). Cela est illustré à la figure\ref{fig:imagexp}. Le spectre de visibilité obtenu lors de l’alignement avec ce dernier est présenté à la figure \ref{fig:vis}.

    \begin{figure}[!htpb]
        \centering
        \includegraphics[width=1.0\textwidth]{visibility_2.png}
        \caption{Dispositif expérimental pour la partie imaginaire de la valeur faible}
        \label{fig:vis}
    \end{figure}

    \noindent Nous avons décidé d’aligner l’interaction faible avec le troisième pique du spectre de visibilité de l’interféromètre. Cela signifie que nous plaçons l’état de polarisation horizontale ($\ket{H}$) sur ce pic, de sorte que l’effet du décalage temporel introduit soit maximal à cet endroit. Cette configuration nous permet ensuite d’obtenir une information sensible sur les états de polarisation circulaire, en particulier à partir des variations observées sur le deuxième pic du spectre. Cependant, le spectre de visibilité seul ne possède pas la résolution nécessaire pour détecter les petits décalages fréquentiels causés par l’interaction faible. Nous revenons donc dans le chapitre suivant sur une analyse spectrale plus poussée.

    \noindent Ensuite, on superpose les impulsions sous le même état de polarisation. Comme mentionné précédemment, le polariseur est réglé sur un état de polarisation verticale pendant notre mesure projective. À l'aide d'une lame demi-onde, nous pouvons autoriser des interférences avec le bras non faiblement mesuré dans le MZ et l’utiliser comme interférence de référence pour identifier quels pics sont causés par des interférences. 

    \noindent Nous cherchons à détecter les variations du spectre de puissance entre chaque état de polarisation, où un état de polarisation circulaire devrait entraîner le décalage fréquentiel maximal par rapport à un état de polarisation linéaire. Cependant, nous avons rencontré des difficultés qui seront abordées dans le chapitre suivant. 

\end{doublespace}

\subsubsection{Résultats attendus théoriques de la partie imaginaire de la valeur faible}

\begin{doublespace}
    
    \noindent Nous allons maintenant écrire les résultats attendus pour la partie imaginaire de la valeur faible pour chaque état d’entrée. 
    
    \begin{equation}
        \mathcal{I}\Bigl( \expval{\hat{\pi}_W} \Bigr) \equiv \frac{\expval{\hat{\omega}}}{\tau} \propto \ket{\psi_i}
    \end{equation}
    
    \noindent En commençant par $\ket{\psi_{i}^{1}}$ :

    \begin{equation}
        \expval{\hat{\omega}} = 0
    \end{equation}

    \noindent Ensuite, $\ket{\psi_{i}^{2}}$:

    \begin{equation}
        \expval{\hat{\omega}} = \frac{i\tau}{4\sigma^2}(cos(2\theta)sin(2\theta))
    \end{equation}

    \noindent Et finalement, $\ket{\psi_{i}^{3}}$:

    \begin{equation}
        \expval{\hat{\omega}} = \frac{\tau}{8\sigma^2}(sin^2(2\theta)-cos^2(2\theta))
    \end{equation}


    \noindent Considérons le terme proportionnel dans la partie imaginaire de la valeur faible:

    
    \begin{equation}
        \frac{\mathcal{I}(\expval{\hat{\pi}_W})}{\tau} = \expval{\hat{\omega}} \propto \frac{\tau}{4\sigma^2}
    \end{equation}
    
    \noindent Cela signifie que, pour une période temporaire de notre laser de 10 ns, associée à un retard de 167 PS, on doit observer un retard en fréquence de seulement 417,5 kHz. Cette valeur est extrêmement petite par rapport à la fréquence de notre laser, qui se situe dans les térahertz, et qui est difficile à mesurer. Les mesures interférométriques régulières effectuées dans un laboratoire ne présentent qu’une résolution en MHz, ce qui nécessite un délai extrêmement long, ce qui dépasse la marge de mesure minimale ainsi que sa facilité d’utilisation dans un laboratoire. En effet, pour effectuer la mesure dans un interféromètre de Michelson, les distances nécessaires dépassent la longueur de cohérence du laser de 0,2 mm. Des spectromètres ou d’autres méthodes photoniques, telles que les combs de fréquence, qui atteignent cette résolution sont vraiment coûteux qui contredit l’objectif de créer un dispositif dans un laboratoire commun pour la caractérisation d'un état quantique. 
    
    \noindent Cela résume nos approches expérimentales menées dans le cadre de ce projet de maîtrise. Le chapitre suivant présentera nos méthodes d'analyse, nos résultats et les implications que nous et d'autres pourrions rencontrer. 
\end{doublespace}