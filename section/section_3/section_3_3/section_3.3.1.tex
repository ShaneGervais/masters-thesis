\begin{doublespace}
    %À partir de l'expérience de la partie réelle, notre interaction 
    %faible est réalisée en introduisant un décalage temporel faible entre 
    %les deux composantes de la polarisation du système. Ce décalage modifie la 
    %forme temporelle du paquet d’ondes (pointeur) de manière cohérente, 
    %sans perturber fortement l’état quantique. Ces effets se manifestent 
    %dans le domain conjugué du domaine temporelle c'est à direle 
    %spectre d’interférence sous la forme d’un déplacement de la 
    %position des peignes de fréquences du spectre. 
    %Comme le temps et la fréquence sont des 
    %quantités conjuguées au sens de Fourier, le principe d’incertitude 
    %d’Heisenberg se manifeste lors d’une interaction faible avec le 
    %système. 

    %\noindent Pour observer ce spectre, on peut soit utiliser un 
    %spectromètre ou un interféromètre avec une résolution suffisante. 
    %Cependant, ces deux méthodes ont leurs limites dans notre expérience. 
    %D’une part, un spectromètre offre une résolution suffisante, mais 
    %il est coûteux et peu pratique à intégrer dans une infrastructure 
    %photonique quantique intégrée. D’autre part, la faible longueur de 
    %cohérence du laser (entre $2$ et $8$ $\mu m$) limite 
    %l’interférométrie classique. Cela nécessiterait un système de 
    %translation piézoélectrique motorisé à haute précision. À la place, 
    %nous avons incorporé notre dispositif de mesure faible temporelle 
    %dans un interféromètre de Mach-Zehnder (MZ) afin de caractériser 
    %l’état de polarisation d’un système photonique de manière hétérodyne
    %(voir la figure \ref{fig:imagexp}).

    \noindent Pour observer ce spectre, on peut soit utiliser un 
    spectromètre. Cependant, cette méthode peut offrir une 
    bonne résolution, mais il reste coûteux et difficilement intégrable 
    dans une plateforme photonique intégrée. 
    %En ce qui concerne 
    %l'interférométrie, la faible longueur de cohérence du laser (entre $2$ et $8$ $\mu m$) 
    %rend délicate l’observation d’interférences stables sans un système 
    %de translation piézoélectrique haute précision. 
    Donc, nous avons choisi d’intégrer notre dispositif de mesure faible 
    temporelle (figure \ref{fig:realexp}) 
    dans un interféromètre de 
    Mach-Zehnder (MZ) ce qui permet d’extraire le décallage 
    dans le domaine fréquentiel afin de caractériser 
    des états de polarisation à partir des variations d’intensité 
    en sortie avec sont spectre de puissance (figure \ref{fig:imagexp}).

    \begin{figure}[!htpb]
        \centering
        \includegraphics[width=1.0\textwidth]{partieimagexp.png}
        \caption{Dispositif expérimental pour la partie imaginaire 
        de la valeur faible. Cette configuration utilise le 
        dispositif de la partie réelle dans un interféromètre de 
        Mach-Zedner. Un séparateur de faisceau non polarisant 
        (NPBS) est utilisé avant l'étape de préparation de l'état 
        pour diviser le faisceau en deux voies. L’une de ces voies 
        sert de référence pour interférer avec l’autre voie, qui a 
        subi une mesure directe. Une lame demi-onde a été placée sur 
        le trajet de cette référence afin de modifier son état de 
        polarisation pour permettre des interférences avec le dernier.}
        \label{fig:imagexp}
    \end{figure}

    \noindent Cette technique repose sur l’interférence entre deux 
    impulsions émises par un seul laser, l’une ayant subi une mesure 
    faible (signal expérimental) et l’autre servant de référence (signal de référence)
    (figure \ref{fig:imagexp}). L’interférence entre ces deux signaux 
    modifie le spectre de fréquence du signal mesuré. L’information 
    sur l’état de polarisation se retrouve dans les 
    modifications de ce spectre. Cependant 
    la résolution de l’interféromètre doit être suffisante pour 
    distinguer les variations de fréquence causées par l’interaction 
    faible. Car que notre laser a une faible longueur de cohérence 
    (entre $2$ et $8$ $\mu m$) et qu'il y a plusieurs modes 
    fréquentiels \cite{ThorlabsNPL64B}, c'est difficile d'obtenir 
    une résolution suffisante pour observer directement les franges 
    d'interférence et distinguer les variations de fréquence causées par 
    l’interaction faible. Pour contourner ce problème, nous
    allons observer les variations de fréquence avec l'enveloppe
    de ce spectre, c'est-à-dire le spectre de puissance, qui est
    obtenu en effectuant une transformation de Fourier rapide (FFT)
    sur les données temporelles de l’interféromètre.

    \noindent Nous commençons par régler l’interféromètre de MZ de 
    manière à ce qu’il y ait une visibilité maximale avec le bras 
    correspondant à l'état de polarisation vertical $\ket{V}$ 
    dans la partie mesure faible du 
    dispositif, correspondant à l’absence de délai. Le 
    spectre de visibilité obtenu lors de l’alignement avec ce dernier 
    est présenté à la figure \ref{fig:vis}.

    \begin{figure}[!htpb]
        \centering
        \includegraphics[width=1.0\textwidth]{visibility_2.png}
        \caption{Spectre de visibilité de notre laser. }
        \label{fig:vis}
    \end{figure}

    \noindent Nous avons décidé d’aligner l’interaction faible avec le 
    troisième pique du spectre de visibilité de l’interféromètre. Cela 
    signifie que nous plaçons l’état de polarisation horizontale 
    ($\ket{H}$) sur ce pic, un déplacement du mirroir d'environ 
    $1.2$ $cm$ (allé-retour), de sorte que l’effet du décalage temporel 
    introduit soit maximal à cet endroit. Cette configuration nous permet 
    ensuite d’obtenir une information sensible sur les états de 
    polarisation circulaire, en particulier à partir des variations 
    observées sur le deuxième pic du spectre. Cependant, le spectre de 
    visibilité seul ne possède pas la résolution nécessaire pour détecter 
    les petits décalages fréquentiels causés par l’interaction faible. 
    Nous revenons donc dans le chapitre suivant sur une analyse spectrale 
    plus poussée.

    \noindent Ensuite, on superpose les impulsions sous le même état de 
    polarisation. Comme mentionné précédemment, le polariseur est réglé 
    sur un état de polarisation verticale pendant notre mesure projective. 
    À l'aide d'une lame demi-onde, nous pouvons autoriser des 
    interférences avec le bras non faiblement mesuré dans le MZ et 
    l’utiliser comme interférence de référence pour identifier quels 
    pics sont causés par des interférences. 

    \noindent Les données sont prélevées en mode FFT (transformation de 
    Fourier rapide) de l’oscilloscope sur plusieurs impulsions pendant 
    $400$ $ns$. On les moyenne ensuite à partir de $10$ séries de mesures. 
    Comme nous ne pourrons pas caractériser la partie imaginaire de la 
    valeur faible avec autant d'états d'entrée, comme la partie réelle, 
    nous allons vérifier si nous pouvons mesurer l’écart maximal attendu 
    entre des polarisations linéaires et circulaires qui est attendu par 
    la relation: $\expval{\hat{\omega}} \propto \frac{\tau}{8\sigma^2}$. 
    Les variations du spectre de puissance entre les états de polarisation 
    linéaires et circulaires, où un état de polarisation circulaire 
    devrait entraîner le décalage fréquentiel maximal par rapport à un 
    état de polarisation linéaire, devraient suivre cette relation. 
    Cependant, nous avons rencontré des difficultés qui seront abordées 
    dans le chapitre suivant. 

\end{doublespace}