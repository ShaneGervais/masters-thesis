\begin{doublespace}
    Nous proposons une expérience consistant à incorporer notre dispositif expérimental dans un interféromètre de Mach-Zedner (MZ), impliquant des effets interférométriques hétérodynes. Nous faisons interférer la fréquence bien connue de notre laser avec l’impulsion faiblement mesurée. 

    \begin{figure}[!htpb]
        \centering
        \includegraphics[width=1.0\textwidth]{partieimagexp.png}
        \caption{Dispositif expérimental pour la partie imaginaire de la valeur faible}
        \label{fig:imagexp}
    \end{figure}
    
    \noindent Nous commençons par faire interférer les impulsions sous le même état de polarisation. Comme mentionné précédemment, le polariseur est réglé sur un état de polarisation verticale pendant notre mesure projective. Par conséquent, grâce à une lame demi-onde, nous pouvons permettre des interférences avec le bras non faiblement mesuré dans le MZ. Nous cherchons à détecter les variations d’interférence entre chaque état de polarisation, où un état de polarisation circulaire devrait entraîner le décalage fréquentiel maximal par rapport à un état de polarisation linéaire. Cependant, nous avons rencontré des difficultés qui seront abordées dans le chapitre suivant. 

\end{doublespace}

\subsubsection{Résultats attendus théoriques de la partie imaginaire de la valeur faible}

\begin{doublespace}
    
    \noindent Nous allons maintenant écrire les résultats attendus pour la partie imaginaire de la valeur faible pour chaque état d’entrée. 
    
    \begin{equation}
        \mathcal{I}\Bigl( \expval{\hat{\pi}_W} \Bigr) \equiv \frac{\expval{\hat{\omega}}}{\tau} \propto \ket{\psi_i}
    \end{equation}
    
    \noindent En commençant par $\ket{\psi_{i}^{1}}$ :

    \begin{equation}
        \expval{\hat{\omega}} = 0
    \end{equation}

    \noindent Ensuite, $\ket{\psi_{i}^{2}}$:

    \begin{equation}
        \expval{\hat{\omega}} = \frac{i\tau}{4\sigma^2}(cos(2\theta)sin(2\theta))
    \end{equation}

    \noindent Et finalement, $\ket{\psi_{i}^{3}}$:

    \begin{equation}
        \expval{\hat{\omega}} = \frac{\tau}{8\sigma^2}(sin^2(2\theta)-cos^2(2\theta))
    \end{equation}


    \noindent Considérons le terme proportionnel dans la partie imaginaire de la valeur faible:

    
    \begin{equation}
        \frac{\mathcal{I}(\expval{\hat{\pi}_W})}{\tau} = \expval{\hat{\omega}} \propto \frac{\tau}{4\sigma^2}
    \end{equation}
    
    \noindent Cela signifie que, pour une période temporaire de notre laser de 10 ns, associée à un retard de 167 PS, on doit observer un retard en fréquence de seulement 417,5 kHz. Cette valeur est extrêmement petite par rapport à la fréquence de notre laser, qui se situe dans les térahertz, et qui est difficile à mesurer. Les mesures interférométriques régulières effectuées dans un laboratoire ne présentent qu’une résolution en MHz, ce qui nécessite un délai extrêmement long, ce qui dépasse la marge de mesure minimale ainsi que sa facilité d’utilisation dans un laboratoire. En effet, pour effectuer la mesure dans un interféromètre de Michelson, les distances nécessaires dépassent la longueur de cohérence du laser de 0,2 mm. Des spectromètres ou d’autres méthodes photoniques, telles que les combs de fréquence, qui atteignent cette résolution sont vraiment coûteux qui contredit l’objectif de créer un dispositif dans un laboratoire commun pour la caractérisation d'un état quantique. 
    
    \noindent Cela résume nos approches expérimentales menées dans le cadre de ce projet de maîtrise. Le chapitre suivant présentera nos méthodes d'analyse, nos résultats et les implications que nous et d'autres pourrions rencontrer. 
\end{doublespace}