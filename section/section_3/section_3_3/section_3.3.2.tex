
\begin{doublespace}
    
    Nous allons maintenant écrire les résultats attendus pour 
    la partie imaginaire de la valeur faible, comme nous l'avons fait avec 
    la partie réelle dans la section précédente, pour chaque trajet
    de polarisation. La relation de la partie imaginaire de la valeur faible
    est donnée par l'équation \ref{eq:imaginary_part}. Voici les 
    trajets de polarisation correspondant aux
    états d’entrée que nous avons utilisés pour la
    partie imaginaire de la valeur faible. 
    
    \noindent Nous commençons par l'état de polarisation linéaire
    $\ket{\psi_{i}^{1}} = cos(2\theta)\ket{H} + sin(2\theta)\ket{V}$
    qui est préparé par la lame demi-onde à $\theta$ et plaçons
    ses amplitudes de probabilité dans l'équation
    \ref{eq:imaginary_part} avec une postsélection sur l'état de polarisation
    $\ket{D}$, nous obtenons: 

    \begin{equation}
        \expval{\hat{\omega}} = 0
    \end{equation}

    \noindent ce qui signifie que l'état de polarisation linéaire ne
    devrait pas produire de décalage fréquentiel. Cela est
    attendu, car l'état de polarisation linéaire ne possède pas de 
    différence de phase dans son transport, contrairement aux trajets 
    suivants. Ensuite, pour un état de polarisation elliptique
    $\ket{\psi_{i}^{2}} = e^{\frac{-i\pi}{4}}cos(2\theta)\ket{H} + ie^{\frac{-i\pi}{4}} sin(2\theta)\ket{V}$
    préparé par la lame demi-onde à $\theta$ et une lame quart d'onde 
    à $0$ degré, nous avons:

    \begin{equation}
        \expval{\hat{\omega}} = \frac{\tau}{8\sigma^2}sin(4\theta)\label{eq:expval_omega_circular}
    \end{equation}

    \noindent ce type d'état de polarisation devrait produire un
    décalage fréquentiel proportionnel à l'état d'entrée et à la
    durée de l'interaction faible $\tau$. Enfin, pour l'état de polarisation
    de superposition 
    $\ket{\psi_{i}^{3}} = \frac{1}{2}(((1-i)sin(2\theta)+(i+1)cos(2\theta))\ket{H}
    + ((1+i)sin(2\theta)+(i-1)cos(2\theta))\ket{V})$
    préparé par la lame demi-onde à $\theta$ et une lame quart d'onde à $45$ degrés,
    nous avons:

    \begin{equation}
        \expval{\hat{\omega}} = -\frac{\tau}{8\sigma^2}sin(4\theta)
    \end{equation}

    \noindent ce qui signifie que l'état de polarisation de superposition
    devrait également produire un décalage fréquentiel proportionnel à l'état d'entrée
    et à la durée de l'interaction faible $\tau$.

    \noindent Cela résume nos approches et attendus expérimentaux
    menées dans le cadre de ce projet de maîtrise. Le chapitre suivant
    présentera nos méthodes d'analyse, nos résultats et les implications
    de ce projet.
\end{doublespace}