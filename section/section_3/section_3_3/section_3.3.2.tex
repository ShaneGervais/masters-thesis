\begin{doublespace}
    
    Nous allons maintenant écrire les résultats attendus pour 
    la partie imaginaire de la valeur faible, comme nous l'avons fait avec 
    la partie réelle dans la section précédente, pour chaque état 
    d’entrée. La relation de la partie imaginaire de la valeur faible 
    est donnée par le suivant :
    
    \begin{equation}
        \mathcal{I}\Bigl( \expval{\hat{\pi}_W} \Bigr) \equiv \frac{\expval{\hat{\omega}}}{\tau} \propto \ket{\psi_i}
    \end{equation}
    
    \noindent Voici les trajets de polarsation correspondant aux 
    états d’entrée que nous avons utilisés pour la 
    partie imaginaire de la valeur faible, qui sont calculés de la 
    même façon que dans la dernière section. Pour $\ket{\psi_{i}^{1}}$ :

    \begin{equation}
        \expval{\hat{\omega}} = 0
    \end{equation}

    \noindent Une absence de délai fréquenciel est attendue pour l'état 
    de polarisation linéaire. Ensuite, $\ket{\psi_{i}^{2}}$:

    \begin{equation}
        \expval{\hat{\omega}} = \frac{\tau}{8\sigma^2}(cos(2\theta)sin(2\theta))
    \end{equation}

    \noindent Et finalement, $\ket{\psi_{i}^{3}}$:

    \begin{equation}
        \expval{\hat{\omega}} = \frac{\tau}{8\sigma^2}(sin^2(2\theta)-cos^2(2\theta))
    \end{equation}

    \noindent Cela résume nos approches et attendus expérimentales 
    menées dans le cadre de ce projet de maîtrise. Le chapitre suivant 
    présentera nos méthodes d'analyse, nos résultats et les implications 
    de ce projet. 
\end{doublespace}