\begin{doublespace}
    Maintenant, analysons nos données de l'expérience de la vitesse d’un signal à travers un câble BNC à partir des délais d’impulsion. Nous avons configuré l’oscilloscope en mode EQ-time (temps d'équivalence), avec une durée de $100$ $ns$, une longueur d'enregistrement de $10 000$ points et une résolution de $4$ $ps$. Plus tard, lors d’expériences de mesures faibles, nous constaterons que nous n’avons pas besoin d’une telle durée. Nous commençons par observer l'impulsion typique du laser dans la figure \ref{fig:typical-pulse}. 
\end{doublespace}

\begin{figure}[hp]
    \centering
    \includegraphics[width=1.0\textwidth]{typ_pulse.png}
    \caption{Profil temporel typique de notre 
    impulsion laser ultra-courte NPL64B \cite{ThorlabsNPL64B}, mesurée avec un 
    photodétecteur DET025A à base de Si \cite{ThorlabsDET025A} et acquise 
    à l'aide du mode d'acquisition temporelle EQ-time 
    de l'oscilloscope \cite{TektronixTDS5000}.}
    \label{fig:typical-pulse}
\end{figure}

\begin{doublespace}
    \noindent Observez que le profil temporel des impulsions n’est pas une fonction de Gauss, mais plutôt une fonction de porte. En allongeant la durée de l’impulsion du laser, l’impulsion ressemble de plus en plus à une fonction de port au fur et à mesure. La raison pour laquelle nous souhaitons une forme gaussienne est qu'elle est simplement plus fréquente dans les mesures faibles ayant un sens physique plus naturel et qu'il est plus facile de déterminer la valeur moyenne de la position temporelle de l'impulsion (le moment le plus probable pour détecter un photon de cette impulsion) que nous définissons comme le pic. Par conséquent, notre objectif est de calculer numériquement la dérivée des données de l’impulsion, ce qui nous permettra de localiser précisément le pic et de le comparer à d'autres impulsions.
    
    \noindent Notez que le profil temporel des impulsions n’est pas une fonction gaussienne, mais plutôt une fonction porte. En prolongeant la durée de l’impulsion du laser, l’impulsion tend vers une fonction porte de plus en plus. Nous désirons une distribution gaussienne, car elle se révèle plus fréquente dans les mesures à faible intensité, présentant ainsi une signification physique plus naturelle. De plus, il est plus facile de déterminer la valeur moyenne de la position temporelle de l'impulsion (le moment le plus probable pour détecter un photon de cette impulsion), que nous nommons le \guillemetleft pic \guillemetright. Par conséquent, notre objectif est de calculer numériquement la dérivée des données de l’impulsion. La méthode des différences finies est utilisée comme type de dérivé numérique. Elle s’avère suffisante pour cette expérience. Ce dernier nous permettra de localiser précisément le pic en prenant un ajustement de courbe avec un domaine temporel plus fini pour la localisation du pic. Cette méthode nous permettra aussi de choisir un point temporel pour le temps d’arrivée, qui n’est pas nécessairement situé entre deux points de données. On comparera ensuite chacun de ces temps d’arrivée, pour chacune des longueurs de câble.

    \noindent À partir de là, l’ajustement de la courbe devient un aspect vraiment important dans la localisation du temps d’arrivée de notre signal. Nous effectuons de nombreux calculs numériques pour constater que l’option la plus adéquate était d’utiliser un pas de temps de $1/10000$ de la résolution des oscilloscopes pour l’axe des temps sur lequel nous effectuions l’ajustement. Nous avons remarqué qu’en négligeant environ $40 \%$ des données d’amplitude initiales, l’ajustement était plus susceptible d’être optimal. Le reste du signal ne présentait pas d’intérêt. On prend alors la position temporelle de ce pic et on l’ajoute dans un dictionnaire des temps d’arrivée, puis on la compare à notre câble de référence, qui est considéré comme notre origine, donc une position temporelle $0$ ou sans délai. Le tableau suivant, table \ref{table:BNC-fits}, montre les différents ajustements que nous avons essayés ainsi que le temps moyen d'arrivée pour chaque distance en utilisant ces ajustements. Il montre également l'écart-type de la façon dont la position du temps d'arrivée change pour chaque fichier individuel de cette même longueur de câble et le coefficient de détermination $R^2$ qui a également été utilisé comme paramètre pour déterminer la qualité de l'ajustement ainsi que la visualisation de tous ces ajustements. Nous avons sélectionné $5$ échantillons distincts pour chaque longueur de câble. Nous avions initialement prévu d’en prélever plus de $5$, mais nous avons réalisé que c'était excessif et qu'il suffisait d'en prélever au moins $3$. 

    
\end{doublespace}


\begin{longtable}{p{2.0cm} p{1.5cm} p{3.0cm} p{2.0cm} p{2.5cm}}
    \caption{Résultats des temps d'arrivées et écart-type de différent ajustement de courbe pour l'expérience de vitesse dans les câbles BNC} \\
    \toprule
    \label{table:BNC-fits}
    Type de fit & Longueur du câble (mm) & Temps d'arrivée (ns) & Écart-type (ns) & Qualité du fit \\
    \midrule
    \endfirsthead
    
    \toprule
    Type de fit & Longueur du câble (mm) & Temps d'arrivée (ns) & Écart-type (ns) & Qualité du fit \\
    \midrule
    \endhead
    
    \midrule
    \multicolumn{5}{r}{{Continued on next page}} \\
    \midrule
    \endfoot
    
    \bottomrule
    \endlastfoot
    
    % Paste all your rows here like:
    poly2     & 0            & 7.44310          & 0.00068       & 0.24743 \\
    poly2     & 172          & 8.37596          & 0.00105       & 0.24363 \\
    poly2     & 270          & 8.86583          & 0.00182       & 0.23058 \\
    poly2     & 522          & 10.12025         & 0.00104       & 0.24149 \\
    poly2     & 1032         & 12.68037         & 0.00099       & 0.24337 \\
    poly2     & 3000         & 22.62379         & 0.00354       & 0.24339 \\
    poly3     & 0            & 7.46342          & 0.00222       & 0.23412 \\
    poly3     & 172          & 8.39546          & 0.00278       & 0.23143 \\
    poly3     & 270          & 8.88637          & 0.00227       & 0.21571 \\
    poly3     & 522          & 10.14113         & 0.00196       & 0.22738 \\
    poly3     & 1032         & 12.70091         & 0.00150       & 0.23016 \\
    poly3     & 3000         & 22.64460         & 0.00396       & 0.23005 \\
    poly4     & 0            & 7.45750          & 0.00114       & 0.21840 \\
    poly4     & 172          & 8.38888          & 0.00177       & 0.21391 \\
    poly4     & 270          & 8.88005          & 0.00136       & 0.19823 \\
    poly4     & 522          & 10.13472         & 0.00099       & 0.20934 \\
    poly4     & 1032         & 12.69463         & 0.00132       & 0.21404 \\
    poly4     & 3000         & 22.63830         & 0.00336       & 0.21629 \\
    poly5     & 0            & 7.46484          & 0.00131       & 0.21682 \\
    poly5     & 172          & 8.39631          & 0.00106       & 0.21231 \\
    poly5     & 270          & 8.88844          & 0.00185       & 0.19591 \\
    poly5     & 522          & 10.14197         & 0.00119       & 0.20773 \\
    poly5     & 1032         & 12.70253         & 0.00211       & 0.21218 \\
    poly5     & 3000         & 22.64039         & 0.00294       & 0.21483 \\
    poly6     & 0            & 7.46521          & 0.00202       & 0.21675 \\
    poly6     & 172          & 8.39659          & 0.00153       & 0.21216 \\
    poly6     & 270          & 8.88736          & 0.00307       & 0.19582 \\
    poly6     & 522          & 10.14084         & 0.00182       & 0.20762 \\
    poly6     & 1032         & 12.69214         & 0.00663       & 0.21204 \\
    poly6     & 3000         & 22.29098         & 0.18052       & 0.21469 \\
    poly7     & 0            & 7.33751          & 0.03120       & 0.21499 \\
    poly7     & 172          & 8.21566          & 0.13047       & 0.21160 \\
    poly7     & 270          & 8.52569          & 0.11581       & 0.19487 \\
    poly7     & 522          & 9.72629          & 0.12391       & 0.20682 \\
    poly7     & 1032         & 12.23442         & 0.00780       & 0.21036 \\
    poly7     & 3000         & 22.16566         & 0.01151       & 0.21381 \\
    poly8     & 0            & 6.99619          & 0.00997       & 0.21454 \\
    poly8     & 172          & 7.92408          & 0.01360       & 0.21110 \\
    poly8     & 270          & 8.41125          & 0.01215       & 0.19425 \\
    poly8     & 522          & 9.66966          & 0.01255       & 0.20656 \\
    poly8     & 1032         & 12.23447         & 0.00776       & 0.21055 \\
    poly8     & 3000         & 22.16570         & 0.01148       & 0.21527 \\
    poly9     & 0            & 6.99605          & 0.00983       & 0.21387 \\
    poly9     & 172          & 7.92413          & 0.01359       & 0.21270 \\
    poly9     & 270          & 8.41132          & 0.01220       & 0.19766 \\
    poly9     & 522          & 9.66971          & 0.01257       & 0.20799 \\
    poly9     & 1032         & 12.23498         & 0.00810       & 0.21169 \\
    poly9     & 3000         & 22.16569         & 0.01146       & 0.22500 \\
    fourier1  & 0            & 7.44586          & 0.00075       & 0.22821 \\
    fourier1  & 172          & 8.37913          & 0.00125       & 0.22118 \\
    fourier1  & 270          & 8.86896          & 0.00128       & 0.20785 \\
    fourier1  & 522          & 10.12332         & 0.00084       & 0.21928 \\
    fourier1  & 1032         & 12.68338         & 0.00115       & 0.22325 \\
    fourier1  & 3000         & 22.62676         & 0.00359       & 0.22515 \\
    fourier2  & 0            & 7.46383          & 0.00189       & 0.21728 \\
    fourier2  & 172          & 8.39629          & 0.00139       & 0.21263 \\
    fourier2  & 270          & 8.88716          & 0.00276       & 0.19641 \\
    fourier2  & 522          & 10.14204         & 0.00308       & 0.20802 \\
    fourier2  & 1032         & 12.70221         & 0.00217       & 0.21270 \\
    fourier2  & 3000         & 22.64465         & 0.00462       & 0.21526 \\
    fourier3  & 0            & 7.41817          & 0.06115       & 0.21482 \\
    fourier3  & 172          & 8.37827          & 0.00987       & 0.21003 \\
    fourier3  & 270          & 8.87882          & 0.01193       & 0.19474 \\
    fourier3  & 522          & 10.12619         & 0.01515       & 0.20527 \\
    fourier3  & 1032         & 12.68675         & 0.01349       & 0.21032 \\
    fourier3  & 3000         & 22.61673         & 0.00260       & 0.20832 \\
    fourier4  & 0            & 7.42123          & 0.00635       & 0.21111 \\
    fourier4  & 172          & 8.34673          & 0.00717       & 0.20424 \\
    fourier4  & 270          & 8.84979          & 0.00872       & 0.19021 \\
    fourier4  & 522          & 10.10293         & 0.00667       & 0.20169 \\
    fourier4  & 1032         & 12.66846         & 0.00449       & 0.20740 \\
    fourier4  & 3000         & 22.60242         & 0.00497       & 0.20594 \\
    fourier5  & 0            & 7.42623          & 0.00692       & 0.21058 \\
    fourier5  & 172          & 8.36756          & 0.05475       & 0.20214 \\
    fourier5  & 270          & 8.85968          & 0.01496       & 0.18814 \\
    fourier5  & 522          & 10.10500         & 0.00430       & 0.19736 \\
    fourier5  & 1032         & 12.67241         & 0.00725       & 0.20597 \\
    fourier5  & 3000         & 22.52103         & 0.19931       & 0.20521 \\
    fourier6  & 0            & 7.43023          & 0.01046       & 0.21021 \\
    fourier6  & 172          & 8.41189          & 0.06274       & 0.20157 \\
    fourier6  & 270          & 8.85976          & 0.01835       & 0.18777 \\
    fourier6  & 522          & 10.19384         & 0.19938       & 0.19661 \\
    fourier6  & 1032         & 12.76117         & 0.19404       & 0.20516 \\
    fourier6  & 3000         & 22.52226         & 0.19765       & 0.20480 \\
    gauss1    & 0            & 7.44584          & 0.00059       & 0.22856 \\
    gauss1    & 172          & 8.37897          & 0.00111       & 0.22152 \\
    gauss1    & 270          & 8.86887          & 0.00131       & 0.20805 \\
    gauss1    & 522          & 10.12319         & 0.00082       & 0.21959 \\
    gauss1    & 1032         & 12.68330         & 0.00116       & 0.22352 \\
    gauss1    & 3000         & 22.62690         & 0.00346       & 0.22570
\end{longtable}


\begin{doublespace}
    
    \noindent En plus de la vérification de l'ajustement présentant le moins d'écart entre les fichiers pour chaque longueur de câble possible, le tableau suivant, table \ref{table:speed-BNC-table}, présente les différents types d'ajustement, en testant également la précision de la vitesse des mesures du signal en fonction des ajustements trouvés pour le temps d'arrivée. La vitesse est déterminée en effectuant un ajustement linéaire sur les valeurs moyennes du temps d'arrivée, pour chaque longueur de câble. Nous comparons ensuite ces données à la valeur théorique, qui est d’environ $0,66 c$ \cite{arrl2019}. 

    \noindent Nous en concluons que, selon nos deux tableaux, le meilleur ajustement possible est un polynôme du troisième ordre, une série de Fourier du premier ou du deuxième ordre, ainsi qu’un ajustement gaussien. Nous avons choisi un ajustement gaussien comme type d’ajustement à utiliser, car il présente un bon équilibre des écarts-types à travers des différents fichiers, \ref{table:BNC-fits}. Cela est optimal pour obtenir la meilleure résolution possible pour les procédures directes via mesure faible, ainsi que la forme typique qu’on retrouve. Il présente également un bon pourcentage d’erreur par rapport à la théorie. Vous pouvez trouver les mêmes tableaux que nous avons créés pour mesurer la vitesse de la lumière sur différentes longueurs de miroir à l’annexe A, table \ref{table:AnnexeFits}. 
\end{doublespace}

\begin{longtable}{p{2.0cm} p{2.0cm} p{2.0cm} p{2.0cm} p{2.0cm}}
    \caption{Mesure de la vitesse du signal dans les câbles BNC pour différent ajustement de courbe} \\
    \toprule
    \label{table:speed-BNC-table}
    Type de fit & Vitesse mesurée (m/s) & Vitesse théorique (m/s) & Erreur (\%) & Qualité du fit \\
    \midrule
    \endfirsthead
    
    \toprule
    Type de fit & Vitesse mesurée (m/s) & Vitesse théorique (m/s) & Erreur (\%) & Qualité du fit \\
    \midrule
    \endhead
    
    \midrule
    \multicolumn{5}{r}{{Continued on next page}} \\
    \midrule
    \endfoot
    
    \bottomrule
    \endlastfoot
    
    % Paste all your rows here like:
    poly2     & 198125399     & 197863022           & 0.1326      & 0.0303       \\
    poly3     & 198116966     & 197863022           & 0.1283      & 0.0301       \\
    poly4     & 198117476     & 197863022           & 0.1286      & 0.0298       \\
    poly5     & 198191774     & 197863022           & 0.1662      & 0.0301       \\
    poly6     & 203054996     & 197863022           & 2.6240      & 0.0637       \\
    poly7     & 201780752     & 197863022           & 1.9800      & 0.1358       \\
    poly8     & 198220958     & 197863022           & 0.1809      & 0.0271       \\
    poly9     & 198220258     & 197863022           & 0.1805      & 0.0271       \\
    fourier1  & 198125807     & 197863022           & 0.1328      & 0.0304       \\
    fourier2  & 198124682     & 197863022           & 0.1322      & 0.0302       \\
    fourier3  & 198150017     & 197863022           & 0.1450      & 0.0454       \\
    fourier4  & 198105100     & 197863022           & 0.1223      & 0.0295       \\
    fourier5  & 199361343     & 197863022           & 0.7573      & 0.0386       \\
    fourier6  & 199614259     & 197863022           & 0.8851      & 0.0779       \\
    gauss1    & 198122675     & 197863022           & 0.1312      & 0.0304 
\end{longtable}

\begin{doublespace}
    \noindent Discutons nos données dans une façon plus visuelle. Sur la figure \ref{fig:BNC-pulse}, on voit chaque impulsion provenant de différentes longueurs de câble BNC. Cette expérience ne mesure pas seulement la vitesse de la lumière dans les câbles BNC, elle teste aussi nos paramètres d’ajustement, puisque nous avons utilisé les mêmes paramètres pour la vitesse de la lumière pour les miroirs ajustables ainsi que nos mesure faibles. 
\end{doublespace}


\begin{figure}[h]
    \centering
    \includegraphics[width=1.0\textwidth]{overlay_pulses_with_fits_BNC.png}
    \caption{Profil temporel de la dérivée des données d'impulsion dans l'expérience des câbles BNC RG-58 avec chacun de ses ajustements de courbe. }
    \label{fig:BNC-pulse}
\end{figure}


\begin{doublespace}
    \noindent Nous procédons à l’ajustement des données en leur appliquant une fonction gaussienne, qui s’écrit comme suit :
\end{doublespace}

\begin{equation}
    y(t) = a_0 e^{-(\frac{t-b_0}{c_0})^2}
\end{equation}

\begin{doublespace}
    \noindent Les paramètres d’ajustement $a_0$, $b_0$ et $c_0$ de la fonction avec variable $y$, qui représente l’amplitude, et $t$, qui correspond à la position temporelle des courbes, sont sélectionnés pour optimiser l’ajustement de nos données. Parmi ces données, $40\%$ des points de l’axe d’amplitude et de l’axe temporel sont ignorés. Ces paramètres correspondent le mieux à nos données. Il est difficile d’attribuer une valeur numérique pour évaluer la qualité du réglage de notre courbe, puisque celui-ci a principalement découlé d’une analyse visuelle. Nous avons néanmoins utilisé le coefficient de détermination ($R^2$) comme boussole, mais nous avons tenté d'éviter un ajustement excessif. Ce réglage nous permet maintenant d’identifier la position optimale, qui correspond à une position réelle observée dans nos données. Ensuite, nous comparons chaque position temporelle à celle des distances de référence pour obtenir les délais mesurés pour notre expérience. Ces délais sont tracés en fonction de la distance associée, et, par ajustement linéaire de la courbe, nous pouvons déterminer que la pente correspond à la vitesse du signal. 
\end{doublespace}

\begin{figure}[h]
    \centering
    \includegraphics[width=1.0\textwidth]{speed_of_light_BNC.png}
    \caption{Délais mesurés pour la longueur du câble BNC eux avec son ajustement de courbe.}
    \label{fig:BNC-res}
\end{figure}

\begin{doublespace}
    \noindent Notre résultat pour la vitesse du signal est $197 863 022$ $m/s$, ce qui représente une erreur en pourcentage de $0.1312\%$ correspondant que la vitesse du sginal dans un câble BNC possède un différentes de $0.66$ par rapport à $c$, table \ref{table:speed-BNC-table}, ce qui est réaliste et cohérent avec l'erreur observée lors de l'expérience précédente \cite{arrl2019,thorlabs2021}. En effet, nous avons donc démontré notre capacité à mesurer des délais très précis. C’est un élément essentiel pour pouvoir commencer à mesurer de petits délais entre des états polarisation d'entrée de changement dans le biais de mesures faibles.
\end{doublespace}

\begin{doublespace}
    \noindent Voici les résultats des données issues des impulsions de l’expérience du miroir déplacé, figure \ref{fig:speed-pulses}. Ces données dérivées montrent clairement une forme gaussienne avec un pic maximal nettement visible. Cela facilite grandement son identification de sa position temporelle. 
\end{doublespace}

\begin{figure}[!hptb]
    \centering
    \includegraphics[width=1.0\textwidth]{overlay_pulses_mirror.png}
    \caption{Profil temporel de la dérivée des données d'impulsion pour chacune des distances mesurées et ajustement de la courbe pour l'expérience de la vitesse de la lumière avec un miroir réglable.}
    \label{fig:speed-pulses}
\end{figure}

\begin{doublespace}
    \noindent Le résultat de notre expérience sur la vitesse de la lumière est de $296 991 901$ $m/s$ avec une marge d'erreur de $0.91$ $\%$ par rapport à la valeur actuelle, table \ref{table:AnnexeASpeed}. Cela correspond à une différence de $0,9998$ par rapport à la vitesse de la lumière $c$ dans l’air ($c_{air} = 0.9998c$) \cite{hecht2012optics}.
\end{doublespace}

\begin{figure}[!hptb]
    \centering
    \includegraphics[width=1.0\textwidth]{speed_of_light_mirror.png}
    \caption{Résultats des délais mesurés pendant l'expérience sur la vitesse de la lumière, ainsi que leur ajustement linéaire. Les barres d’erreur horizontales représentent l’incertitude de nos mesures de la distance, soit $\pm 0,5$ $mm$, qui est trop petite pour être visible sur le graphique. Les barres d'erreurs verticales correspondent à l'erreur de l'ajustement Fourier basé sur l'expérience de la vitesse de la lumière dans les câbles BNC $\pm 0,03\%$.}
    \label{fig:speed-res}
\end{figure}

\begin{doublespace}
    \noindent À partir des résultats de ces deux expériences, nous concluons que nous pouvons mesurer des délais temporels très courts, de l'ordre de 2 ps avec une variation inférieure à 1 ps entre les ensembles de données. Cette résolution est suffisante pour mettre en œuvre notre proposition de mesure du temps. Les sections suivantes seront consacrées à la caractérisation des états de polarisation par un petit décalage temporel entre les états de base. 
\end{doublespace}





