Dans cette section, nous discuterons des 
résultats et de l'analyse de l'expérience sur 
la vitesse de la lumière. Commençons par 
examiner la forme d'impulsion typique de notre 
laser. Celle-ci est acquise avec les paramètres 
de l'oscilloscope réglés sur le mode EQ-time, 
une durée de 100ns, une longueur de rec de 10000 
avec une résolution de 4ps. Nous nous rendrons 
compte plus tard, lors d'expériences de mesures 
faibles, que nous n'avons pas besoin d'une telle 
durée. Remarquez que le profil temporel des 
impulsions n'est pas essentiellement une 
fonction de gaussien mais plutôt une fonction de 
port. 

\begin{figure}[hp]
    \centering
    \includegraphics[width=1.0\textwidth]{typ_pulse.png}
    \caption{Profil temporel typique de notre 
    impulsion laser ultra-courte NPL64B 
    fabriquée par thorlabs, mesurée avec un 
    photomultiplicateur à base de Si et acquise 
    à l'aide du mode d'acquisition temporelle EQ 
    de l'oscilloscope.}
    \label{fig:typical-pulse}
\end{figure}

L'impulsion ressemble de plus en plus à 
une fonction de port au fur et à mesure que nous 
augmentons la durée de l'impulsion. Une figure 
de thorlabs montre comment la forme de 
l'impulsion change avec l'augmentation de la 
largeur de l'impulsion. La raison pour laquelle 
nous voulons une forme gaussienne est qu'elle 
est simplement plus fréquente dans les mesures 
faibles et qu'il est plus facile de trouver la 
valeur d'espérance temporelle de l'impulsion 
(le moment le plus probabiliste pour trouver un 
photon de cette impulsion) que nous déterminons 
comme étant le pic. Ainsi, nous voulons prendre 
la dérivée des données de l'impulsion 
numériquement, ce qui nous donne une vision 
claire de l'endroit où se trouve le pic et qui 
est cohérent avec d'autres impulsions.

\begin{figure}[hp]
    \centering
    \includegraphics[width=1.0\textwidth]{speed_pulses.png}
    \caption{Profil temporel des données 
    d'impulsion pour chacune des distances 
    mesurées et ajustement de la courbe pour 
    l'expérience de la vitesse de la lumière 
    avec un miroir ajustable.}
    \label{fig:speed-pulses}
\end{figure}

Voici les 
données numériques dérivées des impulsions de 
l'expérience du miroir déplacé qui montrent la 
forme de l'impulsion après la dérivée avec un 
maximum clair qui peut être trouvé et comment il 
suit clairement une tendance linéaire suspecte 
d'augmentation du retard avec l'augmentation de 
la distance du miroir. Nous soumettons les 
données à un ajustement de courbe en utilisant 
une fonction de série de Fourier du 2e ordre, 
soit :

\begin{equation}
    y(t) = a_0 + \sum_{j=1}^{n} a_{j}cos(jwt) + b_{j}sin(jwt)
\end{equation}

Soit les paramètres d'ajustement $a_i$, $b_i$ et $w$ 
de la série de Fourier, $n$ est l'ordre ($n=2$) 
pour les variables $y$ soit l'amplitude
et $t$ la position temporel des courbes sont 
choisis pour optimiser l’ajustement de nos 
données dont $40$ $\%$ des points de l'axe 
d'amplitude et de l'axe temporel sont ignorés. 
Ce sont ces paramètres qui correspondent le 
mieux à nos données. Il n'y a pas vraiment de 
valeur quantifiée pour évaluer la qualité de 
l'ajustement de notre courbe, car il a été 
effectué principalement à partir d’une analyse 
visuelle. Le coefficient de détermination ($R^2$) 
a également été utilisé pour nous guider, mais 
nous avons tenté d'éviter un ajustement excessif. 
Grâce à cet ajustement, nous pouvons identifier 
la position maximale, laquelle correspond à une 
position réelle mesurée dans nos données. Cette 
position maximale est notre valeur prédite pour 
la position temporelle moyenne des photons pour 
la distance mesurée. Ensuite, chaque position 
temporelle est comparée à celle des distances de 
référence, ce qui donne les retards mesurés pour 
notre expérience. Ces délais sont ensuite tracés 
avec la distance associée, et, par ajustement 
linéaire de la courbe, nous pouvons déterminer 
que la pente correspond à la vitesse de la 
lumière. 

\begin{figure}[hp]
    \centering
    \includegraphics[width=1.0\textwidth]{speed_results.png}
    \caption{Résultats des retards mesurés 
    lors de l'expérience sur la vitesse de la 
    lumière et leur ajustement linéaire}
    \label{fig:speed-res}
\end{figure}

Notre résultat pour la vitesse de la 
lumière de cette expérience est $296 297 418,89$ $m/s$ 
avec un pourcentage d'erreur de $1,15$ $\%$. 
Cela correspond à une différence de $0,9998$ par 
rapport à la vitesse de la 
lumière $c$ dans l’air ($c_{air} = 0.9998c$) \cite{hecht2012optics}. 
La raison pour laquelle la vitesse de la lumière 
est plus lente est ce que nous pensons être due 
à l'alignement de notre installation. En effet, 
selon notre erreur, cet écart correspondrait à 
environ $1 cm$, ce qui est plausible. Voici 
maintenant le résultat de l'expérience avec le 
câble BNC. 

\begin{figure}[h]
    \centering
    \includegraphics[width=1.0\textwidth]{BNC_pulses.png}
    \caption{Profil temporel des données 
    d'impulsion de la vitesse de la lumière dans 
    l'expérience des câbles BNC RG-58 avec 
    chacun de ses ajustements de courbe. }
    \label{fig:BNC-pulse}
\end{figure}

Sur la figure \ref{fig:BNC-pulse}, on voit chaque 
impulsion provenant de différentes longueurs de 
câble BNC. Cette expérience ne mesure pas 
seulement la vitesse de la lumière dans les 
câbles BNC, elle teste aussi nos paramètres 
d’ajustement, puisque nous avons utilisé les 
mêmes paramètres. 

\begin{figure}[h]
    \centering
    \includegraphics[width=1.0\textwidth]{BNC_results.png}
    \caption{Retards mesurés pour la longueur 
    du câble BNC eaxh avec son ajustement de 
    courbe.}
    \label{fig:BNC-res}
\end{figure}

Notre résultat pour la vitesse 
de la lumière est $195 185 070.31$ $m/s$, ce qui 
représente une erreur en pourcentage de $1,35\%$
correspondant que la vitesse de la lumière dans un 
cable BNC possède un différentes de $0.66$ 
par rapport à $c$, soit une erreur de 1cm, 
ce qui est réaliste et cohérent avec l'erreur dans 
l'expérience précédant \cite{arrl2019,thorlabs2021}. 
Nous avons donc démontré notre capacité à 
mesurer des retards très précis. C’est un 
élément essentiel pour pouvoir commencer à 
mesurer de petits retards dans les états 
d'entrée de changement de polarisation par le 
biais de mesures faibles.



