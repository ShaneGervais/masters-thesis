\begin{doublespace}
    Dans cette section, nous discuterons des résultats et de l’analyse de l’expérience sur la vitesse de la lumière. Pour commencer, examinons la forme d’impulsion typique de notre laser.  
\end{doublespace}

\begin{figure}[hp]
    \centering
    \includegraphics[width=1.0\textwidth]{typ_pulse.png}
    \caption{Profil temporel typique de notre 
    impulsion laser ultra-courte NPL64B \cite{ThorlabsNPL64B}, mesurée avec un 
    photodétecteur DET025A à base de Si \cite{ThorlabsDET025A} et acquise 
    à l'aide du mode d'acquisition temporelle EQ-time 
    de l'oscilloscope \cite{TektronixTDS5000}.}
    \label{fig:typical-pulse}
\end{figure}

\begin{doublespace}
    \noindent Nous avons configuré l’oscilloscope en mode EQ-time (temps d'équivalence), avec une durée de $100$ $ns$, une longueur d'enregistrement de $10 000$ points et une résolution de $4$ $ps$. Plus tard, lors d’expériences de mesures faibles, nous réaliserons que nous n’avons pas besoin d’une telle durée. Remarquez que le profil temporel des impulsions n’est pas essentiellement une fonction de Gauss, mais plutôt une fonction de porte.
    \noindent En allongeant la durée de l’impulsion du laser, l’impulsion ressemble de plus en plus à une fonction de port au fur et à mesure. La raison pour laquelle nous souhaitons une forme gaussienne est qu'elle est simplement plus fréquente dans les mesures faibles et qu'il est plus facile de déterminer la valeur moyenne de la position temporelle de l'impulsion (le moment le plus probable pour détecter un photon de cette impulsion) que nous définissons comme le pic. Par conséquent, notre objectif est de calculer numériquement la dérivée des données de l’impulsion, ce qui nous permettra de localiser précisément le pic et de le comparer à d'autres impulsions.
    \noindent Voici les résultats des données issues des impulsions de l’expérience du miroir déplacé, figure \ref{fig:speed-pulses}. Ces données dérivées montrent clairement une forme gaussienne avec un pic maximal nettement visible. Cela facilite grandement son identification de sa position temporelle. 
\end{doublespace}

\begin{figure}[!hptb]
    \centering
    \includegraphics[width=1.0\textwidth]{overlay_pulses_mirror.png}
    \caption{Profil temporel de la dérivée des données d'impulsion pour chacune des distances mesurées et ajustement de la courbe pour l'expérience de la vitesse de la lumière avec un miroir réglable.}
    \label{fig:speed-pulses}
\end{figure}

\begin{doublespace}
    \noindent Nous procédons à l’ajustement des données en leur appliquant une fonction de série de Fourier du 2e ordre, qui s’écrit comme suit :
\end{doublespace}

\begin{equation}
    y(t) = a_0 + \sum_{j=1}^{n} a_{j}cos(jwt) + b_{j}sin(jwt)
\end{equation}

\begin{doublespace}
    \noindent Les paramètres d’ajustement $a_i$, $b_i$ et $w$ de la série de Fourier, où $n$ est l’ordre (fixé à 2) pour les variables $y$, qui représente l’amplitude, et $t$, qui correspond à la position temporelle des courbes, sont sélectionnés pour optimiser l’ajustement de nos données. Parmi ces données, $40\%$ des points de l’axe d’amplitude et de l’axe temporel sont ignorés. Ces paramètres correspondent le mieux à nos données. Il est difficile d’attribuer une valeur numérique pour évaluer la qualité du réglage de notre courbe, puisque celui-ci a principalement découlé d’une analyse visuelle. Nous avons néanmoins utilisé le coefficient de détermination ($R^2$) comme boussole, mais nous avons tenté d'éviter un ajustement excessif. Ce réglage nous permet maintenant d’identifier la position optimale, qui correspond à une position réelle observée dans nos données. Ensuite, nous comparons chaque position temporelle à celle des distances de référence pour obtenir les délais mesurés pour notre expérience. Ces délais sont tracés en fonction de la distance associée, et, par ajustement linéaire de la courbe, nous pouvons déterminer que la pente correspond à la vitesse de la lumière. 
\end{doublespace}

\begin{figure}[!hptb]
    \centering
    \includegraphics[width=1.0\textwidth]{speed_of_light_mirror.png}
    \caption{Résultats des délais mesurés pendant l'expérience sur la vitesse de la lumière, ainsi que leur ajustement linéaire. Les barres d’erreur horizontales représentent l’incertitude de nos mesures de la distance, soit $\pm 0,5$ $mm$, qui est trop petite pour être visible sur le graphique. Les barres d'erreurs verticales correspondent à l'erreur de l'ajustement Fourier basé sur l'expérience de la vitesse de la lumière dans les câbles BNC $\pm 0,03\%$.}
    \label{fig:speed-res}
\end{figure}

\begin{doublespace}
    \noindent Le résultat de notre expérience sur la vitesse de la lumière est de $296 991 901$ $m/s$ avec une marge d'erreur de $0.91$ $\%$ par rapport à la valeur actuelle. Cela correspond à une différence de $0,9998$ par rapport à la vitesse de la lumière $c$ dans l’air ($c_{air} = 0.9998c$) \cite{hecht2012optics}. Nous disons que la raison pour laquelle la vitesse de la lumière est plus lente est due à l’alignement de notre expérience, une erreur commune et probable. En effet, selon notre erreur, cet écart correspondrait à environ $\approx 1$ $cm$ $=33$ $ps$, ce qui est plausible. Voici maintenant le résultat de l'expérience avec le câble BNC. 
\end{doublespace}

\begin{figure}[h]
    \centering
    \includegraphics[width=1.0\textwidth]{overlay_pulses_with_fits_BNC.png}
    \caption{Profil temporel de la dérivée des données d'impulsion dans l'expérience des câbles BNC RG-58 avec chacun de ses ajustements de courbe. }
    \label{fig:BNC-pulse}
\end{figure}

\begin{doublespace}
    \noindent Sur la figure \ref{fig:BNC-pulse}, on voit chaque impulsion provenant de différentes longueurs de câble BNC. Cette expérience ne mesure pas seulement la vitesse de la lumière dans les câbles BNC, elle teste aussi nos paramètres d’ajustement, puisque nous avons utilisé les mêmes paramètres pour la vitesse de la lumière pour les miroirs ajustables. 
\end{doublespace}

\begin{figure}[h]
    \centering
    \includegraphics[width=1.0\textwidth]{speed_of_light_BNC.png}
    \caption{Délais mesurés pour la longueur du câble BNC eux avec son ajustement de courbe.}
    \label{fig:BNC-res}
\end{figure}

\begin{doublespace}
    \noindent Notre résultat pour la vitesse de la lumière est $197 946 443$ $m/s$, ce qui représente une erreur en pourcentage de $0,04\%$ correspondant que la vitesse de la lumière dans un câble BNC possède un différentes de $0.66$ par rapport à $c$, soit une erreur de 1cm, ce qui est réaliste et cohérent avec l'erreur observée lors de l'expérience précédente \cite{arrl2019,thorlabs2021}. En effet, nous avons donc démontré notre capacité à mesurer des délais très précis. C’est un élément essentiel pour pouvoir commencer à mesurer de petits délais entre des états polarisation d'entrée de changement dans le biais de mesures faibles.
\end{doublespace}



