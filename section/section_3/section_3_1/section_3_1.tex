\begin{doublespace}

    Nous souhaitons évaluer notre capacité à mesurer des délais 
    temporels avec précision. À cet effet, nous allons déterminer 
    notre précision de mesure temporelle en mesurant la vitesse d'un
    signal électrique qui se déplace dans un câble BNC de différentes
    longueurs. Nous utiliserons ensuite cette méthode pour
    mesurer la vitesse de la lumière dans un système photonique,
    qui nous permettra de tester notre précision de mesure temporelle
    dans un système plus complexe. Ces expériences nous permettront
    de caractériser notre capacité à mesurer des délais
    temporels avec précision, ce qui est essentiel pour la
    caractérisation des états de polarisation avec une mesure faible
    temporelle. Dans la figure \ref{fig:speed-of-light}, nous 
    présentons le dispositif expérimental que nous utiliserons pour 
    les deux prochaines expériences. 

    \begin{figure}[!hpbt]
    \centering
    \includegraphics[width=1.0\textwidth]{speed_of_light_exp.png}
    \caption{
        Représentation de notre dispositif expérimental pour 
    évaluer la précision de nos mesures temporelles en mesurant la 
    vitesse d'un signal dans des câbles BNC et la vitesse de la lumière. 
    L’impulsion du laser est d’abord réglée en 
    intensité par une lame demi-onde et un séparateur 
    de faisceau polarisant (PBS) qui divise les états de polarisation 
    horizontaux et verticaux en deux 
    voies orthogonales. Le faisceau réfléchi, représenté par l’état de base de 
    polarisation verticale $\ket{V}$, sera notre 
    signal de référence pour déclencher l’oscilloscope. 
    Le faisceau transmis, correspondant à un état de polarisation
    horizontal $\ket{H}$ est le faisceau que nous allons mesurer, le signal
    expérimental.
    Il passe par un autre séparateur de faisceau polarisant (PBS)
    qui le divise en deux voies, la voie réfléchie ne sera pas mesurée 
    car elle ne présente pas d'intérêt,
    tandis que la voie transmise est dirigée vers un miroir fixe qu'on déplace
    selon les différentes distances à évaluer. Il
    est ensuite converti en un état de polarisation verticale à l'aide
    d'une lame quart d'onde pour être réfléchi
    vers le photodétecteur pour mesurer son temps d'arrivée.
    Pour l'expérience de la vitesse d'un signal dans un câble BNC, nous 
    utilisons simplement différentes longueurs de câble attachés 
    sur notre photodétecteur, désignées par $\Delta L$. Pour l'expérience de la 
    vitesse de la lumière, nous
    plaçons le miroir à des différentes distances, désignée par $\Delta x$.
    }
    \label{fig:speed-of-light}
\end{figure}

    \noindent Notre laser est une source à diode pulsée
    émettant des impulsions de l'ordre de la nanoseconde, 
    de $5$ à $39$ $ns$ \cite{ThorlabsNPL64B}.
    Nous utilisons une impulsion de $10$ $ns$
    pour nos expériences, car les intervalles de temps plus longs  
    rendent la mesure moins précise. Cette perte de précision se présente 
    dans la mesure de la partie imaginaire. À ces longueurs d'impulsion,
    la distribution temporelle du signal devient plus similaire à celle d’une
    fonction porte \cite{ThorlabsNPL64B}. Or, les discontinuités de cette
    fonction entrainent la présence de bruit à hautes fréquences ce qui
    rend l'analyse moins facile. Donc, nous avons opté à identifier le 
    temps d'arrivée de l'impulsion comme la variable temporelle moyenne
    du pointeur pour éviter ce bruit et faciliter l'analyse.
    Le laser possède une longueur d’onde comprise entre 
    $640 \pm 10$ $nm$, avec une énergie d’impulsion maximale de 
    $2,0$ $nJ$. Sa puissance de crête s’élève à un maximum de 
    $50$ $mW$ avec une puissance moyenne de $20$ $mW$. 
    Nous fixons le taux de répétition à $1$ $MHz$, pour avoir un temps 
    suffisamment long entre les impulsions. 
    
    \noindent Nous recueillons les
    données à l'aide d'un photodétecteur rapide à base de Si 
    \cite{ThorlabsDET025A}, qui possède une bande passante de 
    $2$ $GHz$, ce qui est suffisant pour mesurer les impulsions de
    $10$ $ns$. Le photodétecteur est connecté à un câble BNC de 
    $50$ $\Omega$ \cite{ThorlabsBNC} qui est ensuite connecté à 
    un oscilloscope pour l'enregistrement des données.
    Nous utilisons un oscilloscope Tektronix TDS 5000 \cite{TektronixTDS5000}, 
    qui est un oscilloscope capable d’acquérir des signaux à 
    des vitesses allant jusqu’à 
    $500$ $GS/s$ (gigéchantillons par seconde), et qui possède une 
    bande passante de $500$ $MHz$.  
    Ensuite, nous analysons les données avec \textsc{MATLAB} et \textsc{Python}
    pour en tirer des conclusions sur la précision de nos mesures
    temporelles.
    

\end{doublespace}

