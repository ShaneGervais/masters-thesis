\begin{doublespace}
    Nous vous invitons à évaluer notre capacité à mesurer des délais temporels avec précision. Pour ce faire, nous devons déterminer notre précision de la vitesse de la lumière via des délais temporels. Nous allons utiliser un laser pulsé ultra-court de type nanoseconde dans le cadre de nos expériences \cite{ThorlabsNPL64B}. Ce dispositif laser est capable de générer des impulsions allant de $5$ à $39$ $ns$. Nous avons opté pour une impulsion de $10$ $ns$ dans cette plage, car les intervalles de temps plus longs ont tendance à présenter une distribution temporelle similaire à celle d’une fonction porte. Nous cherchons une impulsion dans le domaine temporel qui ressemble à une fonction gaussienne, ce qui se produit lorsque les impulsions du laser sont plus courtes. Cette forme est souvent utilisée dans les mesures de faibles \cite{OpticalNetworks, Lundeen_Direct_Measurement,Hairiri,Guilleaum,Brunner_2004} pour faciliter l’identification de la position maximale de l’impulsion, que nous identifierons comme correspondant à la position temporelle moyenne de l’impulsion, plus sur ceci se suit. Le laser possède une longueur d’onde comprise entre $640 \pm 10$ $nm$, avec une énergie d’impulsion maximale de $2,0$ $nJ$. Sa puissance de pointe s’élève à $50$ $mW$ lorsque le taux de répétition et la largeur d’impulsion maximale sont utilisés. Toutefois, dans le cadre de notre protocole, nous fixons la fréquence de répétition à $1$ $MHz$, assurant ainsi une fréquence constante tout au long de l’expérience. Cela n’affectera pas l’expérience elle-même. Nous recueillons nos données à l'aide d'un oscilloscope \cite{TektronixTDS5000} et les analyserons sur un logiciel de programmation soit MATLAB ou Python à volonté (plus ceux-ci se suivent). Voici un diagramme du dispositif d’expérimentation que nous utiliserons pour les deux prochaines expériences qui se suivent, visible à la figure \ref{fig:speed-of-light}.
\end{doublespace}

\begin{figure}[!hpbt]
    \centering
    \includegraphics[width=1.0\textwidth]{speed_of_light_exp.png}
    \caption{Représentation de notre dispositif expérimental pour évaluer la précision de nos mesures temporelles en mesurant la vitesse d'un signal dans les câbles BNC et la vitesse de la lumière en déplaçant un miroir. L’impulsion du laser est d’abord réglée en intensité par une lame demi-onde, puis dirigée vers un séparateur de faisceau polarisant (PBS) qui divise les états de polarisation horizontaux et verticaux de base de l’impulsion d’entrée en deux voies orthogonales. Celui qui est réfléchi par le PBS sera notre signal de référence pour déclencher l’oscilloscope déçu. L’autre subit encore un autre PBS. Une des voies sera ensuite ignorée à l'aide d'un bloc. Nous définissons les états de polarisation comme suit : le faisceau réfléchi représente l’état de base de polarisation verticale $\ket{V}$ de l’état d’entrée $\ket{\psi}$, tandis que le faisceau transmis correspond à l’état de base horizontal $\ket{H}$. Nous orientons l’état horizontal vers un miroir, que nous réglerons en fonction des différentes distances à évaluer pour l'expérience de la vitesse de la lumière. Il est ensuite renvoyé vers le PBS pour y être réfléchi. Ce procédé utilise une lame quart d’onde pour convertir l’état de polarisation $\ket{H}$ en un état $\ket{V}$ pour qu'il soit réfléchi. Il est ensuite détecté avec un photodétecteur rapide à base de Si \cite{ThorlabsDET025A} puis interprété par notre oscilloscope. Pour l'expérience de la vitesse d'un signal dans un câble BNC, nous utilisent simplement de différentes longueurs de câble attachées sur notre photodétecteur.}
    \label{fig:speed-of-light}
\end{figure}