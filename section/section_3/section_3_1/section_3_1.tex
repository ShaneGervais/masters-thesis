Nous vous invitons à évaluer notre habileté à 
mesurer des délais temporelles avec précision 
en déterminant la vitesse de la lumière sur 
différentes distances. Nous allons utiliser un 
laser pulsé ultra-court de type NPL64B, fabriqué 
par Thorlabs, pour nos expériences. Ce système 
laser peut produire des impulsions de 5 à 39 
nanosecondes. Nous avons choisi une impulsion 
de 10 nanosecondes dans cette gamme, car les 
intervalles de temps plus longs ont tendance à 
avoir une distribution temporelle similaire à 
celle d’une fonction porte. Nous voulons une 
impulsion dans le domaine temporel qui ressemble 
à une fonction gaussienne, ce qui se produit 
lorsque les impulsions du laser sont plus 
courtes. Cette dernière est souvent utilisée 
dans les mesures de faibles, facilitant ainsi 
l’identification de la position maximum de 
l'impulsion, que nous identifierons comme 
correspondant à la position temporelle moyenne 
de l'impulsion. Le laser utilisé possède une 
longueur d'onde comprise entre 630 et 650 nm, 
avec une énergie d'impulsion maximale de 
2,0 nanojoules. Sa puissance de pointe atteint 
50 milliwatts lorsque le taux de répétition 
maximal et la largeur d'impulsion maximale 
sont utilisés. Toutefois, pour notre protocole, 
nous réglons le taux de répétition à 1 MHz, ce 
qui garantit une fréquence constante tout au 
long de l'expérience. Pour nos tests, nous 
utiliserons un miroir pour régler des 
intervalles de distance variables et analyserons 
les résultats à l’aide d'un oscilloscope fourni 
par Tektronix, le modèle étant le TSD5000B. Un 
schéma de l'appareil expérimental est présenté 
dans la figure \ref{fig:speed-of-light}.

\begin{figure}[hp]
    \centering
    \includegraphics[width=1.0\textwidth]{speed_of_light_exp.png}
    \caption{Schéma de notre dispositif expérimental pour 
    évaluer la fiabilité de nos mesures temporelles en 
    mesurant la vitesse de la lumière. Un laser 
    impulsionnel passe dans une lame demi-onde pour régler 
    l'intensité d'entrée, avant d'être dirigé vers un 
    séparateur de faisceau polarisant (PBS) qui divise les 
    états de polarisation horizontaux et verticaux de base 
    de l'impulsion d'entrée en deux voies orthogonales, 
    l'une d'entre elles étant ignorée par un bloc. Nous 
    définissons les états de polarisation comme le faisceau 
    réfléchi étant l'état de base de polarisation verticale 
    de l'état d'entrée, et le faisceau transmis étant 
    l'état de base horizontal. L'état horizontal se dirige 
    vers un miroir que nous ajusterons en fonction des 
    différentes distances à mesurer. Il est ensuite renvoyé 
    vers le PBS pour y être réfléchi. Ce dernier se fait 
    avec une lame quart d'onde qui modifie l'état de 
    polarisation $\ket{H}$ en un état $\ket{V}$. 
    L'impulsion traverse 
    ensuite une demi-plaque d'onde et un polariseur. Il est 
    réglé pour transmettre uniquement l'état verticale 
    lors de la réflexion du PBS. Il est ensuite 
    détecté avec un photomultiplicateur rapide à base de Si 
    fabriqué par Thorlabs, puis interprété par notre 
    oscilloscope.}
    \label{fig:speed-of-light}
\end{figure}