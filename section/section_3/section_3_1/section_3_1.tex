\begin{doublespace}
    Nous vous invitons à évaluer notre capacité à mesurer des délais temporels avec précision. Pour ce faire, nous devons déterminer notre précision de la vitesse de la lumière via des délais temporels. Nous allons utiliser un laser pulsé ultra-court de type nanoseconde dans le cadre de nos expériences \cite{ThorlabsNPL64B}. Ce dispositif laser est capable de générer des impulsions allant de $5$ à $39$ $ns$. Nous avons opté pour une impulsion de $10$ $ns$ dans cette plage, car les intervalles de temps plus longs ont tendance à présenter une distribution temporelle similaire à celle d’une fonction porte. Nous cherchons une impulsion dans le domaine temporel qui ressemble à une fonction gaussienne, ce qui se produit lorsque les impulsions du laser sont plus courtes. Cette dernière est souvent utilisée dans les mesures de faibles \cite{OpticalNetworks, Lundeen_Direct_Measurement,Hairiri,Guilleaum,Brunner_2004} pour faciliter l’identification de la position maximale de l’impulsion, que nous identifierons comme correspondant à la position temporelle moyenne de l’impulsion. Le laser utilisé possède une longueur d’onde comprise entre $640 \pm 10$ $nm$, avec une énergie d’impulsion maximale de $2,0$ $nJ$. Sa puissance de pointe s’élève à $50$ $mW$ lorsque le taux de répétition et la largeur d’impulsion maximale sont utilisés. Toutefois, dans le cadre de notre protocole, nous fixons la fréquence de répétition à $1$ $MHz$, assurant ainsi une fréquence constante tout au long de l’expérience. Cela n’affectera pas l’expérience elle-même. Pour nos tests, nous utiliserons un miroir pour régler des intervalles de distances variables et analyserons les résultats avec notre oscilloscope \cite{TektronixTDS5000}. Voici un diagramme du dispositif d’expérimentation, visible à la figure \ref{fig:speed-of-light}.
\end{doublespace}

\begin{figure}[!hpbt]
    \centering
    \includegraphics[width=1.0\textwidth]{speed_of_light_exp.png}
    \caption{Représentation de notre dispositif expérimental pour évaluer la précision de nos mesures temporelles en mesurant la vitesse de la lumière. L’impulsion du laser est d’abord réglée en intensité par une lame demi-onde, puis dirigée vers un séparateur de faisceau polarisant (PBS) qui divise les états de polarisation horizontaux et verticaux de base de l’impulsion d’entrée en deux voies orthogonales. Celui qui est réfléchi par le PBS sera notre signal de référence pour déclencher l’oscilloscope décu. L’autre subit encore un autre PBS. Une des voies sera ensuite ignorée par un bloc. Nous définissons les états de polarisation comme suit : le faisceau réfléchi représente l’état de base de polarisation verticale $\ket{V}$ de l’état d’entrée $\ket{\psi}$, tandis que le faisceau transmis correspond à l’état de base horizontal $\ket{H}$. Nous orientons l’état horizontal vers un miroir, que nous réglerons en fonction des différentes distances à évaluer. Il est ensuite renvoyé 
    vers le PBS pour y être réfléchi. Ce procédé utilise une lame quart d’onde pour convertir l’état de polarisation $\ket{H}$ en un état $\ket{V}$. Il est ensuite détecté avec un photodétecteur rapide à base de Si \cite{ThorlabsDET025A} puis interprété par notre oscilloscope.}
    \label{fig:speed-of-light}
\end{figure}