\begin{doublespace}

    Nous souhaitons d'évaluer notre capacité à mesurer des délais 
    temporels avec précision. À cet effet, nous allons déterminer 
    notre précision de mesure temporelle en mesurant la vitesse d'un
    signal électrique qui se déplaçant dans un câble BNC de différentes
    longueurs. Nous utiliserons ensuite cette méthode pour
    mesurer la vitesse de la lumière dans un système photonique,
    qui nous permettra de tester notre précision de mesure temporelle
    dans un système plus complexe. Ces expériences nous permettront
    de caractériser notre capacité à mesurer des délais
    temporels avec précision, ce qui est essentiel pour la
    caractérisation des états de polarisation avec une mesure faible
    temporelle. Dans la figure \ref{fig:speed-of-light}, nous 
    présentons le dispositif expérimental que nous utiliserons pour 
    les deux prochaines expériences. 

    \begin{figure}[!hpbt]
    \centering
    \includegraphics[width=1.0\textwidth]{speed_of_light_exp.png}
    \caption{
        Représentation de notre dispositif expérimental pour 
    évaluer la précision de nos mesures temporelles en mesurant la 
    vitesse d'un signal dans les câbles BNC et la vitesse de la lumière. 
    L’impulsion du laser est d’abord réglée en 
    intensité par une lame demi-onde, puis dirigée vers un séparateur 
    de faisceau polarisant (PBS) qui divise les états de polarisation 
    horizontaux et verticaux en deux 
    voies orthogonales. Le faisceau réfléchi, représenté par l’état de base de 
    polarisation verticale $\ket{V}$, sera notre 
    signal de référence pour déclencher l’oscilloscope, à gauche. 
    Le faisceau transmis, correspondant à l’état de base 
    horizontal $\ket{H}$, subit encore un autre PBS et se revoit avec un miroir fixe 
    ainsi qu'une lame quart d'onde pour convertir l'état de polarisation
    $\ket{H}$ en un état $\ket{V}$ pour qu'il soit réfléchi. Il est 
    ensuite détecté avec un autre photodétecteur puis 
    mesuré par notre oscilloscope, à droite.
    Pour l'expérience de la vitesse d'un signal dans un câble BNC, nous 
    utilisons simplement différentes longueurs de câble attachées 
    sur notre photodétecteur, désigner par $\Delta L$. Pour l'expérience de la 
    vitesse de la lumière, nous
    orientons l'état de polarisation horizontal vers un miroir,
    que nous réglerons en fonction des différentes distances à 
    évaluer pour l'expérience de la vitesse de la lumière, désigner par $\Delta x$. 
    Il est 
    ensuite renvoyé vers le PBS pour y être réfléchi. Ce procédé 
    utilise une lame quart d’onde pour convertir l’état de polarisation. 
    }
    \label{fig:speed-of-light}
\end{figure}
    
    \noindent Notre laser est une source à diodes
    pulsés nanosecondes, capable de générer des impulsions 
    allant de $5$ à $39$ $ns$. Nous utilisons une impulsion de $10$ $ns$
    pour nos expériences, car les intervalles de temps plus longs  
    rendent la mesure moins précise. Cette perte de précision se présente 
    dans la mesure de la partie imaginaire. À ces longueurs d'impulsion,
    la distribution temporelle du signal devient plus similaire à celle d’une
    fonction porte \cite{ThorlabsNPL64B} Or, les discountinuités de cette 
    fonction entrainent la présence de bruit à hautes fréquences ce qui 
    rend l'analyse moins facile. 
    Nous voulons une impulsion de forme gaussienne, car cette
    forme est souvent utilisée dans
    les mesures faibles \cite{OpticalNetworks, Lundeen_Direct_Measurement,Hairiri,Guilleaum,Brunner_2004}
    et pour faciliter l’identification de la position maximale de
    l’impulsion, que nous identifierons comme correspondant à la
    position temporelle moyenne de l’impulsion. Le laser possède une longueur d’onde comprise entre 
    $640 \pm 10$ $nm$, avec une énergie d’impulsion maximale de 
    $2,0$ $nJ$. Sa puissance de crête s’élève pas plus 
    qu’à $50$ $mW$, dont sa puissance moyenne est de $20$ $MW$. 
    Nous fixons le taux de répétition à $1$ $MHz$, pour avoir un temps 
    suffisamment long entre les impulsions. Nous recueillons les
    données à l'aide d'un photodétecteur rapide à base de Si 
    \cite{ThorlabsDET025A}, qui possède une bande passante de 
    $2$ $GHz$, ce qui est suffisant pour mesurer les impulsions de
    $10$ $ns$. Le photodétecteur est connecté à un câble BNC de 
    $50$ $\Omega$ \cite{ThorlabsBNC} qui est ensuite connecté à 
    un oscilloscope pour l'enregistrement des données.
    Nous utilisons un oscilloscope Tektronix TDS 5000 \cite{TektronixTDS5000}, 
    qui est un oscilloscope numérique à mémoire profonde, 
    capable d’acquérir des signaux à des vitesses allant jusqu’à 
    $500$ $GS/s$ (gigéchantillons par seconde), et qui possède une 
    bande passante de $500$ $MHz$.  
    Ensuite, nous analysons les données avec \textsc{MATLAB} et/ou \textsc{Python}
    pour en tirer des conclusions sur la précision de nos mesures
    temporelles.
    

\end{doublespace}

