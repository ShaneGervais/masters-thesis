\begin{onehalfspace}
    Pour caractériser les états de polarisation avec une mesure faible 
    temporelle, nous devons d'abord tester notre capacité à mesurer des 
    délais ultra-courts avec une précision fiable afin de pouvoir 
    caractériser l'état de polarisation d'un signal.
    Le premier objectif de ce chapitre est de déterminer le modèle
    d'ajustement permettant de mesurer des délais le plus précisément
    possible, tout en restant indépendant de tout paramètre
    spécifique au montage. Pour ce faire, nous allons réaliser des expériences en mesurant la 
    vitesse d’un signal électrique se propageant sur différentes longueurs de câbles BNC,
    un système isolé dont on peut aisément changer \textcolor{red}{ces délais pour obtenir
    une analyse indépendante des facteurs extérieurs de l'expérience.}
    Cette expérience nous permettra de déterminer le modèle d'ajustement
    le plus précis pour mesurer des délais ultra-courts.
    Nous testerons ensuite la précision de cette 
    méthode dans une expérience de mesure de la vitesse de la lumière en mesurant
    différents parcours de la lumière dans un système photonique,
    afin de vérifier que la méthode s'applique bien aux impulsions
    lumineuses. Par la suite, nous mettrons en place un dispositif expérimental
    intégrant un interféromètre de polarisation réalisant une mesure faible
    temporelle. Avec la méthode établie pour mesurer des délais,
    nous pourrons caractériser la partie réelle de la valeur faible
    pour différents trajets de polarisation. Nous définirons ces trajets de polarisation et calculerons la partie
    réelle attendue de la valeur faible, utilisée pour relier le décalage
    temporel mesuré \textcolor{red}{par} l'observable de polarisation. Pour accéder à la partie
    imaginaire de la valeur faible, nous insérerons notre dispositif
    de mesure faible temporelle dans un interféromètre de Mach-Zehnder.
    Nous décrirons la procédure de mesure de la partie imaginaire de la valeur faible
    et calculerons les valeurs attendues pour les différents trajets de polarisation.
\end{onehalfspace}
