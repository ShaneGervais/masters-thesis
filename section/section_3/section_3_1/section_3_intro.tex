\begin{doublespace}
    Pour caractériser les états de polarisation avec une mesure faible temporelle, nous devons d'abord tester notre capacité à mesurer des délais très courts et la précision avec laquelle nous le faisons. Pour ce faire, nous allons réaliser des expériences en mesurant la vitesse d’un signal se déplaçant sur différentes longueurs de câble, car il s’agit d’une zone fermée que nous pouvons isoler pour analyser le signal et où nous pouvons déterminer notre précision pour les mesures temporelles. Nous testerons ensuite la précision de cette méthode dans une expérience sur la vitesse de la lumière avec des déplacements de miroir variables, qui servira de précurseur à la caractérisation de la polarisation à l’aide de retards temporels. Par la suite, nous discuterons de nos appareils expérimentaux pour mesurer à la fois la partie réelle et la partie imaginaire de la valeur faible de notre système photonique.
\end{doublespace}