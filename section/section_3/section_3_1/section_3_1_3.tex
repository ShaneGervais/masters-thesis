\begin{doublespace}

    Le dispositif expérimental, figure \ref{fig:speed-of-light} 
    est conçu pour mesurer la vitesse de la 
    lumière dans les câbles BNC, en utilisant un miroir ajustable pour 
    créer un délai temporel. Le principe de l'expérience repose sur le 
    fait que le signal parcourt une distance connue dans un câble BNC, 
    et en mesurant le temps que met le signal à parcourir cette 
    distance, nous pouvons calculer sa vitesse. Peux être que utiliser
    pour mesurer la vitesse de la lumière. Ce dernier ce fait en
    déplaçant le miroir ajustable, ce qui modifie la distance parcourue
    par le signal. En mesurant le temps que met le signal à parcourir 
    cette distance, nous pouvons calculer la vitesse de la lumière dans notre
    milieu. Je vais maintenant présenter les résultats de cette expérience,
    ainsi que les données dérivées de l'impulsion mesurée.

    \noindent J'analyse les données de l'expérience en utilisant la
    méthode que nous avons acquise précédemment, en effectuant un ajustement de
    la courbe de la dérivée de l'impulsion mesurée. Cette approche
    permet de déterminer le temps d'arrivée des impulsions. Le tableau 
    \ref{table:fits_light} présente les résultats de l'expérience,
    qui présente les temps d’arrivée mesurés pour différentes distances 
    parcourues par les impulsions, ainsi que les écarts-types et la 
    qualité des ajustements de courbe. 


\begin{longtable}{p{2.0cm} p{1.5cm} p{3.0cm} p{2.0cm} p{2.5cm}}
    \caption{ Résultats des temps d’arrivée mesurés pour différentes parcout des impulsions et types d’ajustement de courbe, avec leurs écarts-types et qualités des adjustements pour l'expérience de la vitesse de la lumière} \\
    \toprule
    \label{table:fits_light}
    Type de fit & Parcours (cm) & Temps d'arrivée (ns) & Écart-type (ns) & Qualité du fit \\
    \midrule
    \endfirsthead
    
    \toprule
    Type de fit & Parcours (cm) & Temps d'arrivée (ns) & Écart-type (ns) & Qualité du fit \\
    \midrule
    \endhead
    
    \midrule
    \multicolumn{5}{r}{{\dots}} \\
    \midrule
    \endfoot
    
    \bottomrule
    \endlastfoot
    
    % Paste all your rows here like:
    poly2     & 0            & 4.71331          & 0.00540       & 0.65535 \\
poly2     & 5.08         & 4.89051          & 0.00260       & 0.67606 \\
poly2     & 10.16        & 5.06522          & 0.00269       & 0.62149 \\
poly2     & 20.32        & 5.39393          & 0.00350       & 0.68970 \\
poly2     & 40.64        & 6.07951          & 0.00302       & 0.61930 \\
poly2     & 60.96        & 6.76310          & 0.00531       & 0.66085 \\
poly2     & 81.28        & 7.44000          & 0.00568       & 0.62715 \\
poly2     & 101.6        & 8.11442          & 0.00226       & 0.66930 \\
poly2     & 121.92       & 8.79659          & 0.00352       & 0.65540 \\
\midrule
poly3     & 0            & 4.72475          & 0.01273       & 0.65280 \\
poly3     & 5.08         & 4.90564          & 0.00548       & 0.67392 \\
poly3     & 10.16        & 5.07506          & 0.00277       & 0.62064 \\
poly3     & 20.32        & 5.41906          & 0.00608       & 0.68383 \\
poly3     & 40.64        & 6.08960          & 0.00444       & 0.61804 \\
poly3     & 60.96        & 6.76516          & 0.01389       & 0.65857 \\
poly3     & 81.28        & 7.45457          & 0.01290       & 0.62294 \\
poly3     & 101.6        & 8.12238          & 0.01007       & 0.66779 \\
poly3     & 121.92       & 8.80146          & 0.01551       & 0.65261 \\
\midrule
poly4     & 0            & 4.72134          & 0.00366       & 0.64088 \\
poly4     & 5.08         & 4.89551          & 0.00187       & 0.66138 \\
poly4     & 10.16        & 5.07134          & 0.00393       & 0.60899 \\
poly4     & 20.32        & 5.31017          & 0.21437       & 0.67474 \\
poly4     & 40.64        & 6.08475          & 0.00196       & 0.60412 \\
poly4     & 60.96        & 6.76839          & 0.00235       & 0.64813 \\
poly4     & 81.28        & 7.44816          & 0.00308       & 0.61352 \\
poly4     & 101.6        & 8.11833          & 0.00449       & 0.65716 \\
poly4     & 121.92       & 8.80287          & 0.00408       & 0.64269 \\
\midrule
poly5     & 0            & 4.72634          & 0.00810       & 0.64049 \\
poly5     & 5.08         & 4.80277          & 0.20913       & 0.66133 \\
poly5     & 10.16        & 4.99172          & 0.18918       & 0.60878 \\
poly5     & 20.32        & 5.11453          & 0.26545       & 0.67455 \\
poly5     & 40.64        & 6.08785          & 0.00248       & 0.60394 \\
poly5     & 60.96        & 6.96124          & 0.26574       & 0.64786 \\
poly5     & 81.28        & 7.35435          & 0.21549       & 0.61320 \\
poly5     & 101.6        & 8.12024          & 0.00382       & 0.65707 \\
poly5     & 121.92       & 8.80513          & 0.00999       & 0.64219 \\
\midrule
poly6     & 0            & 4.72661          & 0.00817       & 0.63983 \\
poly6     & 5.08         & 4.97781          & 0.18094       & 0.66097 \\
poly6     & 10.16        & 5.07586          & 0.00696       & 0.60835 \\
poly6     & 20.32        & 5.40079          & 0.00798       & 0.67433 \\
poly6     & 40.64        & 6.08851          & 0.00310       & 0.60386 \\
poly6     & 60.96        & 6.86658          & 0.21361       & 0.64768 \\
poly6     & 81.28        & 7.35519          & 0.21417       & 0.61316 \\
poly6     & 101.6        & 8.12101          & 0.00479       & 0.65679 \\
poly6     & 121.92       & 8.80259          & 0.00943       & 0.64184 \\
\midrule
poly7     & 0            & 4.71966          & 0.01125       & 0.63930 \\
poly7     & 5.08         & 4.81158          & 0.19912       & 0.66074 \\
poly7     & 10.16        & 5.07004          & 0.00832       & 0.60817 \\
poly7     & 20.32        & 5.40111          & 0.00790       & 0.67401 \\
poly7     & 40.64        & 6.08673          & 0.00729       & 0.60372 \\
poly7     & 60.96        & 6.67275          & 0.21349       & 0.64746 \\
poly7     & 81.28        & 7.44627          & 0.00925       & 0.61295 \\
poly7     & 101.6        & 8.11835          & 0.00488       & 0.65661 \\
poly7     & 121.92       & 8.81123          & 0.30484       & 0.64079 \\
\midrule
poly8     & 0            & 4.72029          & 0.01048       & 0.63920 \\
poly8     & 5.08         & 4.81604          & 0.19913       & 0.66011 \\
poly8     & 10.16        & 5.07020          & 0.00792       & 0.60801 \\
poly8     & 20.32        & 5.40480          & 0.00838       & 0.67384 \\
poly8     & 40.64        & 6.08686          & 0.00920       & 0.60324 \\
poly8     & 60.96        & 6.67051          & 0.19889       & 0.64680 \\
poly8     & 81.28        & 7.43096          & 0.33318       & 0.61283 \\
poly8     & 101.6        & 7.92769          & 0.20935       & 0.65582 \\
poly8     & 121.92       & 8.42229          & 0.13732       & 0.64046 \\
\midrule
poly9     & 0            & 4.72486          & 0.33006       & 0.63897 \\
poly9     & 5.08         & 4.70720          & 0.22780       & 0.65972 \\
poly9     & 10.16        & 4.88903          & 0.22563       & 0.60786 \\
poly9     & 20.32        & 5.20094          & 0.25460       & 0.67365 \\
poly9     & 40.64        & 5.80773          & 0.23018       & 0.60284 \\
poly9     & 60.96        & 6.38899          & 0.10911       & 0.64666 \\
poly9     & 81.28        & 7.09352          & 0.15865       & 0.61282 \\
poly9     & 101.6        & 7.69569          & 0.08447       & 0.65868 \\
poly9     & 121.92       & 8.38099          & 0.05428       & 0.64209 \\
\midrule
fourier1  & 0            & 4.71572          & 0.00457       & 0.64231 \\
fourier1  & 5.08         & 4.89408          & 0.00229       & 0.66186 \\
fourier1  & 10.16        & 5.06706          & 0.00226       & 0.60996 \\
fourier1  & 20.32        & 5.39963          & 0.00321       & 0.67576 \\
fourier1  & 40.64        & 6.08169          & 0.00254       & 0.60503 \\
fourier1  & 60.96        & 6.76347          & 0.00355       & 0.64888 \\
fourier1  & 81.28        & 7.44291          & 0.00361       & 0.61461 \\
fourier1  & 101.6        & 8.11617          & 0.00191       & 0.65769 \\
fourier1  & 121.92       & 8.79741          & 0.00198       & 0.64380 \\
\midrule
fourier2  & 0            & 4.72877          & 0.01125       & 0.64064 \\
fourier2  & 5.08         & 4.89736          & 0.00518       & 0.66119 \\
fourier2  & 10.16        & 4.99303          & 0.19275       & 0.60877 \\
fourier2  & 20.32        & 5.40637          & 0.00155       & 0.67472 \\
fourier2  & 40.64        & 6.08968          & 0.00487       & 0.60397 \\
fourier2  & 60.96        & 6.77192          & 0.00455       & 0.64791 \\
fourier2  & 81.28        & 7.45179          & 0.00598       & 0.61331 \\
fourier2  & 101.6        & 7.93620          & 0.25209       & 0.65693 \\
fourier2  & 121.92       & 8.89100          & 0.19440       & 0.64262 \\
\midrule
fourier3  & 0            & 4.63993          & 0.18354       & 0.63971 \\
fourier3  & 5.08         & 4.89476          & 0.00496       & 0.66074 \\
fourier3  & 10.16        & 5.07082          & 0.01047       & 0.60822 \\
fourier3  & 20.32        & 5.30076          & 0.21511       & 0.67417 \\
fourier3  & 40.64        & 6.08743          & 0.00893       & 0.60372 \\
fourier3  & 60.96        & 6.76993          & 0.00735       & 0.64743 \\
fourier3  & 81.28        & 7.44726          & 0.01016       & 0.61299 \\
fourier3  & 101.6        & 8.12069          & 0.01187       & 0.65622 \\
fourier3  & 121.92       & 8.71801          & 0.19372       & 0.64159 \\
\midrule
fourier4  & 0            & 4.72830          & 0.01700       & 0.63849 \\
fourier4  & 5.08         & 4.82117          & 0.19003       & 0.65918 \\
fourier4  & 10.16        & 5.16231          & 0.16442       & 0.60734 \\
fourier4  & 20.32        & 5.32363          & 0.21445       & 0.67340 \\
fourier4  & 40.64        & 6.18942          & 0.19740       & 0.60294 \\
fourier4  & 60.96        & 6.85717          & 0.18283       & 0.64564 \\
fourier4  & 81.28        & 7.47014          & 0.00785       & 0.61182 \\
fourier4  & 101.6        & 8.01239          & 0.20774       & 0.65498 \\
fourier4  & 121.92       & 8.80669          & 0.04605       & 0.63934 \\
\midrule
fourier5  & 0            & 4.71827          & 0.03783       & 0.63602 \\
fourier5  & 5.08         & 4.81994          & 0.18129       & 0.65602 \\
fourier5  & 10.16        & 4.99705          & 0.21753       & 0.60598 \\
fourier5  & 20.32        & 5.30335          & 0.20536       & 0.67253 \\
fourier5  & 40.64        & 6.09325          & 0.01973       & 0.60150 \\
fourier5  & 60.96        & 6.60507          & 0.26801       & 0.64472 \\
fourier5  & 81.28        & 7.44084          & 0.06341       & 0.61077 \\
fourier5  & 101.6        & 7.96551          & 0.26103       & 0.65341 \\
fourier5  & 121.92       & 8.70882          & 0.22482       & 0.63844 \\
\midrule
fourier6  & 0            & 4.67487          & 0.21258       & 0.63513 \\
fourier6  & 5.08         & 4.87912          & 0.30044       & 0.65545 \\
fourier6  & 10.16        & 5.01425          & 0.15367       & 0.60510 \\
fourier6  & 20.32        & 5.40065          & 0.04759       & 0.67185 \\
fourier6  & 40.64        & 6.02640          & 0.20386       & 0.60054 \\
fourier6  & 60.96        & 6.82890          & 0.21281       & 0.64294 \\
fourier6  & 81.28        & 7.56540          & 0.18348       & 0.61024 \\
fourier6  & 101.6        & 8.01208          & 0.20554       & 0.65208 \\
fourier6  & 121.92       & 8.67386          & 0.20449       & 0.63782 \\
\midrule
gauss1    & 0            & 4.71424          & 0.00450       & 0.64553 \\
gauss1    & 5.08         & 4.89228          & 0.00232       & 0.66632 \\
gauss1    & 10.16        & 5.06612          & 0.00245       & 0.61264 \\
gauss1    & 20.32        & 5.39707          & 0.00293       & 0.67962 \\
gauss1    & 40.64        & 6.08066          & 0.00273       & 0.60835 \\
gauss1    & 60.96        & 6.76292          & 0.00380       & 0.65146 \\
gauss1    & 81.28        & 7.44163          & 0.00415       & 0.61718 \\
gauss1    & 101.6        & 8.11554          & 0.00192       & 0.66113 \\
gauss1    & 121.92       & 8.79674          & 0.00190       & 0.64651
\end{longtable}

\noindent Le tableau \ref{table:speed_light} 
présente les résultats de la mesure de la vitesse de la lumière pour 
différents ajustements de courbe. Les valeurs mesurées sont comparées
 à la vitesse théorique de la lumière dans le vide, qui est de 
 $299792458$ $m/s$. Les erreurs en pourcentage sont calculées par 
 rapport à cette valeur théorique, et la qualité du fit est 
 indiquée par le coefficient de détermination ($R^2$) pour chaque 
 ajustement.

\begin{longtable}{p{2.0cm} p{2.0cm} p{2.0cm} p{2.0cm} p{2.0cm}}
    \caption{Mesure de la vitesse de la lumière pour différent ajustement de courbe} \\
    \toprule
    \label{table:speed_light}
    Type de fit & Vitesse mesurée (m/s) & Vitesse théorique (m/s) & Erreur (\%) & Qualité du fit \\
    \midrule
    \endfirsthead
    
    \toprule
    Type de fit & Vitesse mesurée (m/s) & Vitesse théorique (m/s) & Erreur (\%) & Qualité du fit \\
    \midrule
    \endhead
    
    \midrule
    \multicolumn{5}{r}{{À suivre sur la prochaine page}} \\
    \midrule
    \endfoot
    
    \bottomrule
    \endlastfoot
    
    % Paste all your rows here like:
    poly2     & 298948041     & 299792458           & 0.2817      & 0.0048       \\
    poly3     & 299593785     & 299792458           & 0.0663      & 0.0061       \\
    poly4     & 297557352     & 299792458           & 0.7456      & 0.0361       \\
    poly5     & 291020596     & 299792458           & 2.9260      & 0.1405       \\
    poly6     & 302282720     & 299792458           & 0.8307      & 0.0605       \\
    poly7     & 297220542     & 299792458           & 0.8579      & 0.0440       \\
    poly8     & 320556240     & 299792458           & 6.9261      & 0.0943       \\
    poly9     & 326791560     & 299792458           & 9.0059      & 0.0690       \\
    fourier1  & 299104313     & 299792458           & 0.2295      & 0.0042       \\
    fourier2  & 299334235     & 299792458           & 0.1528      & 0.0874       \\
    fourier3  & 298537843     & 299792458           & 0.4185      & 0.0546       \\
    fourier4  & 300580478     & 299792458           & 0.2629      & 0.0943       \\
    fourier5  & 303509743     & 299792458           & 1.2400      & 0.0682       \\
    fourier6  & 301689685     & 299792458           & 0.6328      & 0.0907       \\
    gauss1    & 299025597     & 299792458           & 0.2558      & 0.0044     
\end{longtable}


    Je remarque que les ajustements de courbe sont de bonne qualité, avec des valeurs similaires
    que l'expérience précédente, ce qui indique que les données sont
    cohérentes et fiables. Les écarts-types sont également faibles, ce qui
    suggère que les mesures sont précises et répétables. Je procède ensuite
    aux résultats des données issues des impulsions de l’expérience du miroir déplacé, 
    figure \ref{fig:speed-pulses}.
    
    \begin{figure}[!hptb]
        \centering
        \includegraphics[width=1.0\textwidth]{fits_light.png}
        \caption{Profil temporel de la dérivée des données d'impulsion pour chacune des distances mesurées et ajustement de la courbe pour l'expérience de la vitesse de la lumière avec un miroir réglable.}
        \label{fig:speed-pulses}
    \end{figure}

    \noindent D'ici je place les résultats des délais obtenue dans l'expérience
    de la vitesse de la lumière et les comparant à le temps d'arrivée
    d'un impulsion où le miroir est fixe, ce qui nous permet
    de déterminer les délais pour la vitesse de la lumière. La figure
    \ref{fig:speed-res} présente le résultat de notre expérience.

    \begin{figure}[!hptb]
    \centering
    \includegraphics[width=1.0\textwidth]{speed_light.png}
    \caption{Résultats des délais mesurés pendant l'expérience sur la vitesse de la lumière, ainsi que leur ajustement linéaire. Les barres d’erreur horizontales représentent l’incertitude de nos mesures de la distance, soit $\pm 0,5$ $mm$, qui est trop petite pour être visible sur le graphique. Les barres d'erreurs verticales correspondent à l'erreur de l'ajustement Fourier basé sur l'expérience de la vitesse de la lumière dans les câbles BNC $\pm 0,03\%$.}
    \label{fig:speed-res}
    \end{figure}

    \noindent En utilisant les résultats de l'expérience, nous avons
    calculé la vitesse de la lumière dans les câbles BNC. La vitesse
    mesurée est de $299 025 597$ $m/s$ avec un écart-type de 
    $0.03\%$ par rapport à la valeur théorique de la vitesse de la
    lumière dans l'aire libre, qui est de $0.9998c$. À partir de ces 
    résultats de ces deux expériences, nous concluons que nous pouvons 
    mesurer des délais temporels très courts, de l'ordre de $4$ $ps$ avec 
    une variation inférieure à $2$ $ps$ entre les ensembles de données. 
    Les sections suivantes seront 
    consacrées à la caractérisation des états de polarisations en utilisant
    les méthodologies que nous avons développées dans les sections précédentes.
    
\end{doublespace}





