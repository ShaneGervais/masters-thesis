La façon que nous acquérons nos données est 
importante, car nous devons nous assurer que 
nous utilisons la méthode la plus précise pour 
acquérir la position temporelle moyenne de 
l'état en vue d'une analyse ultérieure des 
mesures faibles. Nous faiscon deux expériences
pour mesuré la vitesse de la lumière. L'une est
de mesurer la vitesse de la lumière à partir 
d'un miroir et prenons des mesures à 
différentes distances de 
$2,52$, $5,48$, $10,10$, $20,19$, $30,29$, 
$40,38$, $50,48$ et $65,62$ $cm$.
(mesuré à l'oeuil avec une règle). Le délai mesuré est 
à partir d'une 
distance de référence appelée position 0, 
située à $11,5$ $cm$ du PBS. La lumière 
doit voyager dans les deux 
sens à partir du séparateur de faisceau, de 
sorte que la distance parcourue est le double de 
la distance envoyée. L'autre expérience
est le même principe mais de mesurer la vitesse de 
la lumière dans nos cable BNC RG-58 de Thorlabs à des différentes longeurs.
Cette expérience facilitant nos mesurer car les cables on 
une longeurs déterminer par le manufacturier. Le délai
est puisque dans la différences de longeurs 
entre les différentes
longeurs de cables. Les différences de longeurs 
de cables sont $17$, $27$, $52$, $100$ et $300$ $cm$. Les délais 
mesuré sont à partir d'une prise de mesure 
avec un cable inférieur au
premier. Pour ce dernier, le miroir ajustable est fix.
Une implication importante est la façon 
dont l'oscilloscope 
acquiert des données pour le domaine temporel. 
Le signal est détecté par le photomultiplicateur 
rapide à base de Si et est introduit dans 
l'oscilloscope à l'aide d'un câble BNC et est 
déclenché par le front montant gauche de notre 
signal de référence. Le signal de référence que 
nous déclenchons est utilisé comme temps $0$ pour 
l'oscilloscope afin de mesurer la position 
temporelle pour chaque distance et de les 
comparer à notre position temporelle de 
référence lorsque le miroir est à notre position 
$0$. De cette façon, nous isolons l'expérience 
pour n'observer que ce qui se passe lorsque 
nous déplaçons le miroir de sa position initiale 
à des distances plus grandes. Les performances 
de l'oscilloscope dépendent également de la 
manière dont il acquiert le signal d'entrée 
via ses commandes d'acquisition. Ce paramètre 
traite le nombre de formes d'onde que nous 
pouvons spécifier dans la forme d'onde acquise, 
créant ainsi une forme d'onde moyenne de notre 
signal d'entrée. Nous calculons la moyenne de 
plus de $10000$ formes d'onde pour obtenir un 
signal propre afin de réduire et d'omettre le 
bruit de fond et d'obtenir une mesure plus 
précise de la position temporelle moyenne d'un 
signal. L'expérience est également réalisée 
dans l'obscurité pour réduire le bruit de fond, 
mais nous avons également trouvé que cette 
étape n'était pas nécessaire et que nous n'avons 
pas constaté de changements radicaux dans les 
résultats, mais nous l'avons quand même fait. 
Un autre aspect important de l'oscilloscope est 
son mode d'échantillonnage qui contrôle la façon 
dont l'oscilloscope prend ses échantillons. Ces 
modes sont l'échantillonnage en temps réel, 
l'interpolation et le temps d'équivalence. En 
mode d'échantillonnage en temps réel, 
l'oscilloscope numérise tous les points qu'il 
acquiert après un événement déclencheur. Ce mode 
est principalement utilisé pour les mesures 
d'acquisition unique ou les changements en temps 
réel du signal. Le mode d'interpolation 
interpole entre les points d'échantillonnage en 
créant des points qui aident à combler les 
lacunes de sorte qu'il y ait une ligne droite ou 
une onde sinusoïdale entre les points pour une 
courbe plus lisse. Nous ne souhaitons pas faire 
cela car nous ne voulons pas créer un maximum 
interpolé du signal. Enfin, le mode 
d'échantillonnage par équivalence de temps 
augmente le taux d'échantillonnage au-delà du 
taux d'échantillonnage maximum en temps réel, 
ce qui permet d'obtenir le taux 
d'échantillonnage complet de l'oscilloscope, 
soit 500 Géch/s. Notez que si le déclenchement 
n'est pas en mode externe et que vous avez votre 
état d'entrée dans un canal et le signal de 
référence dans un autre, votre taux 
d'échantillonnage maximum est maintenant divisé 
en deux. L'importance de la fréquence 
d'échantillonnage maximale est de permettre à 
l'oscilloscope d'avoir sa résolution temporelle 
maximale pour notre signal, qui s'avère être de 
$4$ $\pm$ $2$ $ps$, ce qui permet des mesures temporelles 
précises. Nous enregistrons ensuite les formes 
d'ondes de sortie dans un fichier CSV sur un 
ordinateur pour une analyse plus approfondie 
des données.

\begin{figure}[hp]
    \centering
    \includegraphics[width=1.0\textwidth]{EQ _time_oscillo_fig.png}
    \caption{Schéma du fonctionnement du mode 
    d'acquisition du temps d'équivalence de 
    l'oscilloscope. L’oscilloscope acquiert 
    quelques échantillons par événement de 
    déclenchement pour obtenir le signal complet 
    de notre impulsion. Il va ensuite 
    enregistrer un certain nombre d'échantillons 
    d’acquisition. Une fois terminé, 
    l'oscilloscope combine plusieurs 
    échantillons d'acquisition d'un signal 
    répétitif. Il émet ensuite la densité 
    d'échantillonnage du signal d'entrée pour un 
    enregistrement de forme d'onde lisse et 
    complet. Le taux d'échantillonnage est 
    supérieur à celui de son homologue en temps 
    réel. Le type d'échantillonnage par 
    équivalence de temps que prend 
    l'oscilloscope est appelé échantillonnage 
    aléatoire par équivalence de temps, qui est 
    effectué dans le cadre d'événements 
    aléatoires définis par l'horloge 
    d'échantillonnage de l'oscilloscope, qui 
    fonctionne de manière asynchrone par rapport 
    au signal d'entrée et au signal de 
    déclenchement. \cite{TektronixTDS5000}}
    \label{fig:EQ-time}
\end{figure}