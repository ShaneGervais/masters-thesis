\begin{doublespace}
    
    La façon dont nous acquérons nos données est importante, car nous 
    devons nous assurer que nous utilisons la méthode la plus précise 
    pour déterminer la position temporelle moyenne de l’état en vue
    d’une analyse ultérieure. Des implications 
    importantes à considérer pour la précision de nos mesures est la 
    façon dont l’oscilloscope est configuré:
    mode d'acquisition des données, résolution temporelle 
    et mode de déclenchement. 
    %look back here^

    \noindent Pour mesurer des délais temporels, nous avons 
    besoin d'un signal de 
    référence et un signal expérimental. Nous obtenons ce signal 
    en prélevant une partie du signal du laser à l'aide d'un PBS (voir 
    la figure \ref{fig:speed-of-light}). Le signal de référence est
    envoyé à un photodétecteur rapide, qui est utilisé pour
    déclencher l'oscilloscope. Le signal expérimental est envoyé
    à un autre photodétecteur rapide, qui est utilisé pour
    mesurer le délai temporel par rapport au signal de référence. 
    Le signal de référence est déclenché sur son front montant et 
    est connecté à l'entrée auxiliaire afin de ne pas affecter le taux 
    d'échantillonnage de l'oscilloscope. 
    %Je utilise le port externe de l’oscilloscope 
    %pour déclencher l’acquisition à partir d’un 
    %signal externe qui permet de synchroniser l’acquisition avec 
    %le signal de référence et on obtient la plus haute résolution
    %temporelle possible pour l'acquisition des données.       
    Nous avons établi qu’il 
    fallait au moins $30$ $\mu W$ d’intensité de signal pour 
    déclencher l’oscilloscope sur ce port \cite{TektronixTDS5000}. 
    Nous déterminons le délai de chaque acquisition par rapport à 
    une référence, qui correspond à la longueur initiale du câble
    ou la position initiale du miroir. De cette manière, nous isolons 
    l’expérience pour observer uniquement ce qui se produit lorsqu'on modifie la 
    longueur du câble ou lorsqu’on déplace le miroir. 

    \noindent Le mode d'acquisition est un paramètre crucial pour la résolution
    des données. Notre oscilloscope possède plusieurs modes d'acquisition: 
    échantillonnage direct, détection 
    de pic, enveloppe, haute résolution et moyennage. Nous choisissons
    le mode d'acquisition moyen, car il permet de réduire le bruit 
    dans la prise de données. Ce réglage effectue la moyenne sur $N$ 
    signaux expérimentaux générant un signal moyenné. Pour obtenir 
    un signal clair et éliminer le bruit de fond,
    nous devions effectuer la moyenne sur $1000$ signaux expérimentaux.

    \noindent Le mode d'échantillonnage est un autre paramètre important 
    pour la façon dont l'oscilloscope collecte des données. Ces modes sont 
    l’échantillonnage en temps réel, 
    l’interpolation et en temps équivalent
    (\guillemetleft Equivalent-time sampling\guillemetright \space en anglais). En mode 
    d’échantillonnage en temps réel, l’oscilloscope numérise tous 
    les points après un événement déclencheur. Ce mode 
    d'échantillonnage est principalement utilisé pour les mesures 
    ponctuelles où les variations du signal en temps réel sont 
    importantes \cite{TektronixTDS5000}. Le mode d’interpolation crée 
    des points intermédiaires 
    entre les points d’échantillonnage, ce qui permet de combler 
    les éventuelles lacunes. Les options d'interpollation sont linéaire ou 
    sinusoidale entre les points, ce qui rend la courbe plus lisse \cite{TektronixTDS5000}. 
    Cependant, la résolution
    temporelle est insufisante pour nos mesures. Le mode d
    ’échantillonnage en temps équivalent permet d’augmenter 
    numériquementle taux d’échantillonnage au-delà du taux d’échantillonnage 
    maximum de l'oscilloscope. La figure \ref{fig:EQ-time} 
    illustre le fonctionnement de ce mode. 
    
    \begin{figure}[!hptb]
        \centering
        \includegraphics[width=1.0\textwidth]{EQ _time_oscillo_fig.png}
        \caption{Diagramme illustrant le fonctionnement du mode 
        d’échantillonnage en temps équivalent de l’oscilloscope 
        \cite{TektronixTDS5000}. Cet appareil collecte un petit 
        nombre d’échantillons au moment où l’événement de 
        déclenchement se produit, ce qui lui permet d’obtenir le 
        signal complet de notre impulsion. Le taux d’échantillonnage 
        est supérieur à celui du mode en temps réel. 
        L’oscilloscope en mode temps équivalent
        effectue un échantillons à chaque déclenchement du signal
        de référence. Il enregistre un certain nombre d’échantillons
        d’acquisition, puis combine plusieurs échantillons d’un signal
        répétitif en cours d’acquisition. Il régule ensuite la
        fréquence d’échantillonnage du signal d’entrée pour un
        enregistrement du signal complet.}
        \label{fig:EQ-time}
    \end{figure}

    \noindent %En utilisant ce mode, 
    %nous acquérons un petit nombre d’échantillons au moment
    %où l’événement de déclenchement se produit, ce qui lui permet
    %d’obtenir le signal complet de notre impulsion. 
    Dans ce mode, 
    il est possible 
    d’obtenir le taux d’échantillonnage complet de $500$ $GS/s$ 
    (gigéchantillons par seconde) et un taux d'échantillonnage électronique maximum
    de $20$ $GS/s$. Si le déclenchement n’est pas en mode externe 
    et que le signal de reférence se trouve sur un port, 
    le taux d’échantillonnage maximal 
    sera désormais divisé par deux. L’échantillonnage maximal est 
    crucial pour l’oscilloscope, car il permet d’atteindre une 
    résolution temporelle maximale de 
    $4 \pm 2 $ $ps$. Cela nous assure des mesures temporelles 
    précises. 

    \noindent Avec ces paramètres, nous avons la meilleure résolution 
    possible avec l’équipement servant à collecter des données sur 
    notre signal. Nous enregistrons ensuite les signaux d’onde de 
    sortie au format \textsc{CSV}. Une mesure d'acquisition 
    prend environ trois minutes et
    nous prenons $10$ différentes acquisitions pour chaque different délai. 
    Nous mesurons la vitesse d'un signal dans un 
    câble BNC RG-58 \cite{ThorlabsBNC} avec des délais temporels implémentés
    en faisant parcourir le signal expérimental sur
    différentes longueurs de câbles. Cette expérience 
    facilite nos mesures, car les câbles ont une longueur déterminée 
    par le fabricant. Le délai est lié aux variations de longueur entre 
    les différentes longueurs de câbles $\Delta L$. Ces dernières sont de 
    $172$, $270$, $522$, $1032$ et $3000$ $ \pm 0.5$ $mm$ 
    (mesuré à l’œil avec une règle). Les délais 
    commencent par une mesure avec un câble possèdant la longueur 
    la plus courte; dans ce cas, le miroir ajustable est fixe. Pour 
    l'expérience de mesures de la vitesse de la lumière, nous introduisons
    un délai en déplaçant le miroir à différentes distances $\Delta x$: 
    $2,52~$, $5,48~$, $10,10~$, $20,19~$, $30,29~$, $40,38~$, $50,48$ 
    et $65,62$ $cm$ $\pm 2$ $\mu m$ (converti à partir des pouces, car 
    notre platine de translation linéaire manuelle est en unités 
    impériales \cite{EdmundOpticsStage}). Le délai mesuré 
    part d’une référence, appelée position $0$, correspondant à 
    la position initiale du miroir. Comme le montre
    la figure \ref{fig:speed-of-light}, la lumière doit parcourir 
    une distance double car elle parcourt deux fois la 
    distance entre le PBS et le miroir qui se déplace. 
    Ensuite, les données sont analysées afin de 
    trouver le délai obtenu pour chaque distance du miroir.
    
\end{doublespace}
