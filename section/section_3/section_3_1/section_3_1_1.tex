\begin{doublespace}
    
    La façon dont nous acquérons nos données est importante, car nous 
    devons nous assurer que nous utilisons la méthode la plus précise 
    pour déterminer la position temporelle moyenne de l’état en vue
    d’une analyse ultérieure des mesures faibles. Des implications 
    importantes à considérer pour la précision de nos mesures est la 
    façon dont l’oscilloscope est déclenché, ça méthode qu'elle 
    acquiert des données pour le domaine temporel, ainsi que ça 
    méthode d'échantillonnage et ce qu'on définit comme un délai. 

    \noindent Le signal est détecté par le photodétecteur rapide, 
    puis il est introduit dans l’oscilloscope à l’aide d’un câble BNC, 
    figure \ref{fig:speed-of-light}. Il est ensuite déclenché par le 
    front montant gauche de notre signal de déclenchement. Ce signal 
    correspond à notre déclencheur, qui constitue une origine 
    temporelle pour l’oscilloscope. Nous insérons le signal de 
    déclenchement dans la porte externe (auxiliaire) de l’oscilloscope 
    afin d’obtenir la meilleure résolution possible lors de 
    l’acquisition et de l’échantillonnage. Nous avons établi qu’il 
    fallait au moins $30$ $\mu W$ d’intensité de signal pour 
    déclencher l’oscilloscope dans cette porte \cite{TektronixTDS5000}. 
    Nous déterminons le délai de chaque acquisition par rapport à 
    notre référence temporelle de départ, qui correspond à la position 
    du miroir à l’origine, c’est-à-dire lorsqu’il se trouve à 
    l’emplacement $0$. De cette manière, nous isolons l’expérience 
    pour observer uniquement ce qui se produit lorsqu'on modifie la 
    longueur du câble ou lorsqu’on déplace le miroir de son emplacement 
    initial vers des positions plus éloignées. 

    \noindent Les capacités et la résolution de l’oscilloscope sont 
    influencées par sa méthode d’acquisition du signal entré. 
    L’oscilloscope possède plusieurs modes d'acquisition: 
    échantillonnage (\guillemetleft Sample\guillemetright), détection 
    de pic, enveloppe, haute résolution et moyenne. Nous avons choisi 
    le mode d'acquisition moyen, car il nous fournit le signal de 
    sortie le plus typique du laser. Ce réglage gère le nombre de 
    signaux sinusoïdaux que nous pouvons définir dans le signal 
    d’entrée, ce qui permet de créer une moyenne des ondes sinusoïdales 
    acquises \cite{TektronixTDS5000}. Pour obtenir un signal propre et 
    éliminer le bruit de fond, nous devions acquérir en moyenne plus de 
    $10 000$ formes d’ondes. Cela nous permet d’obtenir une mesure plus 
    précise de la position temporelle moyenne du temps d'arrivée d'un 
    signal. Nous avons aussi réalisé l’expérience dans l’obscurité 
    pour réduire le bruit de fond, mais nous avons constaté que cette 
    étape n’était pas nécessaire. Les résultats n’ont pas été 
    radicalement différents, mais nous l’avons quand même fait. 

    \noindent Le mode d'échantillonnage est un autre aspect important 
    de la façon dont l'oscilloscope collecte des données dans le 
    domaine temporel. Ces modes sont l’échantillonnage en temps réel, 
    l’interpolation et l’équivalence temporelle. En mode 
    d’échantillonnage en temps réel, l’oscilloscope numérise tous 
    les points acquits après un événement déclencheur. Ce mode 
    d'échantillonnage est principalement utilisé pour les mesures 
    ponctuelles où les variations du signal en temps réel sont 
    importantes. Le mode d’interpolation crée des points intermédiaires 
    entre les points d’échantillonnage, ce qui permet de combler 
    les éventuelles lacunes. Cela donne une ligne droite ou une onde 
    sinusoïdale entre les points, ce qui rend la courbe plus lisse. 
    Nous ne voulons pas faire cela, car nous ne souhaitons pas que 
    le signal soit surchargé d’interpolations. Enfin, le mode d
    ’échantillonnage par équivalence de temps permet d’augmenter 
    le taux d’échantillonnage au-delà du taux d’échantillonnage 
    maximum en temps réel. Voyez la figure \ref{fig:EQ-time} pour 
    comprendre comment cela fonctionne. Ainsi, il est possible 
    d’obtenir le taux d’échantillonnage complet de l’oscilloscope, 
    soit $500$ $GS/s$ (gigéchantillons par seconde), en utilisant 
    ce mode. Notez que, si le déclenchement n’est pas en mode externe 
    et que votre état d’entrée se trouve sur un canal différent de 
    celui du signal de référence, votre taux d’échantillonnage maximal 
    sera désormais divisé par deux. L’échantillonnage maximal est 
    crucial pour l’oscilloscope, car il permet d’atteindre sa 
    résolution temporelle maximale pour notre signal, qui est de 
    4 $\pm 2 $ $ps$. Cela nous assure des mesures temporelles 
    précises. 
    
    \begin{figure}[!hptb]
        \centering
        \includegraphics[width=1.0\textwidth]{EQ _time_oscillo_fig.png}
        \caption{Diagramme illustrant le fonctionnement du mode d’acquisition du temps d’équivalence de l’oscilloscope \cite{TektronixTDS5000}. Cet appareil collecte un petit nombre d’échantillons au moment où l’événement de déclenchement se produit, ce qui lui permet d’obtenir le signal complet de notre impulsion. Le taux d’échantillonnage est supérieur à celui de son homologue en temps réel. L’oscilloscope fonctionne en mode équivalence de temps en effectuant un échantillonnage aléatoire, qui est déclenché par des événements aléatoires définis par l’horloge d’échantillonnage de l’instrument. Cette horloge fonctionne de manière asynchrone par rapport au signal d’entrée et au signal de déclenchement. Il enregistre ensuite un certain nombre d’échantillons d’acquisition. Après cela, l’oscilloscope combine plusieurs échantillons d’un signal répétitif en cours d’acquisition. Il régule ensuite la fréquence d’échantillonnage du signal d’entrée pour un enregistrement d’ondes régulières et complètes. }
        \label{fig:EQ-time}
    \end{figure}

    \noindent Avec ces paramètres, nous avons la meilleure résolution 
    possible avec l’équipement servant à collecter des données sur 
    notre signal. Nous enregistrons ensuite les signaux d’onde de 
    sortie sous forme de tableau CSV sur un ordinateur, ce qui 
    facilitera une analyse plus approfondie des données. Ce processus 
    d’acquisition prend environ trois minutes. Par la suite, discutons 
    des procédures et des résultats de nos expériences pour valider 
    notre capacité de mesurer des délais ultra-courts. 

    \noindent L'une consiste à mesurer la vitesse d'un signal dans un 
    câble BNC RG-58 \cite{ThorlabsBNC} avec des délais temporels à 
    l'aide de différentes longueurs de câbles. Cette expérience 
    facilite nos mesures, car les câbles ont une longueur déterminée 
    par le fabricant. Le délai est lié aux variations de longueur entre 
    les différentes longueurs de câbles. Ces dernières sont de 
    $172$, $270$, $522$, $1032$ et $3000$ $mm$. Les délais commencent 
    par une mesure avec un câble plus court que le premier. Dans ce 
    cas, le miroir ajustable est fixe.

    \noindent L’autre expérience consiste en un principe semblable, 
    soit la mesure de la vitesse de la lumière à partir d’un miroir 
    et à prendre des mesures à différentes distances 
    ($2,52$, $5,48$, $10,10$, $20,19$, $30,29$, $40,38$, $50,48$ 
    et $65,62$ $cm$ (mesuré à l’œil avec une règle)). Le délai mesuré 
    part d’une référence, appelée position $0$, située à $11,5$ $cm$ 
    du PBS. La lumière doit parcourir une distance double de celle 
    envoyée dans les deux sens à partir du séparateur de faisceau. 
    Ensuite, analyser les données afin de trouver le délai obtenu 
    pour chaque distance du miroir.
    
\end{doublespace}
