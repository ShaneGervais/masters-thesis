\label{sec:imaginary_results}
\begin{doublespace}
    Notre méthode pour mesurer la partie imaginaire de la valeur faible 
    est assez simple. Nous faisons interférer dans un 
    interféromètre de MZ une impulsion faiblement 
    mesurée avec un signal d'interférence 
    correspondant au spectre du laser, voir la figure \ref{fig:imagexp}. 
    Ainsi, nous pouvons accéder 
    après interférences au spectre du signal qui aura été modifié 
    par la mesure faible. Cependant, 
    certaines implications comme 
    la faible résolution de la mesure et la faible longueur 
    de cohérence du laser rendent difficile la 
    mesure de la partie imaginaire de la valeur faible.
\end{doublespace}


\begin{doublespace}

     \noindent À partir de l'équation \ref{eq:imaginary_part}, le 
     déplacement fréquentiel maximal attendu pour la partie imaginaire de la valeur faible est donné par:
    $\expval{\hat{\omega}}_{max} = \frac{\tau}{8\sigma^2}$. 
    Cela signifie que, pour une impulsion avec une durée de 
    $10$ $ns$, et un délai de $40,16$ $ps$ (correspondant au 
    deuxième pic de visibilité mentionné dans la section 
    \ref{sec:imaginary_propose}) appliqué sur le système, on 
    s'attend à un déplacement dans le spectre de fréquence de seulement 
    $7,989$ kHz (avec $\omega \equiv 2\pi f$). Cette valeur est 
    extrêmement petite par rapport à la fréquence de notre laser, qui se 
    situe dans l'ordre $THz$ avec plusieurs modes \cite{ThorlabsNPL64B}. 
    %Les mesures interférométriques 
    %régulières effectuées dans un laboratoire ne présentent qu’une 
    %résolution de l'ordre $MHz$, ce qui nécessite des bras 
    %d'interféromètres assez longs. 
    Donc, pour améliorer l'effet par l'interaction faible,
    il faut réduire la durée de l'impulsion. 
    Nous allons choisir une longueur d'impulsion de $4$ $ns$ (le minimum 
    que notre laser peut effectuer) pour un attendue de $49,932$ $kHz$.
    
    \subsubsection{Vérification de la partie imaginaire}

    Nous avons commencé par déterminer si nous pouvions observer
    des déplacements fréquentiels dans la densité spectrale de
    puissance (DSP), qui représente la 
    distribution de la puissance d'un signal 
    en fonction de la fréquence. Elle permet d’identifier les 
    fréquences qui sont présentes à chaque portion d'un signal et 
    avec quelle intensité. Nous avons utilisé
    les états de polarisation $\ket{H}$, $\ket{V}$, $\ket{R}$,
    $\ket{A}$, $\ket{L}$ et $\ket{D}$ comme états d'entrée.
    Les séries on été prises avec un délai de $40,16$ $ps$, sur 
    plusieurs impulsions de $4$ $ns$ pendant $2$ $\mu s$. Pour chaque état
    d'entrée pendant la prise de données, nous prenons une série de 
    mesures avec des interférences
    et sans interférences, puis calculons la différence. Ensuite, 
    nous effectuons une transformation de Fourier à haute résolution 
    pour obtenir la DSP avec la fonction \textit{FFT} de l'oscilloscope
    sur une étendue spectrale analysée de $1,4062$ $GHz$ \cite{TektronixTDS5000}. L'analyse
    de la DSP est présentée dans la figure \ref{fig:spectre}, 
    qui montre le spectre de puissance total.

    %Voici ce spectre que nous avons obtenu à partir de nos données 
    %expérimentales (figure \ref{fig:spectre}).

    \begin{figure}[!htpb]
        \centering
        \includegraphics[width=1.0\textwidth]{spectre.png}
        \caption{DSP totale résultant d'une mesure faible temporelle. Ici, les 
        lignes verticales représentent la position des pics de puissance 
        dans le spectre. La fréquence centrale est de 
        $703,12$ $MHz$ pour une durée spectrale de $1,4062$ $GHz$ avec une 
        résolution temporelle de $4$ $ps$.
        La figure \ref{fig:spectre1054} montre un zoom 
        sur le pic de puissance à $175,8$ $MHz$.}
        \label{fig:spectre}
    \end{figure}

    \noindent Pour trouver des
    déplacements induits par la mesure faible, nous analysons
    le déplacement de chacun des pics du spectre. Pour
    chacun de ces pics, nous effectuons un ajustement de paramétrique
    polynomiale du quatrième ordre, comme dans nos dernières analyses 
    pour la mesure de la vitesse d'un signal électrique et de la lumière
    dans la section \ref{sec:lightspeed}, pour estimer 
    précisément la position fréquentielle du pic. En
    comparant chacune de ces fréquences à celles de l’état $\ket{R}$,
    nous obtenons le tableau suivant pour le pic à la fréquence $175,8$ $MHz$:

    \begin{longtable}{p{3.0cm} p{3.0cm} p{3.0cm}}
    \caption{Tableau récapitulatif des pics de puissance dans le spectre de
    puissance, avec les états d'entrée, les fréquences mesurées et les
    déplacements fréquentiels. Les fréquences sont mesurées en
    mégahertz (MHz) et les déplacements fréquentiels en kilohertz (kHz) 
    par rapport à l'état $\ket{R}$. } \\
    \toprule
    \label{table:imaginare-table}
    État d'entrée & Fréquence (MHz) & Déplacement fréquentiel (kHz)\\
    \midrule
    \endfirsthead
    
    \toprule
    État d'entrée & Fréquence (MHz) & Déplacement fréquentiel (kHz)\\
    \midrule
    \endhead
    
    \midrule
    \multicolumn{3}{r}{{\dots}} \\
    \midrule
    \endfoot
    
    \bottomrule
    \endlastfoot
    
    $\ket{H}$ & 175.79 &  8.2416  \\
    $\ket{V}$ & 175.79 &  10.851 \\
    $\ket{R}$ & 175.78 &   0.000 \\
    $\ket{A}$ & 175.74 &  -39.963 \\
    $\ket{L}$ & 175.79 &  2.0604 \\
    $\ket{D}$ & 175.78 & 0.96152 \\

    %\midrule


    \end{longtable}

    \noindent Nous avons observé que les déplacements fréquentiels
    mesurés sont généralement faibles, de l'ordre de quelques kilohertz,
    ce qui est conforme à nos attentes théoriques. Nous pouvons
    visualiser ces déplacements dans la figure \ref{fig:spectre1054}.

    \begin{figure}[!htpb]
        \centering
        \includegraphics[width=1.0\textwidth]{spectre_175.png}
        \caption{Zoom sur le pic de puissance à $175,8$ $MHz$. On peut 
        observer la présence de plusieurs pics de puissance, avec leur 
        ajustements de paramétriques polynomiales du quatrième ordre pour 
        determiner chaque position fréquentielle des états d'entrée.}
        \label{fig:spectre1054}
    \end{figure}

    \noindent Bien que la différence prévue soit de $49,932$ $kHz$ (du minimum au maximum), ce qui 
    n’a pas été strictement observé, l’analyse de Fourier révèle 
    effectivement un déplacement de certains pics dans le spectre, 
    mesuré dans l'ordre de kilohertz. Cependant, ces déplacements ne 
    semblent pas être cohérents ni suivre une tendance correspondant à 
    notre théorie, ainsi que ceux des autres pics dans la figure \ref{fig:spectre}.
    Nous pouvons attribuer ce problème à un manque de stabilité
    du laser, ce qui rend difficile la mesure de la partie 
    imaginaire de la valeur faible.

    \subsubsection{Tentatives de mesurer la partie imaginaire}

     Considérons une prise de données similaires
    à ceux effectuées pour la partie réelle (voir la section \ref{sec:real_experiment}). 
    Cette fois ici, prenons sur une 
    durée de $100$ $ns$ avec un délai de $180,71$ $ps$, un délai 
    qui correspond au dernier pic de visibilité (voir la figure \ref{fig:vis}), 
    et une durée d'impulsion de $10$ $ns$. À partir de ces données, nous effectuons
    une analyse de Fourier pour extraire la DSP directement de 
    l'interferomètre de polarisation. Ensuite, nous calculons comment le
    centroïde de chaque pic se déplace en fonction de l'état
    d'entrée d'écrit par:

    \begin{equation}
        f_{cent.} = \frac{\sum_{i=1}^{n} f_i \cdot P_i}{\sum_{i=1}^{n} P_i}
        \label{eq:centroid}
    \end{equation}
    \noindent où $f_{cent.}$ est la fréquence du centroïde, $n$ est le
    nombre de pics dans le spectre, $f_i$ est la fréquence de chaque pic 
    et $P_i$ est la puissance
    de chaque pic. 
    Le centroïde du spectre est une mesure de la position moyenne des pics
    de puissance dans le spectre, ce qui nous permet de mesurer les
    déplacements fréquentiels causés par la mesure faible.
    Nous remarquons que \ref{eq:centroid} est similaire à l'équation \ref{eq:imaginary_part}
    qui décrit la partie imaginaire de la valeur faible.
    Nous pouvons donc utiliser cette équation pour calculer
    la partie imaginaire de la valeur faible en utilisant les déplacements
    du centroïde. 
    
    \noindent Nous avons calculé la différence entre le
    centroïde de chaque état d'entrée et celui de l'état $\ket{V}$
    car on s'attend quelle soit minimale (ainsi que $\ket{H}$).
    Nous évaluons la partie imaginaire de la valeur faible de les
    trajets de polarisation discutés dans le chapitre \ref{chap:3}. Voici les
    résultats que nous avons obtenus pour les déplacements fréquentiels
    pour le trajet de polarisation $\ket{H} \to \ket{R} \to \ket{V} \to \ket{L} \to \ket{H}$
    présenté dans la figure \ref{fig:imaginary_part_HRVLH}.

    \begin{figure}[!htpb]
        \centering
        \includegraphics[width=1.0\textwidth]{imag_weak_value_path_4 copy 3.png}
        \caption{Déplacement fréquentiel mesuré pour les trajets de polarisation 
        $\ket{H} \to \ket{R} \to \ket{V} \to \ket{L} \to \ket{H}$, 
        avec un délai de $180,71$ $ps$ et une durée d'impulsion de $10$ $ns$. 
        Les barres d'erreur verticales représentent l'incertitude de la mesure,
        calculée à partir de la variation des fréquences
        mesurées pour chaque état d'entrée. Les déplacements sont mesurés en
        kilohertz (kHz) par rapport à l'état $\ket{V}$. La courbe en tirets 
        représente la valeur absolue
        de la prédiction théorique (le signe n’est pas accessible avec notre méthode
        de centroïde basée sur l’intensité spectrale seule).}
        \label{fig:imaginary_part_HRVLH}
    \end{figure}

    \noindent Ces données ont été obtenues en calculant, pour chaque état d'entrée, 
    le centroïde de chaque pic à partir de la DSP à l'aide de l'équation 
    \ref{eq:centroid}, puis en comparant cette valeur à la fréquence du centroïde 
    de l'état $\ket{V}$, $\Delta f(\theta) = f_{\text{centroid}}(\theta) - f_{\text{centroid}}(\ket{V})$. 
    Ce choix s'explique par le fait que les états de 
    polarisation $\ket{H}$ et $\ket{V}$ ne présentent pas de décalage fréquentiel 
    attendu, et que l'état $\ket{V}$ sert de référence non décalée temporellement.
    Dans nos données, on ne voit que des décalages positifs, alors que la
    théorie (voir équation \ref{eq:expval_omega_circular}) prévoit des 
    décalages positifs et négatifs selon l’angle.
    Si l’on considère la valeur absolue $|\Delta f(\theta)|$, l’ajustement
    est \emph{bon sur la première moitié} du balayage
    ($\theta \approx 0^\circ$–$90^\circ$) et \emph{moins bon sur la seconde}
    ($90^\circ$–$180^\circ$), précisément là où la courbe théorique devrait
    devenir négative. Ceci s'explique par l'absence de référence de phase et par le fait
    que le centroïde est extrait d’un spectre d’intensité 
    positif $P(f_i) \ge 0$ : le traitement
    est donc insensible au signe du décalage observé et ne conserve que sa magnitude.
    Cela renforce l’idée que le problème provient
    de l'absence des valeurs négatives.
    Le déplacement fréquentiel maximal mesuré
    (par rapport à l'état $\ket{V}$) est de $640,708$ $kHz$, ce qui reste 
    largement supérieur à la valeur attendue théoriquement.
    La tendance de ces déplacements ne correspond 
    pas directement à nos attentes théoriques de $|sin(2\theta)|$, 
    mais ils suivent une tendance similaire.
    Néanmoins, ces déplacements dans la DSP indiquent que la
    partie imaginaire de la valeur faible est présente et elle suit
    une tendance sinusoïdale absolue et permet l'état soit caractérisable à partir de
    ces résultats. Nous pouvons effectivement
    calculé les amplitudes de probabilité (voir figure \ref{fig:prob_amp_imaginary_part_HRVLH})
    pour chaque état d'entrée en utilisant les relations suivantes:

    \begin{align}
        |a| &= \sqrt{\mathcal{I}(\expval{\hat{\pi}_W})} \label{eq:amplitude_a_im}\\
        |b| &= \sqrt{1-|a|^2} \label{eq:amplitude_b_im}
    \end{align}

    \begin{figure}[!htpb]
        \centering
        \includegraphics[width=1.0\textwidth]{imaginary_probability_path_4.png}
        \caption{
        La partie imaginaire des amplitudes de probabilité mesurées pour le trajet de polarisation
        $\ket{H} \to \ket{R} \to \ket{V} \to \ket{L} \to \ket{H}$.
        Les barres d'erreur sont propagées à partir de l'incertitude
        de la mesure des déplacements fréquentiels de la partie imaginaire de ce
        trajet de polarisation. Ces amplitudes proviennent d’une mesure de la
        valeur absolue de la partie imaginaire; le signe est perdu dans notre
        procédure actuelle.
        }
        \label{fig:prob_amp_imaginary_part_HRVLH}
    \end{figure}

    \noindent Ces résultats démontrent que la partie imaginaire de la valeur faible
    est effectivement mesurable, mais que les déplacements ne correspondent pas
    avec nos attentes théoriques. Cependant, nous avons observé des
    déplacements dans le spectre de fréquence, ce qui indique que la
    partie imaginaire de la valeur faible est présente et elle suit
    une tendance sinusoïdale. À ce moment présent, les résultats présentés
    sont nos meilleures
    résultats pour la partie imaginaire de la valeur faible et que les autres
    tentatives de mesure n'ont pas été concluantes.
    
    \subsubsection{Implications pour la partie imaginaire}

    Les résultats des décalages fréquentiels mesurés présentent un 
    comportement imprévisible, signe d’une instabilité dans notre 
    source laser. De plus, la longueur des bras de l’interféromètre 
    de MZ rend l’appareil sensible aux vibrations et aux 
    fluctuations environnementales: la moindre fluctuation de 
    phase dans un bras décale le profil du spectre de puissance, 
    aggravant ainsi l’instabilité observée. Ces facteurs combinés 
    peuvent affecter la précision des mesures.
    On entend par effet fréquentiel un déplacement dans le spectre 
    de fréquence causé par la mesure faible, déplacement qui se 
    manifeste dans la DSP. Si nos données révèlent bien un décalage 
    du centroïde spectral, nous n’avons pas pu mesurer la partie 
    imaginaire de la valeur faible: les déplacements restent trop 
    faibles et incohérents pour coïncider avec les prédictions 
    théoriques.
    Pour assurer une meilleure cohérence, il est nécessaire de 
    stabiliser ou de modifier la source laser. Des études futures 
    examineront l’intégration d’une source impulsionnelle 
    cohérente; dans ces conditions, la partie imaginaire devrait 
    devenir mesurable, permettant une caractérisation complète.

    \noindent Enfin, soulignons que la manipulation avec la source utilisée ne permet 
    \emph{déjà pas} d’observer les décalages négatifs attendus pour la 
    partie réelle. L’absence de cohérence suffisante supprime cette observation. 
    Dans ces conditions, il est \emph{inutile} de chercher 
    à « améliorer » la partie imaginaire : cette voie expérimentale 
    est abandonnée. Les résultats de la partie imaginaire présentés 
    ici sont fournis à titre indicatif pour la discussion, sans prétention 
    quantitative. Dans le prochain chapitre, nous allons 
    conclure sur ces résultats et proposer des perspectives pour de futures 
    recherches.

\end{doublespace}

