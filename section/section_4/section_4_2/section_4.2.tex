\begin{doublespace}
    Notre méthode pour mesurer la partie imaginaire de la valeur faible 
    est assez simple. Nous faissons interférer dans un 
    inertféromètre de MZ une impulsion faiblement 
    mesurée avec un signal d'interférence 
    correspondant au spectre du laser, voir la figure \ref{fig:imagexp}. 
    Ainsi, nous pouvons accéder 
    après intferérences au spectre du signal qui aura été modifié 
    par la mesure faible. Cependant, 
    certaines implications comme la faible visibilité de l'interférence, 
    la faible résolution de la mesure et la faible longueur 
    de cohérence du laser rendent difficile la 
    mesure de la partie imaginaire de la valeur faible.
\end{doublespace}

\subsubsection{Implications pour la partie imaginaire}


\begin{doublespace}

     À partir de la relation proportionelle à la partie imaginaire 
     de la valeur faible:

    \begin{equation}
        \frac{\mathcal{I}(\expval{\hat{\pi}_W})}{\tau} = \expval{\hat{\omega}} \propto \frac{\tau}{8\sigma^2}
    \end{equation}
    
    \noindent Cela signifie que, pour une impulsion avec une durée de 
    $10$ $ns$, et un délai de $40$ $ps$ appliqué sur le système, on 
    s'attend à un déplacement dans le spectre de fréquence de seulement 
    $7,96$ kHz (avec $\omega \equiv 2\pi f$). Cette valeur est 
    extrêmement petite par rapport à la fréquence de notre laser, qui se 
    situe dans l'ordre des térahertz. Les mesures interférométriques 
    régulières effectuées dans un laboratoire ne présentent qu’une 
    résolution en $MHz$, ce qui nécessite un délai extrêmement long. Cela 
    dépasse la marge de mesure minimale ainsi que sa facilité 
    d’utilisation dans un laboratoire. En effet, pour effectuer la mesure 
    dans un interféromètre de Michelson, les distances nécessaires 
    dépassent la longueur de cohérence du laser de $0,2$ $mm$. Cependant, 
    nous allons choisir une longueur d'impulsion de $4$ $ns$ (le minimum 
    que notre laser peut effectuer) pour un attendue de $49,77$ $kHz$, 
    encore très petits. 
    
    \subsubsection{Mesure de la partie imaginaire}
    
    Pour trouver des déplacements induits par la 
    mesure faible, nous calculons la densité spectrale du spectre de 
    puissance (DSP) sur un ensemble d'impulsions du signal expérimentale.  
    La fonction \textit{pspectrum} de \textsc{MATLAB} facilite se calcule pour 
    nous, en effectuent une transformation de Fourier à haute résolution sur 
    plusieurs impulsions pour obtenir la DSP. 
    Voici ce spectre que nous avons obtenu à partir de nos données 
    expérimentales (figure \ref{fig:spectre}).

    \begin{figure}[!htpb]
        \centering
        \includegraphics[width=1.0\textwidth]{powerdensityspectrum.png}
        \caption{DSP résultant d'une mesure faible temporelle. Ici, les 
        lignes verticales représentent la position des pics de puissance 
        dans le spectre. La figure \ref{fig:spectre1054} montre un zoom 
        sur le pic de puissance à $1054$ $MHz$.}
        \label{fig:spectre}
    \end{figure}

    \begin{figure}[!htpb]
        \centering
        \includegraphics[width=1.0\textwidth]{PSD_zoom_1054Mhz.png}
        \caption{Zoom sur le pic de puissance à $1054$ $MHz$. On peut 
        observer la présence de plusieurs pics de puissance, avec leur 
        ajustement de courbe polynomiale du quatrième ordre pour 
        determiner chaque position fréquentielle des états d'entrée.}
        \label{fig:spectre1054}
    \end{figure}

    \noindent Nous analysons ensuite le spectre pour repérer les pics
    avec une amplitude supérieure à $1\%$ de la puissance maximale. Pour 
    chacun de ces pics, nous effectuons un ajustement de courbe 
    polynomiale du quatrième ordre, comme dans nos dernières analyses 
    pour estimer précisément la position fréquentielle du pic. En 
    comparant chacune de ces fréquences à celles de l’état $\ket{V}$, 
    nous obtenons le tableau suivant:

    \begin{longtable}{p{3.0cm} p{3.0cm} p{3.0cm} p{3.0cm}}
    \caption{Mesure de la vitesse du signal dans les câbles BNC pour 
    différents ajustements de courbe} \\
    \toprule
    \label{table:imaginare-table}
    Pic & État d'entrée & Fréquence (MHz) & Déplacement fréquentiel (kHz)\\
    \midrule
    \endfirsthead
    
    \toprule
    Pic & État d'entrée & Fréquence (MHz) & Déplacement fréquentiel (kHz)\\
    \midrule
    \endhead
    
    \midrule
    \multicolumn{4}{r}{{À suivre sur la prochaine page}} \\
    \midrule
    \endfoot
    
    \bottomrule
    \endlastfoot
    
1 & $\ket{H}$ & 175.8   &  -5.4971 \\
1 & $\ket{V}$ & 175.8   &   0      \\
1 & $\ket{R}$ & 175.8   &  -8.2456 \\
1 & $\ket{A}$ & 175.8   &  -8.2456 \\
1 & $\ket{L}$ & 175.81  &   8.2456 \\
1 & $\ket{D}$ & 175.81  &   8.2456 \\

2 & $\ket{H}$ & 351.57  &  -5.4971 \\
2 & $\ket{V}$ & 351.58  &   0      \\
2 & $\ket{R}$ & 351.56  & -16.491  \\
2 & $\ket{A}$ & 351.48  & -101.52  \\
2 & $\ket{L}$ & 351.56  & -21.988  \\
2 & $\ket{D}$ & 351.55  & -24.737  \\

3 & $\ket{H}$ & 527.42  &  27.485  \\
3 & $\ket{V}$ & 527.39  &   0      \\
3 & $\ket{R}$ & 527.41  &  16.491  \\
3 & $\ket{A}$ & 527.30  & -90.701  \\
3 & $\ket{L}$ &   NaN   &   NaN    \\
3 & $\ket{D}$ &   NaN   &   NaN    \\

4 & $\ket{H}$ & 878.82  &  32.982  \\
4 & $\ket{V}$ & 878.78  &   0      \\
4 & $\ket{R}$ & 878.73  & -52.222  \\
4 & $\ket{A}$ &   NaN   &   NaN    \\
4 & $\ket{L}$ & 879.02  & 236.200  \\
4 & $\ket{D}$ & 878.95  & 161.990  \\

5 & $\ket{H}$ & 1054.7  &  13.743  \\
5 & $\ket{V}$ & 1054.7  &   0      \\
5 & $\ket{R}$ & 1054.7  &  24.737  \\
5 & $\ket{A}$ & 1054.7  & -16.491  \\
5 & $\ket{L}$ & 1054.6  & -93.450  \\
5 & $\ket{D}$ & 1054.6  & -46.725  \\


\end{longtable}

    \noindent Bien que la différence prévue soit de $50$ $kHz$, ce qui 
    n’a pas été strictement observé, l’analyse de Fourier révèle 
    effectivement un déplacement de certains pics dans le spectre, 
    mesuré dans l'ordre de kilohertz. Cependant, ces déplacements ne 
    semblent pas être cohérents ni suivre une tendance correspondant à 
    notre théorie. Les interférences présentent un comportement 
    imprévisible, ce qui indique que nous avons dépassé la longueur de 
    cohérence du laser. En conséquence, nous observons des interférences 
    constructives et destructives qui se produisent de manière inconsistante. 
    %Malgré cette limitation, nos résultats indiquent qu’un 
    %effet fréquentiel est causé par la mesure faible. 
    On entend par
    effet fréquentiel un déplacement dans le spectre de fréquence
    causé par la mesure faible. Ce déplacement se manifeste dans la DSP
    cependant, nous n'avons pas pu
    mesurer la partie imaginaire de la valeur faible qui suit nos attentes
    théoriques car les
    déplacements sont trop petits et incohérent pour obtenir une
    mesure fiable.
    Pour assurer une meilleure cohérence, il est nécessaire de modifier 
    notre source.
    Des études futures 
    examineront l’intégration d'une source impulsionnelle cohérente et 
    permettront de mesurer la partie imaginaire de la valeur faible pour 
    une caractérisation complète. 

\end{doublespace}

