\begin{doublespace}
    Notre méthode pour mesurer la partie imaginaire de la valeur faible est assez simple. Nous faisons passer une impulsion faiblement mesurée à travers un interféromètre MZ en y ajoutant un signal de référence, soit celui du laser à une fréquence connue $f_0 = 4,68\cdot10^{14}$ $Hz$. Ce dernier fait en sorte que la fréquence mesurée sera $f_0+\expval{\hat{\omega}}$. Cependant, certaines implications nous empêchent de mesurer la partie imaginaire de la valeur faible.
\end{doublespace}

\subsubsection{Implications pour la partie imaginaire}


\begin{doublespace}

    \noindent Le terme proportionnel dans la partie imaginaire de la valeur faible:

    \begin{equation}
        \frac{\mathcal{I}(\expval{\hat{\pi}_W})}{\tau} = \expval{\hat{\omega}} \propto \frac{\tau}{8\sigma^2}
    \end{equation}
    
    \noindent Signifie que, pour une période temporaire de notre laser de $10$ $ns$, associée à un retard de $40$ $ps$, on doit observer un déplacement dans le spectre de fréquence de seulement 417,5 kHz. Cette valeur est extrêmement petite par rapport à la fréquence de notre laser, qui se situe dans les térahertz, et qui est difficile à mesurer. Les mesures interférométriques régulières effectuées dans un laboratoire ne présentent qu’une résolution en MHz, ce qui nécessite un délai extrêmement long, ce qui dépasse la marge de mesure minimale ainsi que sa facilité d’utilisation dans un laboratoire. En effet, pour effectuer la mesure dans un interféromètre de Michelson, les distances nécessaires dépassent la longueur de cohérence du laser de 0,2 mm. Des spectromètres ou d’autres méthodes photoniques, telles que les combs de fréquence, qui atteignent cette résolution sont vraiment coûteux qui contredit l’objectif de créer un dispositif dans un laboratoire commun pour la caractérisation d'un état quantique. 

    \noindent Nous pourrions envisager de mesurer si un pic se produit lors de l’interférence entre un état de polarisation linéaire et circulaire. Le trajet de polarisation qui nous permet de bien vérifier ceci est $\ket{H} \to \ket{R} \to \ket{V} \to \ket{L}$ et $\ket{D} \to \ket{R} \to \ket{A} \to \ket{L}$, nous poursuivons avec le dernier car elle possède une superposition de nos états de base, combinés linéairement avec $\ket{D}$ et $\ket{A}$ et circulairement avec $\ket{R}$ et $\ket{L}$. Comment pouvons-nous observer ce pic ? Avec le spectre de puissance, car nous pouvons simplement le mesurer directement à l’aide de l'oscilloscope et qu’il démontre l’enveloppe du spectre. On suppose que, grâce à l’enveloppe du spectre fréquentiel (le spectre de puissance), nous pouvons observer s'il y a enfet un déplacement fréquenciel, et que nous pourrons mesurer le décalage fréquentiel du spectre fréquentiel avec la résolution nécessaire à l’avenir. Cela nous permettra de déterminer que la partie imaginaire comme étant réelle. 

    \noindent Le spectre de puissance est directement lié à la transformation de Fourier de l’intensité :

    \begin{equation}
        P(\omega) = \int_{-\infty}^{\infty} I(t)e^{-i\omega t}dt
    \end{equation}

    \noindent Comme nous souhaitons maximiser notre capacité à atteindre le potentiel de voir le pic, nous avons réduit la taille du faisceau au minimum du laser, soit 4 ns. Notre analyse consiste à effectuer une transformation de Fourier rapide (fft fast Fourier transform) à l’aide de la bibliothèque numpy pour python. Il y a d’autres bibliothèques qu’on peut utiliser, comme celle de scipy, mais celle de numpy suffit. Pour une réduire les fuites spectrales lors de la transformée de Fourier rapide, nous appliquons une fenêtre de Hanning au signal, car il présente une discontinuité de l'amplitude du signal. Donc, on suppose que le signal est périodique et fini, ce qui réduit le bruit à 0. La fenêtre de Hanning diminue les oscillations en appliquant un coefficient sur chaque élément, ce qui donne un signal plus lisse. Ensuite, la figure A illustre les données du spectre de puissance pour le trajet de polarisation B. 

    \begin{figure}[!h!t!p!b]
        \centering
        \includegraphics[width=1.0\textwidth]{power_spectrum_DRAL.png}
        \caption{Le spectre de puissance d’une impulsion avec un taille temporel de $4$ $ns$ interférente avec un état faiblement mesuré avec un délai de $167$ $ps$ dont le degré $3$ est l'état $\ket{D}$, $25$ est $\ket{R}$, $48$ est $\ket{A}$ et $70$ est $\ket{L}$, puis $93$ est de retour à $\ket{D}$. On remarque que les états linéaires, c’est-à-dire $\ket{D}$ et $\ket{A}$, restent à $3,3$ $MHz$, alors que $\ket{R}$ et $\ket{L}$ se situent à $5,8$ $MHz$. Dont le pic initial (le plus à gauche) soit le pic central du spectre}
        \label{fig:DRAL_power_spectrum}
    \end{figure}

    \noindent Cela montre qu’un pic existe dans une polarisation linéaire (D et A) par rapport à (R et L), correspondant à une différence de 250 MHz. Ce résultat ne suit pas le modèle de notre partie imaginaire, mais il démontre qu'il est possible de mesurer la partie imaginaire. 

\end{doublespace}

