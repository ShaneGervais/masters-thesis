Notre expérience a été optimisée pour être automatisée. 
Tout d’abord, notre appareil expérimental subit un 
étalonnage consistant en une seule demi-plaque d’onde 
pendant la phase de préparation de la procédure 
expérimentale. Cette dernière est ensuite tournée pour 
envoyer la polarisation V correspondant à notre position 
temporelle moyenne de référence, qui est notre état 
d’entrée à retard 0. L’oscilloscope prend des données 
pour notre position 0, puis notre code fait tourner la 
demi-plaque d’onde pour envoyer le retard maximal, soit 
l’état de polarisation H. Notez qu’un passage de V à H, 
qui correspond à un changement de 90 degrés par rapport 
aux états de base, revient en réalité à un décalage de 
45 degrés sur la demi-plaque d’onde. Chaque état de 
polarisation d’étalonnage (H et V) est mesuré séparément, 
puis les 10 résultats sont moyennés. L’oscilloscope est 
évidemment utilisé pour mesurer les temps d’équivalence 
et il calcule ensuite la moyenne de plus de 10000 formes 
d’onde. Après l’étalonnage, nous pouvons effectuer chacun 
des chemins de polarisation énoncés plus haut dans ce 
chapitre. Pour changer de chemin de polarisation, nous 
devons ajouter un quart de plaque d’onde à la préparation 
de l’état d’entrée, en fonction du chemin que nous 
caractérisons. Le premier chemin consiste seulement en 
une demi-plaque d’onde. Nous prenons des données tous les 
5 degrés, ce qui représente 2,5 degrés en réalité. Pour 
chaque état d’entrée de polarisation, nous prenons 
3 fichiers distincts que nous moyennons, puis nous 
comparons leur position temporelle moyenne avec le 
dossier d’étalonnage pour obtenir le retard associé à cet 
état. Voici les résultats de chaque délai pour les différentes 
état d'entrée pour le trajet 
$\ket{H} \rightarrow \ket{D} \rightarrow \ket{A} \rightarrow \ket{V}$.

\begin{figure}[!h!t!p!b]
    \centering
    \includegraphics[width=1.0\textwidth]{real_weak_value_measurement_3.pdf}
    \caption{Dispositif expérimental pour la partie réel de la valeur faible}
    \label{fig:HDAV_delay_results}
\end{figure}

Cette série de données a été prise avec une 
interaction faible correspondant à un délai de 
$167$ $ps$, obtenu en allongeant le bras $\ket{H}$ 
de $2,5$ $cm$ 
par rapport au bras $\ket{V}$. Avec les mêmes techniques 
d'ajustement que celles utilisées pour 
l'expérience sur la vitesse de la lumière, nous 
mesurons les retards indiqués dans la figure \ref{fig:HDAV_delay_results}. 
L’erreur résulte de l’incohérence de nos 
modifications physiques dans notre expérience, 
que nous ne pouvons pas compenser, telles que 
les vibrations de la table et les variations 
d’alignement du laser pendant les longues 
périodes d’acquisition de données pour chaque 
état d’entrée. Voici les résultats pour le 
chemin $\ket{H} \rightarrow \ket{R} \rightarrow \ket{L} \rightarrow \ket{V}$.

\begin{figure}[!h!t!p!b]
    \centering
    \includegraphics[width=1.0\textwidth]{real_weak_value_measurement_4.pdf}
    \caption{Dispositif expérimental pour la partie réel de la valeur faible}
    \label{fig:HDAV_delay_results}
\end{figure}

Ces résultats suivent la même courbe que ceux 
du chemin précédent. C’est logique, puisque les 
états de polarisation R et L possèdent toujours 
autant d’états de base ($\ket{D}$ et $\ket{A}$). 
On ne peut pas 
déterminer juste à partir de la partie réelle de 
la valeur faible si l’état initial est elliptique. 
L’information concernant le changement de phase 
ou d’ellipticité des états de polarisation se 
trouve dans la partie imaginaire de la valeur 
faible. Voici les résultats pour le chemin 
$\ket{H} \rightarrow \ket{D} \rightarrow \ket{R} \rightarrow \ket{A}$.

\begin{figure}[!h!t!p!b]
    \centering
    \includegraphics[width=1.0\textwidth]{real_weak_value_measurement_5_2.pdf}
    \caption{Dispositif expérimental pour la partie réel de la valeur faible}
    \label{fig:HDAV_delay_results}
\end{figure}

Ceci est plus intéressant, car c'est une preuve 
directe que nous avons toujours la même partie 
réelle de la valeur faible. Par conséquent, 
toute l’information nécessaire au décodage entre 
les polarisations se trouve dans la variable 
complexe du décalage temporel, c’est-à-dire le 
décalage de fréquence obtenu en mesurant la 
valeur minimale. De plus, à partir de ces 
données pour chaque trajet, nous pouvons 
calculer directement les amplitudes de 
probabilité de l’état quantique à partir de ces 
mesures de retard comme nous avons mentionné dans la section 2.3. 