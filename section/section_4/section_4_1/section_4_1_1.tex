\label{sec:real_results}
\begin{doublespace}
    Nous avons optimisé notre expérience pour qu’elle soit 
    automatisée avec un code \textsc{Python} qui contrôle les supports de 
    rotation motorisés pour les lames d'onde à l'étape d'entrée, 
    ainsi que l'oscilloscope qui contrôle les acquisitions de données 
    \cite{pylablib_thorlabs_kinesis,TektronixTDS5000,oscilloscope_coding}. 
    Nous collectons des données à chaque 
    $2,5$ degrés. Pour chaque état d’entrée de polarisation, nous 
    prenons cinq fichiers distincts que nous moyennons, puis nous 
    comparons leur temps d'arrivée moyen avec le dossier de 
    calibration pour obtenir le délai associé à cet état. 
    Nous combinons 
    les résultats pour la partie 
    réelle de la valeur faible à nos valeurs théoriques 
    calculées dans le chapitre précédent dans les équations : 
    \ref{eq:expval_t_1}, \ref{eq:expval_t_2} et \ref{eq:expval_t_3}.
    Pour simplifier la notation des équations, nous redéfinissons 
    l'angle $\theta$ en posant $\theta \leftarrow 2\theta_{\rm phys}$,
    où $\theta_{\rm phys}$ est l'angle physique de la lame demi-onde,
    de sorte que $H$ corresponde à $\theta=0^\circ$ (ou $180^\circ$) et $V$ à $90^\circ$.
    Voici les résultats de chaque délai pour les différents états d’entrée pour
    le trajet
    $\ket{H}\rightarrow\ket{D}\rightarrow\ket{V}\rightarrow\ket{A}\rightarrow\ket{H}$
    (figure \ref{fig:HDAV_delay_results}).
\end{doublespace}

\begin{figure}[!h!t!p!b]
    \centering
    \includegraphics[width=1.0\textwidth]{real_weak_value_path_3.png}
    \caption{Résultats expérimentaux pour la partie réelle de la valeur 
    faible du trajet de polarisation 
    $\ket{H}\rightarrow\ket{D}\rightarrow\ket{V}\rightarrow\ket{A}\rightarrow\ket{H}$. 
    Rappelez-vous que
    nous avons défini l'angle $\theta$ de sorte que $\theta \leftarrow 2\theta_{\rm phys}$,
    où $\theta_{\rm phys}$ est l'angle physique de la lame demi-onde.
    Les barres d’erreurs horizontales représentent l’erreur de
    la position de l’angle induite par notre
    support de rotation motorisé de $\pm 0.3 \degree$ \cite{ThorlabsROT}. Elles ne sont pas visibles, car elles sont
    trop petites pour être prises en
    considération, alors nous les avons négligées.
    Les barres verticales représentent l'incertitude de la partie réelle.
    Chaque point est la moyenne de $5$ mesures pour chaque état d'entrée.
    Les données sont normalisées par le
    délai de référence $\tau$ de $167$ $ps$.
    }
    \label{fig:HDAV_delay_results}
\end{figure}

\begin{doublespace}
    \noindent Cette série de données a été collectée avec une 
    interaction faible correspondant à un délai de $167$ $ps$, obtenu en 
    allongeant le bras $\ket{H}$ de $2,5$ $cm$ par rapport au bras 
    $\ket{V}$. En utilisant les mêmes méthodes d’ajustement que celles
    employées dans l’expérience sur la vitesse de la lumière, nous
    mesurons les délais de chaque état $\ket{\psi(\theta)}$. Nous avons
    également constaté que les données expérimentales de 
    la partie réelle de la valeur faible
    de ce trajet de polarisation est bien représentée par une fonction
    cosinus carrée mais pas parfaitement centrée.
    En effet, la partie réelle de la valeur faible est donnée par l'équation
    \ref{eq:expval_t_1}, qui est une fonction cosinus carrée, avec un terme
    additionnel de $2sin(\theta)cos(\theta)$, plus sur lequel nous allons
    revenir plus tard à la sous-section \ref{sec:implications_real}.
    %la courbe
    %théorique de cette partie réelle est une valeur cosinus carrée en 
    %ignorant la partie $2sin(\theta)cos(\theta)$ de l'équation \ref{eq:expval_t_1}. 
    %On suppose que cette partie est un artéfact perviennant dans
    %l'évolution de nos calculs théoriques, car elle ne contribue pas à la
    %partie réelle de la valeur faible.
    %Cette partie de notre résultat théorique pour ce trajet de polarisation
    %n’a pas de signification 
    %physique; car lorsque nous passons de $\ket{V}$ à $\ket{H}$, nous 
    %devrions pas avoir de délai négatif, car nous avons
    %déjà un délai de référence de $0$ pour l’état $\ket{V}$ au cours de la 
    %calibration. Ce dernier, est donc un terme de 
    %correction qui n’est pas nécessaire pour 
    %caractériser la partie réelle de la valeur faible. En effet,
    %nous avons constaté que la partie réelle de la valeur faible
    %est bien représentée par la fonction cosinus carrée,
    %ce qui est conforme à nos attentes théoriques.
    %Une erreur additionnelle à noter est l’incohérence de nos 
    %modifications physiques dans notre expérience, que nous ne pouvons 
    %pas compenser numériquement. Par exemple, les vibrations de la table 
    %et les variations d’alignement du laser sont des facteurs que nous 
    %ne pouvons pas prendre en compte pendant les longues périodes 
    %d’acquisition de données pour chaque état d’entrée. Cette erreur se 
    %trouve dans le trajet à venir. 
    Voici les résultats pour le trajet 
    $\ket{H} \rightarrow \ket{R} \rightarrow \ket{V} \rightarrow \ket{L} \rightarrow \ket{H}$ 
    (figure \ref{fig:HRAL_delay_results}).
\end{doublespace}

\begin{figure}[!h!t!p!b]
    \centering
    \includegraphics[width=1.0\textwidth]{real_weak_value_path_4.png}
    \caption{Résultats expérimentaux pour la partie réelle de la valeur 
    faible du trajet de polarisation 
    $\ket{H} \rightarrow \ket{R} \rightarrow \ket{V} \rightarrow \ket{L} \rightarrow\ket{H}$.
    }
    \label{fig:HRAL_delay_results}
\end{figure}

\begin{doublespace}
    \noindent Ces résultats sont similaires à ceux du trajet précédent, 
    ce qui est logique, car les états de polarisation $\ket{R}$ et 
    $\ket{L}$ possèdent toujours les mêmes états de base que $\ket{D}$ 
    et $\ket{A}$, mais avec une composante imaginaire. Cette dernière 
    n’affecte pas la partie réelle de la valeur faible, car il ne 
    s’agit que d’un décalage de phase, soit de $e^{i\pi/2}$,
    qui ne modifie pas la partie réelle de la valeur faible. 
    En effet, la partie réelle de la valeur faible est toujours représentée
    par la fonction cosinus carrée, ce qui est conforme à nos attentes
    théoriques trouvées dans l'équation \ref{eq:expval_t_2}. Nous
    pouvons noter que les résultats expérimentaux du trajet de polarisation
    $\ket{H} \rightarrow \ket{D} \rightarrow \ket{V} \rightarrow \ket{A} \rightarrow\ket{H}$
    sont légèrement décalés par rapport à l'axe des états de polarisation d'entrée,
    en comparaison avec les résultats du trajet de polarisation
    $\ket{H} \rightarrow \ket{R} \rightarrow \ket{V} \rightarrow \ket{L} \rightarrow\ket{H}$,
    où celui circulaire se conforme mieux à une fonction cosinus carrée (à revenir).
    

    \noindent Ensuite, voici les résultats pour le trajet 
    $\ket{R} \rightarrow \ket{A} \rightarrow \ket{L} \rightarrow \ket{D} \rightarrow \ket{R}$, 
    qui est le même trajet que le précédent, mais en débutant avec l'état
    $\ket{R}$ et en terminant avec l'état $\ket{R}$,
    ce dernier est représenté
    par le trajet de polarisation de superposition démontré dans
    l'équation \ref{eq:state3} et à la figure \ref{fig:path5sphere}.
    Les résultats sont présentés dans la figure \ref{fig:DRAL_delay_results}.
\end{doublespace}

\begin{figure}[!h!t!p!b]
    \centering
    \includegraphics[width=1.0\textwidth]{real_part_5.png}
    \caption{Résultats expérimentaux pour la partie réelle de la valeur 
    faible du trajet de polarisation 
    $\ket{R} \rightarrow \ket{A} \rightarrow \ket{L} \rightarrow \ket{D} \rightarrow \ket{R}$.
    La courbe théorique est une constante de $1/2$.}
    \label{fig:DRAL_delay_results}
\end{figure}

\begin{doublespace}
    \noindent Cela s’avère intéressant, car cela démontre 
        clairement que nous conservons toujours la même composante 
        réelle de la valeur faible. 
        Ainsi, nous constatons que
        nos données expérimentales sont conformes à une relation
        constante d'environ $1/2$ pour la partie réelle de la valeur faible.
        On ne peut cependant pas
        déterminer à partir de la partie réelle de la valeur faible si
        l’état initial est elliptique ou circulaire. Les informations sur
        l’ellipticité des états de polarisation se
        trouvent dans la partie imaginaire de la valeur faible, où un 
        décalage fréquentiel est observé (maximal pour les états
        circulaires, nul pour les états
        linéaires et intermédiaire pour les états elliptiques), ce qui
        permet de distinguer ces états de polarisation. Par conséquent, l’ensemble des données 
        sur l’ellipticité entre ces polarisations est contenu dans la 
        variable conjuguée du décalage temporel, c’est-à-dire le décalage 
        de fréquence obtenu en mesurant la partie imaginaire de la valeur 
        faible. C’est ce que nous allons
        mesurer dans la section \ref{sec:imaginary_results}.
\end{doublespace}

\subsubsection{Caractérisation des amplitudes de probabilité à partir de la partie réelle de la valeur faible}\label{sec:prob_amplitude_results}

\begin{doublespace}        
    
    \noindent En outre, à partir de ces données expérimentales 
    pour chaque trajet, nous 
    pouvons directement calculer les modules d'amplitudes de 
    probabilité de l’état 
    quantique à partir de ces mesures de délai, comme nous l’avons 
    mentionné dans la section \ref{sec:proposition_exp} avec les équations: 
    \ref{eq:amplitude_a} et \ref{eq:amplitude_b}.
    Nous allons maintenant
    présenter les résultats expérimentaux pour 
    les amplitudes de probabilité pour chaque trajet de polarisation
    que nous avons mesuré. Celui du trajet de polarisation
    $\ket{H} \rightarrow \ket{D} \rightarrow \ket{V} \rightarrow \ket{A} \rightarrow \ket{H}$ 
    sera présenté dans la sous-section \ref{sec:implications_real}. Voici les résultats
    pour le trajet de polarisation
    $\ket{H} \rightarrow \ket{R} \rightarrow \ket{V} \rightarrow \ket{L}$ 
    dans la figure \ref{fig:HRVL_prob_amp}.
\end{doublespace}

\begin{figure}[!h!t!p!b]
    \centering
    \includegraphics[width=1.0\textwidth]{real_probability_path_4.png}
    \caption{Résultats de l'amplitude de probabilité pour le trajet de 
    polarisation 
    $\ket{H} \rightarrow \ket{R} \rightarrow \ket{V} \rightarrow \ket{L}$.
    Les données sont calculées à partir de la partie réelle de la valeur faible
    par les équations \ref{eq:amplitude_a} et \ref{eq:amplitude_b}.
    Les courbes théoriques provenant des amplitudes de probabilités
    de l'équation \ref{eq:expval_t_2}.}
    \label{fig:HRVL_prob_amp}
\end{figure}


\begin{doublespace}
    \noindent Les barres d'erreur sont les incertitudes proviennent de l'erreur 
    de la propagation d'erreur à partir de notre 
    partie réelle de la valeur faible de la façon suivante \cite{taylor1997error}:

    \begin{align}
        \sigma_{a} &= \frac{\sigma_{\mathcal{R}(\expval{\hat{\pi}}_W)}}{2|a|}\\
        \sigma_{b} &= \frac{\sigma_{\mathcal{R}(\expval{\hat{\pi}}_W)}}{2|b|},
    \end{align}

    \noindent où $\sigma_{\mathcal{R}(\expval{\hat{\pi}}_W)}$ est l'incertitude de la partie réelle de la valeur faible,
    et $\sigma_{a}$ et $\sigma_{b}$ sont les incertitudes des amplitudes de probabilité
    $|a|$ et $|b|$ respectivement.
    \noindent Ces résultats sont conformes à nos attentes théoriques,
    car les amplitudes de probabilité sont bien représentées par celles
    des équations \ref{eq:amplitude_a} et \ref{eq:amplitude_b},
    ainsi que les attentes expérimentales pour les amplitudes de probabilité
    calculées pour ce trajet de polarisation: \ref{eq:state2} dont 
    la courbe théorique est donnée par
    $\sqrt{|cos^2(\theta)|}$ pour $a$
    et $\sqrt{|sin^2(\theta)|}$ pour $b$.
    Nous avons ensuite mesuré les amplitudes de probabilité pour le trajet 
    $\ket{R} \rightarrow \ket{A} \rightarrow \ket{L} \rightarrow \ket{D} \rightarrow \ket{R}$
    (figure \ref{fig:DRAL_prob_amp}):
\end{doublespace}

\begin{figure}[!h!t!p!b]
    \centering
    \includegraphics[width=1.0\textwidth]{real_probability_path_5.png}
    \caption{Résultats de l'amplitude de probabilité pour le trajet de 
    polarisation 
    $\ket{R} \rightarrow \ket{A} \rightarrow \ket{L} \rightarrow \ket{D} \rightarrow \ket{R}$.
    Les données sont calculées à partir de la partie réelle de la valeur faible
    par les équations \ref{eq:amplitude_a} et \ref{eq:amplitude_b}.
    Les courbes théoriques provenant de les amplitudes de probabilités
    de l'équation \ref{eq:expval_t_3} soit
    $1/\sqrt{2}$ pour $|a|$ et $|b|$.}
    \label{fig:DRAL_prob_amp}
\end{figure}

\begin{doublespace}
    \noindent Ces résultats démontre bien que les amplitudes de probabilité,
    soit $\mathcal{R}(a)$ est approximativement constant a une valeur de $\frac{1}{\sqrt{2}}$, 
    qui est très interessant, car cela démontre une superposition constante 
    des états de bases $\ket{H}$ et $\ket{V}$ dans ce trajet de polarisation
    dont la partie réelle de la valeur faible est toujours
    $\frac{\tau}{2}$, ce qui est conforme à nos attentes expérimentales et théoriques.
\end{doublespace}