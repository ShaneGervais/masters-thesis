\begin{doublespace}
    Nous remarquons que les données expérimentales pour les trajets 
    $\ket{H} \rightarrow \ket{D} \rightarrow \ket{V} \rightarrow \ket{A} \rightarrow \ket{H}$ 
    et $\ket{H} \rightarrow \ket{R} \rightarrow \ket{V} \rightarrow \ket{L} \rightarrow \ket{H}$ 
    suivent une relations expérimentale d'un cosinus carré et que le 
    trajet $\ket{R} \rightarrow \ket{A} \rightarrow \ket{L} \rightarrow \ket{D} \rightarrow \ket{R}$
    suit une relation constante de $\tau/2$ pour la partie réelle de la valeur faible
    dans les figures \ref{fig:HDAV_delay_results}, \ref{fig:HRAL_delay_results} et \ref{fig:DRAL_delay_results} respectivement, 
    comme nous l'avons vu dans la section précédente. Ainsi, les
    amplitudes de probabilité réelles de ces trajets de polarisation
    dans les figures \ref{fig:HDVA_prob_amp}, \ref{fig:HRVL_prob_amp} et \ref{fig:DRAL_prob_amp}
    suivent respectivement les relations dans les équations
    \ref{eq:amplitude_a} et \ref{eq:amplitude_b}. 
    
    \noindent Cependant, 
    nous avons remarqué que les données expérimentales pour les trajets
    dans les figures \ref{fig:HDAV_delay_results} et \ref{fig:HRAL_delay_results}
    sont légèrement décalées par rapport à l'axe des états de polarisation
    d'entrée. Nous consacrons que ce décalage est dû au terme additionnel
    mentionné précédemment, notamment le terme le terme 
    $2sin(\theta)cos(\theta)$ qui est présent dans l'équation \ref{eq:expval_t_1}.
    Ce terme additionnel apporte des résultats de la partie réelle de la valeur faible
    négative pour les états d'entrée à ce trajet de polarisation,
    ce qui est conforme à nos attentes théoriques.
    Cependant, nous avons remarqué que les données expérimentales pour ce trajet
    ne comportent pas de données négatives mais bien un décalage au niveau de 
    la position selon l'axe des états de polarisation d'entrée.
    Revenant sur l'équation \ref{eq:t_norm_interférence}, ce terme 
    additionnel provient de l'interférence entre le pointeur décalé et 
    non décalé, lors de la mesure de la partie réelle de la valeur faible.
    Or, nous savons que le laser impulsionnel possède une faible 
    longueur de cohérence (voir figure \ref{fig:vis}). 
    Nous démontrons
    l'effet de ce terme en fonction de la visibilité avec l'équation:

    \begin{equation}
        \expval{\hat{t}}_{\text{vis}} = \frac{\tau}{2}(cos^2(\theta) + 2\mathcal{V} sin(\theta)cos(\theta))\label{eq:vis_eq}
    \end{equation}
    
    \noindent où $\mathcal{V}$ est la visibilité de ce terme d'interférence.
    Ainsi, lorsque la visibilité est faible, le terme d'interférence
    est faible et donc négligé. Nous démontrons
    des différentes valeurs de visibilité et comment le terme d'interférence
    affecte la partie réelle de la valeur faible dans la figure \ref{fig:vis_real}.

    \begin{figure}[!htpb]
        \centering
        \includegraphics[width=0.8\textwidth]{real_part_visibility.png}
        \caption{Effet de la visibilité de l'interférence sur la partie réelle de la valeur faible.}
        \label{fig:vis_real}
    \end{figure}

    \noindent Ainsi, ce 
    terme génère du bruit et provoque le décalage observé entre les équations
    \ref{eq:expval_t_1} et \ref{eq:expval_t_2} et dans les figures
    \ref{fig:HDAV_delay_results} et \ref{fig:HRAL_delay_results} 
    respectivement. Nous supposons le même phénomène pour le trajet
    $\ket{R} \rightarrow \ket{A} \rightarrow \ket{L} \rightarrow \ket{D} \rightarrow \ket{R}$ 
    dans la figure \ref{fig:DRAL_delay_results}. 
    
    \noindent Nous allons maintenant examiner les valeurs des visibilités
    nécessaires pour les trajets de polarisation linéaire avec un
    ajustement paramétrique avec l'équation \ref{eq:vis_eq} dans la
    figure \ref{fig:vis_real_path_3}.

    \begin{figure}[!htpb]
        \centering
        \includegraphics[width=0.8\textwidth]{real_weak_value_vis_3_analysis.png}
        \caption{Trajet de polarisation $\ket{H} \to \ket{D} \to \ket{V} \to \ket{A} \to \ket{H}$
        avec ajustement paramétrique pour le terme d'interférence.}
        \label{fig:vis_real_path_3}
    \end{figure}

    \noindent Nous calculons que la visibilité du terme additionelle $2sin(\theta)cos(\theta)$ 
    est de $\mathcal{V} = 0.11$ ce qui démontre que ce terme peut 
    déplacer la partie réelle de la valeur faible. Ce décalage est d'environ $11$ degrés du centre.
    Avec cette visibilité, nous pouvons caractériser les amplitudes de probabilité
    de ce trajet de polarisation. Voici les résultats de l'amplitude de probabilité
    pour le trajet de polarisation
    $\ket{H} \rightarrow \ket{D} \rightarrow \ket{V} \rightarrow \ket{A} \rightarrow \ket{H}$ 
    dans la figure \ref{fig:HDVA_prob_amp}.

    
    \begin{figure}[!h!t!p!b]
    \centering
    \includegraphics[width=1.0\textwidth]{real_probability_path_3.png}
    \caption{Résultats de l'amplitude de probabilité pour le trajet 
    de polarisation 
    $\ket{H} \rightarrow \ket{D} \rightarrow \ket{V} \rightarrow \ket{A} \rightarrow \ket{H}$. 
    Les données sont calculées à partir de la partie réelle de la valeur faible
    par les équations \ref{eq:amplitude_a} et \ref{eq:amplitude_b}
    .}
    \label{fig:HDVA_prob_amp}
    \end{figure}

    
    \noindent Les amplitudes de probabilité sont bien représentées par celles
    des équations \ref{eq:amplitude_a} et \ref{eq:amplitude_b} ainsi que 
    les attentes expérimentales pour les amplitudes de probabilité
    calculées pour ce trajet de polarisation: \ref{eq:state1}.
    Pour le trajet de polarisation en superposition, nous abservons une visibilité
    plus faible que le trajet de polarisation linéaire $\mathcal{V} \approx 0$ 
    car les résultats sont approximativement constant à $1/2$,
    qui pose une difficulté à mesurer. Or, nous sommes en mesure de caractériser les états de 
    polarisation d'entrée avec succès. Il y a cependant 
    l'information de la partie complexe de l'état quantique que nous devons mesurer. 
    La section suivante traitera de nos tentatives de mesure de la 
    partie imaginaire. 

    %En effet, le 
    %chemin 1 suit moins bien la théorie que le chemin 2 car le chemin 1 
    %est celui qui possède ce terme d'interférence dont la visibilité est 
    %faible mais non nulle.
    %Ces résultats correspondent à nos attentes théoriques 
    %concernant la norme des amplitudes de probabilité réelles de chaque 
    %trajet de polarisation. Cette concordance nous permet de conclure 
    %que nous avons correctement mesuré la partie réelle de la valeur 
    %faible et nous l'avons utilisée pour caractériser les amplitudes de 
    %probabilité de nos états initiaux. 
    %Or, nous sommes en mesure de caractériser les états de 
    %polarisation d'entrée avec succès. Il y a cependant 
    %l'information de la partie complexe de l'état quantique que nous devons mesurer. 
    %La section suivante traitera de nos tentatives de mesure de la 
    %partie imaginaire. 
\end{doublespace}