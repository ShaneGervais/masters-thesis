\begin{doublespace}
    Nous avons calculé les amplitudes de probabilité en appliquant les 
    principes fondamentaux étudiés dans notre chapitre consacré à la 
    théorie. Ces résultats correspondent à nos attentes théoriques 
    concernant la norme des amplitudes de probabilité réelles de chaque 
    trajet de polarisation. Cette concordance nous permet de conclure 
    que nous avons correctement mesuré la partie réelle de la valeur 
    faible et nous l'avons utilisée pour caractériser les amplitudes de 
    probabilité de nos états initiaux. En conclusion, nous sommes en mesure de caractériser les états de 
    polarisation d'entrée avec succès. Il y a cependant 
    l'information de la partie complexe de l'état quantique que nous devons mesurer. 
    La section suivante traitera de nos tentatives de mesure de la 
    partie imaginaire. 
\end{doublespace}