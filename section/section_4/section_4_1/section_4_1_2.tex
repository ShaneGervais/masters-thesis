\begin{doublespace}
    Nous remarquons que les données expérimentales pour les trajets 
    $\ket{H} \rightarrow \ket{D} \rightarrow \ket{V} \rightarrow \ket{A} \rightarrow \ket{H}$ 
    et $\ket{H} \rightarrow \ket{R} \rightarrow \ket{V} \rightarrow \ket{L} \rightarrow \ket{H}$ 
    suivent une relations expérimentale d'un cosinus carré et que le 
    trajet $\ket{D} \rightarrow \ket{R} \rightarrow \ket{A} \rightarrow \ket{L} \rightarrow \ket{D}$
    suit une relation constante de $\tau/2$ pour la partie réelle de la valeur faible
    dans les figures \ref{fig:HDAV_delay_results}, \ref{fig:HRAL_delay_results} et \ref{fig:DRAL_delay_results} respectivement, 
    comme nous l'avons vu dans la section précédente. Ainsi, que les 
    amplitudes de probabilité réelles de ces trajets de polarisation
    dans les figures \ref{fig:HDVA_prob_amp}, \ref{fig:HRVL_prob_amp} et \ref{fig:DRAL_prob_amp}
    suivent respectivement les relations dans les équations
    \ref{eq:amplitude_a} et \ref{eq:amplitude_b}. Cependant, 
    nous avons remarqué que les données expérimentales pour les trajets
    dans les figures \ref{fig:HDAV_delay_results} et \ref{fig:HRAL_delay_results}
    sont légèrement décalées par rapport à l'axe des états de polarisation
    d'entrée. Nous concacrons que ce décalage est dû au terme additionnel
    mentionné précédemment, notamment le terme le terme 
    $2sin(\theta)cos(\theta)$ qui est présent dans l'équation \ref{eq:expval_t_1}.
    Ce terme additionnel apporte des résultats de la partie réelle de la valeur faible
    négative pour les états d'entrée à ce trajet de polarisation,
    ce qui est conforme à nos attentes théoriques.
    Cependant, nous avons remarqué que les données expérimentales pour ce trajet
    ne comportent pas de données négatives mais bien un décalage au niveau de 
    la position selon l'axe des états de polarisation d'entrée.
    Revenant sur l'équation \ref{eq:t_norm_interférence}, ce terme 
    additionnel provient de l'interférence entre le pointeur décalé et 
    non décalé, lors de la mesure de la partie réelle de la valeur faible.
    Or, nous savons que le laser impulsionnel possède une faible 
    longueur de cohérence (voir figure \ref{fig:vis}). 
    Par conséquent, ce terme 
    disparait car la visibilité de l'interférence est faible. Ainsi, ce 
    terme génère du bruit et provoque le décalage observée entre les équations
    \ref{eq:expval_t_1} et \ref{eq:expval_t_2} et dans les figures
    \ref{fig:HDAV_delay_results} et \ref{fig:HRAL_delay_results} 
    respectivement. Nous supposons le même phénomène pour le trajet
    $\ket{D} \rightarrow \ket{R} \rightarrow \ket{A} \rightarrow \ket{L} \rightarrow \ket{D}$ 
    dans la figure \ref{fig:DRAL_delay_results}.
    %En effet, le 
    %chemin 1 suit moins bien la théorie que le chemin 2 car le chemin 1 
    %est celui qui possède ce terme d'interférence dont la visibilité est 
    %faible mais non nulle.
    %Ces résultats correspondent à nos attentes théoriques 
    %concernant la norme des amplitudes de probabilité réelles de chaque 
    %trajet de polarisation. Cette concordance nous permet de conclure 
    %que nous avons correctement mesuré la partie réelle de la valeur 
    %faible et nous l'avons utilisée pour caractériser les amplitudes de 
    %probabilité de nos états initiaux. 
    Or, nous sommes en mesure de caractériser les états de 
    polarisation d'entrée avec succès. Il y a cependant 
    l'information de la partie complexe de l'état quantique que nous devons mesurer. 
    La section suivante traitera de nos tentatives de mesure de la 
    partie imaginaire. 
\end{doublespace}