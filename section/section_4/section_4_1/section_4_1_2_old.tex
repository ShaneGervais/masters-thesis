Certains voudrons vérifier théoriquement nos 
résultats expérimentaux sur les états de 
polarisation faiblement mesurés de chaque 
chemin de manière classique par les moyens 
suivants. Pour le chemin $\ket{H} \to \ket{D} \to \ket{V} \to \ket{A}$, nous n'avons 
qu'une seule demi-plaque d'onde appliquée sur 
un état de polarisation $\boldsymbol{H} = \begin{pmatrix}
    1\\0
\end{pmatrix}$ entrant. Une lame demi-onde applique 
l'opération suivant:

\begin{align}
    HWP(\theta) = \begin{pmatrix}
        cos(\theta) & sin(\theta)\\
        sin(\theta) & -cos(\theta)
    \end{pmatrix}
\end{align}

Appliquant sela sur l'état d'entrée pour l'étape de
préparation de l'état avant de subir la mesure faible.

\begin{align}
    HWP(\theta)*\boldsymbol{H} &= \begin{pmatrix}
        cos(\theta) & sin(\theta)\\
        sin(\theta) & -cos(\theta)
    \end{pmatrix} *\begin{pmatrix}
        1\\0
    \end{pmatrix}\\
    &= \begin{pmatrix}
        cos(\theta)\\sin(\theta)
    \end{pmatrix}
\end{align}

Cet-à-dire, l'état quantique $\ket{\psi}$ devrait avoir les
amplitudes de probabilité suivantes $a=cos(\theta)$ et $b=sin(\theta)$.
Ceci nous donnes une vérification classique pour cet effet quantique
et ce n'est pas un résultat pertinant en comparaison que se qu'on obtient pour la
partie réelle de la valeur faible car ils se correspondent bien.
Nous pouvons performer la même opérations pour les deux autres trajets. 
Soit $\ket{H}\to\ket{R}\to\ket{V}\to\ket{L}$ et 
$\ket{D}\to\ket{R}\to\ket{A}\to\ket{L}$, qu'il possède une
lame demi-onde et une lame quart d'onde fixé à $0$ ou $45$ dégrées relativement à l'état $\boldsymbol{H}$ d'entré.
Une lame quart d'onde à $0$ ou $45$ dégrée fait l'opération suivant:

\begin{align}
    QWP(0^{\degree}) &= \begin{pmatrix}
        1 & 0 \\
        0 & i
    \end{pmatrix}\\
    QWP(45^{\degree}) &= \frac{1}{\sqrt{2}}\begin{pmatrix}
        1 & i \\
        i & 1
    \end{pmatrix}
\end{align}

Donc pour le trajet $\ket{H}\to\ket{R}\to\ket{V}\to\ket{L}$ on recoit:

\begin{align}
    QWP(0)*HWP(\theta)*\boldsymbol{H} &= 
    \begin{pmatrix} 
        1 & 0 \\
        0 & i
    \end{pmatrix} 
    *\begin{pmatrix}
        cos(\theta) & sin(\theta)\\
        sin(\theta) & -cos(\theta)
    \end{pmatrix} *\begin{pmatrix}
        1\\0
    \end{pmatrix}\\
    &= \begin{pmatrix}
        cos(\theta)\\isin(\theta)
    \end{pmatrix}
\end{align}

Cela signifie que les amplitudes de probabilité sont $a=cos(\theta)$ et $b=isin(\theta)$.
En comparaison de ce qu'on obtient pour la partie réelle de la valeur faible, nous obtenons 
la même résultat que le trajet procédant. Ce dernier est grace à
la symmétrie de la sphère Poincaré dont quand on parcours un trajet entre
des états symmétriquement orthogonal ou circulaire vont toujours avoir la même partie réel
$\mathcal{R}(\bra{R}\ket{H}) = \mathcal{R}(\bra{D}\ket{H}) = \frac{1}{\sqrt{2}}$
et quand on transition à ces états, nous avons toujours 
la même superposition entre les états de bases.
L'information qui contient la différence les deux trajets est trouvé dans partie imaginaire
qui suppose qu'il exist un shift fréquenciel causé par l'effet d'un intéraction faible entre les deux bases
comme nous avons calculer dans le chapitre 2.
Ensuite pour le trajet 
$\ket{D}\to\ket{R}\to\ket{A}\to\ket{L}$ on 
recoit:

\begin{align}
    QWP(45)*HWP(\theta)*\boldsymbol{H} &= 
    \begin{pmatrix} 
        1 & i \\
        i & 1
    \end{pmatrix} 
    *\begin{pmatrix}
        cos(\theta) & sin(\theta)\\
        sin(\theta) & -cos(\theta)
    \end{pmatrix} *\begin{pmatrix}
        1\\0
    \end{pmatrix}\\
    &= \frac{1}{\sqrt{2}}\begin{pmatrix}
        cos(\theta) + isin(\theta)\\icos(\theta)+sin(\theta)
    \end{pmatrix}
\end{align}

Parreill comme le trajet dernier, ceci aura toujours
une superposition entre les états de bases grace à
la symmétrie de la sphère Poincaré. Ainsi que 
les amplitudes de probabilité ne son pas complètement pareil.
La raison car ceci devient un système quantique
en comparaison d'un système classique c'est causé
par l'intéraction faible grace à un pointer qui
réduit l'état minimalement et 
crée une valeur moyenne faible proportionnelle
à la fonction d'onde qui décrit l'information
sur l'état de polarisation du système. Cependant
une vérification classique n'est pas complètement
inutile car ça nous permet de vérifier si notre expérience
suit la même relation symmétrique que l'optique classique.
Cet-à-dire, que la mesure faible en comparaison des mesures classiques
ou bien l'interferométrie standard, ajoute des détails
sur les phases intermédiaire entre les transitions des
états quantiques. Ainsi, ce calcul nous permet de vérifier
l'état d'entrée pour chaque état préparé selon les trajets.

