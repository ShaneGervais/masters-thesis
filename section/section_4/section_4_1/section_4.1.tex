Comme nous l’avons vu dans les chapitres précédents, nous 
voulons que notre impulsion soit semblable à une impulsion 
de Gauss afin de faciliter la détermination de sa 
position temporelle moyenne. Il est également bien connu 
qu’on utilise les profils d’impulsion de Gauss pour les 
mesures de faiblesse. Par conséquent, nous avons procédé 
au même ajustement des données que celui effectué dans le 
chapitre précédent. Cependant, pour obtenir des délais, 
il faut d'abord calibrer l'expérience. Pour ce faire, 
nous mesurons la position temporelle moyenne sans 
introduction de retard, soit le bras en $\ket{V}$ de l’appareil 
expérimental. Nous mesurons également la position 
temporelle moyenne du retard maximal, soit le bras $\ket{H}$. Une 
fois que l’expérience a été calibrée en fonction du 
délai maximum et minimum, nous pouvons mesurer chaque 
degré d’état d’entrée et trouver le retard qui lui 
correspond selon les positions temporelles calibrées. 
Ensuite, comme nous l’avons vu au chapitre 3, notre 
méthode expérimentale consiste à préparer les états de 
polarisation selon les trajectoires de polarisation 
mesurées, qui subiront ensuite une interaction faible, 
puis seront sélectionnées par un état de polarisation 
diagonale. La section suivante discutera des résultats 
obtenus à partir de nos données concernant la partie 
réelle de la valeur faible et la façon dont nous 
recueillons nos données est expliquée plus en détail.