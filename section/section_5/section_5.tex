\begin{doublespace}

    Enfin, nous allons conclure cette thèse en résumant ces
    résultats expérimentaux obtenus et en discutant de l'importance 
    sur les applications potentielles des mesures faibles temporelles
    et des perspectives futures pour la mesure de la partie
    imaginaire de la valeur faible.

    \subsection{Conclusion sur la thèse}
    
    Nous avons maintenant démontré que nous pouvons 
    caractériser complètement la partie réelle de la valeur faible, 
    tout en montrant l’existence et la mesurabilité de la partie 
    imaginaire. De plus, le résultat obtenu pour la partie réelle de la 
    valeur faible montre que la symétrie quantique pour l’état de 
    polarisation, lorsque les états d’entrée varient, est respectée et 
    suit notre théorie dérivée. En utilisant le domaine temporel comme 
    pointeur nous permet de caractériser la polarisation d'un système 
    photonique quantique, ouvrant ainsi la voie à des applications 
    intégrables dans les technologies quantiques plutôt qu'utiliser 
    d'autres domaines du photon comme pointeur \cite{Hairiri, Guilleaum}. Grâce à notre 
    méthodologie interférométrique des mesures faibles, il est possible 
    d’envisager des applications dans des technologies telles que les 
    systèmes de télécommunication photonique quantique 
    \cite{OpticalNetworks,Brunner_2004}, l’informatique quantique 
    \cite{quantuminfoweak}, la cryptographie quantique 
    \cite{troupe2017quantumcryptographyweakmeasurements}, la métrologie 
    quantique et l'avancement sur nos mesures quantiques précises
    \cite{intlundeen}. 
    
    \subsection{Applications et projets futurs}

    Cette approche pour la caractérisation des états
    quantiques ouvre la voie à de nombreuses applications dans les
    technologies quantiques. En intégrant notre dispositif experimental
    dans les systèmes de télécommunication photonique quantique, il
    devient possible de caractériser les états de polarisation
    des photons de manière efficace et précise. Ce dernier
    diminue l'effondrement complet du système lors de la mesure du
    message, permettant ainsi de préserver l'état quantique des photons
    tout en garantissant la confidentialité des informations
    transmises. De plus,
    en utilisant cette méthode pour caractériser les états quantiques
    dans les ordinateurs quantiques, il devient possible de détecter et
    de corriger les erreurs de manière plus efficace en suivant
    l'évolution temporelle des états quantiques, ce qui est crucial
    pour le développement de systèmes quantiques robustes et fiables.
    En outre, en intégrant cette approche dans les systèmes de
    métrologie quantique, il devient possible de mesurer des
    grandeurs physiques avec une précision sans précédent, ouvrant la
    voie à de nouvelles découvertes.

    \noindent Nos futurs travaux consisteront à déterminer la 
    polarisation non linéaire à l’aide des décalages fréquentiels 
    provoqués par notre décalage temporel du pointeur. Pour ce faire, 
    nous pourrions concevoir un système photonique sophistiqué capable 
    de mesurer de petits décalages fréquentiels. Nous pouvons envisager 
    une expérience où nous remplacerions la source laser par quelque 
    chose qui a une longueur de cohérence plus longue et qui nous 
    permettrait d’introduire un délai encore plus grand, mais dans le 
    régime de mesure faible, comme celui du laser HeNe, sous forme 
    d’impulsions. Nous pourrions effectuer cette expérience en utilisant 
    un modulateur acousto-optique couplé à un générateur d'impulsions. 
    Ensemble, ces dispositifs créent un deuxième signal sous forme 
    d'impulsion du laser HeNe décalé d'une fréquence connue. Cela nous 
    permettrait de mesurer la variation des décalages fréquentiels du 
    signal décalé en fonction des états de polarisation d'entrée et, 
    ainsi, de mesurer avec succès la partie imaginaire de la valeur 
    faible dans un système photonique quantique, comme dans: 
    \cite{imaginary_part}. 

    \noindent En conclusion, nous avons démontré que les mesures
    faibles temporelles peuvent être utilisées pour la caractérisation
    des états quantiques et que cette approche est intégrable dans les
    technologies quantiques \cite{OpticalNetworks,Brunner_2004,quantuminfoweak,troupe2017quantumcryptographyweakmeasurements,intlundeen}. 
    Nous pouvons enfin conclure cette thèse et 
    laissent la porte ouverte à des
    futures perspectives en matière de mesures quantiques pour des
    applications technologiques.

\end{doublespace}