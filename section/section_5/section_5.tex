\begin{doublespace}

    Nous concluons cette thèse avec des mots finaux sur
    notre projet de recherche et des travaux futurs que nous pouvons
    envisager pour la mesure de la partie imaginaire de la valeur faible.

    \subsection{Conclusion sur la thèse}
    L'objectif de cette thèse était d'explorer différentes
    degrés de liberté afin de rendre les mesures faibles
    intégrables dans les technologies quantiques. Nous avons
    choisi le domaine temporel comme pointeur pour
    caractériser les états de polarisation d'un système photonique
    quantique à l'aide de méthodes interférométriques.
    Nous avons démontré qu'il est possible de caractériser
    la partie réelle de la valeur faible, tout en montrant
    la faisabilité de la mesure de sa partie imaginaire. Cependant,
    des défis subsistent, notamment la stabilité du laser ne sont pas encore optimales
    pour mesurer les effets de la partie réelle et imaginaire de la valeur faible.
    En ce qui concerne la partie réelle, nous avons réussi à observer des
    déplacements temporels, mais avec une visibilité limitée. Un terme
    d'interférence est cencé de produire des délais négatifs, mais cela
    n'a pas été possible de les mesurer avec la source utilisée. En revanche pour 
    la partie imaginaire, nous avons
    observé des déplacements fréquentiels, bien que ceux-ci soient
    également limités par la stabilité du laser. Néanmoins, utiliser le
    domaine temporel comme pointeur permet de caractériser la
    polarisation d’un système photonique quantique, ouvrant ainsi la
    voie à des applications intégrables dans les technologies
    quantiques. Grâce à notre méthodologie interférométrique des
    mesures faibles, il est possible d’envisager des applications dans
    des domaines tels que les systèmes de télécommunication photonique
    quantique \cite{OpticalNetworks,Brunner_2004}, l’informatique
    quantique \cite{quantuminfoweak}, la cryptographie quantique
    \cite{troupe2017quantumcryptographyweakmeasurements}, la métrologie
    quantique et l’amélioration de la précision des mesures quantiques
    \cite{intlundeen}.
    
    \subsection{Applications et projets futurs}

    Cette approche pour la caractérisation des états
    quantiques ouvre la voie à de nombreuses applications dans les
    technologies quantiques. En intégrant notre dispositif expérimental
    dans les systèmes de télécommunication photonique quantique, il
    devient possible de caractériser les états de polarisation
    des photons de manière efficace et précise. 
    La mesure faible permet une interaction minimale avec le
    système quantique, et lorsqu’on postsélectionne avec un état connu,
    il est possible de caractériser directement l’état quantique du
    message transmis en observant les déplacements temporels et
    fréquentiels des photons à l’aide de notre pointeur couplé
    \cite{troupe2017quantumcryptographyweakmeasurements}. De plus,
    cette méthode appliquée à la caractérisation des états quantiques
    dans les ordinateurs quantiques permettrait de détecter et
    corriger les erreurs de manière plus efficace en suivant
    l’évolution temporelle des états, ce qui est crucial
    pour le développement de systèmes quantiques robustes et fiables.
    Par ailleurs, en intégrant cette approche dans les systèmes de
    métrologie quantique, il devient possible de mesurer des
    grandeurs physiques avec une précision sans précédent
    \cite{metrology,QuantumLimitedMetrology, RydbergDispersion}, ouvrant la
    voie à de nouvelles découvertes.

    \noindent Les travaux futurs consisteront à caractériser des
    polarisations elliptiques à l’aide des décalages fréquentiels
    provoqués par le décalage temporel du pointeur. Pour ce faire,
    nous pourrions concevoir un système photonique sophistiqué capable 
    de mesurer de petits décalages fréquentiels. Nous envisageons 
    une expérience où la source laser pulsée serait remplacée par
    une source à plus grande longueur de cohérence, facilitant ainsi
    la mesure de la partie imaginaire de la valeur faible. Un laser
    HeNe et un modulateur acousto-optique pourraient être utilisés pour
    générer des impulsions de lumière cohérente. Comme les impulsions
    proviendraient d’un laser à grande longueur de cohérence, le spectre
    d’interférence serait plus net et permettrait de mesurer
    la variation des décalages fréquentiels du signal en fonction
    des états de polarisation d’entrée, permettant ainsi de mesurer
    avec succès la partie imaginaire de la valeur faible
    dans un système photonique quantique.

    \noindent En conclusion, nous avons (i) établi une 
    méthode d’extraction de délais temporels en sélectionnant
    un modèle d’ajustement paramétrique indépendant du montage
    et vérifié que ce modèle s’applique aux impulsions lumineuses en déterminant la 
    vitesse de la lumière, (ii) réalisé un dispositif
    expérimental incorporant des mesures faibles temporelles
    pour caractériser la partie réelle de la valeur faible
    dans un système photonique quantique, (iii) démontré que
    l’on peut caractériser les amplitudes de probabilité
    des états de polarisation d’entrée à partir de la partie réelle
    de la valeur faible, et (iv) proposé une méthode pour mesurer
    la partie imaginaire de la valeur faible en utilisant
    des interférences dans un interféromètre de Mach-Zehnder.
    Cette thèse laisse enfin la porte ouverte à de
    futures perspectives en matière de mesures quantiques pour des
    applications technologiques.


\end{doublespace}