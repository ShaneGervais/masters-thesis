\begin{doublespace}
    \subsection{Conclusion sur la thèse}
    
    \noindent Nous avons maintenant démontré que nous pouvons caractériser complètement la partie réelle de la valeur faible, tout en montrant l’existence et la mesurabilité de la partie imaginaire. De plus, le résultat obtenu pour la partie réelle de la valeur faible montre que la symétrie quantique pour l’état de polarisation, lorsque les états d’entrée varient, est respectée et suit notre théorie dérivée. En utilisant le domaine temporel comme pointeur nous permet de caractériser la polarisation d'un système photonique quantique, ouvrant ainsi la voie à des applications intégrables dans les technologies quantiques plutôt qu'utilisant d'autres domaines du photon comme pointeur. Grâce à notre méthodologie interférométrique des mesures faibles, il est possible d’envisager des applications dans des technologies telles que les systèmes de télécommunication photonique quantique, mais aussi dans d’autres domaines, comme l’informatique quantique ou la métrologie quantique. 
    
    \subsection{Applications et projets futurs}
    \noindent Nos futurs travaux consisteront à déterminer la polarisation non linéaire à l’aide des décalages fréquentiels provoqués par notre décalage temporel du pointeur. Pour ce faire, nous pourrions concevoir un système photonique sophistiqué capable de mesurer de petits décalages fréquentiels. Nous pouvons envisager une expérience où nous remplacerions la source laser par quelque chose qui a une longueur de cohérence plus longue et qui nous permettrait d’introduire un délai encore plus grand, mais dans le régime de mesure faible, comme celui du laser HeNe, mais sous forme d’impulsions. Nous pourrions effectuer cette expérience en utilisant un modulateur acousto-optique couplé à un générateur d'impulsions. Ce dispositif crée une impulsion laser HeNe. Cela nous permettrait de mesurer les décalages fréquentiels entre les états de polarisation d'entrée et, ainsi, de mesurer avec succès la partie imaginaire de la valeur faible dans un système photonique quantique. 
    
    \noindent Nous pouvons conclure avec succès nos résultats expérimentaux au cours de ce projet, et laisser la porte ouverte à de futures perspectives en matière de mesures quantiques pour des applications technologiques. 
\end{doublespace}