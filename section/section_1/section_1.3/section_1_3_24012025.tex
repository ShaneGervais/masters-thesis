Les premiers travaux que nous pouvons trouver sur les 
mesures faibles basées sur le temps et les applications 
sont ceux de Nicolas Brunner et al. dans «Optical 
Telecom Networks as Weak Quantum Measurements with 
Postselection» \cite{OpticalNetworks}. Dans cet article, ils combinent la 
mécanique quantique et la télécommunication dont 
l'état de polarisation de la lumière dans une fibre 
optique est utilisé comme l'état quantique qui subit 
une interaction faible (mesure faible) via la dispersion 
des modes de polarisation (PMD). La PMD est un effet 
d'une fibre optique qui utilise sa biréfringence pour 
introduire un retard optique entre chaque mode de 
polarisation de l'impulsion laser qui traverse la 
fibre. L'autre effet utilisé est la perte dépendante 
de la polarisation (PDL) qui est un effet qui projette 
certains états de polarisation et en laisse passer 
d'autres à travers la fibre optique, pareil comme un 
polariseur. Cependant, il ne s'agit que des travaux 
théoriques qui peuvent d'être réalisé. Sur le plan 
expérimental, ils ont poursuivi leurs travaux avec des 
mesures faibles temporelles en mesurant directement les 
vitesses de groupe supraluminiques dans une fibre 
optique, dans un article intéressant montrant des 
vitesses de groupe pouvant dépasser la vitesse de la 
lumière dans la fibre \cite{Brunner_2004}. D'autres travaux ont été réalisés 
pour mesurer de petits déphasages longitudinaux à 
l'aide de mesures temporelles et ont même été comparés 
à l'interférométrie standard \cite{WeakorStd}. Ce dernier a démontré que 
dans le cas d'états de polarisation non linéaire ou 
de valeurs faibles complexes, les mesures faibles 
peuvent être plus performantes que l'interférométrie 
standard pour mesurer les déphasages longitudinaux \cite{WeakorStd}. 
C'est également ce que montrent les travaux de 
Magana-Loaiza et al, mais dans le domaine des mesures 
faibles spatiales de Jeff Lundeen pour des procédures 
directes \cite{Magaña-Loaiza_2017,Jeff_outperform}. Pour le conjugué complémentaire du domaine 
temporel, la fréquence, Salazar-Serano et al. mesurent 
les délais temporels de sous-largeur d'impulsion 
induits par des décalages fréquentiels, en utilisant 
l'amplification des valeurs faibles \cite{Salazar}. L'amplification 
des valeurs faibles, qui dépassent le cadre de cette 
thèse, est un autre aspect découlant de l'article 
d'AAV selon après l'étape de post-sélection dans la 
mesure directe, le décalage du pointeur induit par la 
valeur faible dépasse largement la plage des valeurs 
propres, d'où la valeur de 100 mesurée pour une 
particule de spin 1/2 peut 
arriver \cite{Aharonov,JordanBook}. En utilisant un 
miroir mobile, ils ont pu introduire de petits décalages 
de fréquence sur les états de bases de l'état quantique 
et mesurer des délais temporels de sous-largeur 
d'impulsion à l'aide de ces décalages fréquentiels \cite{Salazar}. 
Cependant, pour une résolution plus élevée, des travaux 
récents de John C. Howell montrent que, sous le 
régime actuel, un faible décalage de la valeur du 
pointeur dans une base conjuguée a un comportement 
anormal \cite{Howell2022}. 
En termes d'applications, nous avons constaté 
une certaine utilisation dans les réseaux de 
télécommunication et nous en discuterons davantage à 
la fin de cette thèse, mais le problème qui se pose sur 
les mesures faibles positionnelles ou temporelles dans 
les procédures directes, c'est qu'elles utilisent la 
biréfringence des cristaux pour introduire une 
interaction faible sur le système qui ne peut être 
modifié qu'en remplaçant physiquement le cristal dans 
l'installation. Cela serait fastidieux pour une 
application industrielle, alors que peut-on faire pour 
facilement contrôler la faiblesse de l'interaction? 
Ceci nous amène à notre proposition pour cette thèse. 
Nous utiliserons des états de polarisation comme états 
quantiques, comme dans les articles précédents pour la 
simplicité. Nous utiliserons l'interaction faible 
temporelle dans une procédure directe pour mesurer à la 
fois la partie réelle et imaginaire de la valeur faible 
de notre système. Notre interaction faible sur le 
système utilisera des délais temporels via un type 
d'interféromètre où nous pouvons simplement ajuster la 
position des miroirs pour introduire plus ou moins de 
délais entre les états de base de l'impulsion laser. 