Les mesures quantiques sont généralement décrites par 
l'expérience de Stern-Gerlach. Une expérience que toute 
personne ayant suivi un cours de mécanique quantique de base 
connaît bien. Cette expérience est connue pour l'avoir 
découvert l'effet du spin d'une particule quantique. En 1987, 
Aharonov et al. ont montré que le résultat de la mesure du 
spin d'une particule dans un Stern-Gerlach peut être évalué à 
100 \cite{Aharonov}. Pour beaucoup, cela ne devrait pas avoir de sens puisque 
nous savons que le spin d'une particule peut être 
soit $\pm 1/2$. L'article décrit comment la fonction d'onde 
est proportionnelle à ce que l'on appelle la valeur faible. 
La section suivante servira à décrire la valeur faible ainsi 
que les schémas de mesure de cette valeur. 