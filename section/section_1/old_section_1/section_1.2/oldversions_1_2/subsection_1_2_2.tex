C'est que Aharonov, Albert et Vaidman (AAV) se questionaient sur le sujets
quand ils démontrent que le resultat d'une partciule d'un spin de $1/2$
s'évalue à $100$. Dans ce papier ils discutent et démontre   initalement la valeur faible.
C'est démontré que la valeur faible est proportionelle à l'état quantique et 
que c'est une variable complexe. Dans la thèse doctorale de Jeff Lundeen, il généralise les
propriété de la valeur faible ainsi qu'est reliée au opérateur d'annhilation et création.
Jeff et Charles Bamber ont aussi écrit un article sur les procédures pour une mesure quantique.
La mesure faible se décrit comme une intéraction entre le système et le pointeur
se qui est $\ll$ faible $\gg$. Ce qu'on veux dire par cela, c'est qu'on réduit (collapse)
l'état quantique assez faiblement (vraiment peu) que nous avons approxativement le même état 
après la mesure et la projection qu'on avait durant la préparation. Ce dernier, ce sert plus
comme une méthode $\ll$ directe $\gg$ comparativement a ce qu'on appelle la mesure standard une mesure forte.
Comme analogie considéré que vous êtes dans votre appartement et ils y a enfants qui fais du bruit dans le colloir
et t'aimerais s'avoir qu'est ce qu'ils font. Mais à chaque fois qu'on ouvre la porte,
les enfants arrête de faire se qu'ils font (on collapse le système). Une mesure faible est comme
un judas sur la porte qui nous permet de vérifé le système sans complètement le réduire.
Dans cette sous-section, nous allons démontrer la même procédure d'une mesure quantique
mais avec l'application de la notions des mesures faibles. Commmencons avec l'état inital:

\begin{equation}
    \ket{\Psi_i} = \ket{\psi} \otimes \ket{\zeta(x)}
\end{equation}

Nous allons ensuite faire une mesure sur le système dénoté par 
$\hat{\pi_W}$ ($W$ pour $\ll$ weak $\gg$, anglais pour $\ll$ faible $\gg$).
Appliquons l'opérateur d'interaction sur la base $\ket{0}$ et projecter 
avec $\ket{\phi}$ encore mais parce qu'on effectue une mesure faible, 
$\delta \ll \sigma$ donc il vas avoir un chevauchement plus évident au lors 
de la mesure,

\begin{equation}
    |F(x)|^2 = |A|^2 Z(x-\delta) + |B|^2 Z(x) + \bar{A}B\bra{\zeta(x-\delta)}\ket{\zeta(x)} + \bar{B}A\bra{\zeta(x)}\ket{\zeta(x-\delta)}
\end{equation}

Donc, quand nous mesurons l'observable cet variable 
$x$ avec la $\ll$ position $\gg$ moyen de l'observable,

\begin{align}
    \expval{\hat{x}} &= \int_{-\infty}^{\infty} x|F(x)|^2 dx\\
    &= \int_{-\infty}^{\infty} x|A|^2 Z(x-\delta) + x|B|^2 Z(x) + x\bar{A}B\bra{\zeta(x-\delta)}\ket{\zeta(x)} + x\bar{B}A\bra{\zeta(x)}\ket{\zeta(x-\delta)} dx
\end{align}

Chaque intérgrale est évalué dans l'annexe A,

\begin{align}
    \expval{\hat{x}} = \delta|A|^2 + \delta(\bar{A}B + \bar{B}A)e^{-\frac{\delta^2}{8\sigma^2}}
\end{align}

Ainsi que les distributions gaussien son chevauché, nous supposons que $\delta \ll \sigma$ car 
nous effectuons une intéraction qui est le plus faible qu'une interaction du model standard. Cet-à-dire,
nous prenons limite dont $\frac{\delta^2}{\sigma^2} \rightarrow 0$,

\begin{align}
    \lim_{\frac{\delta}{\sigma} \rightarrow 0} \expval{\hat{x}} &= \lim_{\frac{\delta}{\sigma} \rightarrow 0} \delta|A|^2 + \delta(\bar{A}B + \bar{B}A)e^{-\frac{\delta^2}{8\sigma^2}}\\
    &= \delta(|A|^2 + \bar{A}B + A\bar{B})
\end{align}

Il est démontré que ceci correspond à la partie réel de ce qu'AAV renomme la valeur faible
du système $\expval{\hat{\pi}_W}$. Cette valeur introduit sur le système un shift de $\expval{\hat{x}}$ sur le pointeur
ainsi que dans sont complexe conjugué $\expval{\hat{p}}$. Ce dernier, se trouve à l'aide 
d'une transformation de Fourier,

\begin{align}
    F(k) &= \frac{1}{\sqrt{2\pi}}\int_{-\infty}^{\infty} F(x)e^{-i2\pi kx}dx\\
    &= \frac{\sqrt[4]{2}\sqrt{\sigma}}{\sqrt[4]{\pi}}e^{-2\pi k(2\pi k\sigma^2 + i\delta)}(A+Be^{i2\pi k\delta})
\end{align}

Ensuit on repête les mêmes étapes mais pour trouver $\expval{\hat{k}}$,

\begin{align}
    \expval{\hat{k}} &= \int_{-\infty}^{\infty} k|F(k)|^2 dk \\ 
    &= \int_{-\infty}^{\infty} (|A|^2 e^{i2\pi k\delta} + A\bar{B} + B\bar{A}e^{4i\pi k\delta} + |B|^2 e^{i2\pi k\delta})e^{-2\pi k(4\pi k\sigma^2 + i\delta)}kdk\\
    &= \frac{i\delta e^{\frac{-\delta^2}{8\sigma^2}}}{4\sigma^2}(B\bar{A} - A\bar{B})
\end{align}

Parce que ceci est t'une mesure faible, on suppose que $\delta \ll \sigma$,

\begin{equation}
    \expval{\hat{k}} = \frac{i\delta}{4\sigma^2}(B\bar{A} - A\bar{B})
\end{equation}

Ce dernier représente le shift sur le conjugué de notre pointeur qui est représentant de la partie
imaginaire de la valeur faible du système. Ensemble la partie imaginaire s'écrit:

\begin{equation}
    \expval{\hat{\pi}_W} = \frac{1}{\delta}(\expval{\hat{x}} + i4\sigma^2 \expval{\hat{k}})
\end{equation}

Dont pour une mesure faible sur la partie $|0>$,

