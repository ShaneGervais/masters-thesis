\begin{doublespace}
    
    Les travaux récents sur les mesures faibles ont montré 
    leur potentiel dans divers domaines, soit dans 
    le paradoxe de Hardy, l'électrodynamique 
    quantique, télécommunication optique et ainsi 
    suite \cite{Aharonov_2002_Hardy,HardyParadox,Lundeen_thesis,QED,Brunner_2004,OpticalNetworks}. 
    Par exemple, 
    Brunner et al. ont exploré les réseaux de 
    télécommunication optique comme un cadre expérimental 
    pour des mesures faibles. Dans certains cas les mesures faibles peuvent 
    même surpasser des mesures traditionelles 
    \cite{Jeff_outperform,Magaña-Loaiza_2017,WeakorStd}.
    Magana-Loaiza et Lundeen ont 
    étendu ces recherches aux mesures spatiales et 
    quantité de mouvement. Cependant, les mesures faibles 
    positionnelles 
    souffrent des limitations importantes pour les 
    applications technologiques quantiques. Elles 
    nécessitent souvent l’utilisation de cristaux 
    BBO (barium borate (borate de barium))
    d’une taille définie pour ajuster l’interaction faible \cite{Hairiri,Guilleaum,Guilleaum_thesis}. 
    Cette contrainte complique leur adaptabilité à 
    différents systèmes. En revanche, les techniques 
    interférométriques permettent de contrôler directement 
    les délais temporels en ajustant simplement la position 
    des miroirs, supprimant ainsi la dépendance à des 
    cristaux précis. Cette flexibilité fait des mesures 
    faibles temporelles un choix idéal pour les systèmes 
    dynamiques ou industriels. Certain on travailler dans 
    des mesures faibles temporelle mais soit par 
    rapport d'un délai fréquentielle ou 
    pour des aspects théorique \cite{Salazar,OpticalNetworks,Steinberg_prob_div}.
    Cependant, cette thèse propose de surmonter les limitations 
    des mesures faibles positionnelles en développant 
    une approche temporelle utilisant un système photonique 
    quantique. En particulier, l’objectif est de mesurer 
    directement la partie réelle et imaginaire de la valeur 
    faible pour complètement caractériser un état de polarisation. 
    Cette approche exploite la polarisation comme base 
    quantique, car elle est facilement contrôlable et 
    réalisable en laboratoire. En développant une méthode plus efficace pour 
    caractériser les états quantiques directement, cette 
    thèse vise à contribuer à l'avancement des technologies 
    quantiques et à ouvrir de nouvelles possibilités dans 
    les domaines scientifique et industriel.
\end{doublespace}