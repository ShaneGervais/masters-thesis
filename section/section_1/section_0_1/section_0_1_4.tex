Enfin que la fonction d'onde soit la fonction contient
l'information du système quantique peux d'être déterminer 
expériemntalement. Mais comment une fonction d'onde, 
une fonction mathématique
qui se distribue dans l'espace peux d'être déterminer?
Rappel qu'on a mentionné que l'état quantique est
interprèter statistiquement. La règle de Born définie que
la distribution probabilistique pour soit la position 
d'une particule
d'un état quantique $\psi(r,t)$
entre des espaces $a$ et $b$ se trouve:

\begin{equation}
    Prob = \int_{a}^{b} |\psi(r,t)|^2 dr
\end{equation}

Et pour sa position temporel on intègre par $dt$. Pour la quantité
de mouvement d'une particule $p$ on prend la transformation de Fourier de sa
fonction d'onde et on intègre par la quantité de mouvement.

\begin{align}
    \psi(p,t) &= \frac{1}{\sqrt{2\pi\hbar}} \int_{-\infty}^{\infty} e^{-\frac{ipr}{\hbar}} \psi(r,t) dr\\
    Prob_p &= \int_{-\infty}^{\infty} |\psi(p, t)|^2 dp
\end{align}

L'interprétation statistique de la mécanique quantique 
entraîne une incertitude sur la position et la quantité de 
mouvement de l'état, ainsi que sur son temps et son 
énergie, décrite par le principe d'incertitude 
d'Heisenberg.

\begin{align}
    \Delta x \Delta p &\geq \frac{\hbar}{2}\\
    \Delta t \Delta E &\geq \frac{\hbar}{2}
\end{align}

Ce qui nous amène à la question de savoir comment 
nous pouvons obtenir des informations sur le système 
quantique, soit sa position ou d'autre observation réel
qu'on puisque mesurer. Cela se fait à l'aide d'observables. Les 
observables sont des valeurs d'espérance hermitienne 
observable d'une information sur un état quantique, 
telle que sa position ou sa quantité de mouvement, avec 
un résultat réel, qui s'écrit comme suit :

\begin{equation}
    \expval{\hat{Q}} = \bra{\psi}\hat{Q}\ket{\psi}
\end{equation}

Soit pour un observable $\hat{Q}$ dont sa valeur d'espérance
$\expval{\hat{Q}}$ se trouve par un produit de scalare du 
vecteur de la fonction d'onde $\psi$. Ensuite, 
comment mesurer expérimentalement ces observables ? 
Les sections suivantes décriront les mesures quantiques 
traditionnelles, puis une technique que nous explorerons 
dans le cadre de cette thèse, à savoir les mesures faibles. 