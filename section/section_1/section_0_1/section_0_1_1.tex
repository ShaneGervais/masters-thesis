Nous commencerons par la notation. En mécanique 
quantique, il est commun d'utiliser ce que l'on appelle 
la notation braket pour décrire les états quantiques. 
Un ket aura la structure suivante tout au long de cette 
thèse:

\begin{equation}
    \ket{nom^{condition A}_{coniditon B}}_{base}
\end{equation}

Un ket est utilisé pour représenter un vecteur. À 
l'intérieur du ket se trouve le nom du ket qui décrit ce 
qu'est le ket. Il peut y avoir des indices d'un ket 
montrés comme condition A et condition B. La condition 
A sera principalement utilisée pour décrire si un état 
est dans un état initial ou dans son état final. La 
condition B sera principalement utilisée pour décrire 
plus en détail l'état, tel qu'un état qui est un stade 
intermédiaire avec l'indice INT ou un état qui est passé 
par une condition spécifique telle qu'un type de fibre 
optique que nous pourrions écrire avec l'indice FIB. Voici
un exemple d'un ket qui represente un vecteur à $N$ dimension 
avec ${a_n}$ composantes complexe.

\begin{align}
    \ket{a} \to a = \begin{pmatrix}
        a_1\\
        a_2\\
        \vdots\\
        a_N
    \end{pmatrix}
\end{align}

Un bra est la transposée conjuguée d'un vecteur, par 
exemple le ket $\ket{a}$, sa transposée conjuguée est écrite en 
bra :

\begin{equation}
    \bra{a} \to \bar{a} = \begin{pmatrix}
        \bar{a}_1 & \bar{a}_2 \dots \bar{a}_N
    \end{pmatrix}
\end{equation}

La bar represente que ceci est le complex conjugué de $a$. 
Donc, si $a = \alpha + i\beta$, le complex conjugué s'écrit 
$\bar{a} = \alpha - i\beta$ dont $\alpha in \mathcal{R}$ et
$\beta \in \mathcal{I}$. $\mathcal{R}$ et $\mathcal{I}$
represente l'espace réel et imaginaire respectivement. Le produit scalaire de 
deux vecteurs s'écrirait comme suit :

\begin{equation}
    \bra{a}\ket{b} = \bar{a}_{1}b_1 + \bar{a}_{2}b_2 + \dots + \bar{a}_{N}b_N
\end{equation}

L'exemple précédente utilise un vecteur $\ket{b}$ avec 
les mêmes conditions que $\ket{a}$.
Une autre notation que nous verrons tout au long de cette 
thèse est celle d'une fonction représenter avec un ket, 
qui sera écrite comme suit:

\begin{equation}
    \ket{f(x)} \to f(x) = x^2
\end{equation}

Le produit scalare de deux fonction $f(x)$ et $g(x)$ de
$a$ à $b$ s'écrite comme suit:

\begin{equation}
    \bra{f(x)}\ket{g(x)} \equiv \int_{a}^{b} \bar{f(x)}g(x) dx
\end{equation}

Les observables et les opérateurs auront la notation 
suivante (nous reviendrons plus tard sur ces objets 
quantiques) :

\begin{align}
    \expval{\hat{nom}_{conditon A}} &\to observable\\
    \hat{nom}^{condition B} &\to op\acute{e}rateur
\end{align}

Un opérateur et une observable on un chapeau. 
La condition A d'un observable est pour décrire une condition
spécifique du observable et la condition B du opérateur est pour
d'écrire spécifiquement où l'opérateur s'applique. Plus sur les opérateurs et observables se suit.
