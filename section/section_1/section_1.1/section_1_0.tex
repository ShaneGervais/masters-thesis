\begin{doublespace}
    
    Au cours du dernier siècle, les progrès en physique et en 
    technologie ont transformé notre société. Des 
    communications instantanées, ces avancées reposent sur 
    des mesures précises et des outils sophistiqués. 
    Cependant, certains problèmes d’une complexité élevée, 
    notamment ceux liés à la nature quantique, demeurent 
    inaccessibles aux technologies conventionnelles. 
    L’émergence de l’informatique quantique, initialement 
    proposée par Richard Feynman en 1981, et d'autre 
    technologies quantique comme la communication quantique 
    a ouvert une voie pour résoudre ces problèmes à l’aide 
    de calculs exploitant les propriétés de la mécanique 
    quantique \cite{Feynman1982,DiVincenzo_2000,JordanBook}. 
    Malgré leur potentiel, les mesures quantiques 
    suscitent des débats scientifiques et philosophiques 
    depuis la naissance de la théorie quantique. Par exemple, 
    le paradoxe EPR d’Einstein, Podolsky et Rosen questionne 
    la complétude de la théorie, et même 
    John Bell qui confirment la complétude de la théorie et 
    son caractère 
    probabiliste et non déterministe, écrit un article 
    contre les mesures quantiques et ainsi que d'autre scientifiques au cours des années 
    \cite{EPR,Bell1966,Bell_1990,JordanBook,MeasurementProb,MeasurementProb2,Wigner}. 
    Mais la théorie quantique et les mesures quantique ne sont pas 
    contradictoire \cite{Peebles}. Dans la mécanique quantique, les mesures 
    perturbent le système observé, limitant la précision des 
    informations accessibles. Cela reflète la nature 
    probabiliste de la mécanique quantique, contrairement 
    à la mécanique classique où les propriétés d'un système 
    peuvent être prédites avec certitude, les résultats des 
    mesures quantiques suivent une distribution de 
    probabilités relié à ça fonction d'onde du système. Une 
    analogie simple est celle d’un dé : lorsqu’on lance un 
    dé en mécanique classique, toutes les faces ont une 
    probabilité fixe et égale, et le résultat est aléatoire 
    mais déterminé après le lancer. En mécanique quantique, 
    cependant, le système existe initialement dans une superposition 
    d’états dont chaque état étant pondéré par une amplitude 
    complexe. Une fois qu'on lance et mesure l'état du dé,
    on dit que le système se réduit à une de ses états possibles étant
    que si on prend plusieurs mesure du état réduit, le résultat
    est toujours le même. 
\end{doublespace}

\begin{figure}[!h!t!p!b]
    \centering
    \includegraphics[width=1.0\textwidth]{dice.png}
    \caption{L’état du dé en superposition et puisque réduit à une de ses valeurs après avoir été lancer}
    \label{fig:dice}
\end{figure}

\begin{doublespace}
    
    Cette nature probabiliste complique 
    considérablement les mesures, car le résultat n'est pas 
    simplement le reflet de la réalité préexistante, mais 
    plutôt une des nombreuses possibilités définies par la 
    fonction d'onde. Ainsi, mesurer un système quantique 
    perturbe inévitablement son état initial et force 
    l’effondrement de la superposition vers un seul état 
    observable. Le principe d'incertitude d'Heisenberg 
    prolonge cette idée en soulignant une autre limite 
    fondamentale de la mécanique quantique: il est 
    impossible de mesurer simultanément avec précision la 
    position et la quantité de mouvement d'une particule. 
    Ce principe découle directement de la nature ondulatoire 
    et probabiliste des systèmes quantiques, où toute 
    tentative de mesure perturbe inévitablement l'état 
    initial du système. Cette contrainte résulte directement 
    de la nature ondulatoire et probabiliste des systèmes 
    quantiques, où l'acte même de mesurer perturbe l'état 
    initial du système. Cette perturbation survient parce 
    que la mesure quantique, contrairement à une mesure 
    classique, interagit directement avec le système et 
    modifie son état. Les mesures jouent un rôle essentiel en mécanique 
    quantique, car elles sont le principal moyen d’obtenir 
    des informations sur un système. Cependant, cette 
    importance s’accompagne de défis uniques. Dans cette 
    thèse, nous allons explorer les mesures faibles dans
    un système photonique quantique, en utilisant les délais 
    temporels comme pointeur pour extraire des informations 
    sur les états de polarisation. Un système photonique nous
    intéresse car des photons peuvent être manipuler 
    et détecter facilement, ils ont peut de décohérence et
    leur hautes vitesse permet d'avoir des grandes applications
    dans la communications, cyrptographie et informatique 
    quantique \cite{OpticalNetworks}. Nous commencerons par 
    examiner les fondements théoriques de la mécanique 
    quantique et des mesures faibles, puis nous présenterons 
    les aspects expérimentaux liés à la caractérisation des 
    états quantiques. Enfin, nous analyserons les résultats 
    obtenus et discuterons leurs implications pour les 
    applications futures dans le domaine des technologies 
    quantiques.
\end{doublespace}