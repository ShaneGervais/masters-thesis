\subsection{Compréhension analogique des mesures quantiques}
    
\begin{doublespace}
    
    Les mesures en physique sont essentielles au progrès technologique et à l’évolution de l’humanité. Notre capacité à mesurer avec précision et efficacité influence directement les performances de nos innovations. Les processeurs actuels fonctionnent à l’échelle du nanomètre, ce qui nécessite une précision extrême dans les méthodes de mesure. De plus, la diffusion rapide de données dans les communications repose sur une synchronisation temporelle rigoureuse. De plus, la communication sans fil repose sur des mesures spectrales, et d’autres domaines montrent l’importance des mesures physiques pour l’efficacité de nos technologies. Cependant, les technologies actuelles présentent certaines limites ou certains obstacles au niveau quantique, ainsi qu’en ce qui concerne la complexité des calculs. Tout comme dans la fabrication de processeurs, nous cherchons constamment à augmenter le nombre de transistors et à les miniaturiser toujours plus, en respectant la loi de Moore. Cependant, à mesure que nous nous rapprochons du niveau quantique (inférieur à environ 2 nm), la précision de nos mesures devient un problème en raison du principe d’incertitude d’Heisenberg. Ce principe nous empêche de connaître la position et le mouvement d’une particule, comme un électron qui circule dans les transistors avec précision simultanée. À ce stade, les électrons peuvent se trouver dans un état de tunnel quantique, passant d’un niveau à l’autre dans le processeur, ce qui entraîne des courts-circuits et une instabilité. En ce qui concerne le domaine des télécommunications, nous nous efforçons de préserver la confidentialité de nos échanges, de réduire les pertes de données pendant leur transmission et de garantir une réception appropriée des informations transmises par les fibres optiques, sous forme cryptée et sécurisée. Il est important de noter que, lorsqu’ils traversent de longues distances, les photons ont tendance à se dissiper et à être absorbés par le matériau des fibres optiques. Les répéteurs ont la capacité de renforcer un signal, mais cela a pour conséquence d’augmenter le bruit, ce qui affecte les propriétés quantiques des photons. Cela entraine une réduction de la transmission des données et potentiellement une hausse des pertes. Par conséquent, le cryptage de ces données exige une méthodologie traditionnelle, dépendante de la complexité de la factorisation des grands nombres. L’algorithme de Shor possède le pouvoir de mettre en péril cette méthode, ce qui aurait des conséquences néfastes sur notre système politique et nos transactions financières.

    \noindent Il serait captivant d’élaborer une machine où l’effet quantique pourrait être maîtrisé et intégré. Envisagez une machine de Turing universelle capable de résoudre des problèmes complexes en utilisant l’encodage quantique, tout en offrant une infrastructure de communication quantique basée sur des photons. En effet, cet appareil possède effectivement les attributs d’un ordinateur quantique. Il peut être intégré à notre infrastructure pour renforcer la sécurité de nos données et favoriser l’avancée des technologies quantiques. L’avènement de l’informatique quantique et d’autres technologies quantiques, telles que la communication quantique, a ouvert la voie à la résolution de ces problèmes grâce à des calculs exploitant les propriétés de la mécanique quantique Feynman (1982), DiVincenzo (2000) et Jordan (Livre). Bien que leur potentiel soit considérable, l’un des plus grands défis et critères pour les ordinateurs quantiques et les technologies quantiques connexes est la mesure quantique. La théorie quantique a suscité des discussions scientifiques et philosophiques depuis sa création. Elles remettent en question sa validité, malgré la confirmation de sa complétude par des personnalités telles que John Bell, qui a également écrit un article s’opposant aux mesures quantiques, tout comme d’autres chercheurs au fil des ans. Cependant, la théorie quantique et les mesures quantiques ne sont pas contradictoires. Dans la mécanique quantique, les mesures perturbent le système observé, limitant la précision des informations accessibles. Cela met en évidence l’aspect imprévisible de la mécanique quantique en comparaison avec la mécanique classique, où l’on peut prédire avec certitude les propriétés d’un système. Les résultats des mesures quantiques dépendent de la fonction d’onde du système et suivent une distribution de probabilités. 

    \noindent On peut comparer la nature probabiliste des mesures quantiques à celle d’un dé. Classiquement, lorsqu’on le lance, chaque face a une chance égale et prédéterminée de sortir. Le résultat final est donc imprévisible, mais il devient déterminé dès qu’on le dépose sur une surface plane. En mécanique quantique, on propose qu’un système se trouve initialement dans un mélange d’états, chacun ayant un coefficient de probabilité complexe. Une fois qu’on a lancé et mesuré l’état du dé, on dit que le système se réduit à un de ses états possibles, le résultat restant toujours le même après que le système ait été perturbé.
 
\end{doublespace}

\begin{figure}[!h!t!p!b]
    \centering
    \includegraphics[width=1.0\textwidth]{dice.png}
    \caption{L’état du dé en superposition et puisque réduit à une de ses valeurs après avoir été lancer}
    \label{fig:dice}
\end{figure}

\begin{doublespace}
    \noindent Cette nature probabiliste rend complexe toute tentative de mesure, car le résultat ne reflète plus seulement la réalité préexistante, mais plutôt une des nombreuses possibilités définies par la fonction d’onde. Une fois que l’état est perturbé, le résultat reste toujours pareil. Autrement dit, mesurer un système quantique perturbe inévitablement son état initial et force l’effondrement de la superposition vers un seul état observable. Cette perturbation survient parce que la mesure quantique, contrairement à une mesure classique, interagit directement avec le système et modifie son état. Les mesures sont essentielles en mécanique quantique, car elles représentent le principal moyen d’obtenir des informations sur un système. Cependant, cette importance s’accompagne de défis uniques. 
\end{doublespace}

\subsection{La tomographie quantique}

\begin{doublespace}
    
    Traditionnellement, la tomographie quantique 
    est utilisée pour reconstruire la fonction 
    d’onde d’un état quantique à partir d’un 
    ensemble de mesures. Ce processus consiste 
    à effectuer des mesures de projection sur un 
    multiples d’état quantique en utilisant des 
    bases orthogonales variées. Les résultats 
    sont ensuite traités par un algorithme 
    complexe qui reconstruit indirectement la 
    fonction d’onde. Dans le cadre de Paul 
    Kwiat et ses collaborateurs, ils ont 
    développé des protocoles de tomographie 
    d’état photoniques permettant de 
    caractériser précisément l'état de 
    polarisation sous forme d'un qubit \cite{Kwiat}. Pour 
    ce faire, ils mesurent l’état du photon 
    dans plusieurs bases, soit $\{\ket{H}, \ket{V}\}$, 
    $\{\ket{D}, \ket{A}\}$ et 
    $\{\ket{R}, \ket{L}\}$, (polarization horizontale, 
    vertical), (diagonal, anti-diagonal) et 
    (circulaire droit, gauche) respectivement. 
    À partir de ces mesures, il est possible de 
    reconstruire la matrice densité de l’état 
    quantique. La matrice densité est observable
    qui contient tout l'information de l'état quantique
    comme la fonction d'onde mais inclue des propriétés
    statistique que le système pourrais possèder. Elle
    prend la 
    forme $\hat{\rho} = \ket{\psi}\bra{\psi}$
    pour un état $\ket{\psi}$ et on peut 
    aisément vérifier sa pureté de l'état en 
    prenant la trace $Tr(\rho)=1$. Plus d'informations à ce 
    sujet sous peu. Pour approfondir nos 
    connaissance, considérons un exemple 
    plus concret. Supposons un photon préparé 
    dans l’état de polarisation arbitraire soit

    \begin{equation}
        \ket{\psi} = a\ket{H} + b\ket{V}
    \end{equation}

    \noindent avec $a$, $b \in \mathcal{C}$ et $|a|^2 + |b|^2 = 1$
    La matrice densité associée à cet état pur s’écrit

    \begin{equation}
        \hat{\rho} = \begin{pmatrix}
            |a|^2 & a\bar{b}\\
            \bar{a}b & |b|^2
        \end{pmatrix}
    \end{equation}

    \noindent Expérimentalement, le but est de déterminer les 
    coefficients de la matrice $a$ et $b$. Pour ce faire,
    il faut mesurer les probabilités de 
    détection dans différentes bases de 
    polarisation, soit en orientent un polariseur 
    ou avec un séparateur de faisceau polarisant, 
    selon les bases de notre état $\ket{\psi}$ soit
    $\{\ket{H}, \ket{V}\}$. La fraction de photons 
    détectés en sortie $\ket{H}$ correspond 
    alors à $|a|^2$ et en sortie $\ket{V}$ 
    correspond à $|b|^2$. Pour accéder aux 
    termes d’interférence, comme $a\bar{b}$, 
    il faut réaliser des mesures dans des bases 
    complémentaires, telles que $\{\ket{D}, \ket{A}\}$ 
    pour la polarisation diagonale et antidiagonale, 
    ou $\{\ket{R}, \ket{L}\}$ pour la 
    polarisation circulaire. De manière 
    pratique, on insère des lames quart-d’onde 
    et demi-d’onde pour transformer la base $\{\ket{H}, \ket{V}\}$
    vers l’une de ces bases, puis on redirige à 
    nouveau les photons vers un séparateur de 
    faisceau polarisant. Les différences 
    d’intensité observées dans ces diverses 
    configurations expérimentales permettent 
    de reconstruire les éléments de matrice 
    densité. Cette démonstration illustre la 
    puissance de la tomographie quantique, 
    tout en soulignant sa complexité et les 
    ressources expérimentales nécessaires à sa 
    mise en œuvre pour des états de dimension 
    plus élevée. Dans le formalisme de 
    Paul Kwiat, cette matrice densité peut 
    également être exprimée en termes de 
    paramètres de Stokes. Les paramètres de 
    Stokes décrivent complètement l’état de 
    polarisation, et ils sont liés aux 
    probabilités de détection dans 
    différentes bases de polarisation. 
    Les paramètres sont défini par:

    \begin{equation}
        S = \begin{pmatrix}
            S_0\\
            S_1\\
            S_2\\
            S_3
        \end{pmatrix}
        = \begin{pmatrix}
            P_{\ket{H}} + P_{\ket{V}}\\
            P_{\ket{H}} - P_{\ket{V}}\\
            P_{\ket{D}} - P_{\ket{A}}\\
            P_{\ket{R}} - P_{\ket{L}}
        \end{pmatrix}
    \end{equation}

    \noindent Concrètement, $S_0$ représente 
    l’intensité ou probabilité 
    totale du faisceau, $S_1$ représente 
    la différence d’intensité entre les 
    polarisations $\ket{H}$ et $\ket{V}$, $S_2$ 
    représente la différence d’intensité entre 
    les polarisations $\ket{D}$ et $\ket{A}$ et
    $S_3$ représente la différence d’intensité 
    entre les polarisations $\ket{R}$ et $\ket{L}$.
    En notant  les intensités (ou probabilités de mesure) 
    pour chacune de ces bases, on reconstruit la matrice
    densité par les probabilités trouvé avec:

    \begin{equation}
        \hat{\rho} = \frac{1}{2}\sum_{i=0}^{3} S_i \sigma_i
    \end{equation}

    \noindent Soit $\sigma_i$ sont les matrices de Pauli défini comme suit:

    \begin{equation}
        \hat{\sigma}_0 = \begin{pmatrix}
            1 & 0\\
            0 & 1
        \end{pmatrix}
        \hat{\sigma}_1 = \begin{pmatrix}
            1 & 0\\
            0 & -1
        \end{pmatrix}
        \hat{\sigma}_2 = \begin{pmatrix}
            0 & 1\\
            1 & 0
        \end{pmatrix}
        \hat{\sigma}_3 = \begin{pmatrix}
            0 & -i\\
            i & 0
        \end{pmatrix}
    \end{equation}


    \noindent Pour un état pur, comme notre example, on a la 
    propriété , ce qui se traduit par une cohérence 
    quantique maximale. En revanche, un état mixte 
    se décrit par une matrice densité statistique, 
    somme pondérée de matrices densité pures:

    \begin{equation}
        \hat{\rho}_{mixte} = \sum_{i}^{N} p_i\hat{\rho}_i
    \end{equation}


    \noindent avec $N$ états avec des probabilités $p_i$ pour 
    chaque matrice densité $\hat{\rho}_i$ dont $\sum_{i}^{N} = 1$. 
    Dans ce cas, $Tr(\hat{\rho}_{mixte}) < 1$.
    Ainsi, que la distinction de la pureté de la matrice densité peut se définir à 
    partir de la trace de la matrice densité, 
    l’état pur maintient une cohérence parfaite, 
    tandis qu’un état mixte est issu d’un mélange 
    statistique d’états. Les protocoles de tomographie 
    quantique proposés par Kwiat, 
    permettent de déterminer empiriquement ces 
    coefficients (à partir des paramètres de Stokes) 
    expérimentalement, et peut donc reconstruire 
    et caractériser la 
    matrice densité complète d’un photon. Cependant, 
    cette approche présente des 
    inconvénients majeurs: elle est indirecte, 
    complexe et exige un traitement algorithmique 
    intensif, ce qui limite son applicabilité, 
    notamment dans les systèmes dynamiques ou 
    les environnements industriels.
\end{doublespace}

\subsection{La mesure faible}
\begin{doublespace}
    Une alternative intéressante réside dans les mesures 
    faibles, une méthode directe permettant d'accéder à la 
    fonction d'onde d'un système quantique. Introduite par 
    Aharonov, Albert et Vaidman (AAV) dans les années 1980, 
    cette approche repose sur une interaction contrôlée 
    entre un pointeur et un système quantique \cite{Aharonov}. 
    Contrairement aux mesures fortes, qui provoquent un 
    effondrement complet de la fonction d'onde et détruisent 
    la superposition quantique, les mesures faibles 
    préservent cette superposition en minimisant la 
    perturbation du système. Pour 
    illustrer, la différence entre mesures fortes et 
    faibles peut être représentée par une impulsion 
    gaussienne où les états de base sont séparés soit 
    fortement, soit faiblement. Une figure montrant une 
    telle impulsion permettrait de clarifier comment les 
    mesures faibles minimisent l'interaction tout en 
    extrayant des informations précises.
\end{doublespace}
    
\begin{figure}[!h!t!p!b!]
    \centering
    \includegraphics[width=0.9\textwidth]{force_de_mesure.pdf}
    \caption{Représentation visuelle de la 
    différence entre une mesure indirecte et directe sur 
    un système quantique. Supposons un état quantique 
    initialement $\ket{\psi_i}$ avec des états de base $\{ \ket{0};\ket{1} \}$ avec 
    une dispersion d'une distribution de probabilité de l'état $\sigma$ et soit $\delta$ 
    la force de séparation de l'interaction effectuée. a) L'état 
    quantique subit ce que nous appelons une mesure 
    « forte » où l'interaction avec l'état quantique 
    sépare les états de base plus que la distribution de 
    probabilité $\delta \gg \sigma$. Aucune information ne peut donc être 
    récupérée. b) L'état quantique subit une interaction 
    plus faible où ses états de base sont séparés de 
    façon très inférieure à l'écart de distribution des 
    probabilités $\delta \ll \sigma$. L'information réside alors dans le 
    chevauchement de ces états de base, qui peut être 
    récupéré à l'aide d'une mesure projective.
    }
    \label{fig:force_de_mesure}
\end{figure}

\begin{doublespace}
    \noindent Le modèle von Neumann des mesures quantiques fournit 
    le cadre théorique pour comprendre les mesures faibles. 
    Dans ce modèle, le système quantique et le pointeur 
    sont intriqués via un opérateur d’interaction, 
    permettant d’extraire des informations sur la fonction 
    d'onde. Dans le contexte des mesures faibles, la force 
    de l'interaction est choisie pour que le déplacement du 
    pointeur reste plus petit que la largeur de la 
    distribution des probabilités, ce qui permet de mesurer 
    directement les composantes réelles et imaginaires de 
    la valeur faible associée à un état quantique \cite{vonNeumann,Lundeen_Resch,Lundeen_Direct_Measurement}.
    La figure suivante démontre une répresentation du modèle de von Neumann
    utilisée pour les mesures faibles.

    \begin{figure}[!h!t!p!b!]
        \centering
        \includegraphics[width=0.9\textwidth]{fuel_gauge.png}
        \caption{place holder}
        \label{fig:force_de_mesure}
    \end{figure}
    
    \noindent Les mesures faibles sert à un oeuil de juda pour 
    le monde quantique \cite{Peephole}. Ça nous permettons de pertubé
    le système le plus faiblement possible pour obtenir de 
    l'information sur le système quantique. 
    Jeff Lundeen et ses collaborateurs ont joué un rôle 
    crucial dans l’avancement de cette technique. Leur 
    approche typique utilisait un cristal BBO mince pour 
    ajuster la force d’interaction, permettant de réaliser 
    des mesures faibles sur des impulsions lumineuses 
    polarisées. Ils ont notamment démontré que les mesures 
    faibles pouvaient servir à caractériser la matrice 
    de densité et à valider des prédictions fondamentales de 
    la mécanique quantique \cite{Lundeen_Bamber,Lundeen_Direct_Measurement,Lundeen_Resch,Lundeen_thesis}. 
    L’adoption des mesures faibles repose sur plusieurs 
    avantages clés : elles réduisent les perturbations 
    induites sur le système, préservent la cohérence 
    quantique et permettent une approche directe et 
    intuitive pour caractériser des états quantiques.
    
\end{doublespace}
    
\subsection{Motivation de la thèse}

\begin{doublespace}
    
    Les travaux récents sur les mesures faibles ont montré 
    leur potentiel dans divers domaines, soit dans 
    le paradoxe de Hardy, l'électrodynamique 
    quantique, télécommunication optique et ainsi 
    suite \cite{Aharonov_2002_Hardy,HardyParadox,Lundeen_thesis,QED,Brunner_2004,OpticalNetworks}. 
    Par exemple, 
    Brunner et al. ont exploré les réseaux de 
    télécommunication optique comme un cadre expérimental 
    pour des mesures faibles. Dans certains cas les mesures faibles peuvent 
    même surpasser des mesures traditionelles 
    \cite{Jeff_outperform,Magaña-Loaiza_2017,WeakorStd}.
    Magana-Loaiza et Lundeen ont 
    étendu ces recherches aux mesures spatiales et 
    quantité de mouvement. Cependant, les mesures faibles 
    positionnelles 
    souffrent des limitations importantes pour les 
    applications technologiques quantiques. Elles 
    nécessitent souvent l’utilisation de cristaux 
    BBO (barium borate (borate de barium))
    d’une taille définie pour ajuster l’interaction faible \cite{Hairiri,Guilleaum,Guilleaum_thesis}. 
    Cette contrainte complique leur adaptabilité à 
    différents systèmes. En revanche, les techniques 
    interférométriques permettent de contrôler directement 
    les délais temporels en ajustant simplement la position 
    des miroirs, supprimant ainsi la dépendance à des 
    cristaux précis. Cette flexibilité fait des mesures 
    faibles temporelles un choix idéal pour les systèmes 
    dynamiques ou industriels. Certain on travailler dans 
    des mesures faibles temporelle mais soit par 
    rapport d'un délai fréquentielle ou 
    pour des aspects théorique \cite{Salazar,OpticalNetworks,Steinberg_prob_div}.
    Cependant, cette thèse propose de surmonter les limitations 
    des mesures faibles positionnelles en développant 
    une approche temporelle utilisant un système photonique 
    quantique. En particulier, l’objectif est de mesurer 
    directement la partie réelle et imaginaire de la valeur 
    faible pour complètement caractériser un état de polarisation. 
    Cette approche exploite la polarisation comme base 
    quantique, car elle est facilement contrôlable et 
    réalisable en laboratoire. En développant une méthode plus efficace pour 
    caractériser les états quantiques directement, cette 
    thèse vise à contribuer à l'avancement des technologies 
    quantiques et à ouvrir de nouvelles possibilités dans 
    les domaines scientifique et industriel.
\end{doublespace}