\subsection{Introduction}

\begin{doublespace}
    L’émergence de technologies quantiques, telles que les ordinateurs quantiques et la communication quantique, possèdent le potentiel de nous faire entrer dans un nouvel âge quantique. Ces avancées technologiques permettent d’envisager des temps de calcul, nettement plus rapides que ceux d'un ordinateur classique \cite{Feynman1982}. De plus, il sera possible de crypter nos messages avec une cryptographie ultra sécurisée \cite{quantumcomputing40years}, utilisant l’état quantique. Ces technologies profitent également à d’autres domaines bénéficient de ces technologies, comme la télécommunication photonique quantique \cite{photonicInformation,QTapplImpl,QTintelecomindustry}. 
    
    \noindent En effet, le développement des technologies quantiques repose sur la capacité à mesurer et à caractériser les états quantiques avec une précision et une fiabilité accrues \cite{DiVincenzo_2000,QTapplImpl}. Dans ce contexte, le domaine de la photonique quantique occupe une place centrale grâce aux propriétés étonnantes des photons. Contrairement aux systèmes basés sur d’autres matières, les photons présentent une faible décohérence. Ils peuvent être manipulés facilement grâce à leurs divers degrés de liberté (polarisation, domaine positionnel, quantité de mouvement, temporel et fréquentiel), et ils s’intègrent naturellement dans les infrastructures optiques existantes \cite{browne2017quantumopticsquantumtechnologies,photonicInformation}. L’état de polarisation des photons peut être utilisé comme état de base pour des technologies quantiques, par exemple sous forme de qubit pour des ordinateurs ou de circuit quantique \cite{Kwiat}. Cette plateforme possède des caractéristiques idéales pour de nombreuses applications, notamment la communication quantique sécurisée, l’imagerie quantique à haute résolution, la métrologie quantique, le calcul quantique basé sur l’optique linéaire, ainsi que d’autres domaines où les technologies quantiques sont nécessaires \cite{QTintelecomindustry,metrology,Wang_2019}.

    \noindent Par conséquent, il est crucial d’effectuer des mesures approfondies afin de caractériser avec précision les états quantiques dans le cadre de ces avancées technologiques. Cependant, effectuer des mesures quantiques pose un défi en raison de la nature aléatoire de la théorie. Contrairement à la mécanique classique, il est impossible de mesurer simultanément le moment d’arrivée et la fréquence spectrale avec une précision absolue et optimale. Cela est dû au principe d’incertitude d’Heisenberg, qui découle de la nature ondulatoire du système. On prétend que l’état du système se trouve dans une superposition de tous ses états possibles. Une mesure ou une interaction avec le système quantique cause une perturbation qui fait effondrer l’état dans l’un de ses états propres possibles. Une fois perturbé, l’état demeure inchangé. La figure \ref{fig:dice} ci-dessous illustre ce concept de manière classique. Par conséquent, pour caractériser un état quantique, on doit effectuer des mesures projectives pour obtenir certaines informations sur le système. Les techniques conventionnelles, comme la tomographie quantique, permettent une reconstruction complète des états à l’aide de plusieurs mesures projectives. Toutefois, elles deviennent rapidement inadaptées aux systèmes de grande dimension en raison de leur coût computationnel et expérimental exponentiel. Étant que ces techniques reposent sur un grand nombre de mesures de projection, ceci les rend inadaptées à certaines applications nécessitant des mesures en temps réel ou une interaction minimale avec le système.

    \begin{figure}[!htbp]
        \centering
        \includegraphics[width=1.0\textwidth,page=1]{FIGURES.pdf} % Selects page 2
        \caption{Cette analogie s’inspire des célèbres paroles d’Einstein, selon lesquelles \guillemetleft Dieu ne joue pas aux dés \guillemetright quand il met en doute la complétude de la théorie \cite{EPR}, ce qui a été démenti par John Bell \cite{Bell1966}, qui l’a contredit. Envisageons maintenant que nous lancions un dé. Lorsque nous lançons un dé, nous supposons qu’il se trouve dans un état de superposition $\ket{\text{dé}} = c_1\ket{1} + c_2\ket{2} + c_3\ket{3} + c_4\ket{4} + c_5\ket{5} + c_6\ket{6}$, où tous les états propres possibles $\{ \ket{1},\ket{2},\ket{3},\ket{4},\ket{5},\ket{6} \}$ et leurs coefficients de probabilité $c_1, c_2, c_3, c_4, c_5, c_6$ se trouvent dans notre main. Une fois lancés sur la table et en observant leur résultat, nous disons que le système s’effondre vers l’une de ses valeurs possibles, comme l’état $\ket{5}$, qui a une probabilité de $\frac{1}{6}$ de se produire. Une fois qu’il s’est effondré, les mesures ultérieures du système restent les mêmes. L’objectif des mesures quantiques est de caractériser complètement l’état du système quantique du dé. Comme nous sommes limités aux mesures projectives, nous discuterons des techniques possibles, telles que l’effondrement continu du système via la tomographie et la reconstruction indirecte de l’état quantique, ainsi que notre méthode alternative utilisant la mesure faible pour la caractérisation directe de l’état quantique.}
        \label{fig:dice}
    \end{figure}
    

    \noindent Une approche alternative consiste à utiliser des mesures faibles qui permettent d’extraire des informations sur un état quantique directement sans entraîner son effondrement complet. Ce dernier repose sur le modèle de mesure quantique de von Neumann \cite{vonNeumann} et exploite un pointeur couplé au système, dont le déplacement minimal appliqué sur le système est proportionnel à un observable complexe nommé la \guillemetleft valeur faible \guillemetright. Cette technique, introduite par Aharonov, Albert et Vaidman (AAV) \cite{Aharonov}. Bien que les mesures faibles aient été largement étudiées en théorie, leur mise en œuvre expérimentale dans le domaine temporel des photons est relativement peu explorée, en particulier dans le contexte d’applications pratiques sur les technologies quantiques.

\end{doublespace}



\subsection{Motivation de la thèse}

\begin{doublespace}
    
    Les recherches sur les mesures faibles ont démontré leur potentiel dans divers domaines, notamment en ce qui concerne la théorie des mesures quantiques, informatique quantique, cryptographie quantique et la télécommunication optique, entre autres \cite{Lundeen_Resch,QED,OpticalNetworks,Lundeen_thesis,Brunner_2004}. Elles peuvent même se révéler plus efficaces que les méthodes traditionnelles \cite{WeakorStd,Magaña-Loaiza_2017,Jeff_outperform}. Cette thèse se concentre sur la caractérisation d’un état de polarisation dans un système photonique quantique en s’appuyant sur les travaux: \cite{Lundeen_Direct_Measurement,Lundeen_Bamber,Hairiri,Guilleaum}. Ils ont démontré la faisabilité de cette approche en utilisant des mesures faibles et en exploitant le domaine spatial et la quantité de mouvement des photons comme pointeur pour caractériser un système quantique complètement et directement. Toutefois, ces méthodes présentent de graves limites pour les applications et l’intégration dans des technologies quantiques exigeant des mesures en temps réel. Elles exigent généralement l’utilisation de cristaux BBO (bêta-borate de baryum) de taille spécifique pour régler l’interaction faible \cite{Hairiri,Guilleaum,Guilleaum_thesis}. Cette exigence rend leur flexibilité difficile pour s’adapter à divers systèmes. Grâce aux différents degrés de liberté des photons, nous proposons d’utiliser le domaine temporel comme pointeur pour la caractérisation de l’état quantique à l’aide de méthodes interférométriques. Ces dernières offrent en effet un contrôle direct du temps par le réglage de la disposition des miroirs, ce qui élimine le besoin de compter sur des cristaux particuliers. Cette flexibilité rend possible pour l’intégration de mesures faibles temporelle dans des technologies quantiques \cite{kaneda2018highefficiencysinglephotongenerationlargescale,Dai_2020}. Elles peuvent également être facilement implantées dans nos systèmes optiques existant, par exemple dans les télécommunications à fibre optique \cite{OpticalNetworks}. Cette thèse propose de surmonter les limites des mesures faibles en développant une approche temporelle utilisant un système photonique quantique pour caractériser un état quantique. Certains ont déjà travaillé sur des mesures temporelles, mais principalement sous l’angle d’un délai fréquentiel ou à des fins théoriques \cite{Salazar,OpticalNetworks,Steinberg_prob_div}. Notre objectif consiste à évaluer directement la composante réelle et imaginaire de la valeur faible provenant d'un pointeur temporel, ce qui permet une description exhaustive de l’état de polarisation. Cette méthode utilise la polarisation comme base quantique, car elle est facile à contrôler et à mettre en pratique en laboratoire. Cette thèse vise à améliorer la méthode de caractérisation directe des états quantiques, contribuant ainsi à l’avancement des technologies quantiques et à l’exploration de nouvelles opportunités dans les domaines scientifique et industriel.
    
    \noindent Pour résumé, on suggère une nouvelle méthode de mesure faible qui exploite le domaine temporel des photons pour caractériser des états de polarisation. Elle offre un cadre plus robuste pour la caractérisation d’état de polarisation que des mesures faibles basées sur le domaine positionnel. Nous commencerons par une analyse des principes fondamentaux des mesures quantiques. Nous comparerons les approches tomographiques classiques et les mesures faibles en termes de leurs avantages et limites respectifs. Nous présenterons ensuite une méthode innovante, spécialement conçue pour les systèmes photoniques, qui permettra de collecter le plus d'information possible sur l'état quantique tout en limitant l’influence de la mesure (interaction) sur le système. Cette méthode sera évaluée à travers des expériences en laboratoire. Ensuite, nous discuterons des résultats et des implications pour les technologies quantiques émergentes. Enfin, nous examinerons les possibilités d’application de cette technique dans des domaines tels que la communication quantique, l'informatique quantique et autres.
\end{doublespace}