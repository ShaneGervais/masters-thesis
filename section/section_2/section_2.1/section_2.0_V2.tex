\begin{doublespace}
    Ce chapitre explore les fondements théoriques des mesures 
    quantiques, en ce qui concerne la tomographie quantique et les 
    mesures faibles. Nous examinons les bases de la tomographie 
    quantique, puis introduisons les fondements théoriques des mesures 
    faibles selon le formalisme proposé par AAV \cite{Aharonov}, ainsi 
    que dans le cadre expérimental développé dans: \cite{Lundeen_Resch}. 
    Nous dérivons la valeur faible en suivant la procédure directe sur un 
    système quantique dont son résultat final du pointeur est 
    proportionnel à cette valeur en moyenne (avec un état de projection). 
    Cette relation est un outil important dans le processus de 
    caractérisation de l’état quantique. Enfin, nous démontrerons que 
    cette valeur est composée en une partie réelle et une partie 
    imaginaire, dans lesquelles elles peuvent être mesurées en laboratoire.
\end{doublespace}


\subsection{La tomographie quantique}

\begin{doublespace}
        
    Traditionnellement, la tomographie quantique est utilisée pour 
    reconstruire la fonction d’onde d’un état quantique à partir d’un 
    ensemble de mesures projectives. En photonique quantique, ce 
    processus consiste à effectuer des mesures de projection sur divers 
    états quantiques en utilisant des bases orthogonales sélectionnées, 
    soit $\{\ket{H}, \ket{V}\}$, $\{\ket{D}, \ket{A}\}$ et $\{\ket{R}, \ket{L}\}$, 
    (polarisation horizontale, verticale) (diagonale, anti-diagonale) et 
    (circulaire droite, gauche) respectivement. Ensuite, les résultats 
    obtenus sont analysés par un algorithme qui reconstruit implicitement la 
    matrice densitée de l’état quantique. La matrice densitée représente un 
    opérateur hermitien qui renferme toutes les informations sur l’état 
    quantique, y compris la fonction d’onde ainsi que certaines 
    caractéristiques probabilistes qu’un tel système peut présenter. Elle 
    se présente sous la forme $\hat{\rho} =\ket{\psi}\bra{\psi}$ pour un 
    état $\ket{\psi}$ et on peut facilement vérifier sa pureté en prenant 
    la trace de $\hat{\rho}^2$, $Tr(\hat{\rho}^2)=1$. 
    
    Pour approfondir nos connaissances, considérons un exemple 
    arbitraire. Supposons un photon préparé dans l’état de polarisation 
    suivant:
    
    \begin{equation}
        \ket{\psi} = a\ket{H} + b\ket{V}
    \end{equation}

    \noindent où $a$, $b \in \mathcal{C}$, $|a|^2 + |b|^2 = 1$ et dans la 
    base $\{ \ket{H}, \ket{V} \}$. La matrice de densité de cet état, que 
    l'on retrouvera prochainement dans cet exemple, s'écrit comme suit:
    
    \begin{equation}
        \hat{\rho} = \begin{pmatrix}
            |a|^2 & ab^*\\
            a^*b & |b|^2
        \end{pmatrix}
    \end{equation}
    
    \noindent L'objectif d'une tomographie quantique est de déterminer 
    les coefficients de la matrice. Pour ce faire, il faut effectuer des 
    mesures projectives pour obtenir les probabilités ou intensités de 
    détection dans différentes bases de polarisation. La fraction de 
    photons détectés en sortie $\ket{H}$ correspond alors à $|a|^2$ et en 
    sortie $\ket{V}$ correspond à $|b|^2$. Pour accéder aux termes 
    d’interférence, comme $ab^*$, il faut réaliser des mesures dans des 
    bases complémentaires, telles que $\{\ket{D}, \ket{A}\}$ pour la 
    polarisation diagonale et antidiagonale, et/ou $\{\ket{R}, \ket{L}\}$ 
    pour la polarisation circulaire. Les différences d’intensité 
    observées dans ces diverses configurations de mesure projective 
    permettent de reconstruire les éléments de matrice densité. 
    
    \noindent En photonique quantique, cette matrice densité peut 
    également être exprimée en termes des paramètres de Stokes 
    \cite{Kwiat}. Ce derniers décrivent les paramètres de Stokes 
    décrivent complètement l’état 
    de polarisation et ils sont liés aux probabilités de détection dans 
    différentes bases de polarisation \cite{hecht2012optics}. 
    Les paramètres sont définis par:
    
    \begin{equation}
        S = \begin{pmatrix}
            S_0\\
            S_1\\
            S_2\\
            S_3
        \end{pmatrix}
        = \begin{pmatrix}
            P_{\ket{H}} + P_{\ket{V}}\\
            P_{\ket{H}} - P_{\ket{V}}\\
            P_{\ket{D}} - P_{\ket{A}}\\
            P_{\ket{R}} - P_{\ket{L}}
        \end{pmatrix}
    \end{equation}
    
    \noindent Où $P_{\ket{H}}$ la probabilité de détection pour l'état de 
    polarisation horizontale $\ket{H}$ et $P_{\ket{V}}$ la probabilité de 
    détection pour l'état de polarisation verticale $\ket{V}$. Ainsi, les 
    paramètres de Stokes: $S_0$ représente l’intensité ou probabilité 
    totale du faisceau, $S_1$ représente la différence de probabilités 
    entre les polarisations $\ket{H}$ et $\ket{V}$, $S_2$ représente la 
    différence de probabilités entre les polarisations 
    $\ket{D} \equiv \frac{1}{\sqrt{2}}(\ket{H} + \ket{V})$ et $\ket{A} \equiv \frac{1}{\sqrt{2}}(\ket{H} - {V})$ 
    et $S_3$ représente la différence de probabilités entre les polarisations 
    $\ket{R} \equiv \frac{1}{\sqrt{2}}(\ket{H}+i{V})$ et $\ket{L} \equiv \frac{1}{\sqrt{2}}(\ket{H} - i\ket{V})$. 
    En notant les probabilités de mesure pour chacune de ces bases, on 
    reconstruit la matrice densité à partir de:
    
    \begin{equation}
        \hat{\rho} = \frac{1}{2}\sum_{i=0}^{3} S_i \sigma_i
    \end{equation}
    
    \noindent Soit $\hat{\sigma}_i$ sont les matrices de Pauli définies 
    comme suit:
    
    \begin{equation}
        \hat{\sigma}_0 = \begin{pmatrix}
            1 & 0\\
            0 & 1
        \end{pmatrix}
        \hat{\sigma}_1 = \begin{pmatrix}
            1 & 0\\
            0 & -1
        \end{pmatrix}
        \hat{\sigma}_2 = \begin{pmatrix}
            0 & 1\\
            1 & 0
        \end{pmatrix}
        \hat{\sigma}_3 = \begin{pmatrix}
            0 & -i\\
            i & 0
        \end{pmatrix}
    \end{equation}

    \noindent Utilisant l'état arbitraire que nous avons mentionné, 
    trouvons la matrice densité avec les paramètres de Stokes. 
    Commençons par réécrire la matrice densité comme suit:

    \begin{equation}
        \hat{\rho} = \frac{1}{2}(S_0\hat{\sigma}_0 + S_1\hat{\sigma}_1 + S_2\hat{\sigma}_2 + S_3\hat{\sigma}_3)
    \end{equation}

    \noindent Ensuite, trouvons chacun des paramètres de Stokes en 
    projetant les différentes bases sur l'état de polarisation. Les deux 
    premiers paramètres $S_0$ et $S_1$ sont simples; trouvons les 
    probabilités $P_{\ket{H}}$ et $P_{\ket{V}}$.

    \begin{align}
        P_{\ket{H}} &= |\bra{H}\ket{\psi}|^2 = (a\bra{H}\ket{H} + b\bra{H}\ket{V})(a^*\bra{H}\ket{H} + b^*\bra{H}\ket{V})\\
        &= |a|^2\\
        P_{\ket{V}} &= |\bra{V}\ket{\psi}|^2 = (a\bra{V}\ket{H} + b\bra{V}\ket{V})(a^*\bra{V}\ket{H} + b^*\bra{V}\ket{V})\\
        &= |b|^2
    \end{align}

    \noindent Les paramètres de stokes $S_0$ et $S_1$ sont donc les 
    suivants:

    \begin{align}
        S_0 &= P_{\ket{H}} + P_{\ket{V}} = |a|^2 + |b|^2 = 1\\
        S_1 &= P_{\ket{H}} - P_{\ket{V}} = |a|^2 - |b|^2
    \end{align}

    \noindent Pour les deux paramètres suivants $S_2$ et $S_3$, nous 
    devons exprimer les états projetés dans nos états de base 
    $\{\ket{H},\ket{V}\}$. 

    \begin{align}
        P_{\ket{D}} &= |\bra{D}\ket{\psi}|^2 = \Biggl[\biggl(\frac{1}{\sqrt{2}}(a\bra{H}\ket{H} + a\bra{V}\ket{H})\biggr)\\ 
        &+ \biggl(\frac{1}{\sqrt{2}}(b\bra{H}\ket{V} + b\bra{V}\ket{V})\biggr)\Biggr]\Biggl[ \biggl(\frac{1}{\sqrt{2}}(a^*\bra{H}\ket{H} + a^*\bra{V}\ket{H})\biggr)\\
        &+ \biggl(\frac{1}{\sqrt{2}}(b^*\bra{H}\ket{V} + b^*\bra{V}\ket{V})\biggr) \Biggr] = \frac{1}{2}(a + b)(a^* + b^*)\\
        &= \frac{1}{2}(|a|^2 + ab^* + a^*b + |b|^2)\\
        P_{\ket{A}} &= |\bra{A}\ket{\psi}|^2 = \Biggl[\biggl(\frac{1}{\sqrt{2}}(a\bra{H}\ket{H} - a\bra{V}\ket{H})\biggr)\\ 
        &+ \biggl(\frac{1}{\sqrt{2}}(b\bra{H}\ket{V} - b\bra{V}\ket{V})\biggr)\Biggr]\Biggl[ \biggl(\frac{1}{\sqrt{2}}(a^*\bra{H}\ket{H} - a^*\bra{V}\ket{H})\biggr)\\
        &+ \biggl(\frac{1}{\sqrt{2}}(b^*\bra{H}\ket{V} - b^*\bra{V}\ket{V})\biggr) \Biggr] = \frac{1}{2}(a - b)(a^* - b^*)\\
        &= \frac{1}{2}(|a|^2 - ab^* - a^*b + |b|^2)\\
        S_2 &= P_{\ket{D}} - P_{\ket{A}} = ab^* + a^*b = 2\mathcal{R}(ab^*)
    \end{align} 

    \noindent On répète la même technique pour $S_3$:

    \begin{align}
        P_{\ket{R}} &= |\bra{R}\ket{\psi}|^2 = \Biggl[\biggl(\frac{1}{\sqrt{2}}(a\bra{H}\ket{H} + ia\bra{V}\ket{H})\biggr)\\ 
        &+ \biggl(\frac{1}{\sqrt{2}}(b\bra{H}\ket{V} + ib\bra{V}\ket{V})\biggr)\Biggr]\Biggl[ \biggl(\frac{1}{\sqrt{2}}(a^*\bra{H}\ket{H} + ia^*\bra{V}\ket{H})\biggr)\\
        &+ \biggl(\frac{1}{\sqrt{2}}(b^*\bra{H}\ket{V} + ib^*\bra{V}\ket{V})\biggr) \Biggr] = \frac{1}{2}(a + ib)(a^* + ib^*)\\
        &= \frac{1}{2}(|a|^2 + iab^* + ia^*b - |b|^2)\\
        P_{\ket{L}} &= |\bra{L}\ket{\psi}|^2 = \Biggl[\biggl(\frac{1}{\sqrt{2}}(a\bra{H}\ket{H} - ia\bra{V}\ket{H})\biggr)\\ 
        &+ \biggl(\frac{1}{\sqrt{2}}(b\bra{H}\ket{V} - ib\bra{V}\ket{V})\biggr)\Biggr]\Biggl[ \biggl(\frac{1}{\sqrt{2}}(a^*\bra{H}\ket{H} - ia^*\bra{V}\ket{H})\biggr)\\
        &+ \biggl(\frac{1}{\sqrt{2}}(b^*\bra{H}\ket{V} - ib^*\bra{V}\ket{V})\biggr) \Biggr] = \frac{1}{2}(a - ib)(a^* - ib^*)\\
        &= \frac{1}{2}(|a|^2 - iab^* - ia^*b - |b|^2)\\
        S_3 &= P_{\ket{R}} - P_{\ket{L}} = i(ab^* + a^*b) = 2\mathcal{I}(ab^*)
    \end{align}

    \noindent Ensuite, écrivons nous résultats dans notre matrice 
    densité:
    

    \begin{align}
        \hat{\rho} &= \frac{1}{2}(S_0\hat{\sigma}_0 + S_1\hat{\sigma}_1 + S_2\hat{\sigma}_2 + S_3\hat{\sigma}_3)\\
        &= \frac{1}{2}\begin{pmatrix}
            S_0 + S_1 & S_2 -S_3\\
            S_2 + S_3 & S_0 - S_1
        \end{pmatrix}\\
        &= \begin{pmatrix}
            |a|^2 & \mathcal{R}(ab^*) - i\mathcal{I}(ab^*)\\
            \mathcal{R}(a^*b) - i\mathcal{I}(a^*b) & |b|^2
        \end{pmatrix}\\
        &= \begin{pmatrix}
            |a|^2 & ab^*\\
            a^*b & |b|^2
        \end{pmatrix}
    \end{align}

    \noindent Nous avons maintenant reconstruit notre matrice de densité 
    à partir d'un état de polarisation arbitraire en utilisant les 
    paramètres de Stokes. 
    
    \noindent Pour un état pur, comme notre exemple, on a la propriété 
    $Tr(\hat{\rho^2}) = 1$, ce qui se traduit par une cohérence quantique 
    maximale. En revanche, un état mixte se caractérise par une matrice 
    densité statistique, qui est une somme pondérée d'états purs:

    \begin{equation}
        \hat{\rho}_{mixte} = \sum_{i}^{N} p_i\ket{\psi}_i\bra{\psi}_i
    \end{equation}


    \noindent Nous avons $N$ états, chacun étant associé à une 
    probabilité $p_i$, de sorte que $\sum_{i}^{N} p_i = 1 $. Chaque état 
    $\ket{\psi_i}$ correspond à un mélange statistique représenté par 
    une matrice densité mixte. Dans ce contexte, $Tr(\hat{\rho}^{2}_{mixte}) < 1 $. 
    Cela signifie que la pureté d’une matrice densité peut être mesurée 
    par sa trace. Un état pur possède une cohérence parfaite, tandis 
    qu’un état mixte résulte d’un mélange statistique d’états. Il est 
    aussi possible d'évaluer la pureté d’un état à l'aide des paramètres 
    de Stokes, avec la relation suivante (en supposant une normalisation 
    avec $S_0=1$): 

    \begin{equation}
        \sqrt{\sum_{i=1}^{3} S_i^2} = \text{DOP}
    \end{equation}

    \noindent Où \text{DOP}, le dégré de polarisation, peut être soit $1$ 
    pour un état pur (entièrement polarisé) ou strictement inférieur à $1$ 
    pour un état mixte (partiellement polarisé). Il est possible de 
    visualiser l’état de polarisation sur la sphère de Poincaré (figure 
    \ref{fig:spherepoincarre}) une représentation tridimensionnelle où 
    chaque axe correspond à un paramètre de Stokes, excluant ainsi $S_0$. 
    Chaque point sur cette surface représente un état distinct de 
    polarisation complète. 

    \begin{figure}[!h!t!p!b!]
        \centering
        \includegraphics[width=1.0\textwidth]{poincare_sphere.png}
        \caption{La sphère Poincarré}
        \label{fig:spherepoincarre}
    \end{figure}

    \noindent Enfin, les protocoles de la tomographie quantique 
    permettent de déterminer empiriquement ces coefficients de la matrice 
    densité, à partir des paramètres de Stokes, et peuvent donc 
    reconstruire et caractériser entièrement la matrice densité complète 
    d’un état de polarisation. 

    \noindent Toutefois, cette méthode présente plusieurs inconvénients: 
    elle est indirecte, complexe et nécessite un traitement algorithmique 
    intensif \cite{Lundeen_Direct_Measurement}, en particulier lorsque la 
    dimension de l’espace d’état est élevée. De plus, elle ne permet pas 
    d'accéder facilement à des éléments individuels de la matrice 
    densité, car elle se repose sur une reconstruction globale 
    \cite{Guilleaum}. Ces limitations rendent la tomographie peu adaptée 
    aux applications nécessitant un accès direct ou simultané à certains 
    paramètres de la matrice densitée, ou des mesures en temps réel
    \cite{TomoReview}.
\end{doublespace}

\subsection{Introduction aux mesures faibles}

\begin{doublespace}
    
    Une alternative intéressante consiste à utiliser des mesures faibles, 
    une méthode permettant d’accéder à la fonction d’onde d’un système 
    quantique directement. AAV ont proposé cette méthode dans leur 
    article \guillemetleft How the result of a measurement of a component 
    of the spin of a spin-1/2 particle can turn out to be 100 
    \guillemetright (Comment le résultat de la mesure de la composante 
    spin d’une particule ayant un spin-1/2 peut devenir 100) en 1988 
    \cite{Aharonov}. Cette méthode s’inspire du modèle de von Neumann 
    \cite{vonNeumann}, dans lequel un système faiblement lié à un 
    \guillemetleft pointeur \guillemetright subit une interaction 
    (perturbation) faible. La mesure du résultat est représentée par un 
    déplacement du pointeur proportionnel à ce que l’on appelle la 
    \guillemetleft valeur faible \guillemetright. 

    \noindent Le modèle von Neumann des mesures quantiques sert de 
    fondement théorique pour comprendre les mesures faibles. Dans ce 
    modèle, le système quantique et ce qu’on appelle un \guillemetleft 
    pointeur \guillemetright (nommé en référence à l’aiguille d’un 
    instrument de mesure \cite{vonNeumann}) sont entremêlés par un 
    opérateur d’interaction faible, permettant ainsi d’extraire des 
    informations sur la fonction d’onde. Le pointeur indique l'état de la 
    mesure moyenne de l'appareil de mesure \cite{vonNeumann}. Voici un 
    schéma (figure \ref{fig:neumann}) illustrant l’utilisation du modèle 
    de Von Neumann dans le cadre des mesures à faible intensité.

    \begin{figure}[!htbp]
        \centering
        \includegraphics[width=1.0\textwidth,page=2]{FIGURES.pdf} % Selects page 2
        \caption{Une voiture de Formule 1 rapide peut être décrite comme un système faiblement couplé au départ, avec l’indicateur de vitesse comme pointeur. On peut imaginer que ce couplage est rompu lorsque le conducteur relâche le frein en même temps que l'accélérateur. Une fois que le conducteur interagit avec le système, le pointeur est déplacé.}
        \label{fig:neumann}
    \end{figure}

    \noindent Contrairement aux interactions fortes, qui provoquent un 
    effondrement complet de la fonction d'onde et détruisent la 
    superposition quantique des états de base, une interaction faible 
    préserve cette superposition en minimisant la perturbation du système. 
    Pour illustrer la différence entre une interaction forte et faible, 
    on peut représenter celle-ci par un pointeur ayant une forme 
    gaussienne couplée à un état $\ket{\psi}$ où les états de base sont 
    soit fortement, soit faiblement séparés. La figure 
    \ref{fig:interaction} permettrait de clarifier comment les mesures 
    faibles minimisent l’interaction tout en extrayant des informations 
    précises.  

    \begin{figure}[!htbp]
        \centering
        \includegraphics[width=1.0\textwidth,page=3]{FIGURES.pdf} % Selects page 2
        \caption{Considérons qu'un pointeur possède une forme gaussienne 
        avec une position moyenne, démontré par la ligne pointillée, avec 
        une distribution de probabilité $\sigma$ dans l’état 
        $\ket{\psi} = a\ket{H} + b\ket{V}$ et un coefficient d’interaction 
        $\delta$ qui décrivent la force de séparation des états. 
        a) Interaction forte : une interaction forte impliquerait un 
        effondrement complet de l’état séparant complètement les états de 
        base. Cela se produirait lorsque $\delta \gg \sigma$. Par conséquent, 
        nous mesurerions l’un ou l’autre via une mesure projective. 
        b) Interaction faible : Elle consiste en une faible interaction avec 
        le système qui permet aux deux états de base de se chevaucher, de 
        sorte que, lors d’une mesure projective, nous obtenions en retour un 
        état qui comprend essentiellement l’état initial du système. Cela se 
        produit lorsque $\delta \ll \sigma$.}
        \label{fig:interaction}
    \end{figure}

    \noindent Dans le contexte des mesures faibles, la force de l’interaction, 
    soit $\delta$, est choisie pour que le déplacement du pointeur soit 
    inférieur à la largeur de la distribution des probabilités. Cette 
    méthode permet ainsi de mesurer directement le déplacement du pointeur 
    après une mesure projective et d’en obtenir la valeur faible.
    
    \begin{equation}
        \hat{S}_W = \frac{\bra{\psi_f}\hat{S}\ket{\psi_i}}{\bra{\psi_f}\ket{\psi_i}}
    \end{equation}

    \noindent La valeur faible $\hat{S}_W$, pour soit un observable 
    $\hat{S}$ du système, un état d'entrée $\ket{\psi_i}$ et l'état d'une 
    mesure projective $\ket{\psi_f}$, issue d’une mesure faible, est une 
    variable complexe composée d'une partie réelle et imaginaire 
    \cite{Aharonov,Lundeen_Resch}. Ces composantes renferment des 
    informations sur l'observable de la variable du pointeur $\hat{p}$ 
    ainsi que sur sa variable conjuguée $\hat{q}$, permettant une 
    caractérisation complète \cite{Lundeen_Bamber}.

    \begin{equation}
        \hat{S}_W = \frac{1}{\delta}\Biggl( \expval{\hat{p}} + i4\sigma^2 \expval{\hat{q}} \Biggr)
    \end{equation}

    \noindent Les mesures faibles servent d’œil de Judas au monde 
    quantique \cite{Peephole}. Cela nous permet de perturber le système le 
    moins possible pour obtenir de l’information sur le système quantique. 
    L’adoption des mesures faibles repose sur plusieurs avantages clés : 
    elles réduisent les perturbations induites sur le système, préservent 
    la cohérence quantique et permettent une approche directe et 
    intuitive pour caractériser des états quantiques \cite{ApplicationWeak}.

\end{doublespace}

\subsection{Fondamentaux théoriques des mesures faibles}

\begin{doublespace}
    Tout d’abord, nous aborderons les principes théoriques sous-jacents 
    aux mesures faibles, en expliquant comment leur valeur est liée à la 
    fonction d’onde de l’état quantique et peut être mesurée directement. 
    Ces mesures sont une étape clé dans la procédure décrite par l’AAV 
    \cite{Aharonov} et dans \cite{Lundeen_Resch}, qui se compose des 
    éléments suivants. Voici les étapes de la procédure à suivre:

    \begin{itemize}
        \item Préparation de l’état initial
        \item Interaction faible, sujet de discussion dans la présente section
        \item Mesure projective (postsélection), souvent réalisée sur un état possédant une quantité égale des deux états de base de l’état initial.
    \end{itemize}

    \noindent Dans cette section, nous nous attarderons sur l’étape de 
    l’interaction faible, qui correspond à une perturbation faible de 
    l’état quantique. Comme cela a été mentionné précédemment, cette 
    procédure directe via des mesures faibles s’appuie sur le modèle de 
    von Neumann pour les mesures quantiques. Ce modèle implique un 
    système quantique composé de deux objets : le système à mesurer $S$ 
    et l’appareil de mesure (pointeur) $P$. Ces deux objets sont traités 
    comme des objets de la mécanique quantique couplés dans un système 
    total $T$ \cite{Hairiri,vonNeumann}. Lorsqu'un état quantique est 
    soumis à une mesure, il se trouve initialement dans un état superposé 
    inconnu, noté $\ket{\psi}_S = \sum_{j}^{N}c_{j}\ket{s_j}_S$, dont les 
    composantes sont des combinaisons linéaires de bases propres 
    $\ket{s_j}_S$, d'un observable du système $\hat{S}$, avec des valeurs 
    propres $s_j$ (avec coefficients complexes) et entraine un déplacement 
    du pointeur en fonction de la force d’interaction $\delta$ et de la 
    valeur propre $s_j$, $\Delta p =\delta s_j$ appelée la valeur moyenne 
    faible \cite{Lundeen_Bamber,Hairiri,vonNeumann}. 

    \noindent Cette réduction de l’état quantique se traduit par une 
    translation de la position du pointeur, initialement dans l’état 
    $\ket{\xi}_P =\ket{\bar{p} = 0}_P$ dans la base $P$, où $\bar{q}$ 
    représente la valeur moyenne d’une variable $p$ avec une distribution 
    de probabilité $\sigma$ passe de sa position initiale à 
    $\ket{\bar{p} =\delta s_j}_P$. La quantification de cette interaction 
    se définit à travers un opérateur d’interaction, communément désigné 
    sous le nom d’opérateur d’interaction de von Neumann. Ce dernier est 
    exprimé comme suit :

    \begin{equation}
        \hat{U} \equiv exp\Bigl( -\frac{i\mathcal{H}t}{\hbar} \Bigr)
    \end{equation}

    \noindent Il s'agit d'un opérateur d'évolution temporelle, où $t$ 
    représente le temps d'interaction avec le système, $\hbar$ la 
    constante de Planck et $\mathcal{H}$ le hamiltonien du système total 
    $T$, qui est défini comme suit :

    \begin{equation}
        \mathcal{H} \equiv g(\hat{S} \otimes \hat{q})
    \end{equation}

    \noindent Soit $g$ la constante de couplage, qui doit être réelle 
    pour que le hamiltonien soit hermitien, et $\hat{q}$ la variable 
    pointeur conjuguée de l'observable $\hat{S}$. Le pointeur et l’état 
    mesuré sont étroitement liés dans un état décrivant l’ensemble du 
    système $T$, initialement écrit sous la forme :

    \begin{equation}
        \ket{\Psi^i}_T = \ket{\psi}_S \otimes \ket{\bar{p}=0}_P    
    \end{equation}

    \noindent Après la mesure, le pointeur se déplace en fonction de la 
    force d’interaction $\delta\equiv\frac{gt}{\hbar}$. Lorsque combiné, 
    l'état du système évolue comme suit :

    \begin{align}
        \ket{\Psi^f}_T &= \hat{U} \Bigl[\ket{\psi}_S \otimes \ket{\bar{p} = 0}_P\Bigr]\\
        &= \sum_{j}^{N} c_{j}\ket{s_j}_S \otimes \ket{\bar{p} = \delta s_j}_P
    \end{align}

    \noindent Dans un régime de mesures faibles, où l’interaction est 
    plus faible que la distribution de probabilité du pointeur, 
    c’est-à-dire $\delta \ll \sigma$, le système mesuré est faiblement 
    couplé avec le pointeur, entraînant ainsi un effondrement minimal 
    préservant la superposition. La mesure de $\hat{S}$ par la suite 
    déplace légèrement $\hat{p}$ \cite{Hairiri,Lundeen_Resch}. Considérons 
    ce qui suit:
    
    \begin{equation}
        \hat{U}\ket{\Psi^{i}}_T = \hat{U}\Bigl[ \ket{\psi}_S \otimes \ket{\bar{p} = 0}_P \Bigr]
    \end{equation}

    \noindent Examinons une étude plus approfondie du système initial 
    total qui subit une interaction faible. Réécrivons l'opérateur 
    d'interaction de von Neumann sous la forme d'une série de Taylor.

    \begin{align}
        \hat{U}\ket{\Psi^{i}}_T &= \hat{U}\Bigl[ \ket{\psi}_S \otimes \ket{\bar{p} = 0}_P \Bigr]\\
        &= e^{-i\delta(\hat{S} \otimes \hat{q})}\Bigl[ \ket{\psi}_S \otimes \ket{\bar{p} = 0}_P \Bigr]\\
        &= \Bigl( 1 - i\delta(\hat{S} \otimes \hat{q}) + \mathcal{O}(\delta^2) \Bigr)\Bigl[ \ket{\psi}_S \otimes \ket{\bar{p} = 0}_P \Bigr]\\
        &=  \ket{\psi}_S \otimes \ket{\bar{p} = 0}_P - i\delta\hat{S}\ket{\psi}_S\otimes\hat{q}\ket{\bar{p} = 0}_P + \mathcal{O}(\delta^2)\ket{\psi}_S \otimes \ket{\bar{p}=0}_P
    \end{align}

    \noindent Où $\mathcal{O}(\delta^2)$ correspond à les ordres plus 
    élevés de la série de Taylor, que nous négligeons. En suivant la 
    procédure de mesure faible, nous projetterons une mesure projective 
    ultérieure sur le système avec l’état $\ket{\varphi}_S$, qui a les 
    mêmes états de base que $\ket{\psi}_S$. 

    \begin{align}
        \ket{\varphi}_S\bra{\varphi}_S\hat{U}\ket{\Psi^{i}}_T &= \ket{\varphi}_S\bra{\varphi}_S\ket{\psi}_S \otimes \ket{\bar{p} = 0}_P - i\delta\ket{\varphi}_S\bra{\varphi}_S\hat{S}\ket{\psi}_S\otimes\hat{p}\ket{\bar{p} = 0}_P
    \end{align}

    \noindent Normalisent l’état du système total avec le module de la 
    probabilité $\bra{\varphi}_S\ket{\psi}_S =\sqrt{Prob}$, dont 
    $Prob\equiv |\bra{\varphi}_S\ket{\psi}_S|^2$ \cite{Lundeen_Resch,Lundeen_thesis, Steinberg_prob_div}.

    \begin{align}
        \ket{\varphi}_S\frac{\bra{\varphi}_S\hat{U}\ket{\Psi^{i}}_T}{\bra{\varphi}_S\ket{\psi}_S} &= \ket{\varphi}_S\frac{\bra{\varphi}_S\ket{\psi}_S}{\bra{\varphi}_S\ket{\psi}_S} \otimes \ket{\bar{p} = 0}_P - i\delta\ket{\varphi}_S\frac{\bra{\varphi}_S\hat{S}\ket{\psi}_S}{\bra{\varphi}_S\ket{\psi}_S}\otimes\hat{q}\ket{\bar{p} = 0}_P\\
        &= \ket{\varphi}_S \otimes \ket{\bar{p} = 0}_P - i\delta\ket{\varphi}_S\frac{\bra{\varphi}_S\hat{S}\ket{\psi}_S}{\bra{\varphi}_S\ket{\psi}_S}\otimes\hat{q}\ket{\bar{p} = 0}_P
    \end{align}

    \noindent En ce cas, le $\frac{\bra{\varphi}_S\ket{\psi}_S}{\bra{\varphi}_S\ket{\psi}_S}$ 
    sur le côté droit est annulé et nous déplaçons le $\frac{1}{\bra{\varphi}_S\ket{\psi}_S}$ 
    du côté gauche vers le côté droit. L'état final est maintenant le 
    suivant :

    \begin{align}
        \ket{\Psi^f}_T &\equiv \ket{\varphi}_S\bra{\varphi}_S\hat{U}\ket{\psi}_S\\
        &\simeq \bra{\varphi}_S\ket{\psi}_S \Bigl[ \ket{\bar{p} = 0}_P - i\delta\frac{\bra{\varphi}_S\hat{S}\ket{\psi}_S}{\bra{\varphi}_S\ket{\psi}_S} \hat{q}\ket{\bar{p}=0}_P \Bigr] \otimes \ket{\varphi}_S
    \end{align}

    \noindent Les parenthèses carrées représentent l’état final du 
    pointeur, c'est-à-dire le déplacement du pointeur après l’interaction
    faible s'écrivant comme suit:
    %ce qui nous permet de calculer les parties réelles et 
    %imaginaires de $\hat{S}$. 

    \begin{equation}
        \ket{\bar{p} = \delta s_j}_P \equiv \ket{\bar{p} = 0}_P - i\delta\frac{\bra{\varphi}_S\hat{S}\ket{\psi}_S}{\bra{\varphi}_S\ket{\psi}_S} \hat{q}\ket{\bar{p}=0}_P
    \end{equation}

    \noindent Observez que la position finale du pointeur est 
    proportionnelle à ce qui suit :

    \begin{equation}
        \hat{S}_W \equiv \frac{\bra{\varphi}_S\hat{S}\ket{\psi}_S}{\bra{\varphi}_S\ket{\psi}_S}
    \end{equation}

    \noindent Il s’agit de la valeur faible dérivée pour la première 
    fois par AAV, une valeur complexe composée d’une partie réelle et 
    d’une partie imaginaire. Celle-ci correspond au décalage de la 
    variable du pointeur $p$ et à son décalage par rapport à sa variable 
    conjuguée $q$. En d’autres termes, si une particule présente un 
    décalage dans sa position, il y aura également un décalage dans sa 
    quantité de mouvement. Par exemple, le temps d'arrivée d’un photon et 
    sa fréquence centrale varieront l’une par rapport à l’autre lors 
    d’une interaction. Si l’interaction est faible, il est possible de 
    mesurer ces valeurs décalées individuellement lors d’une procédure 
    directe via une mesure faible \cite{Hairiri,Lundeen_Resch,Lundeen_Direct_Measurement,Guilleaum}. 
    Pour conclure, écrivons l’état final avec cette valeur :

    \begin{align}
        \ket{\Psi^f}_T &= \bra{\varphi}_S\ket{\psi}_S \Bigl[ \ket{\bar{p} = 0}_P - i\delta \hat{S}_W \hat{q}\ket{\bar{p}=0}_P \Bigr] \otimes \ket{\varphi}_S\\
        &= \bra{\varphi}_S\ket{\psi}_S\Bigl[ 1 - i\delta \hat{S}_W\hat{p} \Bigr] \ket{\bar{p} = 0} \otimes \ket{\varphi}_S\\
        &= \bra{\varphi}_S\ket{\psi}_S e^{-i\delta \hat{S}_W\hat{q}} \ket{\bar{p} = 0} \otimes \ket{\varphi}_S\\
        &= \ket{\psi}_S e^{-i\delta \hat{S}_W\hat{q}}\ket{\bar{p}=0}
    \end{align}

    \noindent C’est-à-dire que si nous avons une mesure faible parfaite, 
    en prenant la limite que $\delta \to 0 $, nous avons essentiellement 
    l’état initial. Nous pouvons même mesurer et caractériser l'état 
    quantique directement sans aucune reconstruction algorithmique à 
    l'aide des parties réelles et imaginaires de la valeur faible. 
    Démontrons cela \cite{Lundeen_thesis,Lundeen_Resch}:
    
    \begin{align}
        \bra{\bar{p} = \delta s_j}\hat{p}\ket{\bar{p}=\delta s_j} &= -i\delta \mathcal{R}\Bigl(\hat{S}_W\Bigr)\bra{\bar{p}=\delta s_j}(\hat{p}\hat{q} - \hat{q}\hat{p})\ket{\bar{p}=\delta s_j}\\
        &+ \delta \mathcal{I}\Bigl(\hat{S}_W\Bigr)\bra{\bar{p} = \delta s_j}(\hat{p}\hat{q} + \hat{q}\hat{p})\ket{\bar{p}=\delta s_j}\\
        &= \delta\mathcal{R}\Bigl(\hat{S}_W\Bigr) = \expval{\hat{p}}
    \end{align}

    \noindent Ainsi, pour la variable conjuguée:

    \begin{align}
        \bra{\bar{q} = \delta s_j}\hat{q}\ket{\bar{P}=\delta s_j} &= -i\delta \mathcal{R}\Bigl(\hat{S}_W\Bigr)\bra{\bar{p}=\delta s_j}(\hat{q}^2 - \hat{q}^2)\ket{\bar{p}=\delta s_j}\\
        &+ \delta \mathcal{I}\Bigl(\hat{S}_W\Bigr)\bra{\bar{p} = \delta s_j}(\hat{q}^2 + \hat{q}^2)\ket{\bar{p}=\delta s_j}\\
        &= \frac{\delta}{4\sigma^2}\mathcal{I}\Bigl(\hat{S}_W\Bigr)
    \end{align}

    \noindent Ensemble, la valeur faible s'écrit:

    \begin{equation}
        \hat{S}_W = \frac{1}{\delta}\Bigl( \expval{\hat{p}} + i4\sigma^2\expval{\hat{q}} \Bigr)
    \end{equation}

    \noindent En démontrant que la valeur faible est proportionnelle à 
    la fonction d’onde (l'état quantique), comme l’a fait dans 
    \cite{Lundeen_Resch}, et qu’on peut en déduire directement les 
    paramètres, on a ouvert un tout nouveau domaine dans les mesures 
    quantiques et une alternative à la tomographie quantique 
    traditionnelle.

\end{doublespace}

\subsection{Proposition d'une procédure directe avec une mesure faible temporelle}

    
\begin{doublespace}
    Les mesures faibles temporelles exploitent les propriétés temporelles 
    et fréquentielles d’une impulsion lumineuse pour caractériser un état 
    quantique. Cette approche est fondée sur l’hypothèse selon laquelle 
    les délais temporels peuvent être directement liés aux composantes 
    réelles et imaginaires de la valeur faible. Dans les sections 
    suivantes et dans ce projet de thèse, nous nous concentrerons sur la 
    mesure de la valeur faible à partir d’une faible interaction 
    temporelle. Nous effectuerons des mesures faibles répétées sur un 
    grand ensemble d’impulsions identiquement préparées dans un système 
    photonique quantique. Ces mesures nous permettront de déterminer la 
    position temporelle moyenne d’une impulsion gaussienne, utilisée 
    comme pointeur, ainsi que l’effet sur sa variable conjuguée, un 
    décalage fréquentiel variant avec la valeur faible. Ce faisant, nous 
    pourrons caractériser l’état de polarisation du système.
\end{doublespace}

\subsubsection{La partie réelle du système}
    
\begin{doublespace}
    Nous voulons caractériser l'état de polarisation avec une mesure 
    faible temporelle. Pour ce faire, il est nécessaire de calculer 
    l’attendue des composantes de la valeur faible $\hat{\pi}_W$ du 
    système que nous allons implémenter. Pour y parvenir, nous devons 
    d’abord analyser chaque composante de cette valeur. Tout d’abord, 
    examinons la partie réelle en définissant les paramètres de 
    l’expérience potentielle que nous souhaitons éventuellement réaliser. 
    L’état de polarisation du système que nous souhaitons mesurer est 
    défini comme suit :

    \begin{equation}
        \ket{\psi} \equiv a\ket{H} + b\ket{V}
    \end{equation}

    \noindent Où $a$ et $b$ les amplitudes de probabilité des bases 
    $\{ \ket{H}$ , $\ket{V} \}$, respectivement, avec 
    $|a|^2 + |b|^2=1 $, ainsi que $\ket{H}$ et $\ket{V}$ représentent la 
    polarisation horizontale et verticale d’un photon.
    
    \begin{equation}
        \ket{\xi(t)} = \bra{t}\ket{\xi} \equiv \frac{1}{(\sqrt{2\pi}\sigma)^{1/2}}e^{-\frac{t^2}{4\sigma^2}}
    \end{equation}

    \noindent Le pointeur du système utilisé possède un profil temporel 
    gaussien pour un faisceau, caractérisé par une position temporelle 
    moyenne $t$ et une dispersion $\sigma$. L’état initial total s’écrit:

    \begin{equation}
        \ket{\Psi(t)^i} \equiv \ket{\psi} \otimes \ket{\xi(t)}
    \end{equation}


    \noindent Procédons à une faible interaction temporelle sur l’état 
    $\ket{H}$ avec l’opérateur de von Neumann $\hat{U}^H$, dont l’indice 
    $H$ indique qu’il translate la composante horizontale. L’opérateur 
    d’interaction de von Neumann peut être étudié sous la forme 
    $\hat{U} = exp(-\frac{i}{\hbar}\int \mathcal{\hat{H}}dt)$, où 
    $\mathcal{\hat{H}} \equiv g(\hat{\pi}\otimes\hat{E})$, avec 
    $\hat{\pi}$ l'observable du système (l'état de polarisation) et 
    $\hat{E}$ la variable conjuguée du pointeur. Nous voulons appliquer 
    l’opérateur sur un état de base unique, donc l’observable du système 
    $\hat{\pi}$ devient $\ket{J}\bra{J}$, où $J \equiv H,V$. Lorsque nous 
    appliquons l’opérateur, nous obtenons une intéraction sur la partie 
    $\ket{H}$ donnent son observale \cite{Lundeen_Bamber}. Le conjugué de 
    la variable de pointeur temporel est l'énergie $\hat{E}$, puisque, en 
    mécanique quantique, on sait comment le temps et l’énergie se 
    comportent réciproquement \cite{Peebles,Griffiths}. L’opérateur 
    d’énergie s’écrit ainsi :

    \begin{equation}
        \hat{E} = i\hbar\frac{\partial}{\partial t}
    \end{equation}

    \noindent L’interaction de von Neumann pour une interaction 
    temporelle s’écrit alors comme ceci :

    \begin{equation}
        \hat{U}^H = exp\Bigl(-i\tau\ket{H}\bra{H} \otimes \frac{\partial}{\partial t}\Bigr)
    \end{equation}

    \noindent En supposant que la constante de couplage $g$ soit faible, 
    sur un temps d’interaction $\tau$, nous l’appliquons à la partie 
    horizontale de l’état $\ket{H}$, ce qui entraine un décalage du 
    pointeur $exp\Bigl( -\tau \frac{\partial }{\partial t} \Bigr) \xi(t) = \xi(t-\tau)$. 
    L’état évolue ensuite de cette manière:

    \begin{align}
        \hat{U^H}\ket{\Psi(t)^i} &= \hat{U}^H \Bigl[ \ket{\psi} \otimes \ket{\xi(t)} \Bigr]\\
        &= \hat{U}^H \Bigl[ a\ket{H} \otimes \ket{\xi(t)} + b\ket{V} \otimes \ket{\xi(t)}\Bigr]\\
        &= a\ket{H} \otimes \hat{U}^H \ket{\xi(t)} + b\ket{V} \otimes \ket{\xi(t)}\\
        &= a\ket{H} \otimes \ket{\xi(t-\tau)} + b\ket{V} \otimes \ket{\xi(t)}
    \end{align}

    \noindent Lorsque nous interagissons avec un système quantique pour 
    en extraire des informations, nous réalisons une mesure projective à 
    l’aide de l’état superposé $\ket{\varsigma} = \mu\ket{H} + \nu\ket{V}$. 
    Les paramètres $\nu$ et $\mu$ représentent les amplitudes de 
    probabilités respectives des états $\ket{H}$ et $\ket{V}$ de l'état 
    de projection, dont $|\mu|^2 + |\nu|^2 = 1$. L'état s'écrit:
    
    \begin{align}
        \ket{\Psi(t)^f} &= \ket{\varsigma}\bra{\varsigma}\hat{U}^H\ket{\Psi(t)^i} = \Bigl[ \bar{\mu}\bra{H} + \bar{\nu}\bra{V} \Bigr]a\ket{H} \otimes \ket{\xi(t-\tau)} + b\ket{V} \otimes \ket{\xi(t)}\\
        &= \Bigl[\bar{\mu}a\ket{\xi(t-\tau)} + \bar{\nu}b\ket{\xi(t)}\Bigr] \otimes \ket{\varsigma}\\
        &= F(t)\otimes\ket{\varsigma}
    \end{align}

    \noindent Où $F(t) \equiv A\ket{\xi(t-\tau)} + B\ket{\xi(t)}$, 
    $A \equiv a\bar{\mu}$ et $B \equiv b\bar{\nu}$. Trouvons la valeur 
    moyenne de la position temporelle du pointeur $\expval{\hat{t}}$.

    \begin{align}
        \expval{\hat{t}} &= \bra{\Psi(t)^f}\hat{t}\ket{\Psi(t)^f}\\
        &= \int_{-\infty}^{\infty} I(t)tdt
    \end{align}

    \noindent Où $I(t) \equiv |F(t)|^2 $, nous pouvons le normaliser en 
    divisant par $\frac{1}{\bra{\Psi(t)^f}\ket{\Psi(t)^f}}$ :

    \begin{align}
        \expval{\hat{t}^{norm}} &= \frac{\bra{\Psi(t)^f}\hat{t}\ket{\Psi(t)^f}}{\bra{\Psi(t)^f}\ket{\Psi(t)^f}} = \frac{\int_{-\infty}^{\infty} I(t)tdt}{\int_{-\infty}^{\infty} I(t)dt}\\
        &= \frac{\int_{-\infty}^{\infty} |A|^2\Xi(t - \tau)t + |B|^2\Xi(t)t + A\bar{B}\Xi(t, \tau)t + \bar{A}B\Xi(t, \tau)t dt}{\int_{-\infty}^{\infty} |A|^2\Xi(t - \tau) + |B|^2\Xi(t) + A\bar{B}\Xi(t, \tau) + \bar{A}B\Xi(t, \tau) dt}
    \end{align}

    \noindent Où $\Xi(t) \equiv \frac{1}{\sqrt{2\pi}\sigma}e^{-\frac{t^2}{2\sigma^2}}$ 
    et $\Xi(t, \tau) \equiv \frac{1}{\sqrt{2\pi}\sigma}e^{-\frac{2t^2 - 2t\tau + \tau^2}{4\sigma^2}}$. 
    Remarquons qu’en raison d’une interaction faible avec le système, il 
    existe une superposition entre le pointeur et les composantes de 
    polarisation horizontale et verticale. Les solutions de chaque 
    intégrale sont énumérées ci-dessous. Nous allons reprendre notre 
    analyse de la partie réelle de la valeur faible.

    \begin{align*}
        \int_{-\infty}^{\infty} \Xi(t - \tau)t dt &= \tau & \int_{-\infty}^{\infty} \Xi(t) dt &= 1\\
        \int_{-\infty}^{\infty} \Xi(t - \tau) dt &= 1 & \int_{-\infty}^{\infty} \Xi(t, \tau)t dt &= \frac{\tau}{2}e^{-\frac{t^2}{8\sigma^2}}\\
        \int_{-\infty}^{\infty} \Xi(t)t dt &= 0 & \int_{-\infty}^{\infty} \Xi(t, \tau) dt &= e^{-\frac{t^2}{8\sigma^2}}
    \end{align*}

    \noindent Donc, avec ces solutions, la partie réelle se trouve:

    \begin{align}
        \expval{\hat{t}^{norm}} = \tau\frac{|A|^2 + (A\bar{B} + \bar{A}B)e^{-\frac{\tau^2}{8\sigma^2}}}{|A|^2 + |B|^2 + (A\bar{B} + \bar{A}B)e^{-\frac{\tau^2}{8\sigma^2}}}
        \label{eq:expval_t_norm}
    \end{align}

   \noindent Comme nous sommes dans le régime des mesures faibles, 
   prenons la limite où $\tau \ll \sigma$:

    \begin{align}
        \lim_{\frac{\tau}{\sigma} \to 0} \expval{\hat{t}^{norm}} &= \tau\frac{|A|^2 + A\bar{B} + \bar{A}B}{|A|^2 + |B|^2 + A\bar{B} + \bar{A}B}\\
        &\equiv \mathcal{R}(\expval{\hat{\pi}_W})
    \end{align}

    \noindent Ce terme représente la position moyenne temporelle du 
    pointeur lors d’une mesure. Il s’agit de la partie réelle de la 
    valeur faible $\expval{\hat{\pi}_W}$.
\end{doublespace}

\subsubsection{La partie imaginaire du système}
    
\begin{doublespace}
    Comme nous l'avons déjà évoqué, un déplacement de la variable du 
    pointeur, tels que sa position temporelle $t$ par rapport à un $t_0$, 
    devrait entraîner un déplacement de sa position fréquentielle $\omega$, 
    vue que $\hat{E} = \hbar\hat{\omega}$ \cite{Peebles,Griffiths}. 
    Vérifions-le en calculant la partie imaginaire de la valeur faible 
    $\expval{\hat{\pi}_W}$. Tout d’abord, effectuons la transformation de 
    Fourier de la fonction temporelle $F(t)$ de l'état quantique 
    $\ket{\Psi(t)^f}$:

    \begin{align}
        F(\omega) &= \frac{1}{\sqrt{2\pi}}\int_{-\infty}^{\infty} F(t)e^{-i\omega t}dt\\
        &= \frac{\sqrt[4]{2}\sqrt{\sigma}}{\sqrt[4]{\pi}}(A + Be^{i\omega\tau})e^{-\omega^2 \sigma^2 - i\omega\tau}
    \end{align}

    \noindent  Avec ce dernier, l’état quantique s’écrit :

    \begin{equation}
        \ket{\Psi(\omega)^f} = F(\omega) \otimes \ket{\varsigma}
    \end{equation}

    \noindent Ensuite, déterminons la valeur moyenne de la position 
    fréquentielle en suivant les mêmes étapes que celles 
    pour la partie réelle :
    
    \begin{align}
        \expval{\hat{\omega}} &= \bra{\Psi(\omega)^f}\hat{\omega}\ket{\Psi(\omega)^f}\\
        &= \int_{-\infty}^{\infty} S(\omega) d\omega
    \end{align}

    \noindent Où $S(\omega) \equiv |F(\omega)|^2$ et normalisons cette 
    valeur:

    \begin{align}
        \expval{\hat{\omega}^{norm}} &= \frac{\sqrt{2}\sigma}{\sqrt{\pi}}\frac{\int_{-\infty}^{\infty} |A|^2\omega e^{i\omega\tau} + |B|^2\omega e^{i\omega\tau} + A\bar{B}\omega + \bar{A}Be^{2i\omega\tau} \omega d\omega}{\int_{-\infty}^{\infty} |A|^2 e^{i\omega\tau} + |B|^2 e^{i\omega\tau} + A\bar{B} + \bar{A}Be^{2i\omega\tau}d\omega}e^{-2\omega^2 \sigma^2 -i\omega\tau}
    \end{align}

    \noindent En utilisant des méthodes d’intégration similaires, nous 
    arrivons à :

    \begin{equation}
        \expval{\hat{\omega}^{norm}} = \frac{i\tau}{4\sigma^2}\frac{(B\bar{A} - A\bar{B})e^{-\frac{\tau^2}{8\sigma^2}}}{|A|^2 + |B|^2 + \bar{A}B + A\bar{B}}
    \end{equation}

    \noindent Prenons encore la limite dont $\tau \ll \sigma$, qui 
    s’applique au domaine des mesures faibles:

    \begin{align}
        \lim_{\frac{\tau}{\sigma} \to 0}\expval{\hat{\omega}^{norm}} &= \frac{i\tau}{4\sigma^2}\frac{B\bar{A} - A\bar{B}}{|A|^2 + |B|^2 + \bar{A}B + A\bar{B}}\\
        &\equiv \mathcal{I}(\expval{\hat{\pi}_W})
    \end{align}

    \noindent Ce terme correspond à la partie imaginaire de la valeur 
    faible attendue $\expval{\hat{\pi}_W}$.

\end{doublespace}

\subsection{Proposition expérimentale pour la caractérisation de la valeur faible}
    
\begin{doublespace}
    Nous continuons avec notre système photonique quantique, en nous 
    appuyant sur nos résulats théoriques concernant la partie réelle et 
    imaginaire de la valeur faible. Nous pouvons calculer que, pour un 
    état d’entrée, soit :

    \begin{equation}
        \ket{\psi^{in}} = a\ket{H} + b\ket{V} 
    \end{equation}

     \noindent Puisque $a$ et $b$ sont des amplitudes de probabilité 
     pour les états de base $\ket{H}$ et $\ket{V}$ respectivement 
     (c'est-à-dire $a=\bra{H}\ket{\psi^{in}}$ et $b=\bra{V}\ket{\psi^{in}}$), 
     on peut, en pratique, calculer directement les amplitudes de 
     probabilités en fonction des parties de la valeur mesurée. Cette 
     possibilité découle du fait que la valeur faible est proportionnelle 
     à l'état quantique, comme nous l'avons démontré. Pour caractériser 
     l'état de polarisation d'un système quantique, il s'agit de mesurer 
     faiblement $\hat{\pi}^J = \ket{J}\bra{J}$ soit $J= H,V$ \cite{Hairiri,Lundeen_Direct_Measurement,Lundeen_Bamber}, 
     puis de mesurer par projection sur un état intermédiaire tel que 
     $\ket{D} = \frac{1}{\sqrt{2}}(\ket{H}+\ket{V})$. Donc cette opération 
     réussie, elle permettra d’obtenir un ensemble restreint d’essais 
     dont la moyenne des résultats expérimentaux (tels que les 
     déplacements temporels ou fréquentiels du pointeur) sera 
     proportionnelle à la partie réelle ou imaginaire de la valeur 
     faible.
    
     \begin{equation}
        \hat{\pi}_{W}^{J} = \frac{\bra{D}\hat{\pi}^{J}\ket{\psi^{in}}}{\bra{D}\ket{\psi^{in}}} = \sqrt{N}\bra{J}\ket{\psi^{in}}
    \end{equation}

    \noindent Où $N$ est une constante de normalisation indépendante de 
    $J$. On peut écrire l’état quantique en fonction de la valeur faible. 

    \begin{equation}
        \ket{\psi^{in}} = \frac{1}{\sqrt{N}}\Bigl(\hat{\pi}^{H}_{W}\ket{H}+\hat{\pi}^{V}_{W}\ket{V}\Bigr)
    \end{equation}

    \noindent Puisque 
    $N = \Bigl|\hat{\pi}^{H}_{W}\Bigr|^2 + \Bigl|\hat{\pi}^{V}_{W}\Bigr|^2$ 
    ,
    $N = \Bigl|\hat{\pi}^{H}_{W}\Bigr|^2 + \Bigl|1- \hat{\pi}^{H}_{W}\Bigr|^2$: 
    

    \begin{equation}
        \ket{\psi^{in}} = \frac{1}{\sqrt{N}}\Bigl(\hat{\pi}^{H}_{W}\ket{H}+ \Bigl(1-\hat{\pi}^{H}_{W}\Bigr)\ket{V}\Bigr)
    \end{equation}

    \noindent Pour fixer la phase globale, qui varierait selon l'état 
    d'entrée, nous supposons que $a$ est toujours réel. Donc 
    l'ellipticité (ou la phase) se trouve dans $b$ et sera dépendante 
    de la partie imaginaire. À partir des données expérimentales des deux 
    observables $\expval{\hat{t}}$ et $\expval{\hat{\omega}}$, nous 
    pouvons calculer directement les amplitudes de probabilité.  
    
    \begin{align}
        |a|^2 &= \frac{\expval{\hat{t}}}{\tau}\\
        |b|^2 &= 1 - |a|^2
    \end{align}

    \noindent Selon l’état d’entrée, la valeur faible varie. Il est 
    crucial de souligner que le délai $\tau$ correspond au délai maximal 
    que nous utilisons pour interagir avec le système. Ce dernier 
    normalise les amplitudes de probabilité. Lorsque nous modifions les 
    états d’entrée, le délai $\tau$ devrait varier entre l’absence de 
    délai et le délai maximal, c’est-à-dire entre les polarisations 
    $\ket{V}$ et $\ket{H}$. 

\end{doublespace}

\subsection{Mots finaux sur la théorie}
    
\begin{doublespace}
    Ce chapitre a posé les bases théoriques des mesures faibles 
    temporelles et leur potentiel pour les systèmes photoniques. En 
    s’appuyant sur des techniques innovantes et des travaux antérieurs, 
    cette thèse vise à démontrer l’utilité des mesures faibles 
    temporelles pour caractériser directement les états quantiques. Le 
    prochain chapitre abordera les aspects expérimentaux de la mise en 
    œuvre de ces méthodes.
\end{doublespace}