\begin{doublespace}
    Ce chapitre explore les fondements théoriques 
    des mesures faibles temporelles, une technique 
    innovante pour caractériser directement les 
    états quantiques. Nous commencerons par examiner 
    les bases des mesures faibles, introduites par 
    Aharonov, Albert et Vaidman (AAV) dans les 
    années 1980, en soulignant leur différence avec 
    les mesures fortes qui effondrent la fonction 
    d'onde. Les mesures faibles permettent 
    d'extraire des informations tout en préservant 
    la superposition quantique. Nous présenterons ensuite comment la valeur 
    faible, calculée à partir des états d'entrée et 
    états de projection, est proportionnelle à la 
    fonction d’onde du système. Cette relation 
    constitue un outil précieux pour analyser les 
    systèmes photoniques. En particulier, nous 
    montrerons que la valeur faible se décompose en 
    parties réelle et imaginaire, respectivement 
    associées à des délais temporels et des 
    modifications fréquentielles. Dans les sections suivantes, nous fournirons 
    les équations fondamentales pour calculer ces 
    composantes dans un système photonique utilisant 
    des délais temporels comme pointeur.
\end{doublespace}

\subsection{Proposition d'une procédure directe avec une mesure faible temporelle}

\begin{doublespace}
        Nous commencerons par exposer les fondements 
    théoriques essentiels des mesures faibles et que la 
    valeur faible est proportionnelle à la fonction d'onde 
    quantique ainsi qu'elle peut être mesuré directement. 
    Comme indiqué précédemment, les mesures faibles font 
    partie de la procédure directe décrite par Jeff Lundeen 
    et l'AAV, qui comprend les éléments 
    suivants \cite{Lundeen_Bamber,Lundeen_Direct_Measurement,Aharonov}:

    \begin{itemize}
        \item Préparation de l'état d'entrée
        \item Une interaction faible (ce dont nous discuterons dans cette section)
        \item Une mesure projective, généralement effectuée avec un état qui possède une quantité égale des deux états de base de l'état d'entrée.
    \end{itemize}

    L'étape sur laquelle nous nous concentrerons dans cette 
    section est celle de la mesure faible, qui implique une 
    faible réduction de la fonction d'onde quantique, plus 
    sur ceci à suivre. Comme évoqué précédemment, les 
    principes des mesures faibles s’appuient sur le modèle 
    de Von Neumann pour les mesures quantiques. Ce modèle 
    implique l'état quantique que l'on souhaite à mesurer $S$ 
    et le pointeur (l'appareil de mesure) $P$, qui sont traités 
    comme des objets de la mécanique quantique couplée dans un
    système totale $T$ \cite{Hairiri,vonNeumann}. 
    La plupart des mesures quantiques peuvent généralement 
    être décrites par ce modèle. Le modèle de von Neumann 
    décrit que lorsqu'un état quantique est mesuré,
    initialement dans un état de superposition arbitraire
    $\ket{\psi}_S = \sum_{j}^{N}c_{j}\ket{s_j}_S$, soit 
    avec des états propres $\ket{s_j}_S$ en base $S$, 
    valeurs propre $s_j$, coefficient d'amplitude de probabilité,
    une observable $\hat{S}$ qui doit être 
    mesurée et dimension $N$ . Elle subit une 
    réduction à l'un de ses vecteurs propres associés 
    avec sa valeur propre \cite{Griffiths}. 
    La mesure est décrite par un opérateur d'interaction 
    appelé l'opérateur d'interaction de von Neumann :

    \begin{equation}
        \hat{U} \equiv exp\Bigl( -\frac{i\mathcal{H}t}{\hbar} \Bigr)
    \end{equation}

    Il s'agit d'un opérateur d'évolution temporelle soit 
    $t$ le temps d'interaction sur le système, $\hbar$ 
    la constante de Planck et $\mathcal{H}$ le hamiltonien du 
    système $T$ décrit par :

    \begin{equation}
        \mathcal{H} \equiv g(\hat{S} \otimes \hat{p})
    \end{equation}

    Soit $g$ la constante de couplage qui est supposé d'être 
    réelle pour que le hamiltonien reste hermitien et 
    $\hat{p}$ la variable pointeur conjuguée de 
    l'observable mesurée. Cette réduction de l'état 
    quantique est représentée par un déplacement de la 
    position du pointeur soit initialement dans un état 
    $\ket{\xi}_P = \ket{\bar{q} = 0}_P$ en base $P$
    dont $\bar{q}$ la valeur centrale d'une variable $q$
    avec une écart de la distribution de probabilité $\sigma$.
    Ensemble, le pointeur et l'état mesuré sont 
    couplés dans un état décrivant l'ensemble du système 
    $T$, écrit initialement sous la forme :

    \begin{equation}
        \ket{\Psi^i}_T = \ket{\psi}_S \otimes \ket{\bar{q}=0}_P    
    \end{equation}

    Après la mesure le pointeur se déplace en fonction
    de la force de l'intéraction 
    $\delta \equiv \frac{gt}{\hbar}$ et une valeur 
    propres $s_j$ du observable $\hat{S}$, 
    $\Delta q = \delta s_j$. Ce dernier s'écrit 
    dans lequel que l'état du pointeur passe 
    de sa position initiale 
    $\ket{\bar{q} = 0}_P$ à 
    $\ket{\bar{q} = \delta s_j}_P$. Ensemble l'état du système 
    évolue dans la façon suivante:

    \begin{align}
        \ket{\Psi^f}_T &= \hat{U} \Bigl[\ket{\psi}_S \otimes \ket{\bar{q} = 0}_P\Bigr]\\
        &= \sum_{j}^{N} c_{j}\ket{s_j}_S \ket{\bar{q} = \delta s_j}_P
    \end{align}

    L'état final est maintenant intriqué entre le 
    système et le pointeur. Ensuite, pour mesurer et 
    caractériser l'état d'entrée, 
    on lit le $\hat{q}$ du 
    pointeur pour la mesure de $\hat{S}$. 
    Si le décalage du 
    pointeur $\delta$ est plus grand que l'écart de la 
    probabilité de la fonction d'onde $\sigma$ dont 
    $\delta \gg \sigma$, le résultat 
    de la mesure à un sans ambiguïté, détruisant la 
    superposition et réduisant la fonction d'onde à 
    un résultat $s_j$, laissant un 
    seul état $\ket{s_j}_S$ et 
    aucune information sur l'ensemble de la fonction 
    d'onde ne peuvent être récupérés. Cependant, 
    lorsque le décalage du pointeur est inférieur à 
    l'écart de la probabilité de la fonction d'onde, 
    $\delta \ll \sigma$, 
    dans le régime des mesures faibles, le système 
    mesuré n'est plus que très peu intriqué avec le 
    pointeur. La mesure de $\hat{S}$ 
    par la mesure du déplacement de $\hat{q}$ ne perturbe 
    plus que très peu 
    la fonction d'onde \cite{Hairiri,Lundeen_Resch}. 
    Considérons ce qui suit:

    \begin{equation}
        \hat{U}\ket{\Psi^{i}}_T = \hat{U}\Bigl[ \ket{\psi}_S \otimes \ket{\bar{q} = 0}_P \Bigr]
    \end{equation}

    Examinons une étude plus approfondie du système 
    initial total qui subit une interaction de 
    mesure. Réécrivons l'opérateur d'interaction de 
    von Neumann sous la forme d'une série de Taylor.

    \begin{align}
        \hat{U}\ket{\Psi^{i}}_T &= \hat{U}\Bigl[ \ket{\psi}_S \otimes \ket{\bar{q} = 0}_P \Bigr]\\
        &= e^{-i\delta(\hat{S} \otimes \hat{p})}\Bigl[ \ket{\psi}_S \otimes \ket{\bar{q} = 0}_P \Bigr]\\
        &= \Bigl( 1 - i\delta(\hat{S} \otimes \hat{p}) - ... \Bigr)\Bigl[ \ket{\psi}_S \otimes \ket{\bar{q} = 0}_P \Bigr]\\
        &=  \ket{\psi}_S \otimes \ket{\bar{q} = 0}_P - i\delta\hat{S}\ket{\psi}_S\otimes\hat{p}\ket{\bar{q} = 0}_P - ...
    \end{align}

    En suivant la procédure de mesure faible, nous 
    projetterons une mesure projective ultérieure 
    sur le système avec l'état $\ket{\varphi}_S$ qui a les mêmes 
    états de base que $\ket{\psi}_S$. 

    \begin{align}
        \ket{\varphi}_S\bra{\varphi}_S\hat{U}\ket{\Psi^{i}}_T &= \ket{\varphi}_S\bra{\varphi}_S\ket{\psi}_S \otimes \ket{\bar{q} = 0}_P - i\delta\ket{\varphi}_S\bra{\varphi}_S\hat{S}\ket{\psi}_S\otimes\hat{p}\ket{\bar{q} = 0}_P - ...
    \end{align}

    Renomarisons l'état du système total en 
    divisant par le 
    module de l'amplitude de 
    probabilité de $\bra{\varphi}_S\ket{\psi}_S = \sqrt{Prob}$,
    dont $Prob \equiv |\bra{\varphi}_S\ket{\psi}_S|^2$
    \cite{Lundeen_Resch,Lundeen_thesis, Steinberg_prob_div}.

    \begin{align}
        \ket{\varphi}_S\frac{\bra{\varphi}_S\hat{U}\ket{\Psi^{i}}_T}{\bra{\varphi}_S\ket{\psi}_S} &= \ket{\varphi}_S\frac{\bra{\varphi}_S\ket{\psi}_S}{\bra{\varphi}_S\ket{\psi}_S} \otimes \ket{\bar{q} = 0}_P - i\delta\ket{\varphi}_S\frac{\bra{\varphi}_S\hat{S}\ket{\psi}_S}{\bra{\varphi}_S\ket{\psi}_S}\otimes\hat{p}\ket{\bar{q} = 0}_P - ...\\
        &= \ket{\varphi}_S \otimes \ket{\bar{q} = 0}_P - i\delta\ket{\varphi}_S\frac{\bra{\varphi}_S\hat{S}\ket{\psi}_S}{\bra{\varphi}_S\ket{\psi}_S}\otimes\hat{p}\ket{\bar{q} = 0}_P - ...
    \end{align}

    Dans ce cas, le 
    $\frac{\bra{\varphi}_S\ket{\psi}_S}{\bra{\varphi}_S\ket{\psi}_S}$ 
    du coté droit est annulée et nous ramenons le 
    $\frac{1}{\bra{\varphi}_S\ket{\psi}_S}$ du 
    côté gauche au coté droit. 
    L'état final est maintenant le suivant :

    \begin{align}
        \ket{\Psi^f}_T &\equiv \ket{\varphi}_S\bra{\varphi}_S\hat{U}\ket{\psi}_S\\
        &\simeq \bra{\varphi}_S\ket{\psi}_S \Bigl[ \ket{\bar{q} = 0}_P - i\delta\frac{\bra{\varphi}_S\hat{S}\ket{\psi}_S}{\bra{\varphi}_S\ket{\psi}_S} \hat{p}\ket{\bar{q}=0}_P - ... \Bigr] \otimes \ket{\varphi}_S
    \end{align}

    Dans les parenthèses carrées, cela correspond 
    à l'état final du pointeur avec lequel nous 
    pouvons calculer les parties réelles et 
    imaginaires de S. 

    \begin{equation}
        \ket{\bar{q} = \delta s_j}_P \equiv \ket{\bar{q} = 0}_P - i\delta\frac{\bra{\varphi}_S\hat{S}\ket{\psi}_S}{\bra{\varphi}_S\ket{\psi}_S} \hat{p}\ket{\bar{q}=0}_P - ...
    \end{equation}

    Remarquez que la position 
    finale du pointeur est proportionnel à ce qui 
    suit :

    \begin{equation}
        \expval{\hat{S}_W} \equiv \frac{\bra{\varphi}_S\hat{S}\ket{\psi}_S}{\bra{\varphi}_S\ket{\psi}_S}
    \end{equation}

    Il s'agit de la valeur faible dérivée pour la 
    première fois par AAV, une valeur complexe avec 
    une partie réelle et imaginaire correspond au 
    décalage de la variable du pointeur $q$ et à son 
    décalage par rapport à sa variable conjuguée $p$ 
    respectivement. Autrement dit, s'il y a un 
    décalage dans la position d'une particule, il 
    y aura également un décalage dans sa quantité 
    de mouvement, soit comme nous allons explorer, 
    la position temporelle d'un photon et sa 
    position de fréquence se déplaceront l'une par 
    rapport à l'autre lors d'une interaction. Si 
    l'interaction est faible, il est possible de 
    mesurer ces valeurs décalées individuellement 
    lors d'une expérience 
    \cite{Hairiri,Aharonov,Lundeen_Direct_Measurement}.
    Pour terminer, écrivons l'état final avec cette valeur.

    \begin{align}
        \ket{\Psi^f}_T &= \bra{\varphi}_S\ket{\psi}_S \Bigl[ \ket{\bar{q} = 0}_P - i\delta \expval{\hat{S}_W} \hat{p}\ket{\bar{q}=0}_P - ... \Bigr] \otimes \ket{\varphi}_S\\
        &= \bra{\varphi}_S\ket{\psi}_S\Bigl[ 1 - i\delta \expval{S_W}\hat{p} - ... \Bigr] \ket{\bar{q} = 0} \otimes \ket{\varphi}_S\\
        &= \bra{\varphi}_S\ket{\psi}_S e^{-i\delta \expval{S_W}\hat{p}} \ket{\bar{q} = 0} \otimes \ket{\varphi}_S\\
        &= \ket{\psi}_S e^{-i\delta \expval{S_W}\hat{p}}\ket{\bar{q}=0}
    \end{align}

    C’est-à-dire, si nous avons une mesure faible parfaite dont 
    $\delta \ll \sigma$, prenons la limite 
    que $\delta \to 0$, nous avons essentiellement 
    l'état initial. Nous pouvons même mesurer la fonction 
    d'onde directement sans aucune reconstruction algorithmique
    et obtenir les parties réelles et imaginaires de la fonction d'onde 
    à l'aide de la valeur faible. Démontrons cela \cite{Lundeen_thesis,Lundeen_Resch}:

    \begin{align}
        \bra{\bar{q} = \delta s_j}\hat{q}\ket{\bar{q}=\delta s_j} &= -i\delta \mathcal{R}\Bigl(\expval{\hat{S}_W}\Bigr)\bra{\bar{q}=\delta s_j}(\hat{q}\hat{p} - \hat{p}\hat{q})\ket{\bar{q}=\delta s_j}\\
        &+ \delta \mathcal{I}\Bigl(\expval{\hat{S}_W}\Bigr)\bra{\bar{q} = \delta s_j}(\hat{q}\hat{p} + \hat{p}\hat{q})\ket{\bar{q}=\delta s_j}\\
        &= \delta\mathcal{R}\Bigl(\expval{\hat{S}_W}\Bigr) = \expval{\hat{q}}
    \end{align}

    Ainsi pour la variable conjuguée:

    \begin{align}
        \bra{\bar{q} = \delta s_j}\hat{p}\ket{\bar{q}=\delta s_j} &= -i\delta \mathcal{R}\Bigl(\expval{\hat{S}_W}\Bigr)\bra{\bar{q}=\delta s_j}(\hat{p}^2 - \hat{p}^2)\ket{\bar{q}=\delta s_j}\\
        &+ \delta \mathcal{I}\Bigl(\expval{\hat{S}_W}\Bigr)\bra{\bar{q} = \delta s_j}(\hat{p}^2 + \hat{p}^2)\ket{\bar{q}=\delta s_j}\\
        &= \frac{\delta}{4\sigma^2}\mathcal{I}\Bigl(\expval{\hat{S}_W}\Bigr)
    \end{align}

    Ensemble la valeur faible s'écrit:

    \begin{equation}
        \expval{\hat{S}_W} = \frac{1}{\delta}\Bigl( \expval{\hat{q}} + i4\sigma^2\expval{\hat{p}} \Bigr)
    \end{equation}

    En démontrant que la valeur faible est 
    proportionnelle à la fonction d'onde, comme l'a fait 
    AAV et que c'est paramètres peuvent être 
    retrouver directement, on a ouvert un tout 
    nouveau domaine dans les mesures 
    quantiques et une alternative à la tomographie 
    quantique traditionnelle.
\end{doublespace}

\subsection{Mesure faible temporelle d'un système photonique quantique}
    
\begin{doublespace}
        Les mesures faibles temporelles exploitent les 
    propriétés temporelles et fréquentielles d’une 
    impulsion lumineuse pour caractériser un état 
    quantique. L’approche repose sur l’hypothèse 
    que les délais temporels peuvent être 
    directement liés aux composantes réelles et 
    imaginaires de la valeur faible. Pour les sections suivantes et ce projet 
    de thèse, nous allons nous concentrer sur la mesure de la 
    valeur faible à partir d'une interaction faible temporelle. 
    Nous utiliserons un système photonique quantique dans 
    lequel nous caractériserons l'état de polarisation d'un 
    faisceau de photons via les délais temporels d'une mesure 
    faible. Nous avons choisi un système photonique parce 
    qu’il est facilement réalisable en laboratoire avec un 
    laser pulsé. Il permettrait aussi de miniaturiser la 
    force de l'interaction faible, grâce à une sorte de miroir 
    (nous y reviendrons plus tard), et, surtout, les états 
    de base pourraient être bien définis expérimentalement 
    en utilisant les états de polarisation horizontaux et 
    verticaux comme base. Le profil temporel des lasers 
    pulsés peut être utilisé pour voir l'impulsion déplacer 
    son temps d'arrivée lorsque nous tournons une plaque 
    d'onde pour caractériser les différents états de 
    polarisation.
\end{doublespace}

\subsubsection{La partie réelle du système}
    
\begin{doublespace}
        Nous voulons caractériser 
    l'état de polarisation avec une mesure faible temporelle.
    Pour réaliser ce dernier, il faut calculer qu'il faut s'attendre
    à la valeur faible $\expval{\hat{\pi}_W}$. Nous allons calculer chaque
    partie de cette valeur à la fois. Commençons avec la partie réelle
    et par définir les paramètres de cette 
    expérience potentielle que nous voulons eventuellement 
    effectuer. L'état de polarisation de notre système que 
    nous voulons mesurer est défini comme suit:

    \begin{equation}
        \ket{\psi} \equiv a\ket{H} + b\ket{V}
    \end{equation}

    Soit $a$ et $b$ des paramètres de probabilité pour les bases
    $\ket{H}$ et $\ket{V}$ respectivement et $|a|^2 + |b|^2 = 1$, ainsi que $\ket{H}$ et $\ket{V}$ 
    correspond à les polarisation horizontaux et verticaux d'un photon.

    \begin{equation}
        \ket{\xi(t)} = \bra{t}\ket{\xi} \equiv \frac{1}{(\sqrt{2\pi}\sigma)^{1/2}}e^{-\frac{t^2}{4\sigma^2}}
    \end{equation}

    Soit le pointeur du système le profile temporel d'un faisceau,
    généralement gaussien est utilisé, avec position temporel $t$ (par rapport à un temps $t_0$) 
    et $\sigma$ l'écart du profile temporel. L'état totale initial s'écrit:

    \begin{equation}
        \ket{\Psi(t)^i} \equiv \ket{\psi} \otimes \ket{\xi(t)}
    \end{equation}

    Effectuons une interaction faible temporelle sur la partie horizontale de l'état 
    $\ket{H}$ avec l'opérateur de von Neumann $\hat{U}^H$, 
    l'exposant $H$ est pour indiqué 
    que l'opérateur est appliqué sur la partie horizontale.

    \begin{align}
        \hat{U^H}\ket{\Psi(t)^i} &= \hat{U}^H \Bigl[ \ket{\psi} \otimes \ket{\xi(t)} \Bigr]\\
        &= \hat{U}^H \Bigl[ a\ket{H} \otimes \ket{\xi(t)} + b\ket{V} \otimes \ket{\xi(t)}\Bigr]\\
        &= a\ket{H} \otimes \hat{U}^H \ket{\xi(t)} + b\ket{V} \otimes \ket{\xi(t)}\\
        &= a\ket{H} \otimes \ket{\xi(t-\tau)} + b\ket{V} \otimes \ket{\xi(t)}
    \end{align}

    L'interaction de von Neumann subit un délai temporel $\tau$ sur le pointeur couplé 
    avec la partie horizontale. Ensuite Effectuons une mesure projective avec 
    l'état $\ket{\varsigma} \equiv \mu\ket{H} + \nu\ket{V}$ soit $\nu$ et $\mu$ des paramètres
    probabilité pour $\ket{H}$ et $\ket{V}$ respectivement et $|\mu|^2 + |\nu|^2 = 1$. 

    \begin{align}
        \ket{\Psi(t)^f} &= \ket{\varsigma}\bra{\varsigma}\hat{U}^H\ket{\Psi(t)^i} = \Bigl[ \bar{\mu}\bra{H} + \bar{\nu}\bra{V} \Bigr]a\ket{H} \otimes \ket{\xi(t-\tau)} + b\ket{V} \otimes \ket{\xi(t)}\\
        &= \Bigl[\bar{\mu}a\ket{\xi(t-\tau)} + \bar{\nu}b\ket{\xi(t)}\Bigr] \otimes \ket{\varsigma}\\
        &= F(t)\otimes\ket{\varsigma}
    \end{align}

    Soit $F(t) \equiv A\ket{\xi(t-\tau)} + B\ket{\xi(t)}$, $A \equiv a\bar{\mu}$ et $B \equiv b\bar{\nu}$. 
    Trouvons la valeur d'espérance de la position 
    temporel $\expval{\hat{t}}$.

    \begin{align}
        \expval{\hat{t}} &= \bra{\Psi(t)^f}\hat{t}\ket{\Psi(t)^f}\\
        &= \int_{-\infty}^{\infty} I(t)tdt
    \end{align}

    Soit $I(t) \equiv |F(t)|^2$, nous pouvons le normalisé avec $\frac{1}{\bra{\Psi(t)^f}\ket{\Psi(t)^f}}$:

    \begin{align}
        \expval{\hat{t}^{norm}} &= \frac{\bra{\Psi(t)^f}\hat{t}\ket{\Psi(t)^f}}{\bra{\Psi(t)^f}\ket{\Psi(t)^f}} = \frac{\int_{-\infty}^{\infty} I(t)tdt}{\int_{-\infty}^{\infty} I(t)dt}\\
        &= \frac{\int_{-\infty}^{\infty} |A|^2\Xi(t - \tau)t + |B|^2\Xi(t)t + A\bar{B}\Xi(t, \tau)t + \bar{A}B\Xi(t, \tau)t dt}{\int_{-\infty}^{\infty} |A|^2\Xi(t - \tau) + |B|^2\Xi(t) + A\bar{B}\Xi(t, \tau) + \bar{A}B\Xi(t, \tau) dt}
    \end{align}

    Soit $\Xi(t) \equiv \frac{1}{\sqrt{2\pi}\sigma}e^{-\frac{t^2}{2\sigma^2}}$ et $\Xi(t, \tau) \equiv \frac{1}{\sqrt{2\pi}\sigma}e^{-\frac{2t^2 - 2t\tau + \tau^2}{4\sigma^2}}$. 
    Notons que vue que nous effectuons une interaction faible sur
    le système, il y a une superposition entre les pointeurs 
    pour la partie de polarisation horizontale et verticale.
    Les solutions de chaque intégrale sont énumérées 
    ci-dessous et nous reprendrons notre développement de 
    la partie réelle de la valeur faible.

    \begin{align*}
        \int_{-\infty}^{\infty} \Xi(t - \tau)t dt &= \tau & \int_{-\infty}^{\infty} \Xi(t) dt &= 1\\
        \int_{-\infty}^{\infty} \Xi(t - \tau) dt &= 1 & \int_{-\infty}^{\infty} \Xi(t, \tau)t dt &= \frac{\tau}{2}e^{-\frac{t^2}{8\sigma^2}}\\
        \int_{-\infty}^{\infty} \Xi(t)t dt &= 0 & \int_{-\infty}^{\infty} \Xi(t, \tau) dt &= e^{-\frac{t^2}{8\sigma^2}}
    \end{align*}

    Donc avec ces solutions, la partie réelle se trouve:

    \begin{align}
        \expval{\hat{t}^{norm}} = \tau\frac{|A|^2 + (A\bar{B} + \bar{A}B)e^{-\frac{\tau^2}{8\sigma^2}}}{|A|^2 + |B|^2 + (A\bar{B} + \bar{A}B)e^{-\frac{\tau^2}{8\sigma^2}}}
    \end{align}

    Vue que nous sommes dans le régime des mesures faibles, prenons
    la limite $\tau \ll \sigma$:

    \begin{align}
        \lim_{\frac{\tau}{\sigma} \to 0} \expval{\hat{t}^{norm}} &= \tau\frac{|A|^2 + A\bar{B} + \bar{A}B}{|A|^2 + |B|^2 + A\bar{B} + \bar{A}B}\\
        &\equiv \mathcal{R}(\expval{\hat{\pi}_W})
    \end{align}

    Ce dernier est la partie réelle de la valeur faible $\expval{\hat{\pi}_W}$.
\end{doublespace}

\subsubsection{La partie imaginaire du système}
    
\begin{doublespace}
        Comme nous l'avons déjà mentionné, 
    un déplacement de la variable du pointeur, 
    tel que sa position temporelle $t$ par rapport à 
    un $t_0$, devrait entraîner un déplacement de son 
    spectre de fréquence. Vérifions-le en calculant 
    la partie imaginaire de la valeur faible $\expval{\hat{\pi}_W}$. 
    Commençons par prendre la transformation de 
    Fourier de la fonction temporel $F(t)$ de l'état quantique $\ket{\Psi(t)^f}$:

    \begin{align}
        F(\omega) &= \frac{1}{\sqrt{2\pi}}\int_{-\infty}^{\infty} F(t)e^{-i\omega t}dt\\
        &= \frac{\sqrt[4]{2}\sqrt{\sigma}}{\sqrt[4]{\pi}}(A + Be^{i\omega\tau})e^{-\omega^2 \sigma^2 - i\omega\tau}
    \end{align}

    Avec ce dernier la fonction d'onde s'écrit:

    \begin{equation}
        \ket{\Psi(\omega)^f} = F(\omega) \otimes \ket{\varsigma}
    \end{equation}

    Ensuite trouvons la valeur d'espérance pour la position fréquentielle avec des étapes similaires que la partie réelle:

    \begin{align}
        \expval{\hat{\omega}} &= \bra{\Psi(\omega)^f}\hat{\omega}\ket{\Psi(\omega)^f}\\
        &= \int_{-\infty}^{\infty} S(\omega) d\omega
    \end{align}

    Soit $S(\omega) \equiv |F(\omega)|^2$ et normalisons cette valeur:

    \begin{align}
        \expval{\hat{\omega}^{norm}} &= \frac{\sqrt{2}\sigma}{\sqrt{\pi}}\frac{\int_{-\infty}^{\infty} |A|^2\omega e^{i\omega\tau} + |B|^2\omega e^{i\omega\tau} + A\bar{B}\omega + \bar{A}Be^{2i\omega\tau} \omega d\omega}{\int_{-\infty}^{\infty} |A|^2 e^{i\omega\tau} + |B|^2 e^{i\omega\tau} + A\bar{B} + \bar{A}Be^{2i\omega\tau}d\omega}e^{-2\omega^2 \sigma^2 -i\omega\tau}
    \end{align}

    Avec des solutions d'intégrale similaires nous obtenons:

    \begin{equation}
        \expval{\hat{\omega}^{norm}} = \frac{i\tau}{4\sigma^2}\frac{(B\bar{A} - A\bar{B})e^{-\frac{\tau^2}{8\sigma^2}}}{|A|^2 + |B|^2 + \bar{A}B + A\bar{B}}
    \end{equation}

    Prenons encore la limite dont $\tau \ll \sigma$ vue que nous sommes dans le 
    régime des mesures faibles:

    \begin{align}
        \lim_{\frac{\tau}{\sigma} \to 0}\expval{\hat{\omega}^{norm}} &= \frac{i\tau}{4\sigma^2}\frac{B\bar{A} - A\bar{B}}{|A|^2 + |B|^2 + \bar{A}B + A\bar{B}}\\
        &\equiv \mathcal{I}(\expval{\hat{\pi}_W})
    \end{align}

\end{doublespace}

\subsection{Proposition expérimentale pour la caractérisation de la valeur faible}
    
\begin{doublespace}
        Nous continuons avec notre système photonique 
    quantique en nous appuyant sur nos découvertes 
    concernant la partie réelle et imaginaire de la 
    valeur faible. Nous pouvons calculer que pour 
    un état d'entré soit:

    \begin{equation}
        \ket{\psi^{in}} = a\ket{H} + b\ket{V} 
    \end{equation}

    Puisque $a$ et $b$ sont des amplitudes de 
    probabilité pour les états de base $\ket{H}$ et $\ket{V}$ 
    respectivement soit $a=\bra{H}\ket{\psi^{in}}$ et
    $b=\bra{V}\ket{\psi^{in}}$. Expérimentalement, 
    les parties mesurées de la valeur faible 
    peuvent être 
    calculées directement en fonction de la façon 
    dont l'observable change en fonction de l'état 
    d'entrée. Cela est possible car la valeur faible 
    est proportionnelle à l'état quantique, comme le 
    montre la section 2.1. Pour les états de 
    polarisation, il s'agit de mesurer faiblement 
    $\expval{\hat{S}^J} = \ket{J}\bra{J}$ 
    soit $J= H,V$, puis de mesurer par projection sur un état 
    intermédiaire tel que 
    $\ket{D} = \frac{1}{\sqrt{2}}(\ket{H}+\ket{V})$. 
    Si l'on y parvient, 
    on obtient un sous-ensemble d'essais dont le 
    résultat moyen est la valeur faible.

    \begin{equation}
        \expval{\hat{S}_{W}^{J}} = \frac{\bra{D}\hat{S}^{J}\ket{\psi^{in}}}{\bra{D}\ket{\psi^{in}}} = \sqrt{N}\bra{J}\ket{\psi^{in}}
    \end{equation}

    Où $N$ est une constante de normalisation 
    indépendante de $J$. L'état quantique peut être 
    écrit en relation avec la valeur faible écrite. 

    \begin{equation}
        \ket{\psi^{in}} = \frac{1}{\sqrt{N}}\Bigl(\expval{\hat{S}^{H}_{W}}\ket{H}+\expval{\hat{S}^{V}_{W}}\ket{V}\Bigr)
    \end{equation}

    Nous pouvons supposer que 
    $N = \Bigl|\expval{\hat{S}^{H}_{W}}\Bigr|^2 + \Bigl|\expval{\hat{S}^{V}_{W}}\Bigr|^2$ 
    puisque $|a|^2 + |b|^2 = 1$ donc 
    $N = \Bigl|\expval{\hat{S}^{H}_{W}}\Bigr|^2 + \Bigl|1- \expval{\hat{S}^{H}_{W}}\Bigr|^2$. 
    Donc,

    \begin{equation}
        \ket{\psi^{in}} = \frac{1}{\sqrt{N}}\Bigl(\expval{\hat{S}^{H}_{W}}\ket{H}+ \Bigl(1-\expval{\hat{S}^{H}_{W}}\Bigr)\ket{V}\Bigr)
    \end{equation}

    Pour fixer la phase globale qui varierait selon l'état 
    d'entrée, nous supposons que $a$ est toujours réel.
    Cependant $b$ sera dépendant sur la partie imaginaire.
    Avec les données expérimentales des deux 
    observables $\expval{\hat{t}}$ et $\expval{\hat{\omega}}$, nous pouvons calculer 
    directement les amplitudes de probabilité à partir des données expérimentales.  

    \begin{align}
        |a|^2 &= \frac{\expval{\hat{t}}}{\tau}\\
        |b|^2 &= 1 - |a|^2
    \end{align}

    En fonction de l'état d'entrée, la valeur faible 
    varie. Il est important de noter que le délai $\tau$ 
    est le délai maximal que nous utilisons pour 
    interagir avec le système. Ce dernier, normalise
    les amplitudes de probabilité. Lorsque nous 
    changeons les états d'entrée, le délai $\tau$ 
    devrait évoluer entre l'absence de délai et le 
    délai maximal, c'est-à-dire entre les 
    polarisations $\ket{V}$ et $\ket{H}$. 

\end{doublespace}

\subsection{Mots finale sur la théorie}
    
\begin{doublespace}
        Ce chapitre a établi les fondements théoriques 
    des mesures faibles temporelles et leur 
    pertinence pour les systèmes photoniques. En 
    s’appuyant sur des techniques innovantes et des 
    travaux précédents, cette thèse vise à démontrer 
    l’utilité des mesures faibles temporelles pour 
    caractériser directement les états quantiques. 
    Le prochain chapitre présentera les aspects 
    expérimentaux liés à la mise en œuvre de ces 
    méthodes.
\end{doublespace}