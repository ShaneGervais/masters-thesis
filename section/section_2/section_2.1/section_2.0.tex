Ce chapitre explore les fondements théoriques 
des mesures faibles temporelles, une technique 
innovante pour caractériser directement les 
états quantiques. Nous commencerons par examiner 
les bases des mesures faibles, introduites par 
Aharonov, Albert et Vaidman (AAV) dans les 
années 1980, en soulignant leur différence avec 
les mesures fortes qui effondrent la fonction 
d'onde. Les mesures faibles permettent 
d'extraire des informations tout en préservant 
la superposition quantique. Nous présenterons ensuite comment la valeur 
faible, calculée à partir des états d'entrée et 
états de projection, est proportionnelle à la 
fonction d’onde du système. Cette relation 
constitue un outil précieux pour analyser les 
systèmes photoniques. En particulier, nous 
montrerons que la valeur faible se décompose en 
parties réelle et imaginaire, respectivement 
associées à des délais temporels et des 
modifications fréquentielles. Dans les sections suivantes, nous fournirons 
les équations fondamentales pour calculer ces 
composantes dans un système photonique utilisant 
des délais temporels comme pointeur.