Ce chapitre a établi les fondements théoriques 
des mesures faibles temporelles et leur 
pertinence pour les systèmes photoniques. En 
s’appuyant sur des techniques innovantes et des 
travaux précédents, cette thèse vise à démontrer 
l’utilité des mesures faibles temporelles pour 
caractériser directement les états quantiques. 
Le prochain chapitre présentera les aspects 
expérimentaux liés à la mise en œuvre de ces 
méthodes.