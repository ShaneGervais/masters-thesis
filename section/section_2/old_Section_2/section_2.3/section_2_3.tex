Nous continuons avec notre système photonique 
quantique en nous appuyant sur nos découvertes 
concernant la partie réelle et imaginaire de la 
valeur faible. Nous pouvons calculer que pour 
un état d'entré soit:

\begin{equation}
    \ket{\psi^{in}} = a\ket{H} + b\ket{V} 
\end{equation}

Puisque $a$ et $b$ sont des amplitudes de 
probabilité pour les états de base $\ket{H}$ et $\ket{V}$ 
respectivement soit $a=\bra{H}\ket{\psi^{in}}$ et
$b=\bra{V}\ket{\psi^{in}}$. Expérimentalement, 
les parties mesurées de la valeur faible 
peuvent être 
calculées directement en fonction de la façon 
dont l'observable change en fonction de l'état 
d'entrée. Cela est possible car la valeur faible 
est proportionnelle à l'état quantique, comme le 
montre la section 2.1. Pour les états de 
polarisation, il s'agit de mesurer faiblement 
$\expval{\hat{S}^J} = \ket{J}\bra{J}$ 
soit $J= H,V$, puis de mesurer par projection sur un état 
intermédiaire tel que 
$\ket{D} = \frac{1}{\sqrt{2}}(\ket{H}+\ket{V})$. 
Si l'on y parvient, 
on obtient un sous-ensemble d'essais dont le 
résultat moyen est la valeur faible.

\begin{equation}
    \expval{\hat{S}_{W}^{J}} = \frac{\bra{D}\hat{S}^{J}\ket{\psi^{in}}}{\bra{D}\ket{\psi^{in}}} = \sqrt{N}\bra{J}\ket{\psi^{in}}
\end{equation}

Où $N$ est une constante de normalisation 
indépendante de $J$. L'état quantique peut être 
écrit en relation avec la valeur faible écrite. 

\begin{equation}
    \ket{\psi^{in}} = \frac{1}{\sqrt{N}}\Bigl(\expval{\hat{S}^{H}_{W}}\ket{H}+\expval{\hat{S}^{V}_{W}}\ket{V}\Bigr)
\end{equation}

Nous pouvons supposer que 
$N = \Bigl|\expval{\hat{S}^{H}_{W}}\Bigr|^2 + \Bigl|\expval{\hat{S}^{V}_{W}}\Bigr|^2$ 
puisque $|a|^2 + |b|^2 = 1$ donc 
$N = \Bigl|\expval{\hat{S}^{H}_{W}}\Bigr|^2 + \Bigl|1- \expval{\hat{S}^{H}_{W}}\Bigr|^2$. 
Donc,

\begin{equation}
    \ket{\psi^{in}} = \frac{1}{\sqrt{N}}\Bigl(\expval{\hat{S}^{H}_{W}}\ket{H}+ \Bigl(1-\expval{\hat{S}^{H}_{W}}\Bigr)\ket{V}\Bigr)
\end{equation}

Pour fixer la phase globale qui varierait selon l'état 
d'entrée, nous supposons que $a$ est toujours réel.
Cependant $b$ sera dépendant sur la partie imaginaire.
Avec les données expérimentales des deux 
observables $\expval{\hat{t}}$ et $\expval{\hat{\omega}}$, nous pouvons calculer 
directement les amplitudes de probabilité à partir des données expérimentales.  

\begin{align}
    |a|^2 &= \frac{\expval{\hat{t}}}{\tau}\\
    |b|^2 &= 1 - |a|^2
\end{align}

En fonction de l'état d'entrée, la valeur faible 
varie. Il est important de noter que le délai $\tau$ 
est le délai maximal que nous utilisons pour 
interagir avec le système. Ce dernier, normalise
les amplitudes de probabilité. Lorsque nous 
changeons les états d'entrée, le délai $\tau$ 
devrait évoluer entre l'absence de délai et le 
délai maximal, c'est-à-dire entre les 
polarisations $\ket{V}$ et $\ket{H}$. 
