Nous voulons caractériser 
l'état de polarisation avec une mesure faible temporelle.
Pour réaliser ce dernier, il faut calculer qu'il faut s'attendre
à la valeur faible $\expval{\hat{\pi}_W}$. Nous allons calculer chaque
partie de cette valeur à la fois. Commençons avec la partie réelle
et par définir les paramètres de cette 
expérience potentielle que nous voulons eventuellement 
effectuer. L'état de polarisation de notre système que 
nous voulons mesurer est défini comme suit:

\begin{equation}
    \ket{\psi} \equiv a\ket{H} + b\ket{V}
\end{equation}

Soit $a$ et $b$ des paramètres de probabilité pour les bases
$\ket{H}$ et $\ket{V}$ respectivement et $|a|^2 + |b|^2 = 1$, ainsi que $\ket{H}$ et $\ket{V}$ 
correspond à les polarisation horizontaux et verticaux d'un photon.

\begin{equation}
    \ket{\xi(t)} = \bra{t}\ket{\xi} \equiv \frac{1}{(\sqrt{2\pi}\sigma)^{1/2}}e^{-\frac{t^2}{4\sigma^2}}
\end{equation}

Soit le pointeur du système le profile temporel d'un faisceau,
généralement gaussien est utilisé, avec position temporel $t$ (par rapport à un temps $t_0$) 
et $\sigma$ l'écart du profile temporel. L'état totale initial s'écrit:

\begin{equation}
    \ket{\Psi(t)^i} \equiv \ket{\psi} \otimes \ket{\xi(t)}
\end{equation}

Effectuons une interaction faible temporelle sur la partie horizontale de l'état 
$\ket{H}$ avec l'opérateur de von Neumann $\hat{U}^H$, 
l'exposant $H$ est pour indiqué 
que l'opérateur est appliqué sur la partie horizontale.

\begin{align}
    \hat{U^H}\ket{\Psi(t)^i} &= \hat{U}^H \Bigl[ \ket{\psi} \otimes \ket{\xi(t)} \Bigr]\\
    &= \hat{U}^H \Bigl[ a\ket{H} \otimes \ket{\xi(t)} + b\ket{V} \otimes \ket{\xi(t)}\Bigr]\\
    &= a\ket{H} \otimes \hat{U}^H \ket{\xi(t)} + b\ket{V} \otimes \ket{\xi(t)}\\
    &= a\ket{H} \otimes \ket{\xi(t-\tau)} + b\ket{V} \otimes \ket{\xi(t)}
\end{align}

L'interaction de von Neumann subit un délai temporel $\tau$ sur le pointeur couplé 
avec la partie horizontale. Ensuite Effectuons une mesure projective avec 
l'état $\ket{\varsigma} \equiv \mu\ket{H} + \nu\ket{V}$ soit $\nu$ et $\mu$ des paramètres
probabilité pour $\ket{H}$ et $\ket{V}$ respectivement et $|\mu|^2 + |\nu|^2 = 1$. 

\begin{align}
    \ket{\Psi(t)^f} &= \ket{\varsigma}\bra{\varsigma}\hat{U}^H\ket{\Psi(t)^i} = \Bigl[ \bar{\mu}\bra{H} + \bar{\nu}\bra{V} \Bigr]a\ket{H} \otimes \ket{\xi(t-\tau)} + b\ket{V} \otimes \ket{\xi(t)}\\
    &= \Bigl[\bar{\mu}a\ket{\xi(t-\tau)} + \bar{\nu}b\ket{\xi(t)}\Bigr] \otimes \ket{\varsigma}\\
    &= F(t)\otimes\ket{\varsigma}
\end{align}

Soit $F(t) \equiv A\ket{\xi(t-\tau)} + B\ket{\xi(t)}$, $A \equiv a\bar{\mu}$ et $B \equiv b\bar{\nu}$. 
Trouvons la valeur d'espérance de la position 
temporel $\expval{\hat{t}}$.

\begin{align}
    \expval{\hat{t}} &= \bra{\Psi(t)^f}\hat{t}\ket{\Psi(t)^f}\\
    &= \int_{-\infty}^{\infty} I(t)tdt
\end{align}

Soit $I(t) \equiv |F(t)|^2$, nous pouvons le normalisé avec $\frac{1}{\bra{\Psi(t)^f}\ket{\Psi(t)^f}}$:

\begin{align}
    \expval{\hat{t}^{norm}} &= \frac{\bra{\Psi(t)^f}\hat{t}\ket{\Psi(t)^f}}{\bra{\Psi(t)^f}\ket{\Psi(t)^f}} = \frac{\int_{-\infty}^{\infty} I(t)tdt}{\int_{-\infty}^{\infty} I(t)dt}\\
    &= \frac{\int_{-\infty}^{\infty} |A|^2\Xi(t - \tau)t + |B|^2\Xi(t)t + A\bar{B}\Xi(t, \tau)t + \bar{A}B\Xi(t, \tau)t dt}{\int_{-\infty}^{\infty} |A|^2\Xi(t - \tau) + |B|^2\Xi(t) + A\bar{B}\Xi(t, \tau) + \bar{A}B\Xi(t, \tau) dt}
\end{align}

Soit $\Xi(t) \equiv \frac{1}{\sqrt{2\pi}\sigma}e^{-\frac{t^2}{2\sigma^2}}$ et $\Xi(t, \tau) \equiv \frac{1}{\sqrt{2\pi}\sigma}e^{-\frac{2t^2 - 2t\tau + \tau^2}{4\sigma^2}}$. 
Notons que vue que nous effectuons une interaction faible sur
le système, il y a une superposition entre les pointeurs 
pour la partie de polarisation horizontale et verticale.
Les solutions de chaque intégrale sont énumérées 
ci-dessous et nous reprendrons notre développement de 
la partie réelle de la valeur faible.

\begin{align*}
    \int_{-\infty}^{\infty} \Xi(t - \tau)t dt &= \tau & \int_{-\infty}^{\infty} \Xi(t) dt &= 1\\
    \int_{-\infty}^{\infty} \Xi(t - \tau) dt &= 1 & \int_{-\infty}^{\infty} \Xi(t, \tau)t dt &= \frac{\tau}{2}e^{-\frac{t^2}{8\sigma^2}}\\
    \int_{-\infty}^{\infty} \Xi(t)t dt &= 0 & \int_{-\infty}^{\infty} \Xi(t, \tau) dt &= e^{-\frac{t^2}{8\sigma^2}}
\end{align*}

Donc avec ces solutions, la partie réelle se trouve:

\begin{align}
    \expval{\hat{t}^{norm}} = \tau\frac{|A|^2 + (A\bar{B} + \bar{A}B)e^{-\frac{\tau^2}{8\sigma^2}}}{|A|^2 + |B|^2 + (A\bar{B} + \bar{A}B)e^{-\frac{\tau^2}{8\sigma^2}}}
\end{align}

Vue que nous sommes dans le régime des mesures faibles, prenons
la limite $\tau \ll \sigma$:

\begin{align}
    \lim_{\frac{\tau}{\sigma} \to 0} \expval{\hat{t}^{norm}} &= \tau\frac{|A|^2 + A\bar{B} + \bar{A}B}{|A|^2 + |B|^2 + A\bar{B} + \bar{A}B}\\
    &\equiv \mathcal{R}(\expval{\hat{\pi}_W})
\end{align}

Ce dernier est la partie réelle de la valeur faible $\expval{\hat{\pi}_W}$.