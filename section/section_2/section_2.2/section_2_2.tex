Les mesures faibles temporelles exploitent les 
propriétés temporelles et fréquentielles d’une 
impulsion lumineuse pour caractériser un état 
quantique. L’approche repose sur l’hypothèse 
que les délais temporels peuvent être 
directement liés aux composantes réelles et 
imaginaires de la valeur faible. Pour les sections suivantes et ce projet 
de thèse, nous allons nous concentrer sur la mesure de la 
valeur faible à partir d'une interaction faible temporelle. 
Nous utiliserons un système photonique quantique dans 
lequel nous caractériserons l'état de polarisation d'un 
faisceau de photons via les délais temporels d'une mesure 
faible. Nous avons choisi un système photonique parce 
qu’il est facilement réalisable en laboratoire avec un 
laser pulsé. Il permettrait aussi de miniaturiser la 
force de l'interaction faible, grâce à une sorte de miroir 
(nous y reviendrons plus tard), et, surtout, les états 
de base pourraient être bien définis expérimentalement 
en utilisant les états de polarisation horizontaux et 
verticaux comme base. Le profil temporel des lasers 
pulsés peut être utilisé pour voir l'impulsion déplacer 
son temps d'arrivée lorsque nous tournons une plaque 
d'onde pour caractériser les différents états de 
polarisation.