\begin{onehalfspace}
    Ce parcours a connu des hauts et des bas, mais je peux dire avec fierté que j'en suis ressorti avec une meilleure compréhension non seulement de la physique quantique elle-même, mais aussi de moi-même en tant que personne. Ayant un handicap (dysphasie), j'ai plus de difficulté que les autres à comprendre des sujets complexes, à communiquer mes idées et encore plus à rédiger une thèse, surtout en français. Cependant, j'ai trouvé la force, la confiance et l'assurance nécessaires pendant mes études à l'Université de Moncton, où j'ai rencontré divers professeurs qui m'ont inspirée et poussée à dépasser mes limites pour surmonter l'adversité. Pour moi, ce diplôme n'était pas seulement un projet que j'avais choisi, mais quelque chose sur lequel je peux revenir et qui me permet de continuer à progresser pour m'améliorer en tant que scientifique et en tant que personne. 

    \noindent Je tiens tout d'abord à remercier mon directeur de thèse, le Dr Lambert Giner, pour sa grande patience et sa collaboration tout au long de ce projet. Lorsque les choses semblaient ne mener nulle part, que j'étais perdue ou simplement bloquée dans une impasse, il a su me remettre sur la voie de l'objectif réel pour lequel nous travaillons. Je lui suis reconnaissant de m'avoir aidé au laboratoire lorsque le projet semblait dans l'impasse et de m'avoir guidé dans mes compétences de présentation et de rédaction pour une communication scientifique efficace. Bien que j'aie encore beaucoup à apprendre, ses conseils m'ont permis de découvrir mes forces et mes faiblesses, sur lesquelles je peux continuer à travailler. 

    \noindent Je tiens à remercier le Dr Normand Beaudoin de m'avoir inspiré tout au long de ce diplôme. Je n'ai rencontré aucun autre professeur comme lui, sa passion et son envie d'aller au fond des choses m'inspirent. Je me suis reconnue en lui, il m'a donné un aperçu de ce que je peux vraiment devenir en tant que scientifique et en tant que personne. Vouloir comprendre les fondements de l'univers dans lequel nous vivons me motive encore aujourd'hui à continuer, une tâche sans fin à laquelle je suis fière de participer. 

    \noindent Je tiens à remercier le Dr Guilleaume Thekkadeth d'avoir pris le temps de faire partie de ce jury. Ce fut un plaisir d'être évaluée par quelqu'un dont j'ai lu les articles pendant mes études, ce qui a rendu cette expérience vraiment inspirante. 

    \noindent Je tiens à remercier le Dr Alexandre Melanson d'avoir accepté d'être le président du jury et d'avoir cru en moi tout au long de ce projet.

    \noindent Je tiens également à remercier divers autres professeurs qui ont rendu cette expérience vraiment enrichissante pour moi, notamment le Dr Jean-François Bison pour sa positivité et la joie qu'il m'a apportée dans la poursuite de mes études, le Dr Alain Haché pour nos conversations sur le hockey et les conseils qu'il m'a prodigués, le Dr Serge Gauvin pour m'avoir encouragé à remettre en question et à sortir des sentiers battus, le Dr Deny Hamel pour m'avoir inspiré à aller de l'avant et à percer les mystères de l'optique quantique pendant ses cours, les techniciens de notre département : Pierre, pour avoir toujours été là quand j'avais besoin d'aide et pour m'avoir toujours fait rire pendant les pauses, et Julian, pour m'avoir inspiré à enseigner la physique de manière ludique et compréhensible, et pour notre passion commune pour le death metal. 

    \noindent Je tiens à remercier tout particulièrement les personnes suivantes qui m'ont également marqué personnellement tout au long de ce parcours.

    \noindent Ma grand-mère, pour avoir toujours été là pour moi et m'avoir transmis sa passion pour les sciences, en se souvenant des difficultés que j'ai rencontrées pendant mes jeunes années à l'école jusqu'à ce que j'en arrive là où je suis aujourd'hui.

    \noindent Ma mère et mon père, pour leurs conseils que je garderai toute ma vie et pour m'avoir motivé à aller de l'avant malgré les difficultés. Ils m'ont toujours rappelé de continuer à travailler dur et m'ont inspiré à donner le meilleur de moi-même.

    \noindent Ma sœur, pour être à la fois la personne la plus agaçante et la plus attentionnée au monde. Pour avoir compris mes sentiments lorsque j'avais besoin d'être écouté, pour tous les bons moments que nous avons passés ensemble tout au long de mon parcours et pour être la sœur la plus loufoque, la plus agaçante et la meilleure que je puisse souhaiter. 

    \noindent Ma petite amie, pour avoir été là dans les bons comme dans les mauvais moments. Quand tout semblait sombre, elle était là pour m'éclairer. Elle m'a écouté quand personne ne me comprenait, elle a cru en moi quand personne ne le faisait, elle m'a inspiré à être moi-même, elle m'a donné de la force et m'a aidé à tracer la voie pour mes projets futurs. 

    \noindent Mon ami Kastriot, pour toutes les questions difficiles qu'il m'a posées et qui m'ont donné envie d'en savoir toujours plus sur notre univers. 

    \noindent \textit{Je vous aime tous.} 

    \noindent Et toutes les autres personnes que j'ai rencontrées pendant mon master et qui m'ont apporté de la joie, de l'amitié et du courage : Koceila, Mathéo, Elisa, Marie Céline, Paul-Henry, Mathias, Shiva, Chris et les autres.

    \noindent Je vous remercie tous infiniment, car cette réussite est autant la mienne que la leur. Ce fut un moment mémorable pour moi et je chérirai toujours les souvenirs que j'ai créés ici.

\end{onehalfspace}