\begin{onehalfspace}
    %La caractérisation des états quantiques est essentielle au 
    %développement des technologies quantiques. 
    Pour le développement des nouvelles technologies quantiques, telles que les
    ordinateurs quantiques, des capteurs quantiques pour la métrologie
    quantique, ou effectuant des télécommunications quantiques, 
    il est essentiel de pouvoir
    caractériser les états quantiques des systèmes quantiques.
    La caractérisation des états
    quantiques est un domaine de recherche actif, car elle permet de
    comprendre, mesurer et manipuler les systèmes quantiques.

    \noindent Les méthodes de caractérisation traditionnelles, telles 
    que la tomographie quantique, reconstruisent l’état du système 
    quantique en effectuant des mesures projectives
    sur le système, suivies d’une reconstruction de l’état à l’aide
    de la matrice densitée.  Cependant, cette approche est une méthode indirecte qui nécessite
    de nombreuses mesures projectives et ne permet pas de suivre
    l’évolution temporelle de l’état quantique en temps réel car 
    elle requiert l'utilisation d'un algorithme informatique. 
    De plus, elle est limitée par la
    complexité croissante des systèmes quantiques à dimension élevée,
    rendant la reconstruction de l’état du système de plus en plus
    difficile et coûteuse en ressources expérimentales. Elle ne 
    convient pas aux systèmes nécessitant une caractérisation en temps 
    réel, où l’évolution temporelle de l’état est cruciale, que ce soit 
    pour la métrologie quantique ou pour la détection d’erreurs dans les 
    ordinateurs quantiques pendant les calculs. 

    \noindent Une méthode alternative, comme les mesures faibles, 
    présente un bon potentiel pour la caractérisation directe de l’état 
    quantique, sans effondrement complet du système; toutefois, dans sa forme 
    actuelle, elle n’est pas intégrable aux technologies quantiques. 
    Cette thèse vise à explorer les degrés de libertés 
    qui seraient compatibles avec 
    des systèmes quantiques pour identifier la manière de rendre 
    les mesures faibles integrables dans les technologies quantiques futures et
    dans nos infrastructures photoniques existantes.

    \noindent \textcolor{red}{Nous proposons de réaliser des mesures faibles temporelles en exploitant
    le domaine temporel de la propagation des photons pour 
    caractériser des états de polarisation en concevant un dispositif
    expérimental facile à implémenter dans un laboratoire d'optique commun 
    et rentable en ressources. Cela permettrait une intégration
    directe dans notre infrastructure photonique existante et dans les technologies quantiques en permettant
    de déterminer directement, en temps réel, l’évolution temporelle d’un état
    quantique tout en minimisant la perte d’information grâce aux mesures faibles.}

\end{onehalfspace}