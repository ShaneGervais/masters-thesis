\begin{doublespace}
    La caractérisation des états quantiques est essentielle au développement de la technologie quantique. Pour effectuer des calculs complexes, vérifier l’intégrité des messages codés en communication quantique, ou encore créer des systèmes de télécommunication quantique, il est essentiel de connaître l’état des qubits. L'étude sur la caractérisation de ces états quantiques est cruciale pour le développement des technologies quantiques et leur intégration à notre infrastructure existante. 

    \noindent Les méthodes de caractérisation traditionnelles, telles que la tomographie quantique, reconstruisent l’état du système quantique en effondrant l'état en effectuant des mesures projectives dans toutes les bases possibles, ce qui permet d’obtenir une description de l'état du système à l’aide de la matrice densitée. Cette méthode est une approche indirecte qui devient rapidement complexe avec des systèmes quantiques à dimension élevée. Elle ne convient pas aux systèmes nécessitant une caractérisation en temps réel, où l’évolution temporelle de l’état est cruciale, que ce soit pour la métrologie quantique ou pour la détection d’erreurs dans les ordinateurs quantiques pendant les calculs. 

    \noindent Une méthode alternative, comme les mesures faibles, présente un bon potentiel pour la caractérisation directe de l’état quantique, sans effondrement complet du système, mais, dans sa forme actuelle, elle n’est pas intégrable aux technologies quantiques. Nous proposons de réaliser des mesure faible temporels en exploitant le domaine temporel des photons, comme le pointeur, et en concevant un dispositif expérimental dans ce régime pour caractériser un état quantique facile à implanter dans un laboratoire d'optique commun. Cela permettrait une intégration directe dans notre infrastructure photonique existante et dans les technologies quantiques. 

    \noindent Cette technique permet de déterminer directement et en temps réel l’évolution d’un état quantique, en minimisant la perte d’information grâce à une intervention minimale sur le système. Dans cette étude, nous illustrons la théorie requise pour cette méthode de caractérisation et réalisons des expériences visant à mesurer la partie réelle et imaginaire de la valeur faible du système. Enfin, nous élaborant sur les applications potentielles pour ce type de caractérisation dans les technologies quantique, notamment améliorer les computations quantiques, la détection en métrologie quantique, la sécurisation des communications quantiques et les infrastructures photoniques, que ce soit dans les télécommunications quantiques. 
\end{doublespace}