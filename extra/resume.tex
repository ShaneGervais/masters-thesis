\begin{onehalfspace}
    Le développement des technologies quantiques repose sur la capacité à mesurer et à caractériser les états quantiques. Parmi ces technologies, la photonique quantique occupe une place centrale grâce à ses avantages distincts. Grâce à l’utilisation de photons comme particules, la photonique quantique présente plusieurs avantages. Elle génère peu de décohérence, elle est hautement compatible avec les technologies optiques existantes, on peut facilement manipuler plusieurs degrés de liberté, et on peut intégrer des circuits photoniques à grande échelle. Ces propriétés servent de base à une multitude d’applications, notamment la communication quantique ultra-sécurisée, l’imagerie quantique à haute résolution, la métrologie quantique et le calcul quantique basé sur l’optique linéaire.

    Dans ce contexte, maîtriser les techniques de mesure quantique est essentiel pour tirer parti du pouvoir de la photonique quantique. Les techniques traditionnelles, telles que la tomographie quantique, permettent une reconstruction complète des états quantiques, mais elles s’avèrent souvent coûteuses en ressources pour les systèmes à plusieurs dimensions et indirectes. Une solution innovante consiste à utiliser des mesures subtiles, qui fournissent des données sur un système quantique sans perturber significativement son fonctionnement. On nomme cette méthode \guillemetleft mesures faibles \guillemetright. Elle s’appuie sur les postulats de mesure quantique de von Neumann en utilisant un pointeur qui se déplace proportionnellement à la valeur décrite par Aharanov, Albert et Vaidman comme une \guillemetleft valeur faible \guillemetright. Il a ouvert la voie à plusieurs domaines quantiques, et nous proposons une méthode de caractérisation directe, peu coûteuse en ressources et facile à mettre en oeuvre. 

    Dans cette thèse, nous explorons l’utilisation et le concept de bases des mesures faibles dans le domaine temporel des photons en tant qu’approche novatrice pour l’étude et l’optimisation des systèmes photoniques quantiques. Nous commençons par présenter les bases des mesures quantiques, soit la tomographie quantique et celle des mesures faibles. Ensuite, nous proposerons une nouvelle méthodologie qui exploite la dynamique temporelle des photons pour révéler des informations subtiles sur leur état quantique. Nous élaborons le fondement conceptuel, effectuons des expériences en milieu contrôlé, puis interprétons les données recueillies. Nous discutons ensuite des implications de cette approche et de ses perspectives d’application.


\end{onehalfspace}