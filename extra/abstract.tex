\begin{doublespace}

    For the development of new quantum technologies, such as
    quantum computers, quantum sensors for quantum metrology, or
    quantum telecommunications, it is essential to be able to
    characterize the quantum states of a system. The 
    characterization of quantum states is an active area of research,
    as it allows for understanding, measuring and manipulating quantum systems.

    \noindent Traditional characterization methods, such as quantum
    state tomography, reconstructs a quantum system's state by performing projective measurements
    on the system, followed by a reconstruction of the state using the density matrix.
    However, this approach is an indirect method that requires
    many projective measurements and does not allow for real-time 
    tracking of the quantum state evolution, as it requires the 
    use of a computational algorithm. Furthermore, it is limited 
    by the increasing complexity of high-dimensional quantum systems,
    making the reconstruction of the system's state increasingly 
    difficult and resource-intensive. Thus, it is not suitable 
    for systems requiring real-time characterization, where the 
    temporal evolution of a state is crucial, such as in quantum
    metrology or for error detection in quantum computers
    during computational processes.

    \noindent An alternative method, such as weak measurements,
    is a promising approach for the direct characterization of quantum states,
    without complete collapse of the system; however, in its current form,
    it is not integrable into quantum technologies. In this thesis,
    we aim to explore the different degrees of freedom
    that would allow weak measurements to be compatible with
    future quantum technologies and our existing photonic infrastructures.

    \noindent We design an experimental apparatus to 
    perform temporal weak measurements by exploiting
    the temporal domain of photons as our pointer variable. This approach
    allows for the direct characterization of quantum states that
    can enable direct real-time tracking of the temporal
    evolution of a quantum state while minimizing information loss through the weak measurement regime.
    This would allow for direct implementation into our existing photonic infrastructures
    and future quantum technologies, such as in error detection in quantum computers
    or in quantum metrology.


    %\noindent
    %We propose to perform
    %temporal weak measurements by exploiting the temporal domain of photons
    %as our pointer variable. By designing an
    %experimental apparatus in this regime, one can characterize
    %quantum states using a temporal pointer variable enabling real-time
    %tracking of the quantum state evolution. Thus, it would allow
    %for direct integration into existing photonic infrastructures
    %and future quantum technologies, such as in error detection in quantum computers
    %or in quantum metrology.
    
    
    
    
    %By designing an
    %experimental device in this regime to characterize a quantum state, one can 
    %impliment it in futur
    %achieve
    %direct integration into future quantum technologies such as
    %in error detection in quantum computers or in quantum metrology.
    %This could enable direct real-time tracking of the temporal
    %evolution of a quantum state while minimizing information
    %loss through the weak measurement regime.

\end{doublespace}