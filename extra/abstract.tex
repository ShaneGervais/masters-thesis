\begin{doublespace}
    Development of new quantum technologies, such as quantum 
    computers, advanced cybersecurity using quantum communication protocols, 
    and the development of sensitive quantum sensors used in metrology, relies
    on the state characterization of qubits, which are states containing
    the system's information. Similarly, classical bits in computing 
    can be $0$ or $1$; a qubit takes advantage of the fundamental laws of physics,
    namely quantum mechanics, allowing it to have both outcomes in a superposition state.

    The concept of a quantum computer (and later qubits) arises from the famous Richard Feynman, opening the 
    world to new possibilities for technologies. David P. DiVincenzo
    laid the groundwork for the physical implementation of a quantum computer, for which
    measurement is, in fact, crucial. Unlike the classical world, 
    quantum mechanics makes it challenging
    to measure the components of the system.
    Because of the Heisenberg uncertainty principle, one cannot know the precise location
    and momentum of a particle; thus, in quantum mechanics, there’s always a trade-off 
    between what you are trying to observe for full state characterization.

    \noindent Traditionally, quantum state tomography is used for state characterization, 
    where one projectively measures the system in all possible states, which can then be represented
    and reconstructed indirectly. This
    method can become very tedious to consider for systems 
    with many dimensions, and such reconstructions are also resource-intensive. 

    \noindent Alternatively, quantum weak measurements can present an advantage, 
    as they offer a more direct way of measuring the quantum state. Weak measurements take
    advantage of what’s called the "weak value," a value that arises experimentally
    when introducing a weak interaction between a coupled system and a pointer, where the 
    pointer can be thought of as a needle on a measurement apparatus. This strategy,
    famously described by Yakir Aharonov, David Albert, and Lev Vaidman (along with post-selection on a known state)
    can give rise to how the pointer evolves for
    different input states, proportional to this weak value. This could allow for less information loss 
    during characterization, preserving the integrity of the original state, 
    which is essential for quantum communications.

    \noindent However, recent research in quantum weak measurements—particularly in the 
    quantum photonic context—focuses on the position and momentum degrees
    of freedom for polarization state characterization, which is not considered to be 
    practical for quantum technology applications. We propose to look at the 
    time and energy (frequency) domains of photons as a more practically implementable
    approach for future quantum technologies. This would enable real-time 
    tracking of a quantum state, which could bring considerable
    advantages to technologies such as quantum computers and quantum sensors for metrology.

    \noindent In this thesis, we shall explore the fundamentals of weak measurement theory,
    while briefly reviewing quantum state tomography and proposing our theoretical
    expectations for the experimental realization of temporal weak measurements. 
    Experimental verification 
    of our capability to measure small temporal shifts will be carried out 
    by comparing the speed of an electric signal and that of light. 
    We will then perform a temporal weak measurement of a pure state with 
    the objective of measuring the real and imaginary parts of the weak value, 
    which will allow us to directly reconstruct our prepared quantum states.
\end{doublespace}
