\nomenclature{$\ket{H}$}{Polarisation horizontale \begin{pmatrix}    1\\    0\end{pmatrix}}\nomenclchapter{2} 

\nomenclature{$\ket{V}$}{Polarisation verticale \begin{pmatrix}
    0\\
    1
\end{pmatrix}}
\nomenclchapter{2} 

\nomenclature{$\ket{D}$}{Polarisation diagonale \ensuremath{$\frac{1}{\sqrt{2}}$}\begin{pmatrix}
    1\\
    1
\end{pmatrix}}

\nomenclature{$\ket{A}$}{Polarisation anti-diagonale \ensuremath{$\frac{1}{\sqrt{2}}$\begin{pmatrix}
    1\\
    -1
\end{pmatrix}}}

\nomenclature{$\ket{R}$}{Polarisation circulaire droite \ensuremath{$\frac{1}{\sqrt{2}}$\begin{pmatrix}
    1\\
    i
\end{pmatrix}}}

\nomenclature{$\ket{L}$}{Polarisation circulaire gauche \ensuremath{$\frac{1}{\sqrt{2}}$\begin{pmatrix}
    1\\
    -i
\end{pmatrix}}}

\nomenclature{$\rho$}{Matrice densitée d'un système}
\nomenclchapter{2}  

\nomenclature{$\psi$}{Fonction d'onde d'un système état quantique }
\nomenclchapter{2}  

\nomenclature{$S_{0,1,2,3}$}{Paramètres de Stokes}
\nomenclchapter{2}  

\nomenclature{$\sigma_{x,y,z}$}{Matrices de Pauli}
\nomenclchapter{2}

\nomenclature{$\hat{S}$}{Opérateur arbitraire du système}
\nomenclchapter{2}

\nomenclature{$\ket{psi_i}$}{État initial arbitraire}
\nomenclchapter{2}

\nomenclature{$\ket{psi_f}$}{État final arbitraire}
\nomenclchapter{2}

\nomenclature{W}{Indice pour une variable de type \guillemetleft valeur faible \guillemetright}
\nomenclchapter{2}

\nomenclature{$\expval{\hat{S}}_W$}{État initial arbitraire}
\nomenclchapter{2}

\nomenclature{$\hat{p}$}{Observable du système}
\nomenclchapter{2}

\nomenclature{$\hat{q}$}{L'observable conjugué de celui du système}
\nomenclchapter{2}

\nomenclature{$\sigma$}{Écart de la distribution de l'observable du pointeur}
\nomenclchapter{2}

\nomenclature{$i$}{Nombre imaginaire \ensuremath{$\sqrt{-1}$}}
\nomenclchapter{2}

\nomenclature{$S$}{Base du système}
\nomenclchapter{2}

\nomenclature{$P$}{Base du pointeur}
\nomenclchapter{2}

\nomenclature{$g$}{Force de couplage entre le système et le pointeur}
\nomenclchapter{2}

\nomenclature{$\ket{\bar{p}}$}{Position moyenne de l'osbervable du pointeur}
\nomenclchapter{2}

\nomenclature{$\ket{s}$}{Valeur propre du système}
\nomenclchapter{2}

\nomenclature{$\ket{s}$}{Vecteur propre du système}
\nomenclchapter{2}

\nomenclature{$\hbar$}{Constante de Planck réduite}
\nomenclchapter{2}

\nomenclature{$\hat{U}$}{Opérateur d'interaction de von Neumann}
\nomenclchapter{2}

\nomenclature{$\mathcal{O}$}{Ordre analytique d'une série}
\nomenclchapter{2}

\nomenclature{$\ket{\varphi}$}{État projective}
\nomenclchapter{2}

\nomenclature{$\ket{\Psi}$}{État totale incluant le système et le pointeur}
\nomenclchapter{2}

\nomenclature{$\expval{\hat{S}}_W$}{Valeur faible du système}
\nomenclchapter{2}

\nomenclature{$\expval{\hat{\pi}}_W$}{Valeur faible du système de polarisation}
\nomenclchapter{2}

\nomenclature{$a$}{Amplitude de probabilités pour l'état horizontal}
\nomenclchapter{2}

\nomenclature{$b$}{Amplitude de probabilités pour l'état vertical}
\nomenclchapter{2}

\nomenclature{$\ket{\xi}$}{Distribution du pointeur}
\nomenclchapter{2}

\nomenclature{$\ket{t}$}{Domaine temporel du pointeur}
\nomenclchapter{2}

\nomenclature{$\ket{\xi(t)}$}{Fonction décrivant la distribution du domaine temporel du pointeur}
\nomenclchapter{2}

\nomenclature{$\hat{U}^H$}{L'opérateur de von Neumann appliqué sur la partie horizontale de l'état du système}
\nomenclchapter{2}

\nomenclature{$\mathcal{H}$}{Hamiltonien du système}
\nomenclchapter{2}

\nomenclature{$\hat{\pi}$}{Observable d'un état de polarisation}
\nomenclchapter{2}

\nomenclature{$\hat{E}$}{Opérateur d'énergie}
\nomenclchapter{2}

\nomenclature{$t$}{Temps}
\nomenclchapter{2}

\nomenclature{$\tau$}{Délai temporel}
\nomenclchapter{2}

\nomenclature{$\ket{\varsigma}$}{État projective de polarisation}
\nomenclchapter{2}

\nomenclature{$\mu$}{Amplitude de probabilité de l'état projective horizontal}
\nomenclchapter{2}

\nomenclature{$\nu$}{Amplitude de probabilité de l'état projective vertical}
\nomenclchapter{2}

\nomenclature{$\expval{\hat{t}}$}{Position temporel moyen}
\nomenclchapter{2}

\nomenclature{$\expval{\hat{\omega}}$}{Position fréquenciel moyen}
\nomenclchapter{2}

\nomenclature{$A$}{$\mu^*a$}
\nomenclchapter{2}

\nomenclature{$B$}{$\nu^*b$}
\nomenclchapter{2}

\nomenclature{$F(\omega)$}{Fonction du domaine fréquenciel du pointeur}
\nomenclchapter{2}

\nomenclature{$c$}{Vitesse de la lumière}
\nomenclchapter{3}

\nomenclature{PBS}{Séparateur de faisceau polarisant (\guillemetleft Polarizing Beam Splitter \guillemetright)}
\nomenclchapter{3}

\nomenclature{u.a.}{Unités arbitraires}
\nomenclchapter{3}

\nomenclature{$a_0,b_0,c_0$}{Paramètre d'adjustement}
\nomenclchapter{3}

\nomenclature{$\hat{T}$}{Opérateur d'une plaque d'onde}
\nomenclchapter{3}

\nomenclature{$\theta$}{Orientation d'une composante}
\nomenclchapter{3}

\nomenclature{$\phi$}{Orientation d'une composante}
\nomenclchapter{3}

\nomenclature{MZ}{Mach-Zehnder}
\nomenclchapter{3}