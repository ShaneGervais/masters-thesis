 %Dans cette section, on défini le format du document et toute autre fonction qui seront utile pour le document
 \documentclass[superscriptaddress,12pt]{article}
 %\documentclass[aps,pra,notitlepage,superscriptaddress,10pt]{revtex4-2}
 %\documentclass[aps,pra,notitlepage,superscriptaddress,twocolumn,10pt]{revtex4-2}
 \usepackage[utf8]{inputenc} % to insert the character directly by copy paste or as ^+i typed on your keyboard
 \usepackage[T1]{fontenc} % to lower the accent
 \usepackage{graphicx}		%Permet d'ajouter des figures
 \graphicspath{ {./figure/}}
 \usepackage{bm} 			%Permet d'utiliser des charactères gras dans les équations
 \usepackage{amsmath}		%Défini plusieurs fonctions utiles pour les équations
 \usepackage{physics}
 %\usepackage[pdfauthor={Deny Hamel},pdftitle={Title}]{hyperref} %Cette fonction permet de créer des liens dans le document PDF généré. TRÈS utile pour des documents électroniques.
 \usepackage{icomma}			%Enlève l'espace après la virgule dans les nombres 
 \usepackage{setspace}
 \usepackage{multirow}
 \usepackage{booktabs}
 
 
 \makeatletter
 \newcommand{\mathleft}{\@fleqntrue\@mathmargin0pt}
 \newcommand{\mathcenter}{\@fleqnfalse}
 \makeatother
 
 
 \usepackage{gensymb}
 \usepackage{titlesec}
 \titleformat{\section}
   {\normalfont\fontsize{12}{15}\bfseries}{\thesection}{1em}{}
 \titleformat{\subsection}
   {\normalfont\fontsize{12}{15}\bfseries}{\thesubsection}{1em}{}
 \titleformat{\subsubsection}
   {\normalfont\fontsize{12}{15}\bfseries}{\thesubsubsection}{1em}{}
 \usepackage{siunitx}
 
 \usepackage{caption}
 
 
 \usepackage{xcolor}
 
 
 \newcommand\tab[1][0.83cm]{\hspace*{#1}}
 
 
 \usepackage[style=numeric, backend=bibtex, sorting=none]{biblatex}
 \addbibresource{./reference_these.bib}
 
 
 \usepackage{parskip}
 \setlength{\parindent}{4em}
 \setlength{\parskip}{1em}
 \usepackage[ letterpaper, top=2.5cm ,  bottom=2.5cm , left=4cm , right =2.5cm , ]{geometry}
 %\pagestyle{myheadings}
 
 \usepackage[french]{babel}
 \usepackage{hyperref}
 \hypersetup{
     colorlinks,
     citecolor=black,
     filecolor=black,
     linkcolor=black,
     urlcolor=black
 }
 %J'ai changé la numérotation des sections pour avoir une numérotation avec des chiffres arabes. Voir https://tex.stackexchange.com/questions/3177/how-to-change-the-numbering-of-part-chapter-section-to-alphabetical-r
 \renewcommand\thesection{\arabic{section}}
 \renewcommand\thesubsection{\thesection.\arabic{subsection}} 
 \renewcommand\thesubsubsection{\thesubsection.\arabic{subsubsection}} 
 
 
 %\renewcommand\thetable{\arabic{table}} Si on veut aussi une numérotation arabe pour les tableaux.
 
 
 %%Cette partie corrige un problème causé par les définitions ici-haut, qui faisait qu'au lieu de citer par exemple la section 2.1, latex donnait 2 2.1 (ce qui ferait du sens avec le sections romaine de PRA, i.e. II A). Voir : https://tex.stackexchange.com/questions/104486/section-reference-shows-section-then-subsection-number
 \makeatletter
 \def\p@subsection{}
 \makeatother
 \makeatletter
 \def\p@subsubsection{}
 \makeatother
 
 
 %J'ai modifié le format un peu pour que les tables, figures, etc. soient en français
 \def\andname{et} 			%Remplace le "and" dans la liste d'auteur avec un "et"
 \def\tablename{Tableau}	
 \def\figurename{Figure}
 
 
 %La typographie pour la plupart des opérateur mathématique est déjà défini par latex. Or, il y a quelques cas ou il n'y a pas de consensus, comme par exemple exemple pour l'opérateur différentiel d. Dans ces cas, il suffit de définir de nouveaux opérateurs avec la typographie désirée.
 \def\D{\mathrm{d}}
 \def\e{\mathrm{e}}
 \def\i{\mathrm{i}}
 
 
 \usepackage{scalerel}
 
 
 \usepackage{fancyhdr}
 \pagestyle{fancy}
 \fancyhf{}
 \fancyhead[R]{\thepage}
 \renewcommand{\headrulewidth}{0pt}
 
 
 \usepackage{gensymb}
 
 
 
 
 \makeatletter
 \newcommand*{\rom}[1]{\expandafter\@slowromancap\romannumeral #1@}
 \makeatother
 
 
 %%%%%%%%%%%%Le document commence ici%%%%%%%%%%%%%%%%%%%%-----------------------------------------------------------------------------
 \usepackage{titlesec}

\begin{document}
    \pagenumbering{roman}

    \thispagestyle{empty}
    \begin{titlepage}
    \addcontentsline{toc}{section}{\protect\numberline{}Page titre}%
       \begin{center}
           \vspace*{1cm}


           \textbf{Caractérisation directe d'un état de polarisation à l'aide des mesures faibles temporelles}


           \vspace{5cm}


            Thèse présentée à la faculté des sciences de l'université de Moncton \\
            pour l'obtention du grade de \\
            maitrise ès sciences et spécialisation physique (M. Sc.)

           \vspace{5cm}


           \textbf{Shane Gervais} \\
           A00198792


           \vfill

           Département de physique et d'astronomie \\
           Université de Moncton\\
           \textcolor{red}{DATE}

           \vspace{0.8cm}

       \end{center}
    \end{titlepage}




    \section*{Composition du jury}
    \setcounter{page}{2}

    \addcontentsline{toc}{section}{\protect\numberline{}Composition du jury}%


    \vspace{2.5cm}


    Président du jury : \textcolor{red}{noms} \\
    \tab \tab Professeur, \\
    \tab \tab Université de Moncton


    \vspace{3cm}


    Examinateur interne : \textcolor{red}{noms} \\
    \tab \tab Professeur, \\
    \tab \tab Université de Moncton


    \vspace{3cm}


    Examinateur externe : \textcolor{red}{noms} \\
    \tab \tab Professeur, \\
    \tab \tab Université de \textcolor{red}{noms de l'uni}


    \vspace{3cm}


    Directeur de thèse : Lambert Giner \\
    \tab \tab Professeur, \\
    \tab \tab Université de Moncton


    \pagebreak


    \section*{Remerciements}

    \addcontentsline{toc}{section}{\protect\numberline{}Remerciements}%
    \pagebreak

    \section*{Sommaire}

    
    \addcontentsline{toc}{section}{\protect\numberline{}Sommaire}%


    \pagebreak


    \section*{Abstract}

    

    \addcontentsline{toc}{section}{\protect\numberline{}Abstract}%


    \pagebreak


    \singlespacing
    \tableofcontents


    \newpage


    \listoftables


    \addcontentsline{toc}{section}{\protect\numberline{}Liste des tableaux}%


    \pagebreak


    \listoffigures

    \addcontentsline{toc}{section}{\protect\numberline{}Liste des figures}%





    \pagebreak


    \section*{Liste des symboles}


    \addcontentsline{toc}{section}{\protect\numberline{}Liste des symboles}%


    \pagebreak


    \onehalfspacing


    \pagenumbering{arabic}


    \thispagestyle{empty}
    \section{INTRODUCTION DES MESURES QUANTIQUE}

    \subsection{La notion des mesures quantique}
    
    \pagebreak
    
    \subsection{Les procédures de mesure directe utilisant des mesures faibles}	

    \pagebreak

    \subsection{Motivation de la thèse}
    \pagebreak
    
    \thispagestyle{empty}
    \section{LES MESURES QUANTIQUES TEMPORELLES ET FRÉQUENCIELLES}
    
    \subsection{Compréhension des mesures temporelles en quantique}
    
    \pagebreak

    \subsection{Possiblité d'une mesure faible temporelle}
    
    \pagebreak
    
    \subsection{Correlation entre d'observable temporel et fréquenciel}

    \pagebreak

    \thispagestyle{empty}
    \section{MESURE EXPÉRIMENTALE DIRECTE D'UN ÉTAT DE POLARISATION EN UTILISANT UNE MESURE FAIBLE TEMPORELLE}
    
    \subsection{Proposition d'expérience pour mesurer la partie réelle de la valeur faible d'un état quantique}

    \pagebreak

    \subsection{Caractérisation de la partie réelle de la valeur faible}
    
    \pagebreak

    \subsection{Proposition d'expérience pour mesurer la partie imaginaire de la valeur faible d'un état quantique}

    \pagebreak

    \subsection{Caractérisation de la partie imaginaire de la valeur faible}
    
    \pagebreak

    \subsection{Caractérisation générale d'un état de polarisation en utilisant des mesures faibles temporelles}

    \pagebreak

    \thispagestyle{empty}
    \section{CONCLUSION}

    \subsection{Discussion des résultats expérimentaux}

    \pagebreak
    
    \subsection{Conclusion sur la thèse}

    \pagebreak

    \subsection{Applications et projet de future}

    \pagebreak

    \printbibliography	

    \thispagestyle{empty}

    \section*{ANNEXE A}

    \pagebreak
    \thispagestyle{empty}
    \section*{ANNEXE B}
    \pagebreak
    \thispagestyle{empty}

\end{document}