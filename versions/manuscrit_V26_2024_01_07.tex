 %Dans cette section, on défini le format du document et toute autre fonction qui seront utile pour le document
\documentclass[superscriptaddress,12pt]{article}
%\documentclass[aps,pra,notitlepage,superscriptaddress,10pt]{revtex4-2}
%\documentclass[aps,pra,notitlepage,superscriptaddress,twocolumn,10pt]{revtex4-2}
\usepackage[utf8]{inputenc} % to insert the character directly by copy paste or as ^+i typed on your keyboard
\usepackage[T1]{fontenc} % to lower the accent
\usepackage{graphicx}		%Permet d'ajouter des figures
\graphicspath{ {./figure/}}
\usepackage{bm} 			%Permet d'utiliser des charactères gras dans les équations
\usepackage{amsmath}		%Défini plusieurs fonctions utiles pour les équations
\usepackage{physics}
%\usepackage[pdfauthor={Deny Hamel},pdftitle={Title}]{hyperref} %Cette fonction permet de créer des liens dans le document PDF généré. TRÈS utile pour des documents électroniques.
\usepackage{icomma}			%Enlève l'espace après la virgule dans les nombres 
\usepackage{setspace}
\usepackage{multirow}
\usepackage{booktabs}


\makeatletter
\newcommand{\mathleft}{\@fleqntrue\@mathmargin0pt}
\newcommand{\mathcenter}{\@fleqnfalse}
\makeatother


\usepackage{gensymb}
\usepackage{titlesec}
\titleformat{\section}
  {\normalfont\fontsize{12}{15}\bfseries}{\thesection}{1em}{}
\titleformat{\subsection}
  {\normalfont\fontsize{12}{15}\bfseries}{\thesubsection}{1em}{}
\titleformat{\subsubsection}
  {\normalfont\fontsize{12}{15}\bfseries}{\thesubsubsection}{1em}{}
\usepackage{siunitx}

\usepackage{caption}


\usepackage{xcolor}


\newcommand\tab[1][0.83cm]{\hspace*{#1}}


\usepackage[style=numeric, backend=bibtex, sorting=none]{biblatex}
\addbibresource{./reference_these.bib}


\usepackage{parskip}
\setlength{\parindent}{4em}
\setlength{\parskip}{1em}
\usepackage[ letterpaper, top=2.5cm ,  bottom=2.5cm , left=4cm , right =2.5cm , ]{geometry}
%\pagestyle{myheadings}

\usepackage[french]{babel}
\usepackage{hyperref}
\hypersetup{
    colorlinks,
    citecolor=black,
    filecolor=black,
    linkcolor=black,
    urlcolor=black
}
%J'ai changé la numérotation des sections pour avoir une numérotation avec des chiffres arabes. Voir https://tex.stackexchange.com/questions/3177/how-to-change-the-numbering-of-part-chapter-section-to-alphabetical-r
\renewcommand\thesection{\arabic{section}}
\renewcommand\thesubsection{\thesection.\arabic{subsection}} 
\renewcommand\thesubsubsection{\thesubsection.\arabic{subsubsection}} 


%\renewcommand\thetable{\arabic{table}} Si on veut aussi une numérotation arabe pour les tableaux.


%%Cette partie corrige un problème causé par les définitions ici-haut, qui faisait qu'au lieu de citer par exemple la section 2.1, latex donnait 2 2.1 (ce qui ferait du sens avec le sections romaine de PRA, i.e. II A). Voir : https://tex.stackexchange.com/questions/104486/section-reference-shows-section-then-subsection-number
\makeatletter
\def\p@subsection{}
\makeatother
\makeatletter
\def\p@subsubsection{}
\makeatother


%J'ai modifié le format un peu pour que les tables, figures, etc. soient en français
\def\andname{et} 			%Remplace le "and" dans la liste d'auteur avec un "et"
\def\tablename{Tableau}	
\def\figurename{Figure}


%La typographie pour la plupart des opérateur mathématique est déjà défini par latex. Or, il y a quelques cas ou il n'y a pas de consensus, comme par exemple exemple pour l'opérateur différentiel d. Dans ces cas, il suffit de définir de nouveaux opérateurs avec la typographie désirée.
\def\D{\mathrm{d}}
\def\e{\mathrm{e}}
\def\i{\mathrm{i}}


\usepackage{scalerel}


\usepackage{fancyhdr}
\pagestyle{fancy}
\fancyhf{}
\fancyhead[R]{\thepage}
\renewcommand{\headrulewidth}{0pt}


\usepackage{gensymb}




\makeatletter
\newcommand*{\rom}[1]{\expandafter\@slowromancap\romannumeral #1@}
\makeatother


%%%%%%%%%%%%Le document commence ici%%%%%%%%%%%%%%%%%%%%
\usepackage{titlesec}



\begin{document}
\pagenumbering{roman}

\thispagestyle{empty}
\begin{titlepage}
\addcontentsline{toc}{section}{\protect\numberline{}Page titre}%
   \begin{center}
       \vspace*{1cm}


       \textbf{Contrôle des propriétés d'émission d'un laser à l'aide de miroirs nanostructurés}


       \vspace{5cm}


Thèse présentée à la Faculté des Sciences de l'Université de Moncton \\
pour l'obtention du grade de \\
 Maitrise ès sciences et spécialisation physique (M. Sc.)
            
       \vspace{5cm}


       \textbf{Shawn Lapointe} \\
       A00193525


       \vfill
            
       Département de Physique et d'Astronomie \\
       Université de Moncton\\
       \textcolor{red}{DATE}
            
       \vspace{0.8cm}
     
   \end{center}
\end{titlepage}




\section*{Composition du jury}
\setcounter{page}{2}

\addcontentsline{toc}{section}{\protect\numberline{}composition du jury}%


\vspace{2.5cm}


Président du jury : \textcolor{red}{noms} \\
\tab \tab Professeur, \\
\tab \tab Université de Moncton


\vspace{3cm}


Examinateur interne : \textcolor{red}{noms} \\
\tab \tab Professeur, \\
\tab \tab Université de Moncton


\vspace{3cm}


Examinateur externe : \textcolor{red}{noms} \\
\tab \tab Professeur, \\
\tab \tab Université de \textcolor{red}{noms de l'uni}


\vspace{3cm}


Directeur de thèse : Jean-François Bisson \\
\tab \tab Professeur, \\
\tab \tab Université de Moncton


\pagebreak


\section*{Remerciements}

\addcontentsline{toc}{section}{\protect\numberline{}Remerciements}%
\pagebreak

\section*{Sommaire}

Ce manuscrit se concentre sur l'élimination du phénomène nuisible du creusement spatial pour obtenir une émission monomode d'un laser à milieu actif avec un élargissement homogène dans un résonateur à ondes stationnaires, ainsi que sur la suppression d'un des deux états propres de polarisation dans le résonateur en opérant près d'un point où les états propres coalescent (point exceptionnel). La méthode d'élimination du creusement spatial consiste à éliminer la modulation d'intensité de l'onde stationnaire à l'intérieur du résonateur en ayant des ondes contrapropagatives appartenant à un mode qui ont des états de polarisation orthogonaux, de sorte qu'aucune frange d'interférence n'est produite dans le milieu actif donc que la saturation du niveau excité est atteinte sur toute la longueur du milieu à la même puissance de pompage, empêchant ainsi les autres modes d'atteindre leur seuil d'oscillation. Pour atteindre cet objectif, des miroirs présentant une anisotropie par la gravure de réseau sur la surface sont fabriqués selon des spécifications de réponse optique très spécifiques, consistant en un déphasage entre les états de polarisation TE et TM de $\pi$ et une différence de réflectance entre TE et TM. La différence de réflectance est responsable du comportement des états propres de polarisation, qui se rapprochent l'un de l'autre lorsque l'angle entre les axes principaux des deux miroirs change ($\alpha$).

Or, puisque la réponse optique requise pour obtenir le comportement désiré étant très stricte, nous étudions les conditions pour lesquelles l'émission monomode peut être prédite, peu importe le taux de pompage. La condition de suppression de l'émission des modes longitudinaux appartenant aux mêmes états propres de polarisation en fonction de l'efficacité de suppression de la modulation d'onde stationnaire est rappelée, et la condition de suppression de l'autre mode de polarisation est développée lorsque la coalescence des états n'est pas atteinte. Une exploration des points exceptionnels dans l'espace de polarisation est faite pour différents éléments optiques insérés dans le résonateur, démontrant de nouvelles combinaisons possibles pour l'obtention d'un point exceptionnel.   

En ce qui concerne les expériences réalisées, nous avons testé les propriétés d'émission de notre dispositif laser constitué d'un milieu amplificateur YAG dopé à l'Yb$^{3+}$ dans une cavité de taille réduite avec des miroirs anisotropes fonctionnant dans le régime quasi continu. Les mesures des états propres de polarisation et du seuil d'oscillation sont utilisées pour valider notre approche simple de la détermination des états propres de polarisation basée uniquement sur la réponse optique des miroirs. Les mesures de la qualité du profil transverse du faisceau nous assurent que le fonctionnement est dans le mode transversal fondamental TEM$_{00}$. De plus, la qualité de l'état de polarisation est mesurée par le degré de polarisation, et l'efficacité de la conversion de la puissance de pompe en puissance de sortie est mesurée. Enfin, le spectre d'émission est caractérisé et nous observons une émission monomode pour certains angles $\alpha$ avec une bonne stabilité, montrant que ce dispositif d'architecture simple est une autre façon d'obtenir un laser monomode.

\addcontentsline{toc}{section}{\protect\numberline{}Sommaire}%


\pagebreak


\section*{Abstract}

This manuscript focuses on the elimination of the harmful phenomenon of spatial hole burning to obtain single-mode emission from a laser active-medium with homogeneous broadening in a standing-wave resonator, as well as on the suppression of one of the two polarization eigenstates in the resonator by operating near a point where the eigenstates coalesce (exceptional point). The method of eliminating spatial hole burning consists in eliminating the intensity modulation of the standing wave inside the resonator by having contrapropagating waves belonging to one mode that have orthogonal polarization states, so that no interference fringes are produced in the active medium and saturation of the excited state atomes is reached along the entire length of the medium at the same pumping power, thus preventing the other resonator modes from reaching their oscillation threshold. To achieve this objective, mirrors featuring anisotropy through grating etching on the surface are fabricated to very specific optical response specifications, consisting of a phase shift between the TE and TM polarization states of $\pi$ and a reflectance difference between TE and TM. The difference in reflectance is responsible for the behavior of the polarization eigenstates, which change proximity to each other as the angle between the principal axes of the two mirrors changes ($\alpha$).

However, since the optical response required to obtain the desired behavior is very strict, we investigate the conditions for which single-mode emission can be predicted regardless of the pumping rate. The condition for suppressing the emission of longitudinal modes belonging to the same polarization eigenstates as a function of the suppression efficiency of the intensity modulation of the standing wave is recalled, and the condition for suppressing the other polarization mode is developed when coalescence of the two states is not reached. A theoretical exploration of exceptional points in polarization space is done for different optical elements inserted in the resonator, demonstrating new possible combinations for obtaining an exceptional point.   

With regard to the experiments carried out, we tested the emission properties of our laser device consisting of a Yb$^{3+}$-doped YAG amplifier medium in a reduced-size cavity with anisotropic mirrors operating in the quasi-continuous regime. Polarization eigenstate and oscillation threshold measurements are used to validate our simple approach to polarization eigenstate determination based solely on the optical response of the mirrors. Measurements of the transverse beam profile quality assure us that operation is in the fundamental TEM$_{00}$ transverse mode. In addition, the quality of the polarization state is measured by the degree of polarization, and the efficiency of conversion from pump power to output power is measured. Finally, the emission spectrum is characterized and we observe single-mode emission for certain $\alpha$ angles with good stability, showing that this simple architectural device is another way of obtaining a single-mode laser.


\addcontentsline{toc}{section}{\protect\numberline{}Abstract}%


\pagebreak


\singlespacing
\tableofcontents


\newpage


\listoftables


\addcontentsline{toc}{section}{\protect\numberline{}Liste des tableaux}%


\pagebreak


\listoffigures

\addcontentsline{toc}{section}{\protect\numberline{}Liste des figures}%





\pagebreak


\section*{Liste des symboles}


\addcontentsline{toc}{section}{\protect\numberline{}Liste des symboles}%


\pagebreak


\onehalfspacing


\pagenumbering{arabic}


\thispagestyle{empty}
\section{INTRODUCTION AUX LASERS}

	En 1917, Einstein a prédit l'existence de l'émission au cœur du phénomène laser, l'émission stimulée \cite{1917_Einstein,stimulated_emission_Einstein}, ce qui déclencha plusieurs travaux de recherche qui menèrent à l'invention du laser. Bien que les lasers et d'autres sources de lumière produisent tous de l'émission stimulée, elle est cruciale pour le fonctionnement du laser. En effet, comparé à l'émission stimulée produite par une source thermique comme les diodes électroluminescentes et tubes fluorescents, l'émission stimulée dans un laser est amplifiée et est concentrée sur une petite gamme de fréquence qui sont les fréquences de résonance du résonateur.  La première source fonctionnant sur ce principe a été inventée 38 ans après la prédiction d'Einstein par Charles Townes sous le nom de maser ("microwave amplified stimulated emission radiation") \cite{maser_Townes}. Comparée à plusieurs processus d'émission spontanée causés par la désexcitation spontanée des atomes, qui produisent chacun un photon pas nécessairement dans le même mode, c'est-à-dire des photons discernables l'un de l'autre, l'émission stimulée causée par l'interaction entre photons et un atome excité de la même énergie produit deux photons qui sont dans le même mode (photons indiscernables) à la fin du processus. Ce processus donne la possibilité d'augmenter le nombre de photons dans le même mode par l'interaction de ces photons avec d'autres atomes excités. Évidemment, l'amplification par émission stimulée requiert un premier photon qui est produit par l'émission spontanée. Ce premier photon doit être de la même énergie que la différence d'énergie d'une transition électronique de l'atome excité pour permettre l'interaction entre le photon et l'atome excité et dans une direction qui peut mener à l'amplification par le confinement et l'interaction avec d'autres atomes.  

Cependant, l'amplification de l'émission stimulée requiert un milieu où il y a une plus grande population d'atomes excités que d'atomes dans un niveau bas d'énergie, ce que l'on nomme l'inversion de population. L'interprétation de cette condition est qu'il y a un plus grand nombre d'atomes excités disponibles pour l'émission stimulée que d'atomes disponibles pour l'absorption, l'effet net étant qu'il y a plus de photons produits qu'absorbés. Cette condition peut seulement être satisfaite pour des systèmes d'atomes à au moins trois niveaux d'énergie puisque la puisque la section efficace spectroscopique d'absorption et d'émission entre deux transitions électroniques est égale \cite{stimulated_emission_Einstein}. Cela signifie qu'un système à deux niveaux ne peut atteindre qu'une proportion égale d'atomes excités et d'atomes dans le niveau bas, quelle que soit l'intensité du pompage \cite{laser_chp6}. Par contre, cette discussion ne tient pas compte des oscillations de Rabi, qui est un effet qui peut ce produire dans un système quantique à deux niveaux en présence d'un champ électromagnétique qui a pour effet de créer une variation périodique de la population du niveau excité pouvant atteindre, pour de courte durée de temps, 100\% de toute la population dans un niveau excité en théorie. Pour que ces oscillations soit un effet important dans le fonctionnement du laser, le temps de vie du niveaux excité doit être asser grand pour que les oscillations de Rabi puisse ce produire sans avoir une grande partie des atomes qui se sont désexcité, c'est à dire que le temps de cohérence du niveau excité soit beaucoup plus grand que la période des oscillations Rabi. Par ailleurs, la fréquence des oscillations de Rabi augmente avec l'intensité du champ appliqué, de sorte que dans un régime de pompage intense, les oscillations de Rabi peuvent devenir un phénomène important \cite{siegman_225_rabi}. Lorsque l'intensité du champ électromagnétique est faible et que le temps de vie du niveau est petit, la modélisation du système laser par les équations de débit de Statz-DeMars est applicable, et un système à quatre niveaux d'énergie est idéal pour obtenir une absorption négative (gain). En ce qui concerne la condition d'oscillation, elle est remplie lorsque le taux d'émission stimulée dans un mode est égal au taux de pertes de photons dans ce mode du résonateur \cite{laser_chp_12}. Dans le cas de la cavité simple à deux miroirs, un milieu amplificateur et une source de pompage, la réflectance et la géométrie des miroirs jouent un grand rôle pour atteindre la condition d'oscillation en permettant de réduire le taux de perte de photons.

\subsection{Les modes d'une cavité laser}

Les modes de cavité sont des solutions des équations de Maxwell qui se reproduisent après un cycle complet dans la cavité, en accumulant seulement une phase multiple de $2\pi$. Ces solutions peuvent admettre différentes distributions de champ dans le plan perpendiculaire (transversal) à la direction de propagation, différentes orientations d'oscillation du champ (polarisation), différente direction de propagation et fréquences. La condition de résonance dans un résonateur simple à onde stationnaire est écrite comme \cite{laser_chp4}: 
	
\begin{equation}\label{mode_resonateur}
	\frac{2\pi\nu_{mnp}L}{c}-(m+n+1)(\psi(z_{2})-\psi(z_{1})) = p\pi
	\end{equation}	
	
	où $m$ et $n$  sont des entiers positifs représentant les différents modes transverses, $p$ aussi un entier positif représentant les modes longitudinaux de la cavité, $\nu_{mnp}$ est la fréquence d'un mode transverse et longitudinal quelconque, $L$ la longueur optique du résonateur, $c$ est la vitesse de la lumière dans le vide et $\psi(z_{2})$, $\psi(z_{1})$, le déphasage de Gouy aux positions $z_{1}$ et $z_{2}$ correspondant à l'interface des deux miroirs  \cite{laser_chp4}. La longueur optique diffère de la longueur physique mesurée, $L_{mat}$ par $L=nL_{mat}$ où n est l'indice de réfraction du matériau, car la longueur d'onde dans un matériau est plus petite et l'onde effectue donc plus de périodes que si elle parcourait la même distance dans le vide.
	
	Dans les prochaines discussions, on verra que le nombre de modes oscillant avec de différent indice $m$, $n$, $p$ et état de polarisation dans lequel un laser émet est intimement lié à sa cohérence. Nous pouvons diviser la notion de cohérence en deux catégories, la cohérence spatiale et la cohérence temporelle. La cohérence spatiale nous indique la capacité d'un faisceau à interférer avec lui-même entre deux positions transverses à la propagation $x$ et $x'$ au même instant et la cohérence temporelle définit la capacité d'un faisceau à interférer avec lui-même entre deux instants $t$ et $t'$ sur la même position du front d'onde ou bien deux position axial $z$ et $z'$ au même moment. Le phénomène d'interférence se produit seulement lorsqu'il existe une relation de phase stationnaire entre deux points spatiaux ou temporels des deux ondes \cite{Etch_P406}.

	Avant l'invention du laser, pour obtenir une source spatialement cohérente, on imitait le rayonnement d'une source ponctuelle en faisant passer par un trou le rayonnement d'une source thermique peu cohérente. Il est simple de voir que pour une source ponctuelle, il existe une relation de phase stationnaire entre toutes les positions du front d'onde produites par cette source, puisqu'elles ont toutes la même origine. On constate alors que l'addition de plusieurs sources ponctuelles émettant des fronts d'onde qui ne sont pas en phase entre eux à l'effet de brouiller les franges d'interférence, c'est-à-dire que la visibilité des franges diminue. Avec l'invention des lasers, la possibilité d'autres types de front d'onde non sphérique, mais quand même parfaitement cohérent est devenue une possibilité. En théorie, tous les modes $m$ et $n$ sont des modes parfaitement cohérents spatialement. Cela nous amène à un fait intéressant concernant la divergence et son lien avec la cohérence spatiale.  \textcolor{red}{On dit souvent qu'une cohérence spatiale élevée conduit à un faisceau faiblement divergent, mais cela n'est pas tout à fait vrai, car lorsque l'ordre $m$ et $n$ augmente, le facteur de qualité du faisceau $M^{2}$, exprimé par le produit de la divergence $\theta$ et de la taille au pincement $w_{0}$, $M^{2}=\frac{w_{0}\theta}{\lambda} $ augmente} \cite{beam_quality_M_Seigman}.
	
	 Approfondissons d'avantage la caractérisation de la composante transversale des modes. Nous pouvons la diviser en deux parties: la distribution du profil d'amplitude et de phase et l'orientation de l'oscillation du champ électrique (polarisation). Dans l'émission laser, plusieurs profils transverses de mode peuvent exister en même temps. La sélection d'un mode transverse peut être faite soit en ajoutant des pertes ou un gain distribués spatialement dans le plan transverse à la propagation de manière à favoriser une distribution transversale plutôt qu'une autre. Puisque les modes transverses n'ont pas la même taille dans une cavité et divergent plus rapidement à mesure que les indices $m$ et $n$ augmentent, il est possible de sélectionner le mode fondamental TEM$_{m=0,n=0}$ (transverse électromagnétique) en créant plus de perte pour tous les autres modes en ajoutant un iris de la taille du mode fondamental. L'autre façon d'obtenir le mode TEM$_{00}$ consiste à exciter uniquement les atomes de la taille du mode fondamental. Par exemple, en pompant optiquement le milieu amplificateur le long du faisceau laser et en ayant le faisceau de pompe de la même taille que le mode fondamental du faisceau de sorties, nous créons plus de gain pour le mode de cette taille et moins pour tout les autre mode transverse qui sont de plus grande taille. 
	
En ce qui concerne l'aspect de la polarisation des modes, les lasers n'adoptent qu'un état bien polarisé qu'en présence d'une anisotropie provenant soit d'un élément diatténuant ou biréfringent. Ceci étant, les lasers adoptent souvent un état de polarisation puisqu'il est difficile de n'avoir aucune anisotropie. Voici quelque exemple d'effets qui causent de l'anisotropie ou de la biréfringence dans la cavité. Il y a certains matériaux amplificateurs anisotropes où la section efficace d'émission, donc le gain, pour un état de polarisation, est plus grande que pour un autre comme dans les matériaux de YVO$_4$ dopée au Nd \cite{gain_diff_pol}, saphir dopée au titane et LiSGaF dopée au Cr$^{3+}$ \cite{Sorokin2004}; nous avons aussi de l'anisotropie induite par les phénomènes de saturation dans un milieu gazeux\cite{polarisation_preference_induce_by_saturation_DELANG,saturation_1pol}, l'application d'un champ magnétique dans les matériaux qui on une réponse magnéto-optique  \cite{magnétique_pol} qui peut causer une anisotropie dans la cavité, ect. Dans certains cas, un laser He-Ne sans fenêtre de Brewster peut avoir un état de polarisation bien polarisé, mais qui varie en fonction du temps qui est peut être causé par la rotation de l'axe avec moins de pertes qui varie en fonction du temps. Dans le chapitre 2, on ferra une analyse des états propres de polarisation avec l'insertion de matériaux anisotrope.
	
Sur le sujet de la cohérence temporelle, une onde sinusoïdale (monochromatique) est parfaitement cohérente temporellement, car il existe une relation de phase stationnaire entre tous les instants $t$ et $t'$. Toutefois, ceci est seulement une construction mathématique, car pour qu'une relation de phase constante soit définie à chaque instant, l'onde doit également avoir existé pendant un temps infini. Pour un exemple plus réaliste, examinons la relation de phase entre des trains d'ondes, c'est-à-dire une séquence temporelle d'ondes sinusoïdales de longueur finie et de phase aléatoire. Il est clair qu'il n'existe une relation de phase constante que sur la durée d'un train d'ondes, puisque par définition le train d'ondes suivant subit un saut de phase. En outre, l'émission laser d'un mode longitudinal peut être considérée comme l'addition d'un grand nombre de trains d'ondes avec des sauts de phase aléatoires. En regardant la corrélation entre cette fonction décrivant le champ de l'émission laser et la même fonction décalée dans le temps, on peut définir la fonction d'autocorrélation. La largeur de la fonction d'autocorrélation est d'autant plus large que la durée d'un train d'ondes sans saut de phase est grande. Cette fonction d'autocorrélation est liée à la densité spectrale de puissance par une transformation de Fourier, comme l'indique le théorème de Wiener-Khinchin \cite{Wiener_Khinchin}. Dans l'exemple donné, on constate que plus la durée entre les sauts de phase est longue, plus on se rapproche d'une onde monochromatique, et donc plus le degré de cohérence est élevé. Il existe donc une relation inverse entre le temps de cohérence et la largeur spectrale \cite{Fundametals_p481}. Puisque la coexistence de modes est intrinsèquement accompagnée d'un accroissement de la largeur spectrale, nous avons besoin de l'émission monomode pour avoir un long temps de cohérence et donc un spectre étroit. Pour clarifier, le terme monomode signifie que l'émission est distribuée autour d'une fréquence centrale et selon un profil transverse et un seul état de polarisation. Cependant, il y a toujours de l'élargissement spectral de l'émission due à des variations de phase et d'amplitude de l'onde causée majoritairement par des effets de changement de longueur optique de la cavité et en petite partie par du bruit quantique qui provient de l'émission spontanée. Si l'on élimine ses derniers effets, on arrive à la limite ultime de la largeur spectrale provenant de l'émission spontanée dans le mode donc qui ne prend en compte que les limitations quantiques et a été faite par Schawlow et Townes \cite{limit_Schawlow–Townes}.		
	
	\noindent
Pour la prochaine discussion, il est intéressant de montrer l'équation suivante qui découle de l'équation \ref{mode_resonateur} pour un mode transverse quelconque:
	
	\begin{equation}\label{eq:ISL}
	\Delta\nu_{ISL}=\frac{c}{2L}
	\end{equation}	
\noindent	
où $\Delta\nu_{ISL}$ est l'intervalle spectral libre du laser qui représente l'espacement des modes axiaux consécutifs de la cavité.
	
	 Souvent dans les lasers, l'ISL est plus petit que la largeur spectrale du profil de gain, donc plusieurs modes sont susceptibles d'osciller. Par contre, l'un de ces modes aura plus de gain que les autres et sera donc dominant. Quand le gain de ce mode est égal aux pertes, on a l'oscillation laser et la classe d'atomes excités participant à l'émission laser n'augmente plus avec l'énergie de pompage \cite{siegman_laser_saturation}. Ceci est le phénomène de saturation. Au-delà de ce taux de pompage, tous les photons produits dans ce mode sont produits par la conversion directe de l'énergie de la pompe à des photons du mode. Dans les milieux amplificateurs où les atomes excités répondre tous de la même façon au pompage optique, que l'on nomme un milieu à élargissement homogène, la population globale d'atomes excités n'augmente pas avec la puissance de pompage après l'émission du mode dominant, ce qui empêche l'augmentation du gain des autres modes susceptibles d'osciller. L'effet sur la courbe de gain avec l'augmentation du taux de pompage à partir d'un taux de pompage inférieur au seuil d'oscillation jusqu'au seuil est que la courbe de gain augmente globalement et atteint un maximum quand on atteint le seuil. L'autre type de milieu amplificateur est à élargissement inhomogène et est défini à titre indicatif car il n'est pas utilisé dans cette thèse, où les populations d'atomes excités sont différentes, de sorte qu'ils n'ont pas la même fréquence de résonance pour leur transition électronique. Ceci cause seulement une certaine classe d'atomes à participer à l'émission d'un mode. L'effet sur le comportement du laser est que seule une classe d'atomes participant à l'émission atteint la saturation, tandis que d'autres populations d'atomes peuvent continuer à augmenter. Au-dessus du seuil de pompage pour le mode dominant, le profil de gain cesse d'augmenter seulement pour la classe d'atomes qui participe à l'émission donc elle sature seulement localement ce qui cause d'autre mode à pouvoir dépasser leurs seuils d'oscillation.
	
 	Dans le paragraphe ci-dessus, on suppose que toute la population d'atomes subit la même saturation par l'intensité du faisceau créé dans le résonateur. Ce n'est pas le cas dans un résonateur à onde stationnaire, c'est à dire un résonateur ou l'énergie du mode varie selon la composante $z$. Les positions à l'intérieur du résonateur où l'intensité du mode le long de l'axe de propagation est minimale créent alors des populations d'atomes excités qui ne voient que la puissance de la pompe, mais pas celle du faisceau produit. Ces populations d'atomes excitées ne subissent alors aucune saturation et peuvent continuer à augmenter avec la puissance de pompage. Cela permet à un autre mode de la cavité d'une fréquence différente de sonder les zones où la population d'atomes excités est élevée, du fait que ces nœuds et ces ventres sont positionnés à des points différents du mode qui oscille déjà dans la cavité. Cela signifie que ce mode non dominant peut exploiter ces régions où la population d'atomes excités est élevée et obtenir un gain assez élevé pour atteindre son seuil d'oscillation. Par cet effet au nom de $creusement$ $spatial$, l'émission multimode se produit même dans des milieux à élargissement homogène et devient plus multimode à mesure que la puissance de pompage augmente, comme le montrent expérimentalement les articles \cite{multimode_creusement_JF,multimode_SHB_1995}. 
 	
\pagebreak
	
	  \begin{figure}[h!]%Le h! assure que la figure reste proche d'ici. autre options t-top, b-bottom, p - float page. Voir par exemple: https://tex.stackexchange.com/questions/39017/how-to-influence-the-position-of-float-environments-like-figure-and-table-in-lat
\centering
\includegraphics[scale=0.8]{taux_pompage_mulitimode.jpg}
\caption{Comportement multimode d'un laser Nd:YLF dû à l'effet du creusement spatial en augmentant le taux de pompage où $r$ est le nombre de fois au-dessus du seuil de pompage. Figure tirée de la référence \cite{multimode_SHB_1995}.}
\label{grap:multi_mode_taux_pompage}
\end{figure}


	Un autre phénomène qui est néfaste à l'émission monomode est les sauts de modes. Les sauts de modes sont causés par des fluctuations de la longueur optique dues à des vibrations mécaniques et du changement de l'indice de réfraction par changement de température \cite{temperature_mode_hopping}. Le saut se produit, car les modes se déplacent dans le profil de gain lorsque la longueur optique change. Quand les modes se déplacent, le mode dominant cède sa place à un mode voisin qui a maintenant le plus de gain. Ce changement apparait comme un saut abrupt dans le spectre de fréquence \cite{mode_hopping_fig}. Aussi, les sauts de mode peuvent être causés par des réflexions parasites provenant de l'extérieur du résonateur qui font entrer le faisceau de nouveau dans le résonateur, ou par des fluctuations de la puissance de pompage \cite{mode_hopping_fig}.

\subsection{Restriction des modes d'une cavité laser}	

Le creusement spatial est un effet indésirable, que l'on retrouve dans tous les résonateurs à ondes stationnaires dont les matériaux sont élargis de manière homogène, car il empêche l'émission monomode. Dans cette thèse, nous nous intéressons à éliminer cet effet à la source. Les premières méthodes discutées c-dessous ne s'attaquent pas à la source du problème, mais aident à rendre les lasers monomode. 

Premièrement, l'insertion de composants dispersifs tel qu'un filtre est une façon d'obtenir un laser monomode \cite{monomode_Fabry_Perot}. Une méthode bien connue consiste à ajouter un étalon Fabry Pérot dans le résonateur dont l'une de ces fréquence de résonance est la même que celle du résonateur, ce qui introduit beaucoup de pertes pour les autres modes qui ne sont pas en résonance avec l'étalon Fabry Pérot. L'intervalle spectral libre (ISL) d'un tel étalon Fabry Pérot est plus grand que celui du résonateur du laser, mais quand le profil de gain est plus large que l'ISL du Fabry Pérot, plusieurs modes sont susceptibles d'osciller. Dans ce cas, il faut ajouter un second étalon Fabry Pérot plus mince que le premier, c'est-à-dire un étalon avec des modes plus espacées que le premier Fabry Pérot et que ces deux étalons Fabry Pérot aient une fréquence de résonance commune qui est aussi un fréquence de résonance du résonateur. D'ailleurs, la difficulté associée avec cette méthode est d'avoir la bonne épaisseur des étalons pour obtenir une fréquences de résonance commune entre les deux étalon en le résonateur \cite{Delsart_monomode}. Sur le même principe que l'insertion d'étalon dans la cavité, on peut insérer un prisme dispersif pour créer un trajet légèrement différent pour différentes fréquences qui cause une diminution du gain des autres modes \cite{monomode_Fabry_Perot}. Il existe également diverses méthodes interférométriques qui ont toutes un effet similaire à l'insertion d'une cavité Fabry Pérot dans le résonateur \cite{mode_selection_Smith1972}.

	Une autre façon d'obtenir un seul mode est d'avoir un grand espacement des modes de la cavité comparé à la largeur du profil de gain. L'analyse du pompage maximal permettant d'obtenir une émission monomode en fonction de l'épaisseur de la cavité est présentée dans l'article \cite{longeur_cavité_pour_single_mode}. De plus, pour obtenir un seul mode dans le profil de gain du matériau amplificateur utilisé dans cette thèse qui est le $\text{Y}_3\text{Al}_5\text{O}_{12}$ (YAG) dopée à l'$\text{Yb}^{3+}$, il faut réduire la taille à moins de 100 microns \cite{petit_monomode}. Le problème d'avoir de si petits résonateur est la diminution de l'efficacité d'absorption de la puissance de pompe. Pour obtenir une puissance de pompe absorbée suffisante, il faut donc utiliser différentes méthodes de pompage afin d'obtenir plusieurs passages du faisceau de pompe dans le milieu amplificateur. Il existe différentes géométries de pompage qui pompent le milieu amplificateur à un angle différent de la normale, ce qui permet d'utiliser d'autres miroirs pour réfléchir le faisceau pompe non absorbée sur le milieu \cite{multipass_pumping}. Enfin, des effets non linéaires comme dans les matériaux absorbants saturables \cite{Q_switch_single_mode} ou la génération de second harmonique \cite{second_harmonique_sinlge_mode}, peuvent contribuer à améliorer la pureté spectrale.

D'autres méthodes s'appuient sur les résonateurs à deux cavités, comme les oscillateurs maîtres couplés à une cavité amplificatrice  (MOPA), qui comportent un oscillateur maître émettant dans un seul mode à faible puissance de sortie, alimentant un amplificateur \cite{Zawischa1999AllsolidstateNS}. Le verrouillage par auto-injection peut également améliorer la pureté spectrale du laser \cite{injection_locking_single_mode,Takano_injection_lock_single_mode}, qui consiste d'avoir une cavité active faiblement couplée à une cavité passif qui à des modes de cavité de largueur spectrale plus petite que la cavité active \cite{explanation_self_injection_locking}. 

d'autre type de laser comme les lasers à rétroaction distribuée (DFB) obtiennent une émission monomode par le fait qu'il on un réseaux périodique ou un multicouche de Bragg présent dans toutes le long de la cavité ce qui à pour effet de créer un rétroaction optique tout le long du résonateur causant une condition de résonance plus strict \cite{Kitamura_84_DFB,DFB_single_mode}. Il y aussi les lasers à cavité verticale émettant par la surface (VCSEL) \cite{VCSEL_single_mode}, qui ont une bonne efficacité de conversion de l'énergie et un grand espacement des modes grâce à la courte longueur de leur cavité, et les lasers à cristaux photoniques émettant par la surface (PCSEL) \cite{PCsel}. D'autres ont choisi de combiner des méthodes pour obtenir une largeur de raie plus étroite et un meilleur contrôle du spectre d'émission, par exemple en utilisant le verrouillage par auto-injection d'un laser DFB sur un microrésonateur en anneau \cite{self_injection_locked_DFB_laser,self_injection_locked_DFB_laser_nature} ou une diode laser couplée à un filtre Vernier composé de deux microrésonateurs annulaires \cite{Vernier_filter_self_injection_locked_laser}. 

Des développements d'architectures de laser exotiques issues de la mécanique quantique et de la physique des solides sont aussi une perspective intéressant pour obtenir l'émission monomode. Les lasers tels que les nanorésonateurs BIC (bound-state-in-a-continuum), qui exploitent les interférences de type Fano pour obtenir un confinement élevé du rayonnement \cite{BIC_single_mode,BIC_single_mode_2}, les lasers à isolant topologique, qui assurent une propagation unidirectionnelle sans l'application d'un champ magnétique \cite{topological_laser}, et les cavités à microanneau qui ont une symétrie parité-temps (PT), où le gain et la perte sont ajustés spatialement  \cite{PT_symmetric_laser_gain_loss} sont toutes des méthodes qui ont permis d'obtenir une grande pureté spectrale au prix d'une architecture plus complex, mais en gagnant sur la compacité du résonateur.
	
Une façon d'éliminer le creusement spatial à la source est d'utiliser des résonateurs en anneau qui éliminent la propagation de la lumière dans une direction. En ayant une onde se propageant dans une seule direction, aucune onde stationnaire n'est formée dans la cavité, ce qui élimine le creusement spatial. Pour obtenir une préférence de direction de propagation, un rotateur de Faraday, un polariseur et une lame quart d'onde sont insérés dans la cavité \cite{first_unidirectional_laser,unidirection_ring_laser_experiment}. Le rotateur de Faraday et la lame quart d'onde garantissent que les deux ondes contra-propagatrices (contra-directionnelle) n'ont pas les mêmes états propres de polarisation, et le polariseur est orienté à un angle qui élimine l'un de ces états propres de polarisation. Les chercheurs Thomas Kane et Robert Byer ont mis en œuvre avec succès cette méthode pour obtenir un laser monomode dans leur conception de laser en anneau unidirectionnel monolithique \cite{ring_laser_single_mode}. Ils réalisent la fonctionnalité du rotateur de Faraday en appliquant un champ magnétique à un milieu actif de Nd:YAG (matériau qui a une constante de Verdet non nule), la fonctionnalité de l'élement biréfringent (diretardant) est obtenue par le fait que la propagation du faisceau à l'intérieur du millieu actif n'est pas dans un plan et par les réflexions totales internes sur les surfaces coupées du cristal qui fait en sorte que les polarisation s et p n'ont pas le même déphasage à la réflection, et la réflexion sur une des surfaces où la réflexion totale interne ne se produit pas agit comme l'élément diatténuant,  c'est-à-dire une différence de réflectance entre les polarisations s et p. Ce dispositif fonctionne très bien et est encore utilisée aujourd'hui, mais elle exige une grande précision dans la géométrie du cristal, car aucun alignement des surfaces réfléchissantes n'est possible une fois le cristal taillé. Malgré son grand succès, cette méthode n'est pas dépourvue de problèmes. En effet, l'un des problèmes qui limite le taux de puissance de sortie dans le régime monomode est la biréfringence thermique et l'effet de lentille thermique qui peut faire osciller d'autres modes transversaux. \cite{transverse_mode_uni_direct_laser}. Une façon de contrer ces effets est d'avoir un résonateur instable à des taux de pompage faibles qui devient stable à des taux de pompage élevés grâce à l'effet de lentille thermique \cite{ring_laser_single_mode_high_power}. À mesure que l'on trouve des solutions pour obtenir l'émission monomode à de plus haute puissance de sortie, les architectures des lasers deviennent plus complexes, difficiles et dispendieuses. 		  
	
	Les travaux pionniers de Siegman et Evtuhov utilisent une autre méthode qui élimine le creusement spatial à la source, non pas en éliminant une des onde ce propageant dans de direction opposé, mais en éliminant les franges d’interférence produite par les deux ondes contra-propagatrices \cite{Siegman_twisted_mode}. L'idée repose sur la modification de l'état de polarisation des deux modes pour que les ondes contra-propagatrices d'un mode soient mutuellement orthogonales. Les états propres sont modifiés par l'ajout de deux lames quart d'onde placées de part et d'autre du milieu amplificateur, créant deux ondes contra-propagatrices ayant toutes deux une polarisation circulaire droite, ou une polarisation circulaire gauche. Ces ondes contra-propagatices sont mutuellement orthogonales et ceci produit une intensité uniforme selon la direction de propagation du faisceau dans le résonateur. Cependant, puisqu'il n'y a pas d'élément diatténuant dans la cavité, les deux modes de polarisation ont les mêmes pertes donc sont tous les deux susceptibles d'osciller. Avec le désalignement des axes principaux des lames quart d'onde, les modes propres de polarisation accumuleront une différence de phase qui est reliée à la phase géométrique, puisque les deux modes de polarisation font un parcours de sens inverse sur la sphère de Poincaré. Par conséquent, une différence de phase entre les modes propres de polarisation fait en sorte que les fréquences de résonance vont être à de différentes fréquences pour chaque mode propre. Bien que le phénomène de saturation avec l'onde stationnaire éliminée devrait supprimer les autres modes dans la cavité d'osciller, la compétition féroce en raison de leur gain presque égal fait en sorte que les deux modes de polarisation peuvent co-osciller. 
	
\pagebreak
Une suite logique pour l'amélioration de ce type de résonateur est l'ajout d'élément diatténuant pour augmenter les pertes de l'état de polarisation qu'on souhaite éliminer. La première solution repose sur le fait que les modes de polarisation entre le miroir et la lame quart d'onde sont rectiligne; donc, placer un polariseur à cet endroit dans le résonateur peut éliminer un mode propre de polarisation et laisser l'autre mode inchangé \cite{gain_SHB_schumann,twisted_mode_Zhang_2010,twisted_mode_frequency_mixing_2012,twisted_mode_Adams_93}. Cette architecture de résonateur rend difficile l'intégration de tous les composants optiques dans la fonction des miroirs, puisque les axes principaux de la diatténuation et de la biréfringence (diretardance) doivent être à $45\degree$ l'un de l'autre; ainsi la conception d'un élément optique aillant ces propriétés est inconnu à notre connaissance. Pour cette raison, ces cavités doivent être plus longues et il n'est pas possible de les concevoir en une seule pièce, ce qui limite la stabilité de ces lasers et leur miniaturisation. 

La deuxième solution consiste à placer l'élément diatténuant entre le miroir et la lame quart d'onde, mais cette fois en l'alignant sur les axes principaux de la lame quart d'onde. \cite{Point_except_JF}. Cette solution apporte un comportement des états de polarisation complètement différent de la première. La première chose à constater est que les états de polarisations sont modifiés et la modulation de l'onde stationnaire varie avec l'angle relatif entre les axes propres des lames quart d'onde $\alpha$ \cite{Point_except_JF}. De plus, les états de polarisation ne sont plus orthogonaux et on verra qu'ils peuvent coalescer. Un des grands bénéfices de ce type de résonateur est la facilité d'insertion de l'élément diatténuant et biréfringent dans un seul composant comme une couche mince \cite{controle_gain_SHB_JF}. Ainsi la miniaturisation de la cavité devient une perspective intéressante et envisageable. Cette architecture de résonateur est celle sur laquelle nous nous concentrons dans ce manuscrit et sera étudiée plus en détail dans les chapitres suivants.	

	\subsection{But de la thèse}		
	Ce manuscrit s'intéresse à l'obtention d'un seul mode dans une cavité laser par le fonctionnement du laser opérant près d'un point exceptionnel, c'est-à-dire un point où les deux états propres de polarisation coalescent et avec la suppression de l'onde stationnaire dans la cavité en favorisant des ondes contra-propagatrice de polarisations orthogonales. Le comportement des états propres de ce laser avec l'angle $\alpha$ est analogue à d'autres systèmes optiques présentant une symétrie parité-temps, mais dans l'espace de polarisation. Le groupe de recherche de Jean-François Bisson a démontré l'existence de la symétrie parité temps dans l'espace de polarisations par l'observation de la coalescence de mode de polarisation \cite{Point_except_JF}. Cependant, comme les modes de la cavité sont très proches en termes d'espacement des fréquences et que l'architecture du résonateur ne permet pas un plus grand espacement des modes, ils obtenaient des sauts de mode. Une architecture qui permet un raccourcissement du résonateur et qui intègre tout des parties de façon monolithique est intéressante pour avoir un laser monomode et une bonne stabilité de la longueur du résonateur, donc une bonne stabilité fréquentielle. Nous réalisons ceci à l'aide de réseaux de diffraction gravés à la surface des miroirs \cite{LiLapointe}. Ces réseaux de diffraction résonants permettent d'avoir un miroir anisotrope et biréfringent à une incidence normale.

\pagebreak
\thispagestyle{empty}
 \section{MODES PROPRES DE POLARISATION D'UN RÉSONATEUR}

	Le but de ce chapitre est de formuler une théorie qui nous permet de prédire le comportement complexe d’une cavité laser. Le comportement d’émission d’un laser, notamment les fréquences, polarisations et profils transverses des modes, est affecté par de nombreux phénomènes différents tels que le creusement spatial, la compétition entre différents modes transversaux, la compétition entre les modes de polarisation par divers phénomènes et les effets thermiques comme changement de la longueur optique, effets de lentilles thermiques et biréfringence induite thermiquement, ce qui rend difficile à expliquer le comportement du laser. Nous proposons une approche simple pour prédire son comportement en utilisant un formalisme de Jones pour étudier les états propres de polarisation de la cavité. Ce formalisme nous permet de calculer la modulation de l'onde stationnaire par l'interférence qui se produit entre un état propre de polarisation et son onde contra-propagatrice, et de simuler la proximité des états propres de polarisation et la différence de perte entre les deux modes de polarisation, qui sont de bonnes mesures pour prédire la pureté de l'émission. 
	
	Dans la section \ref{sec:Mode propre de polarisation dans les résonateurs avec élément biréfringent}, nous étudions les états propres de polarisation d'un résonateur contenant deux lames à retard (élément diretardant). On montre que cela élimine l'effet du creusement spatial dans le résonateur. Ensuite dans la section \ref{sec:Mode propre de polarisation dans les résonateurs avec élément diretardant et diatténuant}, on ajoute un élément qui a une différence de réflectance pour deux polarisations orthogonal (élément diaténuant). Cela nous amène à un phénomène intéressant qui ce produit dans ces résonateur quand la diretardance et diaténuation des éléments dans le résonateur sont d'une tel valeur et que l'orrientation des axes principaux des différents élément sont à une certain angle, une coalescence des états propres de polarisation est possible. Cela est explorer dans la section \ref{sec:Point exceptionnel dans les résonateurs}. La dernière section \ref{sec:condition d'émission monomode dans les résonateurs à onde stationnaire} on démontre les conditions pour avoir l'émission monomode en considèrant que le creusement spatial est l’effet de plus grande importance qui rend possible l'émission de plusieur mode avec des indices longitudinaux différent dans les milieux à élargissement homogène. On quantifiera le degré auquel le creusement spatial doit être réduit pour garantir l'émission monomode, peu importe le taux de pompage. Dans cette analyse, on considère que la technique de "mode matching" employée pour supprimer les modes transverses supérieurs fonctionne, peu importe les autres parramètres de fonctionnement du laser. Elle consiste à adapter la taille du faisceau de pompe sur le milieu amplificateur à la même taille du mode transverse fondamental TEM$_{00}$ du faisceau produit.

	D'une façon générale, un état propre d'une matrice est un état qui reste inchangé à une constante près lorsque cette matrice est appliquée sur lui, c'est-à-dire $\textbf{M}\ket{u} = \lambda_{\pm}\ket{u}$ où $\textbf{M}$ est une matrice du trajet complèt, $\ket{u}$ est un vecteur propre de $\textbf{M}$ et $\lambda_{\pm}$ est une valeur propre. La signification des vecteurs propres ou, dans notre cas, des modes propres dans une cavité est un mode qui se réplique après avoir fait une trajectoire complète. On entend par trajectoire complète un parcours dans la cavité qui revient à sa position initiale en ayant la même direction de propagation. La différence entre un élément optique en dehors d'un résonateur et d'un élément optique dans un résonateur est qu'en dehors d'un résonateur, la réponse optique de cet élément dépend de la polarisation d'entrée, alors que dans un résonateur, toutes les polarisations d'entrée produit par l'émission spontanée tendent de plus en plus vers un état propre de polarisation à chaque trajectoire complète. Ainsi, les états propres de polarisation sont les états qui obtiennent le plus d'amplification. En examinant uniquement la composante de polarisation des modes, on peut déduire la meilleure configuration du résonanteur pour obtenir un laser spectralement de meilleure pureté en se basant sur l'efficacité de suppression de la modulation de l'onde stationnaire et de la différence de pertes entre les deux états propres de polarisation.	

	
	\subsection{Mode propre de polarisation dans les résonateurs avec élément diretardant}
	\label{sec:Mode propre de polarisation dans les résonateurs avec élément biréfringent}
	
Le travail fait dans cette section s'inspire des travaux sur le résonateur à mode torsadé de Seigman et Evtuhov \cite{Siegman_twisted_mode} et puis des travaux faits par J.F. Bisson et K. Amouzou sur les modes torsadés avec l'insertion de diatténuation dans la cavité \cite{controle_gain_SHB_JF}. Commençons par analyser un cas plus simple de  résonateur que celui utilisé pour nos expériences avec seulement deux composants biréfringents ajoutés de chaque côté du milieu amplificateur pour montrer l'effet de la biréfringence sur les modes propres. En particulier, nous allons d'abord étudier le cas standard du résonateur à mode torsadé de Siegman et Evtuhov qui inclut dans la cavité isotrope une lame quart d'onde de chaque côté du milieu amplificateur. Chacune de ces lames quart d'onde a deux axes principaux orthogonaux $(x_{p},y_{p})$ et $(x_{s},y_{s})$ où l'indice $s$ désigne la lame plus près du miroir de sortie et l'indice ($p$) désigne celle du miroir de pompe et l'orientation relative entre ces axes est l'angle $\alpha$ figure \ref{grap:mode_torsader}. 
	
	La matrice générale d'une lame biréfringente linéaire avec son axe rapide orienté selon $x_{p}$ ou $x_{s}$ est:
	
	\begin{equation}\label{eq:quart_onde}
	\textbf{B}=e^{i\phi_{B}}\begin{bmatrix} 1 & 0 \\ 0 & e^{i\delta} \end{bmatrix}	
	\end{equation}	
	
	\noindent
	avec $\phi_{B}$ qui représente la phase globale accumulée lors du parcours dans la lame biréfringente, $\delta$ qui représente le déphasage entre les axes propres et avec un $\delta=\pi/2$ on retrouve la matrice d'une lame quart d'onde.
	
	 Pour l'effet des miroirs isotropes, ils ne font que changer la direction de propagation $(x,y,z)\rightarrow(x'=x, y'=-y, z'=-z)$ montré à la figure \ref{grap:mode_torsader}, ajouter une phase globale, $\phi_{M}$ et changer l'amplitude du faisceau à l'intérieur du résonateur par les réflectances, $r$, comme:
	
	\begin{equation}\label{eq:miroir}
	\textbf{M}_{(x,y)\rightarrow(x',y')}=e^{i\phi_{M}} \vert r\vert \begin{bmatrix} 1 & 0 \\ 0 & -1  \end{bmatrix}	
	\end{equation}
	 
\begin{figure}[h!]%Le h! assure que la figure reste proche d'ici. autre options t-top, b-bottom, p - float page. Voir par exemple: https://tex.stackexchange.com/questions/39017/how-to-influence-the-position-of-float-environments-like-figure-and-table-in-lat
\centering
\includegraphics[scale=0.4]{résonateur_avec_axe.jpg}
\caption{Insertion de lame à retard d'onde dans un résonateur à deux miroirs}
\label{grap:mode_torsader}
\end{figure}	

	\noindent
	On choisit de représenter les vecteurs propres et toutes les matrices de transformation dans une base bissectrice des axes propres des deux lames $(x,y,z)$. Un avantage de ce formalisme est démontré par l'équation \ref{eq:changement_base} montrant que toutes les matrices de changement de base $\textbf{R}$ nécessaire sont la même transformation et on peut se convaincre de cela d'après la figure \ref{grap:mode_torsader}.
	
	\begin{equation}\label{eq:changement_base}   
	\begin{gathered}
	\textbf{R}_{\scaleto{(x,y)\rightarrow(x_{s},y_{s})}{12pt}}=\textbf{R}_{\scaleto{(x_{p},y_{p})\rightarrow(x,y)}{12pt}} = \textbf{R}_{\scaleto{(x',y')\rightarrow(x'_{p},y'_{p})}{12pt}}= \textbf{R}_{\scaleto{(x'_{s},y'_{s})\rightarrow(x',y')}{12pt}} \\ \equiv \textbf{R}_{\scaleto{\frac{\alpha}{2}}{12pt}}=\begin{bmatrix} \text{cos}(\alpha/2) & \text{sin}(\alpha/2) \\ -\text{sin}(\alpha/2) & \text{cos}(\alpha/2)  \end{bmatrix}
	\end{gathered}
	\end{equation}
	
	Pour ce qui est de la propagation entre les lames biréfringentes, le milieu amplificateur utilisé, Yb$^{3+}$ :YAG  et l'espace libre sont des milieux isotropes donc cette transformation est proportionnelle à la matrice identité et s'écrit comme
	
	\begin{equation} \label{eq:propagation}
	\begin{gathered}
	\textbf{S}=\vert A\vert e^{i\phi_S}\begin{bmatrix} 1 & 0 \\ 0 & 1  \end{bmatrix}  \\
	 \phi_S=\frac{2\pi}{\lambda}(L_{opt})
	 \end{gathered}
	\end{equation}
	
	où $\phi_S$ représente une accumulation de phase qui dépend de la longueur optique, $\vert A\vert$ sont des pertes distribuées lors de la propagation et  $L_{opt}$ qui prend en compte l'indice de réfraction du matériau traversé et dépend de la longueur d'onde d'émission $\lambda$. Maintenant que nous avons toutes les transformations dans un aller-retour, la matrice aller-retour composée de deux lames diretardantes et deux miroirs isotropes s'écrit comme:
	
	\begin{equation} \label{eq:jone_aller_retour_mode_torsadé_parfait}
	\begin{gathered}
	\textbf{J}_{AR}=\textbf{SR}_{\scaleto{\frac{\alpha}{2}}{12pt}}\textbf{B}_{p}\textbf{M}_{p}\textbf{B}_{p}\textbf{R}_{\scaleto{\frac{\alpha}{2}}{12pt}}\textbf{SR}_{\scaleto{\frac{\alpha}{2}}{12pt}}\textbf{B}_{s}\textbf{M}_{s}\textbf{B}_{s}\textbf{R}_{\scaleto{\frac{\alpha}{2}}{12pt}}  \\
	\text{avec les changements de base suivants, lecture de droite à gauche} \\
	(x,y)\leftarrow(x_{p},y_{p})\leftarrow(x'_{p},y'_{p})\leftarrow(x',y')\leftarrow(x'_{s},y'_{s})\leftarrow(x_{s},y_{s})\leftarrow(x,y)	
	 \end{gathered}
	\end{equation}

	Dans le cas étudié par Siegman des résonateurs à mode torsadé, avec une lame quart d'onde, $\delta=\pi/2$, on a $\textbf{BMB}=rI$ ce qui simplifie la matrice d'aller-retour 
	
	\begin{equation} \label{eq:propagation}
	\begin{gathered}
	\textbf{J}_{AR}=\text{exp}(i\phi)  \vert r \vert \vert A \vert^{2} \begin{bmatrix} \text{cos}(2\alpha) & \text{sin}(2\alpha) \\ -\text{sin}(2\alpha) & \text{cos}(2\alpha)  \end{bmatrix}
	 \end{gathered}
	\end{equation}
	\noindent
	où $\phi$ est la phase globale accumulée lors d'un aller-retour et $\vert r \vert$ englobant la réflexion des deux miroirs. Puisque cette matrice est simplement une matrice de rotation, toute polarisation rectiligne sera tournée d'un angle $2\alpha$ et toute polarisation elliptique aura leurs axes majeurs tournés de $2\alpha$ lors d'un aller-retour. Alors, les seules polarisations qui demeurent inchangées sont les polarisations circulaires droite et gauche
	
	\begin{equation} \label{eq}
	\begin{gathered}
	 \ket{u_{1}}=\frac{1}{\sqrt{2}}\begin{bmatrix} 1 & -i \end{bmatrix}^{T} \\
	 \ket{u_{2}}=\frac{1}{\sqrt{2}}\begin{bmatrix} 1 & i \end{bmatrix}^{T}
	 \end{gathered}
	\end{equation}	
	
	\noindent
	et les valeurs propres respectives sont 
	
	\begin{equation} \label{eq:val_propres_torsadé}
	\begin{gathered}
	 \lambda_{\pm} = \vert r \vert  \vert a \vert^{2}  \text{exp}(i(\phi \pm 2\alpha)) \\
	 \end{gathered}
	\end{equation}
	
		\noindent
Lorsque $\alpha = m\pi /2$ où $m$ est un nombre entier, le système est dégénérer donc la matrice devient l'identité, de sorte que le laser n'aura pas de préférence d'état propre de polarisation.
	
	Il y a quelques informations importantes que l'on peut déduire des valeurs propres, notamment, l'intervalle de fréquence entre les deux modes et les pertes que les modes subissent par le fait que la réflectance des miroirs est différente de 100\%. La différence de phase accumulée lors d'un aller-retour entre les deux états propres est donnée par l'angle entre les deux valeurs propres dans le plan complexe comme:
		
	\begin{equation} \label{eq:diff_phase}
	\begin{gathered}
	\Phi = \text{arctan}\left(\frac{\Im(\lambda_{+})}{\Re(\lambda_{+})}\right)-\text{arctan}\left(\frac{\Im(\lambda_{-})}{\Re(\lambda_{-})}\right)
	 \end{gathered}
	\end{equation}
	
\noindent	
	 où $\text{arctan}$ est la fonction $\text{arctan}$ à quatre quadrants, c'est-à-dire qu’elle différencie entre les quatre quadrants du plan cartésien. La différence de phase nous indique que pour le même mode $m, n$ et $p$, les deux modes de polarisation ne peuvent pas être à la même fréquence. Ceci découle du fait que, pour qu'un état propre de polarisation soit en résonance, la phase accumulée sur un trajet complet doit être un multiple de $2\pi$, la fréquence doit donc s'adapter pour satisfaire à la condition de résonance. On en déduit que l'intervalle spectral entre deux états propres de polarisation est d'une valeur de 
	 
	\begin{equation} \label{eq:diff_freq_mode_pol}
	\begin{gathered}
\Delta \nu_\Phi = \frac{\Phi}{2 \pi} \Delta\nu_{ISL}
	 \end{gathered}
	\end{equation}	 
	 
L'autre information que l'on peut sortir des valeurs propres est les pertes dans un aller-retour que subissent les états propres. La norme des valeurs propres indique le degré d'étirement ou de contraction que subit un vecteur propre lorsque sa matrice propre est appliquée. Dans notre cas, elle indique de combien l'amplitude de l'onde électromagnétique est diminuée lors d'un aller-retour dans le résonateur passif, c'est à dire sans millieu amplificateur. La différence de pertes entre les deux modes en termes de l'intensité du champ, donnée par $\vert (1-\vert \lambda_+ \vert^2) - (1-\vert \lambda_- \vert^2) \vert$, aide à supprimer un des modes de polarisation, mais ne garantit pas l'annulation à cause des effets de creusement spatial. Dans ce cas sans diatténuation et de résonateur à mode torsadé, les deux modes de polarisation ont une norme de leurs valeurs propres égales, équation \ref{eq:val_propres_torsadé}, donc ce laser est susceptible d'émetre dans deux modes de polarisation.

	Une quantité de grande importance pour prédire le comportement de l'émission du laser est la modulation de l'intensité de l'onde stationnaire à l'intérieur du résonateur puisqu'elle nous indique l'ampleur que prend le phénomène du creusement spatial. Cette onde stationnaire est causée par l'interférence qui se produit entre les deux ondes contra-propagatrice d'un mode. Pour quantifier l'interférence qui se produit entre deux ondes, la visibilité des franges d'interférence est utilisée.  Par exemple, pour deux ondes contra-propagatives de polarisations orthogonales, la visibilité sera nulle, c'est-à-dire pas de variation de l'intensité en ce déplacent selon l'axe de propagation des deux faisceaux, en considérant un milieu sans absorption ni de gain. D'autre part, dans l'example de la figure \ref{grap:mode_torsader}, les deux états de polarisation circulaire sont superposés pour produire un état de polarisation rectiligne avec une orientation fixe à une position $z$, mais l'orientation de la polarisation rectiligne change avec $z$. C'est de cela que le nom de résonateur à mode torsadé prend son sens.

Commençons par déterminer l'intensité de l'onde à l'intérieur de la cavité pour un état propre. Utilisons un formalisme de Jones, avec un terme oscillant en $z$ pour que l'on puisse déterminer les oscillations de l'onde stationnaire en $z$. prenons un vecteur de Jones quelconque $E_{z} \exp (i k z)\left|u_{z}\right\rangle_{z}$ et sont onde contra-propagatrice $E_{z'} \exp \left(i k z' \right)\left|u_{z'}\right\rangle_{z}$ avec l'indice $z$ à l'extérieur du ket montrant la base du vecteur et $z'=-z$ et $E_z \text{ et } E_{z'}$ étant l'amplitude de l'onde avec une valeur réel positif. L'intensité totale est donnée par le produit scalaire entre la superposition de ces deux ondes \cite{Pancharatnam}. 
	
	\begin{equation} \label{eq:onde_stationnaire}
	\begin{gathered}
	I(z)= \left(E _ { z } e ^ { - i k z } \left\langle\left.u_{z}\right|_{z}+E_{z^{\prime}} e^{-i k(-z)}\left\langle\left. u_{z^{\prime}}\right|_{z}\right)   \cdot \left(E_{z} e^{i k z)}\left|u_{z}\right\rangle_{z}+E_{z^{\prime}} e^{i k(-z)}\left|u_{z^{\prime}}\right\rangle_{z}\right)\right.\right. \\
=I_{z}+I_{z^{\prime}}+2 \sqrt{I_{z} I_{z^{\prime}}}\left(\left|\left\langle\left. u_{z}\right| u_{z^{\prime}}\right\rangle_{z}\right|\right) \operatorname{Re}\left\{e^{2 i k z+i \arg \left(\left\langle\left. u_{z}\right| u_{z^{\prime}}\right\rangle_{z}\right)}\right\}
	 \end{gathered}
	\end{equation}

\noindent
où $I_{z}$ et $I_{z^{\prime}}$ sont l'intensité des ondes oscillant dans la direction $z$ et $z'$ donné par leur amplitude de champ respectif élever au carré. On remarque que la partie oscillante de l'intensité dépend du produit scalaire entre les ondes contra-propagatrices, ce qu'on nomme le facteur de contraste 

\begin{equation} \label{eq:fact_cont}
	\begin{gathered}
|\gamma|=\vert \bra{u_z}\ket{u_{z\prime}}_z \vert 
	 \end{gathered}
	\end{equation}

\noindent
On note que les états propres de polarisation sont exprimés à des positions spécifiques dans le milieu amplificateur. Pour calculer $\vert \gamma \vert$, on doit avoir le vecteur propre se propageant dans la direction $z'$ qui est donné en propageant le vecteur propre dans la cavité pour qu'il ce retrouve au même endroit mais avec sa direction de propagation inversé.

\begin{equation} \label{eq:propagation}
	\begin{gathered}
	\mathbf{R}_{\frac{\alpha}{2}} \mathbf{B} \mathbf{M}_{(x, y, z) \rightarrow\left(x^{\prime}, 		y^{\prime}, z^{\prime}\right)} \mathbf{B R}_{\frac{\alpha}{2}} E_{z} \exp (i k z)\left|u_{z}			\right\rangle_{z}    =    E_{z^{\prime}} \exp \left(i k z^{\prime}\right)\left|u_{z^{\prime}}\right\rangle_{z^{\prime}}
	 \end{gathered}
	\end{equation}

\noindent
	On remarque que le vecteur obtenu est exprimé dans la base $z'$, il faut donc faire un changement de base qui consiste à faire une rotation de 180$\degree$ des trois axes dans le plan que forment les axes $y,z$. Cette transformation est équivalente à la transformation par la réflexion sur un miroir. 


\begin{equation} \label{eq:propagation}
	\begin{gathered}
	\begin{bmatrix} 1 & 0 \\ 0 & -1  \end{bmatrix}_{(x',y',z')\rightarrow (x,y,z)}\ket{u_{z'}}_{z'}=\ket{u_{z'}}_{z}
	 \end{gathered}
	\end{equation}
	
	
	On a vu que le cas du mode torsadé avec des lames quart d'onde parfaite, la matrice $\mathbf{R}_{\frac{\alpha}{2}} \mathbf{B M}_{(x, y, z) \rightarrow\left(x^{\prime}, y^{\prime}, z^{\prime}\right)} \mathbf{B R}_{\frac{\alpha}{2}}$ réduit à une matrice de rotation, $\mathbf{R}_{\frac{\alpha}{2}} \mathbf{R}_{\frac{\alpha}{2}}$, donc l'état propre de polarisation circulaire demeure inchangé par cette transformation. Un résonateur à mode torsadé aura donc un facteur de contraste de :


	\begin{equation} \label{eq:propagation}
	\begin{gathered}
	\vert\gamma\vert=\vert\bra{u_{z}}_{z}\ket{u_{z'}}_{z}\vert= \frac{1}{\sqrt{2}}\begin{bmatrix} 1 & i \end{bmatrix}\cdot\frac{1}{\sqrt{2}}\begin{bmatrix} 1 \\ -i \end{bmatrix}=0
	 \end{gathered}
	\end{equation}


\subsection{Mode propre de polarisation dans les résonateurs avec élément diretardant et diatténuant}

	\label{sec:Mode propre de polarisation dans les résonateurs avec élément diretardant et diatténuant}

Abordons maintenant le cas de résonateur avec élément diatténuant linéaire. L'effet de la diatténuation va créer plus de perte pour un mode de polarisation que pour l'autre et modifie les états propres de polarisation comparés à un résonateur sans diatténuation, sauf si les états propres de polarisation sont linéaires et alignés selon les axes principaux de la diatténuation. L'augmentation de la différence de pertes entre les deux modes sans modifier leurs états de polarisation est obtenue dans un résonateur à mode torsadé lorsqu'il y a un élément diatténuant inséré entre la lame quart d'onde et un miroir avec les axes principaux de l'élément diaténuant orientés à $\theta=\pm 45\degree$ des axes d'une des lames quart d'onde puisque les états propres sont diagonal et anti-diagonal respectivement à cet endroit dans le résonateur illustrer à la figure \ref{fig: torsader avec diattenuation variable}. Cette méthode est souvent employée avec le résonateur à mode torsadé pour avoir un laser monomode, \cite{Draegert_twisted_mode_pol_45,DEJONG1977_twisted_mode}, mais ce résonateur est toujours susceptible au saut de mode entre différent modes longitudinaux, car l'ajout de lame quart d'onde et d'élément polarisant ou diatténuant augmente beaucoup la longueur optique du résonateur. Le racourcissement de la longueur du résonateur pourrait résoudre ce problème mais cela nécéssite une intégration de la diretardance ainsi que le diatténuation dans les miroirs. L'intégration de ces composantes dans les miroirs n'est pas facilement réalisable pour une couche avec ses axes de diatténuation qui sont à une différente orientation que celle de la biréfringence de la couche.

En guise d'exploration théorique des modes de polarisation dans le cas où l'orientation des axes de diatténuation est variable par rapport aux axes de biréfringence, examinons ce qu'il advient des états propres de la polarisation en fonction des différentes orientations possibles. Il y a deux emplacements possibles de l'élément diatténuant: entre la lame quart d'onde et le miroir et entre les deux lames quart d'onde. Dans cette section, nous nous concentrerons sur le cas où l'élément diatténuant se trouve entre la lame quart d'onde et le miroir, puisque c'est ce que font d'autres chercheurs pour obtenir une émission monomode avec l'orientation $\theta=\pm45\degree$.


Commençons par définir la matrice de transformation d'un miroir diatténuant dans la base de ses axes propres, qui a pour effet de réduire là l'amplitude de l'onde d'une polarisation orientée selon ses axes propres:
	
	\begin{equation} \label{eq:propagation}
	\begin{gathered}
	\textbf{M}_{x_{3},y_{3}}=\begin{bmatrix} \sqrt{R_{TE}} & 0 \\ 0 & -\sqrt{R_{TM}} \end{bmatrix}
	 \end{gathered}
	\end{equation}
	
	\noindent
	où $R_{T E}$ et $R_{T M}$ sont les valeurs des réflectances des miroirs mesurées par rapport à l'intensité qui sont des mesures accessibles par ellipsométrie. Cette transformation est dans une base qui est tournée d'un angle $\theta$ de la base $\left(x_{s}, y_{s}\right)$, démontré à la figure \ref{fig: torsader avec diattenuation variable}, dont la matrice de changement de base, $R_{\theta}$, est la même que celle pour le changement de base, $R,_{\frac{\alpha}{2}}$ mais avec un angle $\theta$. La matrice d'un aller-retour de ce résonateur s'écrit comme :


	\begin{equation} \label{eq:allerretour_avec axe dia var}
	\begin{gathered}
	\textbf{J}_{AR}=\textbf{SR}_{\scaleto{\frac{\alpha}{2}}{12pt}}\textbf{B}_{p}\textbf{M}_{p}\textbf{B}_{p}\textbf{R}_{\scaleto{\frac{\alpha}{2}}{12pt}}\textbf{SR}_{\scaleto{\frac{\alpha}{2}}{12pt}}\textbf{B}_{s}\textbf{R}_{\theta}\textbf{M}_{x_{3},y_{3}}\textbf{R}_{\theta}\textbf{B}_{s}\textbf{R}_{\scaleto{\frac{\alpha}{2}}{12pt}}
	 \end{gathered}
	\end{equation}
	
	\begin{figure}[h!]%Le h! assure que la figure reste proche d'ici. autre options t-top, b-bottom, p - float page. Voir par exemple: https://tex.stackexchange.com/questions/39017/how-to-influence-the-position-of-float-environments-like-figure-and-table-in-lat
\centering
\includegraphics[scale=0.5]{résonateur avec axe et diatténuation variable.jpg}
\caption{résonateur à mode torsadé avec élément diatténuant, $\textbf{M}_{\{ x_3,y_3\}}$, avec orientation de ses axes variable}
\label{fig: torsader avec diattenuation variable}
\end{figure}
	
	\noindent
 Pour la simulation de ce résonateur, on choisit d'avoir un miroir isotrope avec une lame quart d'onde $\delta = \pi/2$ et, pour les composantes près de l'autre miroir, une lame quart d'onde et le miroir est diatténuant avec $\sqrt{R_{T E}}=0.70$ et $\sqrt{R_{T M}}=1$, un choix arbitraire de diatténuation qui correspond à une valeur facilement atteignable de diatténuation en utilisant une lame de verre incliner et qui correspond au choix fait pour les expériences de la \cite{Point_except_JF}.

En calculant les états propres et valeurs propres du résonateur en fonction de l'angle entre les axes principaux des deux lames diretardante, $\alpha$ et l'angle entre les axes de l'élément diatténuant et de la lame quart d'onde, $\theta$, on calcule la modulation de l'onde stationnaire, figure \ref{grap:fact_cont_heatmap}, qui est d'une grande utilité à prédire les endroits qui sont propices à l'émission monomode. La figure. \ref{grap:fact_cont_heatmap}b),  est une transformation du facteur de contraste par $\frac{\text{log}(10000 f(\alpha,\theta)+1)}{\text{log}(10001)}$, où $f(\alpha,\theta)$ est la fonction que l'on veut transformée ce qui démontre mieux les endroits de facteur de contraste nul. Cette transformation fait en sorte qu'une petite déviation d'un contraste nul produit une grande augmentation de sa valeur et est normalisée pour que la valeur maximale demeure à 1.

\begin{figure}[h!]%Le h! assure que la figure reste proche d'ici. autre options t-top, b-bottom, p - float page. Voir par exemple: https://tex.stackexchange.com/questions/39017/how-to-influence-the-position-of-float-environments-like-figure-and-table-in-lat
\centering
\includegraphics[scale=0.7]{facteur de contrast.jpg}
\caption{Facteur de contraste de l'onde stationnaire d'un état propre de polarisation dans le cas du résonateur mode torsadé avec élément
diatténuant et axe de diatténuation variable. b) transformation du facteur de contraste par l'équation $\frac{\text{log}(10000 f(\alpha,\theta)+1)}{\text{log}(10001)}$ dans une portion de la gamme alpha. Les cercles indiquent des points exceptionnel, voir figure \ref{grap:orthogonalité_heatmap}}
\label{grap:fact_cont_heatmap}
\end{figure}

 Les coupes selon $\theta=-45\degree$ et $45\degree$ sont les endroits où la lame diatténuante ne change pas les états propres, mais crée seulement une différence de perte comme discuté au début de cette section. Pour quantifier, la proximité des vecteurs propres, on prend la norme du produit scalaire entre les deux vecteurs propres normalisés au même endroit dans le résonateur et se propagent dans la même direction, $\left|\left\langle u_{1} \mid u_{2}\right\rangle\right|$ et elle nous donne une valeur quantitative du rapprochement de deux états propres de polarisation, peu importe si les vecteurs sont rectilignes, circulaires ou elliptiques. D'après la figure \ref{grap:orthogonalité_heatmap} on voit apparaitre des points intéressants sur une coupe à $\theta=0^{\circ}$ et $\theta=90^{\circ}$ où les deux vecteurs propres coalescent. Lorsque deux vecteurs propres coalescents, ils ont automatiquement les mêmes valeurs propres puisqu'ils subissent les mêmes transformations par les éléments optiques dans le résonateur. Ces points qui ont la coalescence de deux vecteurs propres et la même valeur propre sont appelés des points exceptionnels. Ces points sont pertinents puisqu'ils ont également un facteur de contraste nul, ce qui est prometteur pour l'émission monomode.


\begin{figure}[h!]%Le h! assure que la figure reste proche d'ici. autre options t-top, b-bottom, p - float page. Voir par exemple: https://tex.stackexchange.com/questions/39017/how-to-influence-the-position-of-float-environments-like-figure-and-table-in-lat
\centering
\includegraphics[scale=0.825]{proximiter.jpg}
\caption{Proximité des vecteurs propres dans le cas du résonateur mode torsadé
avec élément diatténuant et axe de diatténuation variable.}
\label{grap:orthogonalité_heatmap}
\end{figure}

\pagebreak

\newpage
	
	\subsection{Point exceptionnel dans les résonateurs}
	\label{sec:Point exceptionnel dans les résonateurs}


	Dans cette section nous allons explorer les points exceptionnels (PE) qui existent dans l'espace de polarisation dans certaines configurations de résonateurs. Ces points sont intéressants pour l'obtention d'un seul mode puisque par définition ils sont des points où un seul mode de polarisation existe dans le résonateur. Cependant, on verra que les points exceptionnels n'assurent pas l'émission monomode, car il peut toujours exister plusieurs modes longitudinaux et transversaux appartenant au même mode de polarisation. Pour que ces points soient bénéfiques pour l'émission monomode, il faut que le contraste de l'onde stationnaire qu'on associe à la PE soit petit. On verra dans la section \ref{sec:condition d'émission monomode dans les résonateurs à onde stationnaire} qu'il n'est pas nécessaire que le facteur de contraste soit nul pour que l'émission monomode soit possible, peu importe le taux de pompage.
	
Comme démontré dans la figure \ref{grap:orthogonalité_heatmap}, les points encerclés en rouge sont des points dont les états propres se rapprochent très près du même état, mais, avec les simulations numériques des vecteurs propres, on ne peut pas garantir que ces deux états coalescents en un seul état. Une façon de prouver que ses points sont réellement des PE, lorsqu'une solution analytique n'est pas accessible, consiste à faire l'encerclement de ses points dans l'espace des paramètres de contrôle, $\alpha$ et $\theta$. L'encerclement de ces points a un comportement différent de l'encerclement de zones qui ne contiennent pas de PE, tel que montré dans \cite{encerclementPE_Gabriel}. Lorsqu'on encercle un PE et l'on suit un état propre le long de cette trajectoire, en regardant son plus proche voisin point par point sur la sphère de Poincaré, on remarque qu'après un tour complet, l'état final et l'état initial sont interchangés après un tour. Un exemple de cette technique d'identification de PE est montré à la figure \ref{graph:encerclement etats propres} pour le même cas que la carte de proximité de la figure \ref{grap:orthogonalité_heatmap}. Cette discontinuité est aussi observée dans les valeurs propres de ces états, figure \ref{graph:val_propres_real_imag}.


\begin{figure}[h!]%Le h! assure que la figure reste proche d'ici. autre options t-top, b-bottom, p - float page. Voir par exemple: https://tex.stackexchange.com/questions/39017/how-to-influence-the-position-of-float-environments-like-figure-and-table-in-lat
\centering
\includegraphics[scale=0.5]{encerclement etats propres.jpg}
\caption{Exemple de l'effet de l'encerclement d'un PE sur les états propres pour un résonateur composé de lames quart d'onde et une lame diatténuante. a) Proximité des états avec deux trajectoires, une contenant un PE, à gauche et l'autre sans PE, à droite. b) Trajectoire des états propres sur la sphère de Poincaré pour un trajet contenant un PE, figure à gauche, et trajectoire sans PE, figure à droite.}
\label{graph:encerclement etats propres}
\end{figure}
	
	\begin{figure}[h!]%Le h! assure que la figure reste proche d'ici. autre options t-top, b-bottom, p - float page. Voir par exemple: https://tex.stackexchange.com/questions/39017/how-to-influence-the-position-of-float-environments-like-figure-and-table-in-lat
\centering
\includegraphics[scale=0.5]{val_propre_real_imag.jpg}
\caption{Valeur propre des deux états propres, surface bleu et rouge, autour d'un PE pour un résonateur composer de lame quart d'onde et une lame diatténuante. Partie réelle des valeurs propres à gauche et partie imaginaire correspondante à droite.}
\label{graph:val_propres_real_imag}
\end{figure}


\pagebreak
	
Quant à l'identification des PE lors des manipulations expérimentales, il est important de comprendre le comportement d'émission autour du PE. On illustre les valeurs propres d'une façon qui montre mieux le différent comportement de l'émission autour du PE, en traçant la différence de perte entre les deux états, figure \ref{graph:diff_perte_diff_phase_quart_onde_parfait_diat} a) et la différence de phase, figure \ref{graph:diff_perte_diff_phase_quart_onde_parfait_diat} b). Pour les coupes $\theta = 0\degree$ et $90\degree$, on remarque un comportement des valeurs nulles qui s'interchange pour différente valeur d'$\alpha$ entre la différence de pertes et la différence de phase des modes, c'est-à-dire que pour la région entre $\vert \alpha \vert < 5\degree $ et la région $85\degree< \alpha < 95\degree $, la différence de phase est nulle et pour les autres valeurs d'$\alpha$ les états propres ont les mêmes pertes. Lorsque la différence de phase est nulle, mais que la différence de perte est non nulle, l'oscillation des deux états propres de polarisation est bien supprimée et si les états oscillent, ils le feront à des intervalles de fréquences données par l’intervalle spectral libre du laser. Lorsque la différence de pertes est nulle, mais que la différence de phase est non nulle, l'oscillation des deux modes de polarisation est susceptible de se produire, mais ils vont osciller à des intervalles de fréquences qui ne sont pas nécessairement des multiples entiers de l'interval spectral libre.


	\begin{figure}[h!]%Le h! assure que la figure reste proche d'ici. autre options t-top, b-bottom, p - float page. Voir par exemple: https://tex.stackexchange.com/questions/39017/how-to-influence-the-position-of-float-environments-like-figure-and-table-in-lat
\centering
\includegraphics[scale=0.65]{diff_perte_diff_phase_resonateur_lame_quart onde_diat_axe variable.jpg}
\caption{ a), la différence de pertes entre les deux modes de polarisation et b), la différence de phase entre les deux modes de polarisation sur un facteur d'échelle de $2\pi$ \textcolor{red}{GRAPH SELON THETA = 0}}
\label{graph:diff_perte_diff_phase_quart_onde_parfait_diat}
\end{figure}

\pagebreak

Considérons maintenant une autre configuration de résonateur qui sera plus facile à miniaturiser et à fabriquer à l'aide de la technologie de dépôt de couches minces. Cette configuration consiste à inverser la diatténuation et la diretardance au voisinage du miroir de droite figure\ref{graph: miroir_bir_plus_diat}. Cette modification permet plus facilement d'utiliser des miroirs anisotropes identiques. Ces dernier sont également plus faciles à fabriquer par des techniques de déposition à angles rasant qu'une couche purement diatténuante telle que montré à la figure \ref{fig: torsader avec diattenuation variable}. De plus, il est facile d'ajouter en laboratoire un élément purement diatténuant dans le résonateur sous la forme d'une lame de verre inclinée pour que le faisceau soit en incidence différent qu'une incidence normal.  La figure \ref{graph: miroir_bir_plus_diat} montre cette nouvelle configuration avec $M_p$ et $M_s$ qui sont des miroirs diretardants seulement et $D$ un élément diatténuant. La matrice d'aller-retour avec une figure illustrant les axes principaux des composantes est montrée dans le bas de la figure \ref{graph: miroir_bir_plus_diat}. 
\pagebreak
\begin{figure}[h!]%Le h! assure que la figure reste proche d'ici. autre options t-top, b-bottom, p - float page. Voir par exemple: https://tex.stackexchange.com/questions/39017/how-to-influence-the-position-of-float-environments-like-figure-and-table-in-lat
\centering
\includegraphics[scale=0.60]{résonateur avec axe et diatténuation variable_deux miroir_birefring.jpg}
\caption{ Résonateur à mode torsadé avec miroir biréfringent et élément diatténuant.  Représentation graphique des opérations faits lors d'un aller-retour avec l'équation ci-dessous.}
\label{graph: miroir_bir_plus_diat}
\end{figure}

Il faut noter que cette configuration est différente de ce que plusieurs autres chercheurs font pour éliminer un mode de polarisation comme discutée avant avec la technique de placer une lame diatténuante avec ses axes principaux placés à $\pm45\degree$ des axes principaux d'une lame quart d'onde pour éliminer complètement un état de polarisation en particulier. Dans cette nouvelle configuration, cette technique ne fonctionne plus puisque l'ajout d'une lame diatténuant à cette position dans le résonateur où les états propres de polarisation sont circulaires avant l'insertion de la lame diatténuante, va changer les états propres et diminuer l'amplitude de l'onde des deux états propres.  

Commençons avec un cas avec les mêmes éléments optiques que pour les simulations faites précédemment, figures \ref{grap:fact_cont_heatmap},  \ref{grap:orthogonalité_heatmap} pour pouvoir comparer la configuration précédente, figure \ref{fig: torsader avec diattenuation variable} avec celle-ci, figure \ref{graph: miroir_bir_plus_diat}. Ce cas est le cas avec deux miroirs anisotropes avec la même diretardance qu'une lame quart d'onde en transmission (lame demi-onde en réflexion) et un élément diatténuant de la forme de:


\begin{equation} \label{eq:propagation}
D=\begin{bmatrix}
\sqrt{T_{TE}} & 0\\
0 & \sqrt{T_{TM}} 
\end{bmatrix}_{\{H, V\}}
	\end{equation}
	
\noindent	
avec les deux coefficients de transmission d'une valeur de $T_{TE}= 0.7$ et $T_{TM}=1$ ce qui simule le même élément diatténuant que le cas précédent dans un aller-retour puisque la lame est traversée deux fois dans un aller-retour. Selon la coupe de $\theta=0\degree \text{ et } 90\degree$, cette configuration est la même que celle d'avant, car les axes principaux de diatténuation et de la biréfringence ont la même orientation et $\textbf{R}^{-1}_{\theta} \textbf{M}_s \textbf{R}^{-1}_{\theta}= \textbf{I}$. Toutefois, le comportement diffère pour tous les autres orientations. Le comportement différent de l'autre configuration peut être observé en comparant la proximité des états propres et le facteur de contraste de l'onde stationnaire à la figure \ref{graph:Prox_contr_mir_bir_plus_diat_var} a) et b) respectivement, comparé à la configuration d'avant, figures \ref{grap:fact_cont_heatmap} et \ref{grap:orthogonalité_heatmap}. D'après la proximité des états propres, on remarque qu'il y a une \textit{ligne} \textit{exceptionnelle} de chaque côté de $\alpha=0\degree \text{ et } 90\degree$. La vérification de cette ligne exceptionnelle a été faite dans un autre espace de paramètre puisque dans cette espace de paramètre $(\theta,\alpha)$, il n'est pas possible d'encercler un seul point exceptionnel. L'espace de paramètre où chaque point de cette ligne à été encerclé est dans l'espace de $(\delta,\theta)$ ($\delta$ est le déphasage de la lame diretardant) pour plusieurs valeurs d'$\alpha$ et cela a démontré qu’en effet, il semble que cela est une ligne exceptionnelle. Quant au comportement de la modulation de l'onde stationnaire, figure \ref{graph:Prox_contr_mir_bir_plus_diat_var} b), elle varie très peu sur un grande plage de valeur des deux paramètres de contrôle comparés au cas précédent. 

\begin{figure}[h!]%Le h! assure que la figure reste proche d'ici. autre options t-top, b-bottom, p - float page. Voir par exemple: https://tex.stackexchange.com/questions/39017/how-to-influence-the-position-of-float-environments-like-figure-and-table-in-lat
\centering
\includegraphics[scale=0.625]{prox_fact_M1_M2_180_r_x0.7.jpg}
\caption{ a) Proximité des états propres dans le cas de miroir avec déphasage de $180\degree$ et lame diatténuante, $T_{TE}= 0.7$ et $T_{TM}=1$. b) facteur de contraste de l'onde stationnaire à l'endroit où se situe le milieu amplificateur.}
\label{graph:Prox_contr_mir_bir_plus_diat_var}
\end{figure} 

\pagebreak

Cette configuration nous amène à simuler un cas qui peut être conçu expérimentalement, mais avec des paramètres idéaux qui est la fabrication de miroirs diretardant avec des couches minces qui sont identiques. Pour ce cas, on va supposer que ces couches minces sont seulement diretardantes et que la diatténuation dans le résonateur provient des lames diatténuante seulement. Le cas simulé dans la figure \ref{graph:Prox_contr_mir_bir170_plus_diat_var}, est le cas où l'on n'obtient pas la valeur de déphasage de $180\degree$ mais plutôt une valeur de $170\degree$ puisqu'il est difficile d'atteindre une valeur de déphasage spécifique expérimentalement et avec une lame diatténuante ayant la même diatténuation que le cas précédent. D'après la proximité des vecteurs propres, il semble y avoir deux lignes exceptionnelles autour de $\alpha=90\degree$, mais en utilisant la figure de la différence de pertes et la différence de phase, figure \ref{graph:diff_perte_phase_mir_bir170_plus_diat_var} a), c) et b),d) respectivement, on voit qu'il n'y a que quelques points encerclés en rouge où les lignes de pertes égales et les lignes de phase égales se rejoignent indiquant donc des points exceptionnels. En ce qui concerne le facteur de contraste, sa valeur minimale au PE est de 0.1 comparée à zéro pour le cas de lames quart d'onde parfaites. Plusieurs simulations ont été faites avec différentes valeurs de déphasages égales pour les deux miroirs, et l'on constate que le facteur de contraste est d'autant plus élevé que l'on s'éloigne de la valeur de déphasage de $180\degree$ 


\begin{figure}[h!]%Le h! assure que la figure reste proche d'ici. autre options t-top, b-bottom, p - float page. Voir par exemple: https://tex.stackexchange.com/questions/39017/how-to-influence-the-position-of-float-environments-like-figure-and-table-in-lat
\centering
\includegraphics[scale=0.625]{prox_fact_M1_M2_170_r_x0.7.jpg}
\caption{ a) Proximité des états propres dans le cas de miroir avec déphasage de $170\degree$ et lame diatténuante avec $T_{TE}= 0.7$ et $T_{TM}=1$. b) facteur de contraste de l'onde stationnaire à l'endroit où se situe le milieu amplificateur. \textcolor{red}{CERCLE ROUGE AU PE}}
\label{graph:Prox_contr_mir_bir170_plus_diat_var}
\end{figure}


\begin{figure}[h!]%Le h! assure que la figure reste proche d'ici. autre options t-top, b-bottom, p - float page. Voir par exemple: https://tex.stackexchange.com/questions/39017/how-to-influence-the-position-of-float-environments-like-figure-and-table-in-lat
\centering
\includegraphics[scale=0.625]{diff_perte_diff_phase_MS_MP_bir_170_diat_0.7.jpg}
\caption{Comportement de la différence de pertes a), c) et la différence de phase b), d) pour un résonateur ayant des miroirs biréfringents avec un déphasage de $170\degree$ et une lame diatténuante avec $T_{TE}= 0.7$ et $T_{TM}=1$. Les figures c) et d) sont une transformation des figures a) et b) respectivement avec la fonction $\frac{\text{log}(10000 f(\alpha,\theta)+1)}{\text{log}(10001)}$}
\label{graph:diff_perte_phase_mir_bir170_plus_diat_var}
\end{figure}


\pagebreak

Comme le PE de la simulation précédente est très proche de $\theta=0\degree$, il serait intéressant d'essayer de trouver des PE qui se situent à cette valeur de $\theta$, car cela facilite l'intégration de la diatténuation et de la directivité dans une seule pièce. Ceci nous rapproche de la situation expérimentale étudiée dans cette thèse, où le contrôle des axes de diatténuation est perdu car les axes de diatténuation et de diretardance sont les mêmes pour des raisons de symétrie. L'intégration de tous les composants dans les miroirs permet d'avoir un résonateur plus court, ce qui diminue la possibilité de saut de mode. Dans ce cas, est-ce qu'il y a d'autres façons d'obtenir un PE si on n’obtient pas un déphasage entre les polarisation TE et TM de $180\degree$ comme dans les expériences de \cite{Point_except_JF}? Pour cela, il est bénéfique de trouver une solution analytique pour avoir des PE. Commençons par trouver la forme générale des valeurs propres et les vecteurs propres pour une matrice 2X2. 


\begin{equation} \label{eq:propagation}
J_{AR, \{H, V\}}=\begin{bmatrix}
a & b \\
c & d
\end{bmatrix}
\end{equation}


\noindent
Avec $J_{AR}$ une matrice qui décrit le trajet que parcourt un état de polarisation dans un aller-retour et $a,b,c,d$ des nombres complexes. Les valeurs propres sont trouvées à partir de l'équation, $\text{det}(J_{AR}-I\lambda)=0$

\begin{equation} \label{eq:val_propre_jone_PT}
\lambda_{\pm} = \frac{a+d}{2}\pm\sqrt{\left(\frac{a-d}{2}\right)^{2}+bc}
\end{equation}	

\noindent
et les vecteurs propres sont:

\begin{equation} \label{eq:val_propre_jone_PT}
\ket{u_{\pm}}= \begin{bmatrix} b \\ \lambda_{\pm}-a  \end{bmatrix}
\end{equation}

\noindent
On remarque que les deux valeurs propres et deux vecteurs propres sont égaux quand le discriminant est nul: 

\begin{equation} \label{eq:PE condition}
\zeta=\left(\frac{a-d}{2}\right)^{2}+bc = 0
\end{equation}

\noindent
Avec la condition du point exceptionnel identifiée, nous cherchons à trouver la matrice d'aller-retour avec les matrices générales décrivant la réponse optique d'un miroir linéaire diatténuant et biréfringent, qui est donnée par :

 \begin{equation} \label{eq:}
 \begin{gathered}
\textbf{M}_{1,\left\lbrace H,V \right\rbrace}=\begin{bmatrix} r_{11} & 0 \\ 0 & -r_{12}\text{exp}(i\Delta_{1}) \end{bmatrix} \\
\textbf{M}_{2,\left\lbrace H,V \right\rbrace}=\begin{bmatrix} r_{21} & 0 \\ 0 & -r_{22}\text{exp}(i\Delta_{2}) \end{bmatrix}
\end{gathered}
	\end{equation}


\noindent
avec les coefficients $r$ réels positifs, indiquant l'amplitude de la réflectance seulement et $\Delta$ le déphasage entre $TE$ et $TM$. La diatténuation de ces miroirs est donnée par : $r_{11}- r_{12}$ pour $\textbf{M}_1$ et $r_{21}- r_{22}$ pour $\textbf{M}_2$. Pour un rappel, la matrice décrivant un aller-retour est:

\begin{equation} \label{eq:} 
\footnotesize
\setlength{\arraycolsep}{2pt}
\medmuskip = 1mu % default: 4mu plus 2mu minus 4mu 
\begin{gathered} 
\textbf{R}_{\alpha}          \textbf{M}_{1}         \textbf{R}_{\alpha}         \textbf{M}_{2} =  \\
\begin{bmatrix} r_{21}e^{i\Delta_1} (r_{12} \text{sin}^2 (\alpha)   +  r_{11}\text{cos}^2 (\alpha) )     &
 -r_{22}e^{i\Delta_2}( r_{11}\text{sin}(\alpha)\text{cos}(\alpha)  -  r_{12}e^{i\Delta_1}\text{sin}(\alpha) \text{cos}(\alpha))  
 \\ r_{21}(-r_{11} \text{sin}(\alpha)\text{cos}(\alpha)  +  r_{12}e^{i\Delta_1}\text{sin}(\alpha) \text{cos}(\alpha))    &
  r_{22}e^{i\Delta_2}(r_{11}\text{sin}^2(\alpha) +r_{12}e^{i\Delta_1} \text{cos}^2(\alpha) )  \end{bmatrix}
\end{gathered}
	\end{equation}

\noindent
En remplaçant les valeurs $a,b,c,d$ du discriminant de l'équation \ref{eq:PE condition} par les valeurs de cette matrice ci-dessus, on peut résoudre cette équation pour la condition du PE de $\zeta=0$: 


 \begin{equation} \label{eq:}
 \begin{gathered}
 \dfrac{\left( r_{21}-r_{22}e^{i\Delta_{2}} \right)^2     \left(  r_{11}-r_{22}e^{i\Delta_{2}}\right)^2 \text{cos}^2(2\alpha)}{16} + \\
   \dfrac{ (r_{21}-r_{22}e^{i\Delta_{2}})(r_{21}+r_{22}e^{i\Delta_{2}})(r_{11}-r_{12}e^{i\Delta_{1}})(r_{11}+r_{12}e^{i\Delta_{1}})\text{cos}(2\alpha)}{8} + \\
 \dfrac{r_{11}^2 \left( r_{21}+r_{22}e^{i\Delta_{2}} \right)^2 }{16} + \\
 \dfrac{r_{12}e^{i\Delta_{1}} r_{11}(r_{21}^2-6 r_{22}e^{i\Delta_{2}}r_{21}+r_{22}^2 e^{2i\Delta_{2}})}{8} + \\
  \dfrac{r_{12}^2 e^{2i\Delta_{1}}\left(r_{21}+r_{22} e^{i\Delta_{2}} \right)^2}{16} = 0
\end{gathered}
	\end{equation}
 
\noindent
Le but de ceci est d'obtenir un PE accessible par l'angle $\alpha$ pour une configuration de deux miroirs anisotrope. On résout cette équation quadratique pour $\text{cos}(2\alpha)$, on obtient:


 \begin{equation} \label{eq:alpha_PE_genéral}
 \setlength{\jot}{18pt}
 \begin{gathered}
\alpha_{\scaleto{PE}{4pt}}= (\pm)\frac{1}{2}\arccos\Biggl( \frac{ -r_{11}r_{21}-r_{11}r_{22}e^{i\Delta_{2}}-r_{21}r_{12}e^{i\Delta_{1}}-r_{12}r_{22}e^{i(\Delta_{1}+\Delta_{2})}}{(r_{21}-r_{22}e^{i\Delta_{2}})(r_{11}-r_{12}e^{i\Delta_{1}}} \\
\frac{\pm 4\sqrt{r_{11}r_{12}r_{21}r_{22}e^{i(\Delta_{1}+\Delta_{2})}}}{(r_{21}-r_{22}e^{i\Delta_{2}})(r_{11}-r_{12}e^{i\Delta_{1}}}\Biggr)
\end{gathered}
	\end{equation}
	
\noindent
Cette équation indique que pour avoir le PE accessible par l'angle $\alpha$, l'argument de la fonction doit être un nombre réel compris entre -1 et 1. Le $(\pm)$ à l'extérieur de l'argument signifie qu'il y a deux combinaisons pour chacun des signes $\pm$ à  l'intérieur de l'argument. Le cas du PE étudié dans \cite{Point_except_JF} est la solution de PE accessible par $\alpha$ avec des lames quart d'onde parfaite, $\Delta_1=\Delta_2 = \pi$ et une diatténuation non nulle. Un autre cas qui existe de PE accessible par l'angle $\alpha$ avec une plus grande restriction sur les valeurs de diatténuation est le cas de $\Delta_1= -\Delta_2 \equiv \Delta$ et la diatténuation des deux miroirs égale, c'est à dire, $r_{11}=r_{21} = r_{x}$ et $r_{12}=r_{22} = r_y$. Avec ces simplifications, l'équation \ref{eq:alpha_PE_genéral} devient:

 \begin{equation} \label{eq:}
 \begin{gathered}
\alpha_{\scaleto{PE}{4pt}} = (\pm) \dfrac{1}{2}\text{arccos}\left(\frac{ -r_x^2-2r_x r_y \text{cos}(\Delta) - r_y^2 \pm \sqrt{r_y^2 r_x^2}}{r_x^2+r_y^2-2r_x r_y \text{cos}(\Delta)}\right)
\end{gathered}
	\end{equation}
 
On voit que cette équation n'a plus de termes imaginaires donc donnera des valeurs accessibles d'angle $\alpha_{PE}$. D'après cette équation, on peut obtenir de nouveaux points exceptionnels sans avoir d'axes de diatténuation différent de ceux pour la diretardance et sans avoir de déphasage de $\pi$. Il existe également un autre cas très similaire à celui-ci qui peut donner des PE qui s'agit simplement de définir la matrice avec le miroir tourné de 90 degrés. La situation où $\Delta_1=\Delta_2$ est presque identique à celle d'avant, mais avec le déphasage d'un des miroirs de signe inversés comparé au cas précédent. En d'autres termes, le retard de TE sur TM est inversé, de sorte que le signe de la diatténuation doit également être inversé pour qu'il soit identique au cas précédent, donc on a que $r_{11}=r_{22}$ et $r_{12}=r_{21}$. Ce cas n'a pas d'équation élégante facile à démontrer, mais nous calculerons les états propres numériquement avec les valeurs suivantes : $r_{11}=r_{22} = \sqrt{0,7}$, $r_{12}=r_{21}=1$ et $\Delta_1=\Delta_2=170\degree$, figure \ref{graph:PE_avec_delta_diff_de_180_theta_0}

\pagebreak

\begin{figure}[h!]%Le h! assure que la figure reste proche d'ici. autre options t-top, b-bottom, p - float page. Voir par exemple: https://tex.stackexchange.com/questions/39017/how-to-influence-the-position-of-float-environments-like-figure-and-table-in-lat
\centering
\includegraphics[scale=0.7]{prox_fact_cont_M1_M2_170_r_x_sqrt0.7.jpg}
\caption{Point exceptionnel dans le cas de miroir avec les valeurs suivantes: $r_{11}=r_{22} = \sqrt{0.7}$, $r_{12}=r_{21}=1$ et $\Delta_1=\Delta_2=170\degree$. a) Proximité des vecteurs propres, b) facteur de contraste de l'onde stationnaire calculé entre les deux miroirs dans le milieu amplificateur.}
\label{graph:PE_avec_delta_diff_de_180_theta_0}
\end{figure}

\noindent
 On remarque que les vecteurs propres coalescent seulement près de $\alpha=90\degree$ comparé au premier cas où $\theta=0\degree$ on avait des PE autour de $\alpha=0\degree$. Plus on s’éloigne de la valeur d'une lame quart d'onde parfaite, $\Delta=180\degree$, plus que le facteur de contraste de l'onde stationnaire augmente au PE donc il faut avoir une valeur près de déphasage près de $180\degree$ pour que ces PE soient intéressantes pour l'émission monomode. 

\pagebreak
\newpage
				
	\subsection{Condition d'émission monomode dans les résonateurs à onde stationnaire}
\label{sec:condition d'émission monomode dans les résonateurs à onde stationnaire}

Dans cette section, nous montrerons que dans des conditions expérimentales de fabrication de réseaux où le déphasage entre TE et TM de $\pi$ n'est pas exactement atteint, et où les conditions pour obtenir un PE ne sont donc pas remplies, il est néanmoins possible d'obtenir un laser monomode. Commençons par montrer les résultats des mesures ellipsométriques à incidence normale des réseaux gravés sur miroirs de Bragg utilisés dans les tests laser réalisés au cours de ce travail. La fabrication de réseaux avec des rapports cycliques des réseaux légèrement différents vise à compenser les erreurs de fabrication et les incertitudes liées aux valeurs de l'indice de réfraction. Le fonctionnement des miroirs pour obtenir la réponse optique voulue est présenter dans l'annexe B. La mesure des réflectances $R_{TE}$, $R_{TM}$ et du déphasage entre les polarisations TE et TM des miroirs, $\Delta$ est décrite dans l'annexe A. Les réseaux dont le déphasage mesuré est le plus près de $\pi$ sont présentés dans le tableau \ref{tab:ellipso_miroir} et, à titre de comparaison, nous indiquons les valeurs pour le cas idéal que nous utiliserons pour la simulation plus loin dans cette section. Les résultats de simulation suivants sont tous basés sur les valeurs indiquées dans ce tableau.

\begin{table}[h!]
\centering
\caption{Valeurs des paramètres utilisés dans la simulation présentée à la figure 2. Le cas A correspond à la conception idéale, le cas B correspond aux valeurs mesurées à partir d'échantillons fabriqués.}
\label{tab:ellipso_miroir}
\begin{tabular}{ccc|cc}
                                    & \multicolumn{2}{c|}{miroir de pompe}                                      & \multicolumn{2}{c}{miroir de sortie}                                     \\
\multicolumn{1}{c|}{cas}            & \multicolumn{1}{c|}{$\Delta_1 (\degree)$} & $R_{\text{TE}}/R_{\text{TM}}$ & \multicolumn{1}{c|}{$\Delta_2(\degree)$} & $R_{\text{TE}}/R_{\text{TM}}$ \\ \hline
\multicolumn{1}{c|}{A idéal}        & \multicolumn{1}{c|}{180.0}                & \multirow{2}{*}{$0.97/0.93$}  & \multicolumn{1}{c|}{180.0}               & \multirow{2}{*}{$0.89/0.48$}  \\ \cline{1-2} \cline{4-4}
\multicolumn{1}{c|}{B expérimental} & \multicolumn{1}{c|}{178.4}                &                               & \multicolumn{1}{c|}{188.8}               &                              
\end{tabular}
\end{table}

\noindent
Avec ces miroirs, aucun point exceptionnel n'est obtenu dans le résonateur à deux miroirs anisotropes et un milieu amplificateur, comme le montre la figure de proximité des états propres, figure \ref{graph:prox_experi_ideal}, pour le cas expérimental en noir. Ceci nous indique que dans notre cas, il n'existe pas de valeur de $\alpha$ pour laquelle il n'y a qu'un seul état de polarisation, et de plus, comme démontré par les simulations effectuées dans la section précédente, lorsque le déphasage de $\pi$ n'est pas atteint, la modulation de l'onde stationnaire n'est pas complètement éliminée. Mais même si un PE n'est pas atteint et la modulation de l'onde stationnaire pas complètement éliminée, est-il possible d'avoir l'émission monomode dans le résonateur, quel que soit le taux de pompage ? C'est sur cette question que nous nous concentrerons dans la suite de ce chapitre.

\begin{figure}[h!]%Le h! assure que la figure reste proche d'ici. autre options t-top, b-bottom, p - float page. Voir par exemple: https://tex.stackexchange.com/questions/39017/how-to-influence-the-position-of-float-environments-like-figure-and-table-in-lat
\centering
\includegraphics[scale=0.4]{prox_exp_article.fig.jpg}
\caption{Proximité des vecteurs propres dans le cas idéal courbe A en rouge et le cas expérimental avec un déphasage sur les miroirs différents de $\pi$, courbe B en noir. Les propriétés optiques utilisées pour ces simulations sont indiquées dans le tableau ci-dessus.}
\label{graph:prox_experi_ideal}
\end{figure}

\pagebreak
\subsubsection{Condition d'émission d'un seul mode longitudinal appartenant au même état propre de polarisation}
 
En ce qui concerne les meilleurs angles pour lesquels nous pouvons nous attendre à une amélioration de la pureté du spectre d'émission, nous nous basons sur le facteur de contraste de l'onde stationnaire $\vert \gamma \vert$, équation \ref{eq:fact_cont}, qui dicte l'ampleur du creusement spatial. Dans le développement suivant, nous montrerons pourquoi une onde stationnaire provoque l'émission d'un laser dans plusieurs modes longitudinaux et, plus important encore, dans notre cas, nous montrerons que la suppression des autres modes est possible, peu importe le taux de pompage avec une modulation d'onde stationnaire non nulle, c'est-à-dire un facteur de contraste non nul. Ce développement a été fait dans \cite{first_unidirectional_laser,Casperson_creusement_spatial} et le cas d'un facteur de contraste différent de 0 et 1 est présenté dans \cite{controle_gain_SHB_JF}. Commençons par définir l'équation du niveau excité pour un système à quatre niveaux idéal et l'équation de l'intensité de l'onde stationnaire pour le cas où $\vert \gamma \vert = 1$, c'est-à-dire que les ondes contra-propagatrices d'un mode sont dans le même état de polarisation:

 \begin{equation} \label{eq:}
 \begin{gathered}
I_1(z) = 4I_{01}\text{sin}^2(k_qz) \\
N_1(z) = \dfrac{N_{0}}{1+ \dfrac{I_1(z)}{I_{\text{sat,s}}}}
\end{gathered}
	\end{equation}

\noindent
où $I_0$ est le module au carré de l'amplitude du champ d'une des ondes par unité de surface (W/m$^2$) se propageant dans la cavité en supposant que les deux ondes contra-propagatrices sont de la même intensité. $N_0$ est la densité de population du niveau excité non saturée , $I_{\text{sat,s}}$ l'intensité de saturation à la longueur d'onde du faisceau de sorties et $k_q$ le nombre d'onde du mode dominant d'ordre $q$. Ces équations nous montrent que la population $N_1(z)$ est anti-corrélée à l'intensité de l'onde stationnaire. La figure \ref{fig:SHB_popu_Iz}, tirée de \cite{controle_gain_SHB_JF}, démontre ce comportement en fonction d'un taux de pompage normalisé, $r \equiv N_0/N_{th}$, où $N_{th}$ est la population du niveau excité au seuil d'oscillation. On remarque qu'à mesure que l'intensité du mode dominant augmente, la population du niveau excité au nœud de l'onde stationnaire augmente sans jamais être utilisée pour produire de l'émission stimulée pour ce mode.

\begin{figure}[h!]%Le h! assure que la figure reste proche d'ici. autre options t-top, b-bottom, p - float page. Voir par exemple: https://tex.stackexchange.com/questions/39017/how-to-influence-the-position-of-float-environments-like-figure-and-table-in-lat
\centering
\includegraphics[scale=0.75]{SHB_population_Iz.jpg}
\caption{Distribution d'intensité de l'onde stationnaire en fonction de la position à l'intérieur de la cavité (haut) et la distribution de la population d'atomes excitée causées par cette onde stationnaire. Figure tirée de \cite{controle_gain_SHB_JF} }
\label{fig:SHB_popu_Iz}
\end{figure}

\pagebreak

\noindent
On remarque aussi qu'à mesure que le taux de pompage augmente, la population du niveau excité à l'endroit où l'intensité de l'onde stationnaire est maximale diminue en dessous du niveau nécessaire pour dépasser le seuil d'oscillation. Cependant, même si la population du niveau excité augmente avec le taux de pompage au-delà du seuil d'oscillation, le taux d'émission stimulée dans le mode dominant sur l'ensemble du volume du mode demeure toujours égal au taux de pertes de photons. Le taux d'émission stimulée dans un mode est proportionnel à l'intégrale sur le volume du mode de l'intensité de l'onde stationnaire et de la population d'atomes excités, tel que donné par:

 \begin{equation} \label{eq:}
 \begin{gathered}
\Gamma_{st} = \dfrac{\sigma_{es}(\nu_q)V}{h\nu_q}\langle N_1(z) I_1(z) \rangle_z
\end{gathered}
	\end{equation}

\noindent
où $\nu_q$ est, $V$ le volume du mode, $\sigma_{e}$ la section efficace d'émission, $h$ la constante de Plank et $\langle \rangle_z$ la moyenne spatiale selon l'axe $z$ seulement puisqu'on suppose une intensité constante dans le plan transverse à la propagation sur une surface du faisceau $S$. De plus, la moyenne spatiale est fait dans toutes le résonateur puisqu'on prend le cas ou le milieu amplificateur prend tout l'espace à l'intérieur du résonateur. Cela est différent que si le millieu amplificteur  cette moyenne spatiale est donnée par:

\begin{equation} \label{eq:}
\begin{gathered}
\langle N_1(z) I_1(z) \rangle_z = \frac{1}{L} \int_{0}^{L} N_1(z) I_1(z)  \,dz 
\end{gathered}
\end{equation}

\noindent
Nous notons qu'un autre mode $q\pm n$, n un entier quelconque d'une fréquence différente n'a pas de corrélation en moyenne avec la distribution des atomes excités créés par le mode dominant et est capable de mieux exploiter l'inversion de population. Il est à noter que c'est le mode voisin qui est susceptible d'osciller, car c'est celui qui présente la plus grande section efficace d'émission après le mode dominant. Lorsque le taux de pompage augmente, la population moyenne $\langle N_1(z) \rangle_z $ augmente et, lorsque cette valeur est suffisamment grande, le taux d'émission stimulée dans un mode $q\pm1$ sera égal au taux de perte de photons dans ce mode, de sorte que ce mode sera également capable d'osciller. 

Dans le cas où $\gamma < 1$, l'amplitude de la modulation de l'onde stationnaire diminue ce qui aussi crée une diminution de la modulation de l'amplitude de la population $N_1(z)$. La figure \ref{fig:gamma_pop} montre le comportement de la population du niveau excité pour différentes valeurs du facteur de contraste à un taux de pompage normalisé très élevé de $r=100$ avec l'intensité de l'onde qui est donnée d'après l'équation \ref{eq:onde_stationnaire} avec les deux ondes ayant la même intensité et avec le choix d'une intensité nulle à l'origine:


 \begin{equation} \label{eq:intensiter_onde_stationnaire}
 \begin{gathered}
 I_1(z) = 4I_{01}\vert \gamma \vert \text{sin}^2(kz)+2I_{01}(1-\vert \gamma \vert)
\end{gathered}
	\end{equation}


La population moyenne pour chacun de ces cas est calculée numériquement pour donner une idée du gain moyen restant après l'oscillation du mode dominant. Ce gain moyen est proportionnel à la population moyenne du niveau excité pour le cas, $\vert \gamma \vert=0.75$, $\langle N_1(z)\rangle_z/N_{th} = 1.5$, pour le cas de $\vert \gamma \vert=0.5$, $\langle N_1(z)\rangle_z/N_{th} = 1.15$,  pour le cas de $\vert \gamma \vert=0.25$, $\langle N_1(z)\rangle_z/N_{th} = 1.03$ et pour le dernier cas de $\vert \gamma \vert=0$, $\langle N_1(z)\rangle_z/N_{th} = 1$.      

\begin{figure}[h!]%Le h! assure que la figure reste proche d'ici. autre options t-top, b-bottom, p - float page. Voir par exemple: https://tex.stackexchange.com/questions/39017/how-to-influence-the-position-of-float-environments-like-figure-and-table-in-lat
\centering
\includegraphics[scale=0.6]{N_with_gamma.jpg}
\caption{Inversion de population en fonction de la position dans le milieu amplificateur $z$ pour un taux de pompage normalisé très élevé de $r=100$ et pour différents facteurs de contraste}
\label{fig:gamma_pop}
\end{figure}

\pagebreak
Il est montré dans \cite{controle_gain_SHB_JF}, que dans le cas où $\vert \gamma \vert < 1$, la population du niveau excité moyen tend asymptotiquement vers une valeur maximale lorsque le taux de pompage normalisé tend vers l'infini. Dans ce cas, l'émission peut rester monomode lorsque le taux de pompage augmente, à condition que la moyenne de $\langle N_1(z) \rangle_z $ soit suffisamment faible pour que le taux d'émission stimulée d'un autre mode soit inférieur au taux de perte de photon, qui est la condition d'oscillation. La condition d'émission monomode pour une configuration de résonateur en fonction du facteur de contraste est donné par \cite{controle_gain_SHB_JF} :

\begin{equation} \label{eq:gamma_crit}
 \begin{gathered}
\vert \gamma_{\text{crit}} \vert  =  \sqrt{1- \left(\dfrac{\sigma_e(\nu_{q+1})}{\sigma_e(\nu_q)}\right)^2}
\end{gathered}
	\end{equation}

\noindent
où $\vert \gamma_{\text{crit}} \vert$ est le facteur de contraste maximal pour lequel une oscillation monomode est attendue quel que soit le taux de pompage. Ce facteur dépend indirectement de l'intervalle spectral entre les modes donc est inversement proportionnel à la longueur optique de la cavité. 

La section efficace d'émission de $\text{Y}_3\text{Al}_5\text{O}_{12}$ (YAG) dopée à l'$\text{Yb}^{3+}$ autour de la longueur d'onde d'émission de 1030 nm est présentée à la figure \ref{fig:section efficace emission}, ainsi que les longueurs d'onde du mode principal $q$ et du mode voisin $q-1$. Ici on suppose qu'il y a une fréquence de résonance au maximum de la section efficace d'émission, l'équation \ref{eq:gamma_crit} nous donne une valeur d'environ $\vert \gamma_{\text{crit}} \vert \approx 0.13$ pour une longueur optique du résonateur de 2.33mm avec le milieu amplificateur remplissant le résonateur.  

\begin{figure}[h!]%Le h! assure que la figure reste proche d'ici. autre options t-top, b-bottom, p - float page. Voir par exemple: https://tex.stackexchange.com/questions/39017/how-to-influence-the-position-of-float-environments-like-figure-and-table-in-lat
\centering
\includegraphics[scale=0.5]{emission_cross_section_Yb_YAG.jpg}
\caption{Section efficace d'émission mesurée d'une céramique YAG dopée à l'$\text{Yb}^{3+}$ et espacement des modes ($\Delta \lambda \approx 0.23 $nm)  correspondant à un intervalle spectral du résonateur de longueur optique de 2.33 mm. (b) Zoom de la section efficace d'émission. Section efficace d'émission du mode principal $\sigma_e (\lambda_q = 1029.75 \text{ nm}) = 2.068 \text{X}10^{-20} \text{ cm}^2$ et du mode consécutif  $\sigma_e (\lambda_{q-1} = 1029.98 \text{ nm}) = 2.052 \text{X}10^{-20} \text{ cm}^2$.}
\label{fig:section efficace emission}
\end{figure}
\pagebreak

La figure \ref{graph:fact_cont_exp_ideal_et_crit} montre le facteur de contraste en fonction de l'angle de torsion $\alpha$ calculé avec les paramètres mesurés des réseaux utilisés dans nos expériences. Dans cette figure, il y a également une ligne qui montre le facteur de contraste maximal pour lequel on peut s'attendre à la suppression de l'émission multimode provenant d'un seul mode de polarisation dans le cas de la configuration du résonateur qu'on utilise pour les expériences. 

\pagebreak
\begin{figure}[h!]%Le h! assure que la figure reste proche d'ici. autre options t-top, b-bottom, p - float page. Voir par exemple: https://tex.stackexchange.com/questions/39017/how-to-influence-the-position-of-float-environments-like-figure-and-table-in-lat
\centering
\includegraphics[scale=0.4]{fact_cont_exp_ideal_et_crit.jpg}
\caption{Facteur de contraste pour le cas idéal en rouge et le cas expérimental en noir. Les propriétés optiques utilisées pour ces simulations sont indiquées dans le tableau \ref{tab:ellipso_miroir}. La zone d'émission d'un seul mode longitudinal appartenant au même état propre de polarisation étant toute valeur du facteur de contraste inférieur à $\vert \gamma_{\text{crit}} \vert$ (ligne pointillée)} 
\label{graph:fact_cont_exp_ideal_et_crit}
\end{figure}

Dans ce développement, nous supposons que l'une des fréquences de résonance se situe au maximum de la section efficace d'émission, mais dans nos expériences, nous n'avons pas le contrôle nécessaire sur la longueur optique du résonateur pour garantir que cette hypothèse est respectée. Il en résulte une différence plus faible dans la section efficace d'émission des deux modes, ce qui réduit la valeur maximale du facteur de contraste toléré et rend plus difficile l'émission monomode. Un autre hypothèse émise est que les pertes à l'intérieur du résonateur sont faibles et que, par conséquent, l'amplitude des ondes contra-propagatrices est constante et égale à l'intérieur du milieu actif. Dans notre cas, les pertes ne sont pas négligeables, ce qui signifie que les ondes contra-propagatrices n'ont pas la même amplitude à chaque position z, réduisant ainsi la capacité des ondes à s'interférer, c'est-à-dire que la visibilité des franges d'interférence le long de l'axe z diminue. Par conséquent, la valeur critique du facteur de contraste donnée par l'équation \ref{eq:gamma_crit} serait plus élevée si l'on avait tenu compte de l'amplitude variable des ondes.

\pagebreak

\subsubsection{Condition d'émission d'un seul mode de polarisation}
\label{sub sec:condition d'émission d'un seul mode de polarisation}

Passons maintenant à la question de la condition pour avoir la suppression d'un mode de polarisation dans le résonateur.  La suppression d'un mode de polarisation dépend de plusieurs facteurs comme la différence de pertes entre les modes polarisation, du facteur de contraste de l'onde stationnaire et la différence de phase qui affecte l'intervalle de fréquence entre les modes et donc leur position relative sur la courbe de la section efficace d'émission. De plus, à ma connaissance, cette situation n'a pas été abordée dans la littérature scientifique. L'équation \ref{eq:gamma_crit} développé dans \cite{controle_gain_SHB_JF} du facteur de contraste critique ne s'applique pas ici pour les deux modes de polarisation, car nous ne pouvons pas supposer qu'il n'y a pas de corrélation entre $N_1(z)$ créé par le mode de polarisation dominant et l'onde stationnaire de l'autre mode de polarisation susceptible d'osciller. 

Maintenant, une solution analytique pour le taux d'émission stimulée d'un second mode d'un état de polarisation différent est difficile à calculer analytiquement, puisque par rapport au cas précédent, où il n'y avait pas de corrélation entre la distribution de la population du niveau excité créé par le mode dominant et l'intensité de l'onde stationnaire d'un autre mode longitudinal dans le cas de deux modes de polarisation, il existe une corrélation partielle entre $N_1(z)$ et l'intensité de l'onde stationnaire de l'autre mode de polarisation $I_2(z)$, de sorte que nous ne pouvons pas séparer les termes $N_1(z)$ et $I_2(z)$ de la moyenne pour le calcul de l'émission stimulée comme auparavant. La raison pour cette corrélation partielle vient du fait qu'il existe un déphasage entre les ondes des deux modes de polarisation, $\Phi$, donné par l'équation \ref{eq:diff_phase}, qui change à la fois l'emplacement des maxima et minima de l'onde stationnaire et aussi la fréquence de résonance du second mode de polarisation. L'objectif est d'obtenir un critère basé sur le facteur de contraste maximal tolérable pour notre résonateur, qui nous indique les configurations qui sont les meilleures à l'élimination d'un mode de polarisation. Commençons par définir le taux d'émission stimulée du second mode de polarisation $\Gamma_{st,2}$, qui est donné par une moyenne spatiale du produit entre la distribution de la population du niveau excité déjà établi par le mode dominant et l'intensité de l'onde stationnaire du second mode de polarisation avec une faible amplitude de $I_{02}$.

 \begin{equation} \label{eq:taux_stimuler_mode_pol}
 \begin{gathered}
\Gamma_{st,2} = \dfrac{\sigma_{es}(\nu_2)V}{h\nu_2}\langle N_1(z) I_{2}(z) \rangle_z \\
 \text{avec} \\
 I_{2}(z)= 4I_{02}\vert \gamma \vert \text{sin}^2\left(k_2z-\dfrac{\Phi}{4}\right)+2I_{02}(1-\vert \gamma \vert),
\end{gathered}
	\end{equation}

\noindent
où l'indice $2$ représente l'état propre de polarisation non dominante et $\nu_2$ et $k_2$ sont la fréquence et le nombre d'onde qui diffèrent de la fréquence du mode dominant $\nu_1$:

 \begin{equation} \label{eq:freq_diff_mode_pol}
 \begin{gathered}
\nu_2=\nu_1+ \Delta\nu_{\Phi}\\
k_2=2\pi\nu_2/c \\
\text{avec} \\
\Delta\nu_{\Phi}=\dfrac{\Phi}{2\pi}\Delta\nu_{ISL}
\end{gathered}
	\end{equation}

\noindent
où $\nu_1$ est la fréquence du mode dominant qui est supposé au maximum de la section efficace d'émission et $\Phi$ est la différence de phase entre les états propres données par l'équation \ref{eq:diff_phase}. Dans l'équation \ref{eq:taux_stimuler_mode_pol} nous considérons que le déphasage $\Phi$ de l'onde stationnaire est causé de part égale par les deux miroirs, de sorte qu'il existe un minima de la fonction $I_{2}(z)$ et $N_1(z)$ au centre du résonateur peu import la valeur que le déphasage prend et cela est prise en compte par le dénominateur de du terme $Phi$ dans l'équation de $I_2(z)$. La figure \ref{graph:onde_dephaser} montre plusieurs courbes de $I_2(z)$ avec différente valeur de $\Phi$ dans le cas où la phase globale des deux miroirs est identique. Connaître la phase globale de chaque miroir serait nécessaire pour connaître le chevauchement entre $I(z)$ et $N(z)$ en fonction du déphasage ($Phi$) et ferra quelque example de simulation avec de différente phase globale des miroirs plus loin dans la section

\pagebreak

\begin{figure}[h!]%Le h! assure que la figure reste proche d'ici. autre options t-top, b-bottom, p - float page. Voir par exemple: https://tex.stackexchange.com/questions/39017/how-to-influence-the-position-of-float-environments-like-figure-and-table-in-lat
\centering
\includegraphics[scale=0.475]{onde_dephaser.jpg}
\caption{Effet de la variation du déphasage de $\Phi$ sur l'onde stationnaire, en considérant la phase globale des deux miroirs, $M_1$ et $M_2$ égale (cas de deux miroirs identique), avec la valeur de $\Phi$ augmentant de la courbe rouge à la courbe noire puis à la courbe verte.}
\label{graph:onde_dephaser}
\end{figure}

\noindent
la condition pour que ce deuxième mode ne puisse pas osciller peut être décrite comme:

 \begin{equation} \label{eq:condition_seuil_mode_pol}
 \begin{gathered}
\dfrac{ \Gamma_{ \scaleto{ st,2}{7pt}}}{\Gamma_{\scaleto{c,2}{7pt}}} < 1 \\
\text{avec} \\
\Gamma_{\scaleto{c,2}{7pt}} = \dfrac{2 I_{02}N_{\scaleto{th,2}{7.5pt}}V \sigma_{es}(\nu_2)}{h\nu_{2}} 
\end{gathered}
	\end{equation}

\noindent
où $\Gamma_{\scaleto{c,2}{7pt}}$ est le taux de fuite de photon de ce mode donné dans la référence \cite{controle_gain_SHB_JF}. La population non saturée au seuil $N_{\scaleto{th,2}{7pt}}$ pour ce mode dépend de la section efficace d'émission et du temps de vie des photon dans le résonateur $\tau_{c}$ comme:

 \begin{equation} \label{eq:N_th_mode_pol}
 \begin{gathered}
N_{th,2}=\dfrac{1}{\sigma_{es}(\nu_2)\tau_{c,2}c} \\
\text{et} \\
N_{th,2}=N_{th} \dfrac{\sigma_{es}(\nu_1)\tau_{c,1}}{\sigma_{es}(\nu_2)\tau_{c,2}}
\end{gathered}
	\end{equation}
	
En expriment ce paramètre de population non saturée au seuil du mode de polarisation non dominant par rapport à $N_{th}$, on peut dériver le paramètre de pompage normalisé et plus important encore, le rapport entre la section efficace d'émission du mode de polarisation dominant et la section efficace d'émission de l'autre mode de polarisation. On note que ceci nous donne également le rapport entre les temps de vie des photons pour chaque mode qui  diffèrent lorsqu’un mode est mieux confiné qu'un autre, c'est-à-dire  lorsqu'un mode subit moins de pertes que l'autre. Finalement des équations \ref{eq:taux_stimuler_mode_pol}, \ref{eq:freq_diff_mode_pol}, \ref {eq:condition_seuil_mode_pol} et \ref{eq:N_th_mode_pol}, on obtient:


 \begin{equation} \label{eq:inégalité_mode_pol}
 \begin{gathered}
\dfrac{\sigma_{es}(\nu_1)\tau_{c,1}}{\sigma_{es}(\nu_2)\tau_{c,2}} >   \frac{ \langle N_1(z) I_{2}(z)\rangle_z}{2N_{th}I_{02}} 
\end{gathered}
	\end{equation}
	
Les intégrales sont évaluées numériquement en faisant une somme discrete pour différentes valeurs de $\gamma$ en fonction du déphasage $\Phi$. Les résultats présentés dans la figure \ref{graph:gamma_pol_dephasage}, montrent que l'émission d'un seul mode de polarisation est prévue pour toutes les valeurs des courbes en trait plein (côter droit de l'équation \ref{eq:inégalité_mode_pol}), qui sont inférieures à la courbe en trait pointillé (côter gauche de l'équation \ref{eq:inégalité_mode_pol}). Dans ces simulations, les durées de vie des photons des deux états de polarisation sont supposées identiques, $\tau_{c,1}=\tau_{c,2}$ ce qui correspond au cas des états propres de polarisation qui ont les mêmes pertes comme, par exemple, le cas du résonateur à mode torsadé de Siegman \cite{Siegman_twisted_mode}. Lorsque $\Delta\nu_{\Phi}/\Delta\nu_{\text{ISL}}=1$, c'est le cas d'un second mode d'indice longitudinal $q+1$. Lorsque $\Delta\nu_{\Phi}$ est petit, la suppression de l'autre mode de polarisation est difficile, car l'intervalle de fréquences entre ces modes est beaucoup plus petit que $\Delta\nu_{ISL}$, ce qui rend les sections efficaces d'émission des deux modes très proches l'une de l'autre.

\begin{figure}[h!]%Le h! assure que la figure reste proche d'ici. autre options t-top, b-bottom, p - float page. Voir par exemple: https://tex.stackexchange.com/questions/39017/how-to-influence-the-position-of-float-environments-like-figure-and-table-in-lat
\centering
\includegraphics[scale=0.5]{competition_entre_mode_pol_calcule_num.jpg}
\caption{Effet du creusement spatial sur la compétition entre modes de polarisation. Les courbes pleines sont les résultats des intégrale numérique de l'équation \ref{eq:inégalité_mode_pol} et la courbe pointillée, le rapport des sections efficaces d'émission du mode dominant sur l'autre mode de polarisation dans le cas d'un taux de pompage normalisé du mode dominant très élevé, $r=10000$ et pour différente valeur du facteur de contraste de l'onde stationnaire $\vert \gamma \vert$. Toutes les valeurs inférieures à la courbe en pointillé sont celles pour lesquelles l'émission d'un seul mode de polarisation est prédite.}
\label{graph:gamma_pol_dephasage}
\end{figure}

\pagebreak

L'effet de la phase globale des miroirs est investigué en changeant la position du minima de l'onde stationnaire qui ne bouge pas en fonction de $\Phi$. Ceci a pour effet de simuler la phase globale des deux miroirs. On simule quatre cas à la figure \ref{graph:gamma_pol_dephasage_phase_global} pour une valeur de $\vert \gamma \vert=0.2$. Les cas sont: le minima fixe de l'onde stationnaire à l'extérieur de la cavité, courbe en bleu, le minima fixe à l'interface d'un miroir, courbe en rouge, et le minima fixe au centre du résonateur, courbe en violet. On remarque qu'avoir une phase globale des miroirs identique (courbe en violet) est le meilleur cas pour obtenir une plus grande compétition entre les états de polarisation pour toutes valeur de phase $\Phi$ et à mesure que la différence entre la phase globale des deux miroirs augmente, la fréquence des oscillations augmente et le maximum de la fonction augmente.

\begin{figure}[h!]%Le h! assure que la figure reste proche d'ici. autre options t-top, b-bottom, p - float page. Voir par exemple: https://tex.stackexchange.com/questions/39017/how-to-influence-the-position-of-float-environments-like-figure-and-table-in-lat
\centering
\includegraphics[scale=0.5]{gamma_pol_simulation_avec_déphasage_effet_phase_global.jpg}
\caption{Effet de la phase globale des miroirs sur la valeur de l'intégrale de l'équation dans la figure pour un taux de pompage normalisé du mode dominant très élevée ($r=10000$) et $\vert \gamma \vert= 0.2$. Toutes les valeurs inférieures à la courbe en pointillé sont celles pour lesquelles l'émission d'un seul mode de polarisation est prédite.}
\label{graph:gamma_pol_dephasage_phase_global}
\end{figure}

\pagebreak
Maintenant, on peut appliquer l'équation \ref{eq:inégalité_mode_pol} à notre résonateur que l'on utilise pour les expériences. Il faut la connaissance de la différence de phase entre les valeurs propres, le facteur de contraste de l'onde stationnaire et la norme des valeurs propres qui est reliée au pourcentage de perte qui affecte le temps de vie des photons. La figure \ref{graph:perte_et_dephasage_exp_article} a) montre la différence de pertes pour les cas idéal et expérimental et le déphasage entre les deux modes de polarisation ce qui sont des valeurs importantes pour le calcul des angles $\alpha$ qui nous donne une émission d'un seul mode de polarisation. 
 
 \pagebreak
\begin{figure}[h!]%Le h! assure que la figure reste proche d'ici. autre options t-top, b-bottom, p - float page. Voir par exemple: https://tex.stackexchange.com/questions/39017/how-to-influence-the-position-of-float-environments-like-figure-and-table-in-lat
\centering
\includegraphics[scale=0.475]{perte_et_dephasage_exp_article.jpg}
\caption{a) Perte par aller-retour des deux modes de polarisation courbe solide et pointillé, pour le cas idéal en rouge et le cas expérimental en noir. b) Le déphasage entre les deux modes de polarisation pour le cas idéal et expérimental. Les propriétés optiques utilisées pour ces simulations sont indiquées dans le tableau ci-dessus.}
\label{graph:perte_et_dephasage_exp_article}
\end{figure}

\noindent
La relation entre la norme des valeurs propres et le temps de vie est:

 \begin{equation} \label{eq:temps_vie_photon}
 \begin{gathered}
e^{-T_c/\tau_{c,1}}=\vert \lambda_{\pm} \vert^2
\end{gathered}
	\end{equation}
	
\noindent	
où $T_c$ est le temps que prend la lumière pour parcourir un aller-retour et $\lambda_{\pm}$ la valeur propre exprimée en amplitude du champ. La simulation prend en compte que les miroirs sont on la même phase globale ce qui est un hypothèse acceptable puisque les miroirs on des propriéter optique similaire. On voit que la différence de pertes est le facteur qui change énormément la situation comparée au cas où les pertes sont égale. Selon cette prédiction, les modes de polarisation sont fortement supprimés par rapport aux modes longitudinaux appartenant au même mode de polarisation.
\pagebreak
 
\begin{figure}[h!]%Le h! assure que la figure reste proche d'ici. autre options t-top, b-bottom, p - float page. Voir par exemple: https://tex.stackexchange.com/questions/39017/how-to-influence-the-position-of-float-environments-like-figure-and-table-in-lat
\centering
\includegraphics[scale=0.5]{mode_pol_suppression_dans_resonateur_utiliser_MS_C_MP_H_ideal_and_non_ideal.jpg}
\caption{Effet du creusement spatial sur la compétition entre modes de polarisation dans le cas de notre résonateur. Toutes les valeurs inférieures à la courbe pointillée sont celles pour lesquelles l'émission d'un seul mode de polarisation est prédite. a) cas idéal et b) cas expérimental}
\label{graph:suppression_mode_pol_resonateur_exp}
\end{figure}
\pagebreak

En résumant les résultats de la simulation, nous pensons que les angles $\alpha$ optimaux pour l'émission monomode sont ceux pour lesquels le rapport de contraste est inférieur à $\gamma_{\text{crit}}\vert$ et qu'il existe également une différence de pertes entre les modes de polarisation. La différence de perte nécessaire pour supprimer un mode de polarisation est difficile à estimer, mais nous avons trouvé une solution qui nous montre qu'en fait, les modes de polarisation sont plus facilement supprimées du fait qu'ils ont une assez grande différence de perte. Même dans le cas où la différence de pertes est nul montréer dans la figure \ref{graph:suppression_mode_pol_resonateur_exp} a) pour les angles $\alpha \approx 5\degree \text{ à } 85\degree$ ces condition sont montréer possible d'avoir une émission d'un seul mode de polarisation mais la différence entre les deux courbes est très petit donc expérimetalement, il à été démontrer que l'émission des deux mode de polarisation ce produit \cite{Point_except_JF}. Les angles qui sont donc favorables à l'émission monomode pour notre cas de résonateur se situent entre $\alpha = 16\degree$ et $\alpha = 68\degree$, à l'exception d'une petite gamme d'angles autour de $\alpha = 42\degree$ puisque la suppression des modes de polarisation est moins bonne à ces angles.

 \pagebreak
\thispagestyle{empty}


\section{PROPRIÉTÉ D'ÉMISSION D'UN LASER AVEC MIROIRS NANOSTRUCTURÉS}

Ce chapitre présente les résultats expérimentaux de nos tests laser avec des réseaux gravés sur miroir de Bragg. Dans la première section, on passe aux tests laser qui valident l'approche théorique du calcul des états propres en mesurant le seuil d'oscillation et les états propres de polarisation. D'autres mesures sont effectuées pour mesurer la qualité d'émission de notre laser, comme le degré de polarisation, la qualité du profil transversal du faisceau donnée par la mesure du facteur M$^2$, et l'efficacité de conversion de la puissance fournie par le faisceau de pompe en puissance de sortie. Enfin, le spectre d'émission du laser est mesuré. Toutes ces mesures sont effectuées en fonction de l'orientation relative des axes propres des deux miroirs, $\alpha$. Le montage du résonateur, ainsi que le montage optique pour diagnostiquer les caractéristiques d'émission, est illustré dans la figure \ref{fig:montage expérimental}.

\begin{figure}[h!]%Le h! assure que la figure reste proche d'ici. autre options t-top, b-bottom, p - float page. Voir par exemple: https://tex.stackexchange.com/questions/39017/how-to-influence-the-position-of-float-environments-like-figure-and-table-in-lat
\centering
\includegraphics[scale=0.5]{montage expérimental.jpg}
\caption{Figure du montage expérimental de la cavité laser et des outils de diagnostic de l'émission laser. L'angle d'incidence sur la lame de verre, d'environ 10$\degree$, est exagéré sur ce dessin.}
\label{fig:montage expérimental}
\end{figure}
 
\pagebreak

Les expériences laser ont été faites avec une céramique de $\text{Y}_3\text{Al}_5\text{O}_{12}$ (YAG) dopée à 10\% d'$\text{Yb}^{3+}$, c'est-à-dire que 10\% des sites de l'yttrium sont remplacés par des ions d'$\text{Yb}^{3+}$. Ce milieu amplificateur de 1 mm d'épaisseur, revêtu d'une couche antireflet sur les deux faces, est pompé à travers le miroir de pompe par une diode laser émettant à une longueur d'onde de $\lambda= $ 935 nm. La diode laser a été focalisée avec une paire de lentilles pour obtenir une taille de faisceau de 200 $\mu$m dans le milieu actif qui correspond approximativement à la taille du mode fondamental (TEM$_{00}$) dans le milieu amplificateur, empêchant les autres modes transverses d'osciller, ce qui a été vérifié avec une mesure de la qualité du profil d'intensité (M$^2$) montrée plus tard dans ce chapitre. Pour éviter les sauts de modes longitudinaux dus aux instabilités de longueur optique résultant de la vibration des deux miroirs indépendamment l'un de l'autre, les deux miroirs et le milieu actif ont été mis en contact. Afin de ne pas surchauffer le milieu actif, le laser fonctionne en régime quasi continu avec une durée de pompage de 10 ms à un taux de répétition de 10 Hz, c'est-à-dire un rapport cyclique de 10 \%. Pour permettre la caractérisation des propriétés d'émission du laser en fonction de l'angle $\alpha$, le miroir de sortie a été monté sur une platine de rotation avec le faisceau laser de pompe centré sur l'axe de rotation afin que ça soit à peu près la même partie du réseau qui soit sondée. Même si les réseaux sont faits de façon à être homogènes sur toute leur surface, ceci diminue le risque de sonder une partie du réseau qui a un défaut qui change les propriétés des miroirs lors des mesures en fonction d'$\alpha$. 


\subsection{Caractérisation des modes propres de polarisation de l'émission laser et autres caractéristiques de l'émission}


\subsubsection{Mesure reliant les pertes et la norme de la valeur propre}

La caractérisation des modes propres du laser avec miroirs anisotropes a été basée sur de mesures du seuil d'oscillation et mesures d'états de polarisations. Ces deux mesures nous donnent respectivement l'information sur la norme des valeurs propres du résonateur passif et les états propres en fonction de l'angle entre les axes principaux des miroirs $\alpha$. 

La mesure que nous effectuons pour extraire la norme de la valeur propre dont le module est plus élevé est la puissance de pompe au seuil d'oscillation, $P_{th}$, du mode de polarisation dominant, c'est-à-dire le mode qui oscille en premier lorsque la puissance de pompe est augmentée.  Lorsque la puissance de pompe absorbée augmente, la densité de population du niveau supérieur, $N_2$ de la transition laser de $\text{Yb}^{3+}$ augmente linéairement en dessous du seuil d'oscillation. On peut démontrer ceci en regardant les équations de débit pour l'$\text{Yb:YAG}$ qui est un système à quasi quatre niveaux. Ce milieu amplificateur est dit quasi quatre niveaux parce que, bien qu'il y ait quatre niveaux impliqués, les niveaux d'énergie dans un niveau de Stark, $^2F_{5/2}$ et $^2F_{7/2}$, figure \ref{graph:niveau_energy_Yb:YAG}, atteignent l'équilibre thermodynamique dans le régime de picoseconde \cite{Orazio_principal_of_laser} donc seulement deux niveaux globaux ont besoin d'être considérés.

\begin{figure}[h!]%Le h! assure que la figure reste proche d'ici. autre options t-top, b-bottom, p - float page. Voir par exemple: https://tex.stackexchange.com/questions/39017/how-to-influence-the-position-of-float-environments-like-figure-and-table-in-lat
\centering
\includegraphics[scale=0.65]{oraszio_Svelto_principal_of_lasers_Yb_YAG_energy_level.jpg}
\caption{niveau d'énergie de l'$\text{Yb:YAG}$ \cite{Orazio_principal_of_laser_p385}}
\label{graph:niveau_energy_Yb:YAG}
\end{figure}

\noindent
 L'équation de débit pour un tel système peut être écrite comme: 
 
  \begin{equation} \label{eq:population_N2_rate_of_change}
\begin{gathered}
\frac{\text{d}N_2}{\text{d}t} = \frac{I_p}{h\nu_p} \left( \sigma_{ap} N_1 - \sigma_{ep} N_2 \right) - \frac{N_2}{\tau_{rad}} - \frac{I_s}{h \nu_s} \left( \sigma_{es} N_2 - \sigma_{as} N_1  \right) ,
\end{gathered}
\end{equation}


où l'indice $p$ est pour indiquer la pompe et $s$ le faisceau de sortie, $\sigma_{a}$ et $\sigma_{e}$ les sections efficaces effectives respectives d'absorption et d'émission à la longueur d'onde de la pompe ou l'émission laser, $\tau_{rad}$ le taux de relaxation à partir du niveau excité et $N_1$ la population d'atomes dans le niveau bas de Stark, $^2\text{F}_{7/2}$ et $N_2$ la population d'atomes dans le niveau haut de Stark, $^2\text{F}_{5/2}$. Les deux premiers termes du membre de droite concernent l'absorption et l'émission stimulée à la longueur d'onde de la pompe, le suivant l'émission spontanée et autre relaxation non radiative et les deux derniers termes concerne l'émission stimulée et l'absorption à la longueur d'onde du faisceau de sortie. En régime stationnaire $ \frac{\text{d}N_2}{\text{d}t}=0$, avec la relation $N=N_1+N_2$ où N est la concentration d'$\text{Yb}^{3+}$ et le comportement en dessous du seuil d'oscillatio, l'intensité du faisceau de sortie est faible, $I_s \approx 0$ on trouve une équation pour $N_2$ en fonction de l'intensité de pompage:


  \begin{equation} \label{eq:N_2}
\begin{gathered}
N_2 =  \scaleto{ \frac{N\frac{\sigma_{ap}}{(\sigma_{ep}+\sigma_{ap})}}{1+\frac{I_{\text{sat},p}}{I_p}} }{42pt}  \\
\text{avec} \\
I_{\text{sat,}p} = \frac{h\nu_p}{\tau_{\text{rad}}(\sigma_{ap}+\sigma_{ep})}
\end{gathered}
\end{equation} 
  
Le paramètre d'intensité de saturation $I_{sat,p}$, se produit à une valeur de $I_{\text{sat}}=20 \text{kW/cm}^2$ dans l'$\text{Yb}^{3+}$:YAG pour une longueur d'onde de la pompe de $\lambda_p=935 \text{nm}$ \cite{Isat_parameter}. Ce paramètre est beaucoup plus grand que l'intensité de la pompe $I$ mesurée au seuil, de l'ordre de 1 $\text{kW/cm}^2$. Par conséquent, $N_2$ suit une relation approximativement linéaire avec $I_p$ (aussi avec la puissance de pompe $P$) en considérant $I_p \ll I_{sat,p}$. La relation exact entre ces deux variable dépend de la puissance incidente absorbée, du diamètre du faisceau, de la section efficace d'absorption et de la durée de vie de fluorescence du niveau excité.  La population $N_2$ détermine le gain par unité de longueur, $g$, à l'intérieur du milieu actif. Au seuil d'oscillation, le gain dans un aller-retour compense exactement les pertes du mode dominant, qui proviennent principalement de la réflexion des miroirs et des autres sources de pertes fractionnaires, $A$, réparties à l'intérieur du résonateur. La condition de seuil est la suivante :


 \begin{equation} \label{eq:condition seuil}
 \begin{gathered}
\text{max}(\vert\lambda_{\pm}\vert^2)\text{exp}(2gd)(1-A)=1
\end{gathered}
	\end{equation}
	
\noindent
où $d$ est l'épaisseur du milieu actif. On remarque qu'on fait l'opération de $1-A$ puisque $A$ représente les pertes fractionnaires, mais nous avons besoin de ce qui reste après les pertes. Le terme $\text{max}(\vert\lambda_{\pm}\vert^2)$ remplace le terme usuel de la réflectance des deux miroirs et l'élévation au carré de la norme de la valeur propre est pour avoir ce terme exprimé en intensité. Le gain par unité de longueur, $g$, est donné par:

\begin{equation} \label{eq:gain}
\begin{gathered}
g=N_2 \sigma_{es}-N_1\sigma_{as} = N_2(\sigma_{as} + \sigma_{es})-N\sigma_{as}
\end{gathered}
\end{equation}

\noindent
 À partir des équations \ref{eq:condition seuil} et \ref{eq:gain}, nous obtenons la population de l'état excité au seuil, donnée par : 

\begin{equation} \label{eq:populationN2}
\begin{gathered}
N_{th} = \frac{N\sigma_{as}d-\text{ln}(1-A)-2\text{ln}(\text{max}\vert \lambda_{\pm} \vert)}{2d(\sigma_{es}+\sigma_{as})}.
\end{gathered}
\end{equation}

\noindent
Puisqu'au seuil d'oscillation, $N_{th}$ est proportionnel à la puissance de pompe, on obtient:


\begin{equation} \label{eq:Pompe_val_propre}
\begin{gathered}
P_{th}(\alpha) \propto \frac{N\sigma_{as}d-\text{ln}(1-A)-2\text{ln}(\text{max}\vert \lambda_{\pm} \vert)}{2d(\sigma_{es}+\sigma_{as})}
\end{gathered}
\end{equation}


Les valeurs expérimentales de $P_{th}$ en fonction de $\alpha$ sont présentées dans la figure \ref{fig:seuil}, ainsi que les valeurs théoriques calculées à partir de l'équation \ref{eq:Pompe_val_propre} avec un seul facteur d'échelle ajustable puisque le premier terme est déterminé en fonction de la valeur du milieu amplificateur et que le second terme a été trouvé négligeable pour avoir un bon ajustement paramétrique. Les valeurs de $P_{th}$ mesurées expérimentalement suivent bien les prédictions théoriques. Comme on peut le voir sur la figure \ref{fig:seuil}a), les deux modes ont des pertes très différentes autour de $\alpha = 0\degree  \text{ et }  90\degree$ , donc on s'attend à ce qu'un seul état propre de polarisation oscille pour des valeurs de $\alpha$ dans ces voisinages. D'autre part, les pertes par aller-retour sont très similaires entre $\alpha = 30\degree \text{ et } 60\degree$, ainsi qu'entre $\alpha = 120\degree \text{ et } 150\degree$ donc l'émission des deux mode de polarisation est plus probable à ces angles. 

Si la puissance de la pompe est augmentée bien au-delà de $P_{th}$, il est possible de détecter le seuil d'émission du mode de polarisation non dominant lorsque les pertes par aller-retour du mode non dominant ne sont pas trop importantes par rapport à celles du mode dominant, figure \ref{fig:seuil} b). L'émission des deux modes de polarisation peut alors être facilement observée en utilisant un polariseur elliptique, composé d'une lame quart d'onde et d'un polariseur linéaire en raison de l'oscillation non simultanée dans le temps des deux modes de polarisation. En réglant l'angle de la lame quart d'onde et du polariseur, on peut bloquer un mode de polarisation spécifique et transmettre partiellement l'autre. 

\begin{figure}[h!]%Le h! assure que la figure reste proche d'ici. autre options t-top, b-bottom, p - float page. Voir par exemple: https://tex.stackexchange.com/questions/39017/how-to-influence-the-position-of-float-environments-like-figure-and-table-in-lat
\centering
\includegraphics[scale=0.9]{mesure seuil.jpg}
\caption{Mesure de la puissance de pompe au seuil, $P_{th}$ en fonction de l'angle $\alpha$. Les courbes en trait plein et en pointillée sont un ajustement de l'équation de $P_{th}$, aux mesures prises à l'aide d'un facteur d'ajustement global. b) mesure de puissance de pompe au seuil des deux modes de polarisation.}
\label{fig:seuil}
\end{figure}

Comme prévu, l'oscillation des deux modes est bien supprimée près de $\alpha = 0\degree \text{ et } 90\degree$, même à des valeurs de puissance de pompe qui dépassent plusieurs fois la valeur de la puissance de pompe au seuil. Lorsque $\alpha$ s'éloigne de $0\degree$, un second seuil finit par apparaitre, d'abord à une puissance de pompe très élevée, puis diminue pour atteindre des valeurs similaires au premier seuil pour des valeurs d'$\alpha$ autour de $45\degree$. À cet angle, les deux modes ont un seuil d'oscillation presque identique. D'après les calcules théorique fait dans la section \ref{sub sec:condition d'émission d'un seul mode de polarisation} on ne s'attendait pas d'avoir l'émission de l'autre mode de polarisation pour tout angle $\alpha$. Cela nous indique qu'un autre effet autre que le creusement spatial a un impact sur l'émission de l'autre mode de polarisation.

\pagebreak

\subsubsection{Mesure des états propres de polarisation}

En ce qui concerne la mesure des états propres de polarisation, ces mesures ont été effectuées en annulant la puissance transmise à travers une lame quart d'onde suivie d'un polariseur linéaire, figure \ref{fig:montage expérimental}. La lame quart d'onde peut transformer tout état de polarisation elliptique en un état de polarisation linéaire, qui peut ensuite être éliminé par le polariseur. Intuitivement, nous pouvons voir qu'une lame quart d'onde transforme la polarisation elliptique en polarisation linéaire lorsqu'elle est orientée le long des axes majeurs et mineurs de l'état de polarisation, puisque dans la base des axes majeurs et mineurs de l'état de polarisation, les deux ondes sont déphasées d'un quart de période, de sorte que la lame quart d'onde ajoute un quart de période à l'une de ces ondes. Après cette transformation, l'orientation de l'état linéaire est déterminée par le rapport de la longueur des deux axes de l'ellipse, le paramètre d'ellipticité, $\chi$ donc le polariseur doit être à $90\degree$ de cette orientation pour éliminer le signal.  La relation entre les orientations de la lame quart d'onde, $\theta_{\lambda/4}$, et l'orientation de l'axe majeur ou mineur de l'ellipse, $\xi$, et la relation entre l'orientation du polariseur, $\theta_{\text{pol}}$, et l'ellipticité, $\chi$, sont donées par:

\begin{equation} \label{Poincarer_1}
\begin{gathered}
\xi= \theta_{\lambda/4} 
\end{gathered}
\end{equation}

\begin{equation} \label{Poincarer_2}
\begin{gathered}
\chi= \theta_{\text{pol}}-\theta_{\lambda/4}-\pi/2
\end{gathered}
\end{equation}

\noindent
Les paramètres de l'ellipse sont ensuite associés aux coordonnées sur la sphère de Poincaré de sorte que l'ambiguïté de différentes combinaisons de $\theta_{\lambda/4}$ et $\theta_{\text{pol}}$ produisant le même état de polarisation est levée. Pour énumérer les différentes combinaisons donnant le même état, il y a quatre orientations possibles de la lame quart d'onde selon que l'axe rapide de la lame quart d'onde s'aligne avec l'axe majeur ou l'axe mineur de l'ellipse et deux selon l'axe mineur de l'ellipse qui produisent un état linéaire à la sortie et pour chacune de ces orientations, le polariseur à deux orientations possible qui annulent le signal. La correspondance entre les paramètres de l'ellipse et la sphère de Poincaré est donnée par la formule suivante.


\begin{equation} \label{Poincarer_}
\begin{gathered}
\begin{bmatrix}  X \\ Y \\ Z  \end{bmatrix} = 
\begin{bmatrix} \text{cos}(2\xi) \text{cos}(2\chi) \\ \text{sin}(2\xi)\text{sin}(2\chi) \\ -\text{sin}(2\chi) \end{bmatrix},
\end{gathered}
\end{equation}



\noindent
où les points $X=-1 \text{ et } 1 $ sur la sphère représentent la polarisation verticale et horizontale respectivement, $Y=-1 \text{ et } 1 $ représentent la polarisation anti-diagonale et diagonale respectivement et $Z=-1 \text{ et } 1 $ représentent la polarisation circulaire gauche et circulaire droite respectivement. Ces mesures des coordonnées des états de polarisation sur la sphère de Poincaré sont montrées à la figure \ref{fig:mode_pol_mesure}. Pour les valeurs $\alpha$ où les deux modes ont des pertes similaires et peuvent tous deux osciller, nous avons constaté que les deux modes n'oscillaient pas simultanément, mais sautaient plutôt de l'un à l'autre. Cet effet est montré et expliqué à la fin de la section \ref{sec:Spectre d'émission}. En raison de l'oscillation non simultanée des deux modes, l'utilisation du polariseur elliptique et d'une photodiode a suffi pour déterminer l'état exact de la polarisation de chacun des deux modes en observant l'annulation de chaque mode de polarisation sur l'oscilloscope. 

Du point de la mesure, on doit faire le chemin inverse de la lumière pour propager ces états soit à l'intérieur figure \ref{fig:mode_pol_mesure} a) ou juste à l'extérieur du résonateur figure \ref{fig:mode_pol_mesure} b). Ce chemin partant du photodétecteur, figure \ref{fig:montage expérimental}, est composé d'une réflexion sur une lame de verre taillée en biseau qui est légèrement diatténuante en raison des différents coefficients de réflexion des composantes parallèle et perpendiculaire du champ à incidence oblique ($\approx 10\degree$), puis pour l'état à l'intérieur de la cavité, la transmission en travers du miroir de sortie du résonateur qui a été mesuré par ellipsométrie \ref{transmission_miroir_sortie}. Pour les mesures du mode de polarisation présentant les pertes plus élevées, nous avons dû augmenter la puissance de pompage bien au-delà du seuil pour qu'il apparaisse et se prête à une mesure. En se rapprochant des valeurs $\alpha=0\degree \text{ et } 90\degree$, la différence de pertes entre les deux modes de polarisation est devenue trop importante pour permettre une caractérisation du mode ayant les pertes les plus élevées, même au puissance les plus élevées disponible dans notre expérience. De plus, on remarque que, lorsque les deux miroirs sont en contact, l'augmentation de la stabilité de la longueur optique rend l'oscillation du mode non dominant plus difficile à obtenir donc, pour avoir le plus de données possible du mode non dominant, ces mesures ont été faites sans le contact entre les miroirs et le milieu amplificateur, ce qui n'a aucun effet sur les matrices de Jones d'un aller-retour et ne change pas les états propres de polarisation. 


	
\begin{figure}[h!]%Le h! assure que la figure reste proche d'ici. autre options t-top, b-bottom, p - float page. Voir par exemple: https://tex.stackexchange.com/questions/39017/how-to-influence-the-position-of-float-environments-like-figure-and-table-in-lat
\centering
\includegraphics[scale=0.85]{polarisation_mesure_interieur_exterieur.jpg}
\caption{ Représentation des états de polarisation sur la sphère de Poincaré, a) à l'intérieur du résonateur, b) juste à l'extérieur.}
\label{fig:mode_pol_mesure}
\end{figure}

\subsubsection{Autres mesures pertinentes pour la caractérisation d'un laser: le paramètre M$^2$, le degré de polarisation et l'efficacité de conversion optique}


Un autre caractéristique importante du comportement transverse des modes est le profil d'intensité des modes. Dans notre cas, pour avoir seulement le mode fondamental $\text{TEM}_{00}$, on implémente une technique qui consiste à avoir un faisceau de pompe focalisé sur le milieu amplificateur dont la taille est la même que celle du mode fondamental $\text{TEM}_{00}$ du faisceau de sortie. La taille du mode fondamental dépend dans notre cas seulement de l'effet de la lentille thermique et, dans une moindre mesure, de l'effet d'iris douce due à l'absorption à la longueur d'onde du faisceau de sortie dans la périphérie du faisceau pompe puisque nos miroirs sont des miroirs plans. Lorsqu'un faisceau contient non seulement le mode fondamental, mais aussi d'autres modes supérieurs, le produit de sa divergence et de sa taille au pincement sera plus grand. Ce fait nous permet de se rendre compte si un faisceau contient d'autre mode transverse. En effet, une seule mesure du profil d'intensité du faisceau n'est pas suffisante pour démontrer que le faisceau est de forme du mode $\text{TEM}_{00}$ car, une combinaison d'autre mode $\text{TEM}_{m,n}$ peut avoir un profil d'intensité comme le mode fondamental gaussien, mais, quand le faisceau ce propage, cette combinaison de mode diverge plus rapidement que le mode fondamental \cite{Siegman:98}. C'est pour cette raison que l'on mesure la taille du faisceau à plusieurs positions le long de l'axe de propagation du faisceau après focalisation du faisceau avec une lentille de longueur focale $f=15$ cm. La focalisation par une lentille nous permet d'avoir des mesures de la taille du faisceau près du pincement, $W_0$ et aussi en champ lointain, ce qui rend l'ajustement paramétrique sur l'équation de la propagation du faisceau plus exact. l'équation de la propagation d'un faisceau laser est donnée par:


\begin{equation} \label{propagation_faisceau}
\begin{gathered}
W(z)=W_0\sqrt{1+(z-z_0)^2\left( \frac{\lambda M^2}{\pi W_{0}^2} \right)^2 }
\end{gathered}
\end{equation}
 
\noindent
où $W(z)$ est la taille du faisceau (rayon) en fonction de la position $z$ qui est définie comme le rayon qui contient 2 écarts types de l'intensité, qui correspond a la position ou l'intensité diminue à la valeur $1/e^2$ pour un faisceau gaussien, $z_0$ la position du pincement et M$^2$ est le paramètre de qualité du profil transversal du faisceau. Lorsque le M$^2=1$ le seul mode qui oscille est le mode fondamental puisqu’en fixant la taille de $W_0$, on peut voir d'après l'équation \ref{propagation_faisceau} que pour toute valeur supérieure à M$^2=1$ le faisceau diverge plus vite. Ces mesures ont été prises pour les angles $\alpha = 0\degree, 12\degree, 24\degree  \text{ et } 36\degree $, figure \ref{fig:qualiter_faisceau}. Ces mesures doivent être effectuées avec le moins de bruit de fond possible sur la caméra CCD, afin que la taille du faisceau ne soit pas élargie par ce bruit. La grande majorité du bruit de fond sur la caméra est dû au faisceau pompe, donc l'insertion de filtres qui bloquent la longueur d'onde de pompe et l'utilisation d'un iris un peu plus grand que la taille du faisceau de sortie avant la lentille de focalisation aident a avoir une mesure plus exacte de la taille du faisceau de sorties. On remarque que le facteur de qualité du faisceau est près de 1 pour les angles $\alpha = 0\degree \text{ et } 12\degree \text{ et } 24\degree $ mais augmente significativement à $\alpha = 36\degree $. Ceci semble être lié au fait que pour cette valeur de $\alpha$, deux modes de polarisation coexistent. En effet, selon la figure \ref{fig:qualiter_faisceau_2_deux_pol}, lorsque l'on effectue une mesure à $\alpha = 24\degree$, on remarque que lorsque l'on a un seul mode de polarisation, le facteur de qualité est plus près de 1. Cela pourrait être causé par le fait que dans ce cas de l'émission de deux mode de polarisation, puisqu'il y a plus de mode qui oscille dans la cavité et qu'aucun de ces modes sont parfaitement gaussien donc la somme de ces modes diminue la qualité du faisceau. 

\begin{figure}[h!]%Le h! assure que la figure reste proche d'ici. autre options t-top, b-bottom, p - float page. Voir par exemple: https://tex.stackexchange.com/questions/39017/how-to-influence-the-position-of-float-environments-like-figure-and-table-in-lat
\centering
\includegraphics[scale=0.575]{M_carré_alpha_0_12_24_36.jpg}
\caption{ Mesures de la taille du faisceau pour a) $\alpha = 0\degree$, b) $\alpha= 12\degree$, c) $\alpha= 24\degree$, d) $\alpha= 36\degree$, à différentes positions après une focalisation par lentille. la mesure est faite selon deux axes du profil, selon l'axe x en bleu et l'axe y en rouge}
\label{fig:qualiter_faisceau}
\end{figure}


\pagebreak


\begin{figure}[h!]%Le h! assure que la figure reste proche d'ici. autre options t-top, b-bottom, p - float page. Voir par exemple: https://tex.stackexchange.com/questions/39017/how-to-influence-the-position-of-float-environments-like-figure-and-table-in-lat
\centering
\includegraphics[scale=0.55]{M_carré_alpha_12_deux_mode_pol.jpg}
\caption{ Mesure de la taille du faisceau pour $\alpha= 24\degree$, avec l'émission contenant un seul mode de polarisation, a), deux modes de polarisation, b).}
\label{fig:qualiter_faisceau_2_deux_pol}
\end{figure}


\pagebreak


On a vérifié la qualité du profil transverse de notre laser, mais une autre qualité importante est celle de la qualité des états de polarisation de nos modes. On a utilisé une théorie qui néglige complètement la dépolarisation qui pourrait se produire dans notre cavité laser. Est-ce que ceci est justifié. La mesure du degré de polarisation, $P$, est là pour vérifier cette hypothèse que la dépolarisation de nos modes est petite. La première chose à vérifier est la dépolarisation causée par nos miroirs, qui, en première vue, est la cause principale de dépolarisation dans la cavité. En mesurant, les paramètres de Stokes de nos miroirs en réflexion faits dans l’annexe A, nous ont permis de constater que la dépolarisation est faible et nous nous attendons donc à ce que le degré de polarisation de l'émission soit également élevé. Le degré de polarisation, $P$, pour différentes valeurs de $\alpha$ est représenté sur la Fig \ref{fig:degré_pol}. Pour ce faire, nous avons mesuré le rapport d'extinction de polarisation, $\eta$, de notre faisceau, ce qui implique de calculer le rapport entre le signal sur la photodiode pour une orientation de la lame quart d'onde et du polariseur qui minimise le signal et le signal avec le polariseur tourné de $90\degree$ par rapport à la mesure précédente ce qui maximise le signal. Les paramètres $P$ et $\eta$ sont liés par l'équation suivante \cite{polarimetric_characterisation_of_light1}:


\begin{equation} \label{propagation_faisceau}
\begin{gathered}
P=\frac{\eta -1}{\eta +1}
\end{gathered}
\end{equation}


\noindent
Cette équation est bornée de 0 à 1, avec 0 étant un état complètement dépolarisé et 1 un état parfaitement polarisé. Nous avons trouvé que $P$ était supérieur à 0,99 pour la plupart des valeurs $\alpha$, sauf près de $\alpha=0\degree$, où il est toujours supérieur à 0,9. Nous remarquons que le changement de la valeur de $P$ est fortement corrélé avec le changement rapide de l'état de polarisation à l'intérieur du résonateur, de rectiligne à circulaire entre $\alpha=0\degree \text{ et } 10\degree$, figure \ref{fig:mode_pol_mesure} a). La raison pour lequel le degré de polarisation est moins bon autour de $\alpha$ n'est pas bien comprise. En faite, l'effet de la biréfringence thermique devrait être le mieux éliminé à $\alpha = 0\degree$ puisqu’à cet angle, l'état propre de polarisation est linéaire et la configuration du résonateur est similaire à un résonateur utiliser par \cite{W_A_Clarkson_2001} qui insère un polariseur d'un bord du résonateur et une lame quart d'onde de l'autre bord. L'un des effets possibles d'une dépolarisation plus importante autour de $\alpha=0\degree$ est que la dépolarisation causée par les réseaux est plus importante parce que la durée de vie des photons est plus élevée en raison de pertes plus faibles, figure \ref{fig:seuil}, de sorte que les photons passent plus souvent en contact avec les miroirs et ce qui augmente la dépolarisation. Cette hypothèse pourrait être testée en mesurant le degré de polarisation pour un résonateur avec un seul miroir légèrement dépolarisant et différents coupleurs de sortie avec différentes réflectances.  Du coup, ce degré élevé de polarisation proche de 1 valide notre approche consistant à utiliser le formalisme de Jones pour décrire les états de propres de polarisation. 

\begin{figure}[h!]%Le h! assure que la figure reste proche d'ici. autre options t-top, b-bottom, p - float page. Voir par exemple: https://tex.stackexchange.com/questions/39017/how-to-influence-the-position-of-float-environments-like-figure-and-table-in-lat
\centering
\includegraphics[scale=0.5]{degré_polarisation.jpg}
\caption{Degré de polarisation en fonction de l'angle $\alpha$ pour une puissance de pompe deux et trois fois au-dessus de la puissance pompe aux seuils. Courbe tireté est seulement un guide pour l'œil.}
\label{fig:degré_pol}
\end{figure}


\pagebreak
Une autre propriété importante d'un laser est sa capacité de convertir la puissance de pompe en puissance de sortie. Pour ce calcul, nous avons choisi de calculer seulement l'efficacité de conversion du milieu amplificateur, donc les pertes du faisceau de pompe par la réflexion sur le miroir pompent, de 8\%, et l'absorption de seulement 22\% du faisceau pompe par le milieu amplificateur sont pris en compte. En théorie, l'efficacité de conversion de l'énergie des photons de pompe à $\lambda=935$nm en photon de sortie à $\lambda=1030$nm ne sera jamais de 100\% \cite{Lacovara:91_quantum_defect}. Ceci est dû au fait que l'énergie des photons de pompe est plus grande que celle des photons émis donc l'énergie restante est convertie en chaleur. La limite théorique d'efficacité pour notre laser est donnée par: $\lambda_p/\lambda_s=0.91\%.$


La puissance de sortie en fonction de la puissance de pompe absorbée est illustrée à la figure \ref{fig:efficacité} pour plusieurs valeurs $\alpha$. Le seuil d'oscillation et l'efficacité de conversion de la pompe varient avec la valeur $\alpha$. Lorsque cette dernière est augmentée par rapport à $0\degree$, les pertes du mode augmentent, figure \ref{fig:seuil} a), donc la puissance du faisceau de sorties qui fuit par le miroir de sortie augmente, de sorte que le seuil augmente, mais l'efficacité de la pente augmente. La puissance de pompe au seuil est comprise entre 40 et 70 mW, tandis que l'efficacité de conversion de la puissance de pompe absorbée en puissance du faisceau de sorties est comprise entre 50 \% et 76 \%. L'efficacité maximale atteinte est où le creusement spatial est bien supprimé, de sorte que l'émission est monomode donc on remarque une bonne corrélation entre l'efficacité de pompage et l'émission monomode.


\begin{figure}[h!]%Le h! assure que la figure reste proche d'ici. autre options t-top, b-bottom, p - float page. Voir par exemple: https://tex.stackexchange.com/questions/39017/how-to-influence-the-position-of-float-environments-like-figure-and-table-in-lat
\centering
\includegraphics[scale=0.45]{efficaciter_laser.jpg}
\caption{efficacité de conversion de la puissance pompe en puissance de sortie}
\label{fig:efficacité}
\end{figure}

\pagebreak
\subsection{Spectre d'émission}
\label{sec:Spectre d'émission}

Cette section concerne les mesures du comportement de la partie longitudinale des modes, c'est-à-dire le spectre d'émission du laser. Deux méthodes sont utilisées pour déterminer le spectre. La première, qui utilise un analyseur de spectre couplé à une fibre avec un monochromateur à réseau (Ocean Optics modèle USB2000+), a une résolution plus faible, de l'ordre de $\Delta \lambda = 1$ nm, mais couvre la gamme spectrale de l'ensemble de la section efficace d'émission du $\text{Yb:YAG}$. Les résultats du spectre en fonction de l'angle $\alpha$ sont montrés à la figure \ref{fig:spectre_avec_spectromètre}. Cette mesure nous assure que la seule raie d'émission est celle à $\lambda = 1030$ nm puisque pour l'$\text{Yb:YAG}$, l'oscillation peut aussi se produire à $\lambda = 1050$ nm lorsque les pertes du résonateur sont suffisamment petites \cite{Yb:YAG_1050nm}.


\begin{figure}[h!]%Le h! assure que la figure reste proche d'ici. autre options t-top, b-bottom, p - float page. Voir par exemple: https://tex.stackexchange.com/questions/39017/how-to-influence-the-position-of-float-environments-like-figure-and-table-in-lat
\centering
\includegraphics[scale=0.85]{spectre_avec_spectromètre.jpg}
\caption{Spectre d'émission normalisé du laser en fonction d'$\alpha$. La largeur de raie est limitée par la résolution du spectromètre.}
\label{fig:spectre_avec_spectromètre}
\end{figure}

La seconde méthode, qui utilise un étalon Fabry-Pérot, FP, a permis d'obtenir une résolution, $\Delta \lambda$ nm, suffisamment élevé pour résoudre les fréquences des modes individuels ($\Delta \lambda = 0.028$ nm) , mais n'a pas la gamme spectrale de la première méthode. La méthode utilisée pour avoir un continuum de fréquences de résonance transmis par le Fabry-Pérot consiste à utiliser un objectif de microscope pour obtenir un continuum d'angle d'incidence dans le FP qui permet d'avoir la condition de résonance pour différente longueur d'onde. La condition de résonance d'un FP est donnée par \cite{born_wolf_1999_FP}: 

\begin{equation} \label{propagation_faisceau}
\begin{gathered}
p\lambda= 2\, n\, e\, \text{cos}(\theta_{\text{in,p}}) + \frac{ \lambda\left(\phi_{R1} + \phi_{R2}\right)}{2\pi}
\end{gathered}
\end{equation}

\noindent
où $\theta_{in,p}$ est l'angle d'incidence dans le FP, $e $ l'épaisseur du FP, $p$ indique l'ordre de la résonance et $\phi_{R1}$ , $\phi_{R2}$ le déphasage à la réflexion dû au fait qu'il y a un multicouche ou une surface métallique sur les deux surfaces du FP. Or, puisque l'épaisseur du FP est beaucoup plus grande que la longueur d'onde, $\phi_{R1}$ et $\phi_{R2}$ peuvent être négligés. Pour sortir l'information des figures d'interférence, on place une lentille de focale $f$  après de FP qui affine la largeur des anneaux d'interférence sur la caméra CCD et qui nous donne une relation entre l'angle d'incidence à la sortie $\theta_{sor,p}$ et le rayon de ces anneaux donné par \cite{Bisson_thin_film_twisted_mode_exp}:


\begin{equation} \label{propagation_faisceau}
\begin{gathered}
r_p(\nu) = f \, \text{tan}(\theta_{\text{sor},p}(\nu))
\end{gathered}
\end{equation}

et avec la loi de Snell Descartes, on peut faire le lien entre l'angle à l'intérieur et à l'extérieur du FP. L'information sur la fréquence d'émission est difficilement accessible puisque l'information sur l'ordre $p$ de chaque anneau n'est pas connue. Par contre, on sait qu'il existe une fréquence $\nu_0$ qui est en interférence constructive pour un angle d'incidence $\theta_{\text{out},p}=0\degree$ c'est-à-dire qui produit un anneau de rayon $r_p\approx 0$ pour un ordre $p$ quelconque. Pour le même ordre $p$, mais pour un angle d'incidence différent de $0\degree$, l'intervalle entre ces deux fréquences est donc:
 
\begin{equation} \label{propagation_faisceau}
\begin{gathered}
\Delta \nu =  \left( \nu_0 -\frac{\nu_0}{\text{cos}(\theta_{\text{in},p})} \right)
\end{gathered}
\end{equation}

\noindent
L'intervalle spectral libre du Fabry-Pérot de (ISL$_{\text{FP}}=200 \text{GHz}$) est choisi nettement plus grand que le ISL du laser, de l'ordre de 64 GHz, de sorte que l'oscillation de plusieurs modes longitudinaux consécutifs soit facilement interprétée comme une duplication des anneaux d'interférence. De plus, la finesse du FP, définie comme le rapport entre ISL$_{\text{FP}}$ et la pleine largeur à mi-hauteur d'une fréquence élargie par le FP, a une valeur de l'ordre de 25, ce qui est suffisant pour résoudre les fréquences des deux modes de polarisation en fonction de l'angle $\alpha$. En effet, en calculant le rapport entre la l'écart minimal spectral entre deux modes mesurables par le FP de $200\text{GHz}/25 = 8\text{GHz}$, ce qui correspond d'après la figure \ref{graph:perte_et_dephasage_exp_article} de la différence de phase entre les deux modes de polarisation à un déphasage de $0.125\times2\pi$, correspondant aux angles $\vert \alpha \vert > 12\degree \text{ et } \vert \alpha-90\degree \vert > 12\degree $, couvrant toute la gamme des angles que l'on observe deux modes de polarisation. 

Les figures d'interférence et les figures des spectres en fonction d'$\alpha$ sont montrées aux figures \ref{fig:image_interference_FP} et \ref{fig:spetre_fabry_perot_graph_en_fonction_alpha}. En guise de comparaison, une mesure du spectre d'émission pour une cavité avec des miroirs isotropes avec une longueur de cavité et de matériau actif similaires est également présentée en haut à gauche de la figure \ref{fig:image_interference_FP} et en bas de la figure \ref{fig:spetre_fabry_perot_graph_en_fonction_alpha}. On constate qu'il y a trois ordres $p$ pour chaque fréquence et qu'ils sont séparés par l'intervalle spectral libre du FP de (ISL$_{\text{FP}}=200 \text{GHz}$). Pour montrer la stabilité de l'émission, chaque spectre de la figure \ref{fig:spetre_fabry_perot_graph_en_fonction_alpha} contient 10 images superposées avec un temps d’exposition de la durée d'une impulsion de la pompe (10ms). En augmentant les valeurs d'$\alpha$ à partir d'$\alpha=0\degree$, le spectre d'émission passe de trois à deux modes à $\alpha=12\degree$, puis le deuxième mode est presque supprimé à $\alpha=24\degree$. Ensuite, le spectre d'émission présente à nouveau trois modes à $\alpha=42\degree$, dont deux très proches qui correspondent aux deux modes propres de polarisation avec le même indice longitudinal puisque les deux états de polarisation ont des pertes similaires, comme le montre la figure \ref{fig:seuil} a). Cependant, avec un niveau du facteur de contraste le plus bas parmi de tous les angles $\alpha$, figure \ref{graph:fact_cont_exp_ideal_et_crit}, il est inattendu d'avoir plusieurs modes longitudinaux appartenant aux mêmes modes de polarisation. Pour des angles $\alpha$ supérieurs à $42\degree$, nous obtenons des résultats similaires avec $\alpha=48\degree$ dont on a obtenu une émission monomode, mais avec des sauts de mode de polarisation très peu fréquente. La cause de ces sauts de mode est possiblement due à de fluctuations lentes de la longueur de la cavité. Notre meilleur résultat d'émission monomode qui a peu de fluctuation entre deux modes de polarisation lorsqu'un bon contacte entre les composantes est fait a été trouvée à $\alpha=66\degree$.

L'évolution qualitative des spectres d'émission en fonction de l'angle $\alpha$ est en bon accord avec le comportement du facteur de contraste, qui prédit que les angles dans l'intervalle $\alpha = 24\degree \text{ à } 66\degree $ devraient avoir un seul mode longitudinal appartenant à un mode de polarisation. Ce calcul est expliqué en plus de détail dans la section \ref{sec:condition d'émission monomode dans les résonateurs à onde stationnaire}. Or, pour la plupart des mesures dans cet intervalle d'angle $\alpha$ que l'on observe l'émission de deux modes, ce sont deux modes longitudinaux associés à deux états de polarisation différente. Ce résultat est surprenant puisque la théorie prédit que les modes de polarisation pour tous les angles $\alpha$ devraient être plus faciles à éliminer que les modes longitudinaux. Cependant, ce calcule considère le cas d'oscillation simultanée de mode de polarisation est bien éliminé, mais, dans notre cas, les modes s'échangent l'énergie périodiquement. En outre, notre meilleur résultat d'émission monomode à $\alpha=66\degree$ se situe à un angle pour lequel le facteur de contraste $\vert\gamma\vert=0,11$ est inférieur au facteur de contraste maximal $\vert\gamma_{\text{crit}}\vert=0,135$ pour lequel nous prévoyons une émission monomode. De plus, cette mesure présente la plus grande différence de perte entre les deux modes de polarisation parmi les angles dans la plage de $\alpha = 24\degree \text{ à } 66\degree $ qui ont un facteur de contraste inférieur au facteur de contraste maximal $\gamma_{\text{crit}}$.

\begin{figure}[h!]%Le h! assure que la figure reste proche d'ici. autre options t-top, b-bottom, p - float page. Voir par exemple: https://tex.stackexchange.com/questions/39017/how-to-influence-the-position-of-float-environments-like-figure-and-table-in-lat
\centering
\includegraphics[scale=1.05]{image_interference_FP.jpg}
\caption{Anneaux d'interférence sur la caméra CCD obtenus avec la configuration du Fabry-Pérot. Ces mesures de spectre ont été faites à une intensité de pompage $I$ de 2 fois l'intensité du courant de seuil $I_{th}$, dans la diode laser. intervalle spectral libre (FSR)}
\label{fig:image_interference_FP}
\end{figure}

\begin{figure}[h!]%Le h! assure que la figure reste proche d'ici. autre options t-top, b-bottom, p - float page. Voir par exemple: https://tex.stackexchange.com/questions/39017/how-to-influence-the-position-of-float-environments-like-figure-and-table-in-lat
\centering
\includegraphics[scale=1.2]{spetre_fabry_perot_graph_en_fonction_alpha.jpg}
\caption{Spectre obtenu d'après les images des anneaux d'interférence de la figure \ref{fig:image_interference_FP} avec la configuration du Fabry-Pérot.}
\label{fig:spetre_fabry_perot_graph_en_fonction_alpha}
\end{figure}

\pagebreak
Un autre aspect important pour un laser est le comportement de son émission en fonction de la puissance de pompage. En augmentant le taux de pompage, les angles $\alpha = 24\degree \text{ à } 66\degree $ pour lesquels le facteur de contraste est inférieur à $\gamma_{crit}$, on ne s'attend pas à avoir d'autres modes longitudinaux appartenant aux mêmes modes de polarisation. Les mesures en fonction du taux de pompage ont été faites pour une intensité de courant dans la diode laser de pompe de 1.2, 1.5, 2 et 3 fois sur la valeur seuil. Nous nous sommes arrêtés à 3 fois la valeur du seuil en raison de dommages sur les réseaux qui se sont produits à des taux de pompage plus haut. L'observation au microscope des réseaux a montré qu'il y avait plusieurs défauts où l'absorption du faisceau était fortement augmentée. De plus, nous avons remarqué que le réseau de sortie est celui qui subit le plus de dommages optiques, ce qui est en accord avec la simulation de la distribution d'intensité dans les miroirs, figure \ref{graph:distribution_champ_electric_miroir}, qui est 2 fois plus intense dans le réseau de sortie.

 L'évolution des spectres d'émission en fonction de la puissance de la pompe est montrée dans la figure \ref{fig:perte_fabry_perot_graph_effet_taux_pompage} pour le cas de (a) $\alpha = 42\degree$ (b) $66\degree$ , (c) $90\degree$ et (d) un résonateur de géométrie similaire avec des miroirs isotropes conventionnels. L'émission est restée principalement monomode pour $\alpha = 66\degree$, tandis que pour $\alpha = 42\degree$, l'émission est devenue principalement bimodale à partir d'une intensité de pompage égale à deux fois la valeur seuil. Pour le résonateur anisotrope avec $\alpha = 90\degree$, le nombre de modes est passé de 3 à 5, tandis que pour un résonateur isotrope, le nombre de fréquences est passé de 5 à 6 pour la même gamme de pompage. Le fonctionnement hautement multimode observé à proximité d'$\alpha = 0\degree \text{ et } 90\degree$ et l'amélioration du caractère monomode entre ces deux valeurs est cohérents avec la quasi-annulation du creusement spatial loin des valeurs d'$\alpha = 0\degree \text{ et } \alpha 90\degree$, comme on peut le voir sur la figure \ref{graph:fact_cont_exp_ideal_et_crit}, courbe B. De même, la coexistence des états de polarisation, qui est plus fréquemment observée autour de $42\degree \text{ et } 48\degree$ qu'à des valeurs $\alpha $ plus proches de $\alpha = 0\degree \text{ et } 90\degree$, est cohérente avec la différence de pertes, figure \ref{graph:perte_et_dephasage_exp_article} a) et confirmée par les mesures du second seuil dans la figure \ref{fig:seuil} b).

\pagebreak

\begin{figure}[t!]%Le h! assure que la figure reste proche d'ici. autre options t-top, b-bottom, p - float page. Voir par exemple: https://tex.stackexchange.com/questions/39017/how-to-influence-the-position-of-float-environments-like-figure-and-table-in-lat
\centering
\includegraphics[scale=1.15]{spetre_fabry_perot_graph_effet_taux_pompage.jpg}
\caption{Évolution du spectre d'émission en fonction de l'intensité de pompage pour des résonateurs anisotropes (a) $\alpha = 42\degree$ , (b) $\alpha = 66\degree$ et (c) $\alpha = 90\degree$, et (d) pour un résonateur à miroirs isotropes pour une intensité de pompage $I$ allant de 1,2, 1,5, 2, et 3 fois l'intensité du courant de seuil, $I_{\text{th}}$, dans la diode laser.}
\label{fig:perte_fabry_perot_graph_effet_taux_pompage}
\end{figure}

On constate que l'intervalle spectral entre les modes n'est pas toujours un multiple de l'ISL de la cavité laser de $64$ GHz. La raison pour ceci est due au fait que les modes de polarisation n'oscillent pas à la même fréquence, figure \ref{graph:perte_et_dephasage_exp_article} b). Pour démontrer que les différentes fréquences appartiennent à des modes de polarisation différents, on mesure le spectre avec le montage FP, mais avec une lame quart d'onde et un polariseur sur le chemin de cette configuration, ce qui nous permet d'éliminer un état de polarisation et laisser passer une partie de l'intensité de l'autre mode. Ceci a été fait pour l'angle $\alpha = 42\degree$, figure \ref{fig:spectre_double_polarisation}. On remarque que l'intervalle entre les deux modes de polarisation n'est pas ce que l'on attendrait de la différence de phase , figure \ref{graph:perte_et_dephasage_exp_article} b), d'un intervalle de 30GHz, mais plutôt on obtient un intervalle de 94GHz. Cet intervalle correspond bien à ce qui est prévu pour deux modes de polarisation qui diffèrent d'un indice longitudinal $p$ et $p+1$. La raison pour laquelle les indices diffèrent pourrait être liée au creusement spatial qui fait en sorte qu'un mode de polarisation ait plus de gain à une fréquence plus éloigné du mode dominant. 

\begin{figure}[h!]%Le h! assure que la figure reste proche d'ici. autre options t-top, b-bottom, p - float page. Voir par exemple: https://tex.stackexchange.com/questions/39017/how-to-influence-the-position-of-float-environments-like-figure-and-table-in-lat
\centering
\includegraphics[scale=1.25]{spectre_double_polarisation.jpg}
\caption{Spectre d'émission observé à $\alpha=42\degree$ sans polariseur (haut) et avec un polariseur elliptique qui ne transmet que le premier et le deuxième mode propre de polarisation (milieu et bas) à une intensité de pompage de 2,5 fois la valeur seuil.}
\label{fig:spectre_double_polarisation}
\end{figure}
 
La dernière mesure faite pour les spectres consiste à montrer que les modes de polarisation n'oscillent pas simultanément, mais s'échangent l'énergie dans le temps. En plaçant le polariseur elliptique sur le chemin de la photodiode, nous pouvons bloquer l'un des modes de polarisation sur la photodiode tout en continuant à voir le contenu spectral complet du laser sur la caméra CCD. Cela nous permet de voir sur l'oscilloscope les effets du blocage d'un mode, comme le montre la figure \ref{fig:spectre_mode_pol_dans_le_temps}. Un fonctionnement monomode dans l'un ou l'autre état propre de polarisation est visible dans les figures d'interférence du haut et du milieu, tandis qu'une paire d'anneaux due à l'émission des deux modes est visible dans la figure du bas de la figure \ref{fig:spectre_mode_pol_dans_le_temps} (a). Chaque anneau d'interférence peut être converti en fréquence, figure \ref{fig:spectre_mode_pol_dans_le_temps} (b) et chaque spectre est associé à une courbe d'intensité tirée de l'oscilloscope \ref{spectre_mode_pol_dans_le_temps} c). Lorsque l'on bloque un état propre sur la photodiode en utilisant un polariseur elliptique, on ne voit rien (trace du haut), la pleine puissance d'un mode est reçue (trace du milieu), ou la transition d'un mode à l'autre (trace du bas), figure \ref{fig:spectre_mode_pol_dans_le_temps} (c).

\begin{figure}[h!]%Le h! assure que la figure reste proche d'ici. autre options t-top, b-bottom, p - float page. Voir par exemple: https://tex.stackexchange.com/questions/39017/how-to-influence-the-position-of-float-environments-like-figure-and-table-in-lat
\centering
\includegraphics[scale=0.75]{spectre_mode_pol_dans_le_temps.jpg}
\caption{Spectres d'émission observés à $\alpha = 42\degree$. (a) Images de la caméra CCD des anneaux d'interférence du FP à différents moments, (b) analyse correspondante des anneaux pour obtenir le contenu spectral et (c) traces d'oscilloscope correspondantes de l'intensité du faisceau sur la photodiode. Pour cette dernière, un polariseur elliptique est ajusté pour supprimer le mode de polarisation propre (mode 1) sur le chemin de la photodiode.}
\label{fig:spectre_mode_pol_dans_le_temps}
\end{figure}

Une hypothèse possible pour l'oscillation non simultanée des modes de polarisation est que, puisque le creusement spatial est bien supprimé et il y a une différence de perte entre les modes. La population d'atome excité selon l'axe de propagation du faisceau est en majeure partie utilisée par le mode dominant donc il n'y a pas une grande population d'atomes excités non exploitée par le mode dominant qui peut être utilisé pour l'autre mode de polarisation. Avec l'effet de saturation qui limite la croissance de la population d'atome excité après qu'un mode commence à osciller, l'autre mode de polarisation ayant de plus grandes pertes n'atteint pas le seuil d'oscillation et donc par cet effet nous nous attendons à avoir un seul mode qui oscille à la fois ce qui est prédit par la théorie montée dans la figure \ref{suppression_mode_pol_resonateur_exp} b). Par contre, cette théorie ne prend pas en compte que les modes peuvent s'échanger l'énergie et ne pas osciller simultanément.

Il y a deux hypothèses possibles qui peuvent  tenter des explications de ce phénomène. La première hypothèse est qu'il existe des fluctuations de la longueur du résonateur dues soit à des vibrations mécaniques omniprésentes ou soit par un changement de l'indice de réfraction causé par des variations de température dans le milieu amplificateur. Les modes changent légèrement de fréquence se déplaçant dans la courbe de gain de sorte que le mode de polarisation non dominant peut être situé plus près du maximum, ce qui se traduit par un gain plus important pour ce mode. Même si cette hypothèse est vraie, elle n'expliquerait pas pourquoi l'oscillation des deux modes est possible lorsque la différence de pertes entre les deux modes de polarisation devient plus grande.

 La seconde explication, bien documentée dans la littérature \cite{antiphase_dynamics_1992,anti_phase_polarization_2005}, est un effet de couplage entre les deux modes de polarisation. Dans ce modèle, on dit que chaque mode de polarisation est associé à une population d'atomes différents, et un mode de polarisation est susceptible d'interagir faiblement ou fortement avec la population d'atomes de l'autre mode. Il en résulte des effets de saturation et de gain croisés, c'est-à-dire que l'intensité d'un état de polarisation est amplifiée par l'inversion de population associée à l'autre état, et que l'émission stimulée d'un état de polarisation sature la population associée à l'autre état. À l'aide de simulation numérique, ces chercheurs ont trouvé que ce modèle explique bien cette dynamique commutation lente entre les deux modes de polarisation. Nous notons que dans notre cas, l'oscillation d'un mode à l'autre était plus fréquente lorsqu'il n'y avait pas de contact entre les miroirs, ce qui entraînait une plus grande variation de la longueur optique dont la première hypothèse semble plus être en jeux dans notre situation.
	
\pagebreak
\thispagestyle{empty}

\section{CONCLUSION ET PERSPECTIVES DE RECHERCHE}

\subsection{Sommaire des résultats}

Nous avons fabriqué et testé un laser à l'état solide composé d'un milieu amplificateur de YAG dopé à l'Yb$^{3+}$ d'une cavité de taille réduite avec des miroirs anisotropes, où le concept pour obtenir une émission monomode repose sur le fonctionnement près d'un point où les états propres de polarisation coalesce (point exceptionnel) et l'élimination du creusement spatial. Nous montrons que pour avoir un point exceptionnel la réponse optique des miroirs doit présenter la même diretardance qu'une lame demi-onde, en plus d'une différence de réflectance non nulle, et que l'angle entre les axes principaux des miroirs $\alpha$ est ajusté pour obtenir cette coalescence. Nous avons également démontré théoriquement
une autre condition pour obtenir un point exceptionnel avec une réponse optique des deux miroirs présentant la même diretardance et la même différence de réflectance entre TE et TM mais de signe inverse, qui ne sont pas réalisées expérimentalement dans cette thèse. L'anisotropie de la réponse optique des miroirs du laser a été obtenue par la gravure d'un réseau de diffraction sur un miroir multicouche de Bragg fonctionnant en incidence normal. L'avantage d'une intégration de toutes les composantes optiques dans une seule pièce est le raccourcissement de la longueur de la cavité et donc un intervalle spectral plus grand entre les modes ce qui nous permet d'avoir une meilleure suppression des autres modes. 

Expérimentalement, nous avons démontré le fonctionnement d'un laser monomode avec une conception simple qui est applicable à d'autres lasers à milieux à élargissement homogène, où le phénomène imposant l'émission multimode est le creusement spatial, c'est-à-dire la formation d'ondes stationnaires dans le résonateur et la compétition entre états de polarisation, ce qui est largement le cas des lasers à l'état solide. La méthode utilisée pour obtenir cette émission monomode diffère de celle prévue, qui consistait à opérer autour d'un point exceptionnel accessible avec le seul paramètre de contrôle $\alpha$, puisque les spécifications très strictes concernant la réponse optique des miroirs n'ont pas été atteintes. Bien que le meilleur scénario expérimental est l'élimination complète du phénomène de creusement spatial, il est démontré que cela n'est pas nécessaire d’avoir l'émission d'un seul mode longitudinal appartenant à un état propre de polarisation. Nous démontrons également pour la première fois la condition pour l'émission d'un seul mode de polarisation, ce qui suggère que l'émission des deux modes de polarisation dans notre cas de réponse optique obtenue semble être très bien supprimée pour une grande plage d'angle $\alpha$. Les mécanismes responsables de l'amélioration de la pureté d'émission sont la différence de pertes entre les deux états de polarisation et la réduction de la modulation de l'onde stationnaire dans la cavité. De plus, en faisant coïncider la taille du faisceau pompe à celle du mode fondamental TEM$_{00}$, on réussit à éliminer les autres modes transverses de la cavité sauf le mode fondamental TEM$_{00}$ ce qui a été montré par la qualité du faisceau $M^2$ près de la valeur limite de 1. Finalement, nous notons que les mesures de seuil et les mesures des états propres de polarisation, ainsi que les mesures du degré de polarisation avec une valeur supérieure à 0,9 pour tout angle $\alpha$ testé, ont validé notre approche théorique simple pour simuler les états propres de polarisation en termes de réponse optique des miroirs, en se basant sur le formalisme de Jones. Le fait que nous observions des oscillations lentes d'un état propre de polarisation à l'autre dans le temps pour une large gamme d'angles $\alpha$ pour laquelle la théorie montre une grande suppression de l'oscillation des deux modes, suggère qu'un autre effet autre que le creusement spatial est la cause de l'émission des deux états propres de polarisation. Cet effet est bien documenté dans la littérature comme étant un phénomène de saturation croisée entre les populations d'atomes qui sont légèrement différentes \cite{anti_phase_polarization_2005,anti_phase_1991,anti_phase_1999,anti_phase_1991_Winner_takes_all,anti_phase_1992}.
 
 
\subsection{Application et perspective}

La conception proposée démontre la possibilité de remplacer les conceptions plus complexes qui sont habituellement nécessaires pour obtenir une émission monomode par une conception de taille beaucoup plus compacte et qui a la possibilité d'être miniature et monolithique, c'est-à-dire un résonateur conçu dans une seule pièce. En ce qui concerne les limites de notre conception, nous pensons qu'il y a des fluctuations dans la longueur optique de la cavité, due à des vibrations mécaniques et à des modifications de la longueur optique induites par un changement de température. Nous pensons qu'un laser, où les réseaux et les miroirs sont directement intégrés au milieu amplificateur par soit déposer directement ces couches minces sur le milieu amplificateur ou par contact adhésif et que sa température est stabilisée par un système de refroidissement, résoudrait les problèmes observés avec la conception. 

Le contrôle de la température s'accompagne d'un contrôle fin de la longueur de la cavité, ce qui permet de contrôler la fréquence d'émission. En effet, la température affecte l'indice de réfraction du milieu actif et provoque une dilatation du matériau, ce qui modifie la condition de résonance des modes, qui doivent donc s'ajuster pour être toujours en résonance. On note aussi que la courbe de gain se déplace avec la température, mais que cet effet n'est pas pris en compte dans ce calcul. Dans le cas parfait où le creusement spatial est complètement éliminé, le décalage maximal de fréquence possible tout en restant avec une émission monomode est d'un intervalle spectral libre, car en déplaçant la fréquence du mode dominant, de la position du maximum de la section efficace d'émission, d'un demi-intervalle spectral libre, on obtient une position de deux modes qui on la même section efficace d'émission. Avec cette position des modes, l'émission devient bimodale. Pour le cas où la modulation de l'onde stationnaire n'est pas complètement éliminée, le rapport entre les sections efficaces d'émission du mode dominant et non dominant doit être plus grand à mesure que le facteur de contraste augmente pour que la condition de l'émission monomode s'applique, démontré par l'équation \ref{eq:gamma_crit}. Ce comportement du maximum de déplacement de la fréquence en fonction de $\vert\gamma \vert$ est montré à la figure \ref{fig:deplacement_freq_max_en_fonction_gamma}, qui dépend indirectement de la pente de la section efficace d'émission du milieu utilisé (dans notre cas Yb:YAG) et de la longueur de la cavité, ce qui augmente la valeur maximale de ($\gamma_{\text{crit}}$) et l'intervalle spectral libre du laser au fur et à mesure que la longueur de la cavité diminue. L'axe des y à droite est la variation de température nécessaire pour avoir ce déplacement en fréquence qui est donnée par dans la référence \cite{polarisation_stabilized_microchip}, d'une valeur de $\Delta \lambda = 0.011 \text{ nm/ \degree C} $ pour l'Yb$^{+3}$:YAG en considérant que la section efficace d'émission ne change pas avec la température ce qui n'est pas le cas. On note que ce calcul ne considère pas l'émission multimode provenant de l'autre mode de polarisation donc considère que les modes de polarisation sont bien supprimés par une grande différence de pertes.

\pagebreak

\begin{figure}[h!]%Le h! assure que la figure reste proche d'ici. autre options t-top, b-bottom, p - float page. Voir par exemple: https://tex.stackexchange.com/questions/39017/how-to-influence-the-position-of-float-environments-like-figure-and-table-in-lat
\centering
\includegraphics[scale=0.4]{deplacement_freq_max_en_fonction_gamma.jpg}
\caption{Simulation du décalage maximal de la fréquence d'émission, axe des y à gauche, pour rester dans le régime d'émission monomode, en fonction du facteur de contraste obtenu par notre résonateur. L'axe des y à droite représente le changement de température nécessaire pour créer ce décalage de la fréquence. ces simulations ont été faites pour un résonateur de longueur optique de 2.33 mm qui est le résonateur utiliser lors de cette thèse, courbe en bleu, et pour le cas du microchip composant seulement le milieu actif d'une longueur optique de 1.83 mm, courbe en rouge et pour la section efficace d'émission du Yb$^{3+}$:YAG}
\label{fig:deplacement_freq_max_en_fonction_gamma}
\end{figure}

Nous pensons qu'un microhip monolithique utilisant notre architecture de résonateur sera un bon candidat pour les applications de détection et de télémétrie par ondes continues modulées en fréquence \cite{Slinger2012IntroductionTC}. Ce mécanisme de changement de la fréquence n'est pas aussi rapide que ce qui est obtenu par d'autre laser comme les VCSELs qui accomplit cela par modulation du courant \cite{Okano:20_VCSEL} ou dans les lasers à rétroaction distribuée "distributed feedback lasers" verrouillés par injection avec un résonateur en anneaux passifs à haute finesse avec élément électro-optique modulant la fréquence \cite{self_injection_locked_DFB_laser}. Cependant, la puissance maximale de ces dispositifs est généralement très faible, c'est-à-dire inférieure à quelques dizaines de mW montré dans le tableau du document supplémentaire de \cite{self_injection_locked_DFB_laser}, alors que notre dispositif pourrait facilement produire une puissance de l'ordre de plusieurs watts \cite{twisted_mode_space_application,Polynkin:05_twisted_mode_high_power}, ce qui convient aux applications Lidar à plus longue portée puisque la détection du signal par rapport au bruit devient nettement meilleur à mesure que la puissance augmente \textcolor{red}{cite}.

D'autre part, nous devons chercher un moyen d'avoir un mode de polarisation unique par l'élimination de l'effet de saturation croisée. Avec une microchip, la longueur optique sera bien contrôlée, ce qui nous permettra d'étudier cet effet de manière plus isolée. Par rapport à d'autres recherches menées sur ce phénomène, nous disposons d'états de polarisation non orthogonaux et le phénomène de creusement spatial est partiellement éliminé. Nous serons en mesure de comprendre si cela permet de réduire l'effet de la saturation croisée, et donc d'obtenir une émission monomode stable sur de longues périodes.

En ce qui concerne le développement de la théorie présentant d'autres conditions pour avoir un point exceptionnel à la section \ref{sect:Point exceptionnel dans les résonateurs}, nous avons reconnu qu'on obtient un quasi-point exceptionnel avec des restrictions moins sévères sur les conditions des deux miroirs. Ces conditions sont: un déphasage entre TE et TM de soit $\Delta_1= -\Delta_2 $ ou $\Delta_1=\Delta_2$ et une différence de réflectance entre TE et TM pour un des miroirs. Les simulations montrent qu'avoir de la diatténuation d'un seul côté et avoir deux miroirs avec la même diretardance nous amène tout près d'un PE. Ce résultat de la proximité des états propres de polarisation est illustré dans la figure  \ref{graph:prox_fact_cont_M1_M2_170_r_x_sqrt0.7 et M1_170_M2_190} où a) et le facteur de contraste b) dans le cas avec $\Delta_1= \Delta_2 = 170\degree$ et c) et d) le cas avec $ \Delta_1=190\degree \text{ et } \Delta_2=170\degree$. D'autre part, nous remarquons que le facteur de contraste de l'onde stationnaire n'est pas aussi bien éliminé que dans le cas de $\Delta_1 = \Delta_2 = 180\degree$. Cela peut être intéressant pour la fabrication de couches minces anisotropes, puisque des couches minces avec des déphasages très proches les uns des autres pourraient être fabriquées simplement en faisant faire les dépôts sur les miroirs en même temps. Cependant, la manière d'ajouter de la diatténuation sans ajouter de diretardance en utilisant la méthode des couches minces n'est pas si évidente.

\pagebreak

\begin{figure}[h!]%Le h! assure que la figure reste proche d'ici. autre options t-top, b-bottom, p - float page. Voir par exemple: https://tex.stackexchange.com/questions/39017/how-to-influence-the-position-of-float-environments-like-figure-and-table-in-lat
\centering
\includegraphics[scale=0.65]{prox_fact_cont_M1_M2_170_r_x_sqrt0.7 et M1_170_M2_190.jpg}
\caption{Point quasi exceptionnel dans le cas de miroir avec les valeurs suivantes: $r_{11}= \sqrt{0.7}$, $r_{12}=r_{21}=r_{22} =1$ et dans a) et b), $\Delta_1=\Delta_2=170\degree$, dans c) et d), $\Delta_1= 190\degree \text{ et } \Delta_2=170\degree$. a) et c) Proximité des vecteurs propres, b) et d) facteur de contraste de l'onde stationnaire calculé entre les deux miroirs dans le milieu amplificateur.}
\label{graph:prox_fact_cont_M1_M2_170_r_x_sqrt0.7 et M1_170_M2_190}
\end{figure}

Sur cette note d'investigation de quasi-point exceptionnel, de différents réseaux que celle utilisée lors de la thèse qui produisent une proximité des états propres de polarisation meilleure ont été remarqué par leur déphasage près de la condition de $\Delta_1= -\Delta_2$. Les résultats de simulation de la proximité des états de polarisation et le facteur de contraste de l'onde stationnaire est démontré dans la figure, montrant une proximité maximale de 0.9 qui est d'une valeur plus grande que celle du cas de miroirs utilisé dans cette thèse et un facteur de contraste qui est similaire au cas étudié expérimentalement dans cette thèse. Ces miroirs pourraient nous donner une émission plus pure que les résultats obtenus, mais pas de manière significative.

\begin{figure}[h!]%Le h! assure que la figure reste proche d'ici. autre options t-top, b-bottom, p - float page. Voir par exemple: https://tex.stackexchange.com/questions/39017/how-to-influence-the-position-of-float-environments-like-figure-and-table-in-lat
\centering
\includegraphics[scale=0.725]{ MS_C_MP_F.jpg}
\caption{Simulation des vecteurs propres du résonateur fait avec les miroirs de sortie C et miroir de pompe F a) Proximité des vecteurs propres, b) facteur de contraste de l'onde stationnaire calculé entre les deux miroirs dans le milieu amplificateur.}
\label{graph:}
\end{figure}
\pagebreak


\thispagestyle{empty}
\section*{ANNEXE A}
\label{section:ANNEXE A}
\subsection*{Mesure des propriétés optiques des miroirs nanostructurés par ellipsométrie}
L'ellipsométrie optique à incidence normale des miroirs à réseau a été utilisée pour déterminer la réflectance des polarisations TE et TM ainsi que leur déphasage relatif $\Delta$. Le montage ellipsométrique en incidence normal est montré à la figure \ref{graph:montage_ellipso}. L'angle d'incidence normal est très important, car nous essayons de mesure les propriétés de réseau dans les conditions d'opération dans le laser et que l'on sait que les propriétés de réseaux peuvent changer considérablement avec un petit changement dans l'angle d'incidence. Puisque seulement les propriétés autour de $\lambda = 1030$ nm nous intéresse pour le fonctionnement du laser, un laser Yb$^{3+}$:YAG a été construit pour l'ellipsométrie et le laser est passé à travers un polariseur linéaire dont l'axe de transmission est orienté à un angle fixe de $\theta_1 = 45\degree$. Ce faisceau, désormais polarisé diagonalement, se déplace alors le long de la trajectoire illustrée par la figure \ref{graph:montage_ellipso}, en passant en travers un séparateur de faisceau non polarisant, une partie du faisceau est à utiliser comme référence, envoyé sur le photodétecteur $D_1$ tandis que l'autre moitié de la puissance est utilisée pour sonder l'échantillon. En mode réflexion de l'ellipsométrie, le trajet du faisceau de la sonde est réfléchi par l'échantillon, à incidence normale, puis elle est réfléchie par le séparateur de faisceau, et passe à travers l'analyseur, c'est-à-dire un polariseur $P_2$, et est envoyé vers un second photodétecteur  $D_2$. Trois mesures de photo-signal sont effectuées avec $D_2$, le polariseur $P_2$ étant orienté à $\theta_2 = 0\degree, 45\degree \text{ et } 90\degree$. Avant de caractériser les miroirs d'intérêts, une mesure de base avec un échantillon est effectuée pour calibrer le ratio du signal mesurer par les détecteurs $D_1$ et $D_2$, $V_1/V_2$ avec un miroir d'aluminium isotrope dont on la réflectance de $R=0.94 est déja connue$

\begin{figure}[h!]%Le h! assure que la figure reste proche d'ici. autre options t-top, b-bottom, p - float page. Voir par exemple: https://tex.stackexchange.com/questions/39017/how-to-influence-the-position-of-float-environments-like-figure-and-table-in-lat
\centering
\includegraphics[scale=0.95]{montage ellipsomètre.jpg}
\caption{Montage expérimental pour les mesures ellipsométriques des miroirs du réseau en incidence normale à 1030 nm. Les mesures sont effectuées sans et avec la lame quart d'onde (compensateur). L lentille, P polariseur, }
\label{graph:montage_ellipso}
\end{figure}

Ce ratio du signal mesuré est proportionnel à l'intensité qui est décrite en fonction de l'angle de l'analyseur et des propriétés optiques de l'échantillon en question par l'équation. 

 \begin{equation} \label{intensiter_total}
\begin{gathered}
	I(\theta_2)=\dfrac{1}{2} (S_0 +S_1 \cos(2\theta_2) +S_2 \sin(2\theta_2))
\end{gathered}
\end{equation}
 
 où $S_0, S_1 \text{ et } S_2$ sont les trois premiers composants du vecteur de Stokes. Pour un échantillon non dépolarisant avec une matrice de Jones diagonale, c'est-à-dire avec une diatténuation et une biréfringence rectilignes et ses axes principaux orientés dans la base $\lbrace H,V\rbrace$ du laboratoire, et avec un faisceau de sondes polarisées diagonalement, on a:
 
\begin{equation} \label{S0_S1_S2}
\begin{gathered}
    S_0 = \frac{R_{TE} + R_{TM}}{2} \\
    S_1 = \frac{R_{TE} - R_{TM}}{2} \\
    S_2 = \sqrt{R_{TE} R_{TM}} \cos\Delta
\end{gathered}
\end{equation}

Après calibration avec un miroir d'aluminium isotrope, on peut effectuer les mesures de l'échantillon et obtenir les valeurs de $R_{TE}$, $R_{TM}$ et $\cos(\Delta)$. Cependant, cela ne nous permet pas d'avoir le signe de $\Delta$. Pour ce faire, nous devons insérer la lame compensatrice avec ses axes principaux orientés dans la base $\lbrace H,V\rbrace$ du laboratoire, ce qui nous donne l'expression pour l'intensité en fonction de l'angle de l'analyseur comme:   

\begin{equation} \label{S0_S1_S2}
\begin{gathered}
  I(\theta_2) = \frac{1}{2}\left(S_0 + S_1 \cos(2\theta_2) - S_3 \sin(2\theta_2)\right) 
\end{gathered}
\end{equation}

et avec le dernier composant du vecteur de Stokes $S_3$, pour un échantillon non dépolarisant qui à la relation suivante: 

\begin{equation} \label{S0_S1_S2}
\begin{gathered}
    S_3 = \sqrt{R_{TE} R_{TM}} \sin\Delta
\end{gathered}
\end{equation}

avec cette mesure, on peut trouver le signe de $\Delta$. Un autre avantage de faire les mesures avec la lame compensatrice est que nous avons accès à une mesure de consistance entre les deux mesures en calculant $\cos^2(\Delta) + \sin^2(\Delta)$  qui devrait donner une valeur près de 1. Les mesures ellipsométriques des miroirs de pompe et des miroirs de sortie en réflexion sont montrées au tableau \ref{reflection_miroir_pompe} et \ref{reflection_miroir_sortie} respectivement avec les premiers colones de la réflectance montrant les mesures faites sans lame compensatrice, ensuite celle faite avec la lame compensatrice et le les dernières colonnes le résultat du $\Delta$ selon la mesure sans lame compensatrice, avec lame compensatrice, et puis le résultat de calcule de ces deux mesures. Les miroirs utilisés lors des tests laser sont: les miroirs de pompe H et miroir de sortie C.   
 
 \pagebreak
\begin{table}[]
\centering
\caption{Mesures de réflexion des miroirs de pompe. Tous les réseaux ont la même période de 435 nm et ne diffèrent que par leur largeur de trait.}
\label{reflection_miroir_pompe}
\resizebox{\textwidth}{!}{%
\begin{tabular}{l|llllllllll}
Réseaux               & A     & B     & C     & D     & E     & F     & G     & H     & I     & J     \\ \hline
largeur de trait (nm) & 241   & 243   & 268   & 221   & 226   & 220   & 200   & 216   & 233   & 231   \\ \hline
$\cos(\Delta)$             & -0.78 & -0.79 & -0.38 & -0.99 & -1.00 & -0.99 & -0.91 & -1.00 & -0.93 & -0.94 \\
R$_{\text{TE}}$                & 0.96  & 0.94  & 0.93  & 0.97  & 0.97  & 0.97  & 0.99  & 0.97  & 0.98  & 0.95  \\
R$_{\text{TM}}$              & 0.97  & 0.95  & 0.97  & 0.94  & 0.93  & 0.93  & 0.93  & 0.93  & 0.97  & 0.92  \\ \hline
$\sin(\Delta)$             & -0.62 & -0.64 & -0.92 & -0.10 & -0.08 & 0.12  & 0.40  & 0.03  & -0.39 & -0.34 \\
R$_{\text{TE}}$                & 1.05  & 0.93  & 0.93  & 0.95  & 0.96  & 0.95  & 0.99  & 0.97  & 0.98  & 0.95  \\
R$_{\text{TM}}$               & 1.06  & 0.94  & 0.96  & 0.93  & 0.92  & 0.93  & 0.92  & 0.92  & 0.96  & 0.91  \\ \hline
cos²($\Delta$) + sin²($\Delta$)      & 0.99  & 1.03  & 1.00  & 1.00  & 1.00  & 0.99  & 0.98  & 1.00  & 1.01  & 1.00  \\
$\Delta$-cos($\degree$)             & 218   & 218   & 247   & 187   & 185   & 171   & 155   & 177   & 202   & 200   \\
$\Delta$-sin($\degree$)              & 218   & 220   & 248   & 186   & 185   & 173   & 156   & 178   & 203   & 200   \\
$\Delta$-tan($\degree$)             & 218   & 219   & 247   & 186   & 185   & 173   & 156   & 178   & 203   & 200  
\end{tabular}%
}
\end{table}
 
 
 
\begin{table}[]
\centering
\caption{Mesures de réflexion des miroirs de sortie. Tous les réseaux ont la même période de 644 nm et ne diffèrent que par leur largeur de trait.}
\label{reflection_miroir_sortie}
\resizebox{\textwidth}{!}{%
\begin{tabular}{l|llllllllll}
Réseaux               & A    & B     & C     & D     & E     & F     & G     & H     & I     & J     \\ \hline
largeur de trait (nm) & 505  & 499   & 485   & 466   & 469   & 471   & 474   & 463   & 452   & 444   \\ \hline
$\cos(\Delta)$             & 0.31 & -0.58 & -0.97 & -0.53 & -0.64 & -0.52 & -0.80 & -0.45 & -0.10 & -0.21 \\
R$_{\text{TE}}$                & 0.27 & 0.66  & 0.89  & 0.65  & 0.70  & 0.60  & 0.79  & 0.58  & 0.46  & 0.50  \\
R$_{\text{TM}}$                   & 0.30 & 0.44  & 0.48  & 0.41  & 0.51  & 0.51  & 0.55  & 0.49  & 0.71  & 0.89  \\ \hline
$\sin(\Delta)$             & 0.89 & 0.79  & -0.15 & -0.82 & -0.74 & -0.84 & -0.59 & -0.87 & -0.97 & -0.97 \\
R$_{\text{TE}}$                & 0.28 & 0.65  & 0.88  & 0.64  & 0.69  & 0.59  & 0.78  & 0.57  & 0.46  & 0.50  \\
R$_{\text{TM}}$                  & 0.30 & 0.44  & 0.49  & 0.40  & 0.50  & 0.50  & 0.55  & 0.48  & 0.70  & 0.88  \\ \hline
cos²($\Delta$) + sin²($\Delta$)     & 0.88 & 0.96  & 0.96  & 0.94  & 0.96  & 0.97  & 0.98  & 0.96  & 0.96  & 0.98  \\
$\Delta$-cos($\degree$)              & 72   & 125   & 194   & 238   & 231   & 239   & 217   & 243   & 265   & 258   \\
$\Delta$-sin($\degree$)              & 62   & 127   & 189   & 235   & 228   & 237   & 216   & 240   & 257   & 255   \\
$\Delta$-tan($\degree$)              & 71   & 126   & 189   & 237   & 230   & 238   & 216   & 243   & 264   & 258  
\end{tabular}%
}
\end{table} 

Les mesures en transmission du miroir de sortie ont aussi été faites pour pouvoir comparer les résultats de mesure des états propres de polarisations du résonateur laser utiliser lors des expériences laser avec la théorie. Le montage doit être légèrement modifié pour que l'on puisse mesurer les échantillons en transmission. L'encadré dans la figure \ref{graph:montage_ellipso} montre cette  modification, qui consiste à remplacer l'échantillon sur le bras du détecteur D$_2$ et à ajouter un miroir de réflectance connue à l'endroit où l'échantillon est placé lors des mesures de réflexion. $D_2$ Les résultats de mesures des miroirs de sortie en transmission sont montrés dans le tableau \ref{transmission_miroir_sortie} 

\begin{table}[]
\centering
\caption{Mesures de la transmission des miroirs de sortie.}
\label{transmission_miroir_sortie}
\resizebox{\textwidth}{!}{%
\begin{tabular}{l|llllllllll}
Réseaux           & A     & B    & C     & D     & E     & F     & G     & H     & I     & J     \\ \hline
$\cos(\Delta)$           & 0.95  & 0.94 & -0.59 & -0.40 & -0.23 & -0.29 & -0.48 & -0.37 & -0.54 & -0.47 \\
T$_{\text{TE}}$             & 0.67  & 0.25 & 0.06  & 0.27  & 0.24  & 0.29  & 0.21  & 0.36  & 0.50  & 0.32  \\
T$_{\text{TM}}$           & 0.62  & 0.54 & 0.55  & 0.51  & 0.36  & 0.36  & 0.42  & 0.46  & 0.21  & 0.12  \\ \hline
$\sin(\Delta)$         & -0.28 & 0.18 & -0.81 & -0.89 & -0.96 & -0.93 & -0.84 & -0.90 & -0.78 & -0.62 \\
T$_{\text{TE}}$            & 0.66  & 0.26 & 0.06  & 0.27  & 0.23  & 0.29  & 0.21  & 0.34  & 0.48  & 0.31  \\
T$_{\text{TM}}$          & 0.64  & 0.54 & 0.57  & 0.53  & 0.37  & 0.37  & 0.44  & 0.48  & 0.23  & 0.13  \\ \hline
cos²($\Delta$) + sin²($\Delta$) & 0.98  & 0.92 & 0.99  & 0.95  & 0.97  & 0.94  & 0.94  & 0.94  & 0.90  & 0.60  \\
$\Delta$-cos($\degree$)          & 342   & 380  & 234   & 246   & 256   & 253   & 241   & 249   & 238   & 242   \\
$\Delta$-sin($\degree$)          & 344   & 370  & 234   & 242   & 253   & 248   & 237   & 244   & 231   & 218   \\
$\Delta$-tan($\degree$)            & 344   & 371  & 234   & 245   & 256   & 253   & 240   & 248   & 235   & 232  
\end{tabular}%
}
\end{table}
	
	
\section*{ANNEXE B}
\label{section:ANNEXE B}
\subsection*{Réponse optique anisotrope d'un réseau gravé sur miroir}
	
Quant à l'anisotropie dans le résonateur, il y a deux façons qui ont été testées pour en introduire directement sur les miroirs sans augmenter la longueur optique de la cavité. La première façon qui a été envisagée lors des travaux de Koffi Amouzou \cite{Bisson_thin_film_twisted_mode_exp} est la fabrication de couches minces par la méthode de déposition à angle rasant \cite{GLAD_shadowing_1999}. Cette méthode repose sur le dépôt de matériaux présentant une porosité, de sorte que la structure et la quantité de porosité peuvent être contrôlées par l'angle de dépôt, ce qui change la diretardance de la couche à mesure que l'épaisseur augmente. Avec cette méthode, le substrat est incliné par rapport à la direction du dépôt de ce qui cause la croissance de matériaux en colonnes inclinées qui empêchent le dépôt d'autres particules entre les colonnes en raison d'un effet d'ombrage qui cache ces parties \cite{GLAD_shadowing_1999}. Il en résulte une structure non uniforme, c'est-à-dire une anisotropie de forme qui donne lieu à des effets optiques anisotropes tels que la diretardance et la diatténuation.

À point d'illustration de l'effet d'une anisotropie de forme sur la résponse anisotrope d'un matériaux, on modélise les couches composites de deux milieux différents de taille petite par de fines plaques parallèles équidistantes \cite{born_wolf_bhatia_clemmow_gabor_stokes_taylor_wayman_wilcock_1999}, figure \ref{fig:modèle_millieu_effectif}

\begin{figure}[h!]%Le h! assure que la figure reste proche d'ici. autre options t-top, b-bottom, p - float page. Voir par exemple: https://tex.stackexchange.com/questions/39017/how-to-influence-the-position-of-float-environments-like-figure-and-table-in-lat
\centering
\includegraphics[scale=0.6]{modèle_millieu_effectif.jpg}
\caption{modélisation d'un milieu effectif composé de deux matériaux}
\label{fig:modèle_millieu_effectif}
\end{figure}

\noindent
,où $t_1$ et $t_2$ sont les épaisseurs, $\epsilon_1$ et $\epsilon_2$ les constants diélectriques des deux matériaux et $d$ l'épaisseur de la couche. La relation entre la constante diélectrique et l'indice de réfraction dans le cas de faibles coefficients d'extinction est $\epsilon=n^2$.  Ce modèle s'applique d'autant mieux que l'échelle d'hétérogénéité est fine par rapport à la longueur d'onde et fonctionne comme une première approximation de la réponse optique d'un réseau de diffraction ce que l'on vas explorer dans le reste de cette section. Le calcul montre que l'indice de réfraction parallèle aux plaques, $n_{\parallel}$ et perpendiculaire aux plaques $n_{\perp}$ s'obtient:

\begin{equation} \label{eq:index_refraction_millieu_effectif}
\begin{gathered}
n_{\perp}= \frac{n_1^2 n_2^2}{f_1n_2^2+f_2n_1^2} \\
\\
n_{\parallel}= f_1 n_1^2+f_2 n_2^2
\end{gathered}
\end{equation}

\noindent
où $f_1=t_1/(t_1+t_2)$ et $f_2=t_2/(t_1+t_2)$ qui sont les fractions du volume total occuper par le milieu 1 et 2 respectivement. Par ce modèle simple, on peut déterminer l'anisotropie d'un milieu bien particulié dont l'anisotropie de forme est idéalisée. 
  
Dans ce travail, on utilise plutôt un réseau de diffraction qui possède cette propriété des propriétée similaires de milieu effectif lorsque la période du réseau est inférieure à la longueur d'onde de rayonnement incident. Or, ce qui différencie les réseaux d'un simple milieu effectif, ce sont les ordres de diffraction supérieurs qu'ils peuvent produire \textcolor{red}{ORDRE SUPÉRIEUR DANS LE RÉSEAUX EST TOUT DE MÊME POSSIBLE EN TRANSMISSION ?}. Dans notre cas, on s'intéresse à avoir une réponse optique anisotrope en réflexion sous une incidence normale. Dans ce cas, la condition pour avoir des ordres de diffraction supérieurs est donnée par l'équation:

\begin{equation} \label{eq:loi des réseaux}
\begin{gathered}
\Lambda n \sin(\theta) = m \lambda ,
\end{gathered}
\end{equation}

\noindent
où $\Lambda$ est la période du réseau, $n$ l'indice de réfraction du milieu de l'onde incidente sur le réseau, $\theta$ l'angle de diffraction à l'intérieur du guide d'onde, m l'ordre de diffraction qui est un entier relatif et $\lambda$ la longueur d'onde. Par exemple, une façon d'avoir une réflectance plus grande avec un réseau comparé à une simple couche est d'avoir un milieu d'indice élevé en dessous du réseau, qui agit comme un guide d'ondes pour les ordres de diffraction supérieurs. Ceci est illustré à la figure \ref{graph:réseaux_résonant}. 

\begin{figure}[h!]%Le h! assure que la figure reste proche d'ici. autre options t-top, b-bottom, p - float page. Voir par exemple: https://tex.stackexchange.com/questions/39017/how-to-influence-the-position-of-float-environments-like-figure-and-table-in-lat
\centering
\includegraphics[scale=0.6]{réseaux résonant.jpg}
\caption{Illustration de l'effet d'un réseau résonant. Opacité de la flèche indique l'amplitude de l'onde. }
\label{graph:réseaux_résonant}
\end{figure}

Les ordres de diffraction supérieure dans le guide d'onde peuvent alors être de nouveau couplés vers l'extérieur par le réseau dans l'ordre zéro et si les ondes couplées vers l'extérieur sont en interférence constructive avec l'onde réfléchi de la surface du réseau, il est possible d'avoir une réflexion résonante de 100\% \cite{resonant_wave_guide_explanation,Destouches:06}. Par contre, les ondes qui sont diffractées dans les ordres supérieurs dans le guide d'onde, n'accumuleront pas la même phase pour les deux états de polarisation TE et TM. Ceci est dû au fait qu'il y a une réflexion totale interne (RTI) de l'onde dans le guide d'onde sur la surface opposée du réseau. Or, d'après les coefficients de réflexion complexe de Fresnel pour une RTI, les états de polarisation TE et TM ont des coefficients de réflexion qui diffèrent en phase. Donc les deux états de polarisation n'auront pas en général une réflexion résonante pour la même longueur d'onde. Cet effet est responsable de l'anisotropie dans la réponse optique d'une telle structure, comme le montre la figure \ref{graph:réseaux_résonant}, où l'état TM est en résonance donc a une plus grande réflectance que le mode TE, qui n'est pas en résonance. De plus, la diretardance peut être contrôlée en modifiant le déphasage ($\phi$) entre les ondes couplées à l'extérieur du guide d'onde du mode de polarisation TE non résonant, $\phi$.
 
La difficulté de produire un miroir avec la réponse optique voulue est qu'il n'existe pas de méthode algorithmique connue qui fournit les paramètres de structure du miroir de Bragg et du réseau par rapport aux réponses optiques voulues. Pour cette raison, 10 réseaux on été fabriqués avec des valeurs différente du rapport cyclique pour s'approcher des propriétés idéales. Les réseaux sélectionnés pour les expériences laser ont été choisis en fonction de leur proximité avec la valeur du déphasage de $\pi$, afin de s'approcher le plus possible de la condition du point exceptionnel. Ces réseaux qui ont les propriétés optique les plus proches des propriétés idéales sont le réseaux du miroir de pompe H et le réseaux du miroir de sortie C, tableaux \ref{reflection_miroir_pompe} et \ref{reflection_miroir_sortie} de la section \ref{sec:condition d'émission monomode dans les résonateurs à onde stationnaire}. Les détails des mesures ellipsométriques de ces miroirs sont présentés dans l'annexe A. \textcolor{red}{FABRIQUATION DES RÉSEAUX - PROCÉDURE}

Les réseaux doivent également répondre à d'autres critères pour pouvoir être utilisés dans une cavité laser. Premièrement, le coefficient de transmission à la longueur d'onde de la pompe doit être suffisamment élevé pour le réseau de pompe, de sorte qu'une grande partie de l'intensité du faisceau de la pompe entre dans la cavité. Deuxièmement, les coefficients de réflectance à la longueur d'onde du faisceau de sorties pour le miroir de pompe doit être suffisamment élevés pour ne pas avoir de trop grandes pertes dans le résonateur. Enfin, il doit y avoir une grande différence de réflectance entre les états de polarisation TE et TM dans le miroir de sortie, de sorte que le PE se trouve à un angle $\alpha_{\scaleto{PE}{4pt}}$ donné par l'équation \ref{eq:alpha_PE_genéral}, qui est suffisamment grand pour nous permettre de le repérer plus facilement lors d'une expérience.

Bien qu'une modélisation précise soit effectuée à l'aide de simulations numériques avec la méthode modale de Fourier, il est important de comprendre de quel phénomène provient la réponse optique de nos réseaux. L'optimisation des miroirs à été faite par nos collège du Laboratoire Hubert-Currien et l'affinement du modèle par "University of Eastern Finland" et les miroirs optimisés ressemblent à la figure \ref{graph:schéma_miroir} \cite{LiLapointe}.

\pagebreak

\begin{figure}[h!]%Le h! assure que la figure reste proche d'ici. autre options t-top, b-bottom, p - float page. Voir par exemple: https://tex.stackexchange.com/questions/39017/how-to-influence-the-position-of-float-environments-like-figure-and-table-in-lat
\centering
\includegraphics[scale=0.775]{réseaux_figure.jpg}
\caption{Conception des miroirs optimisée pour (a) le miroir de pompe et (b) le miroir de sortie. Le seul paramètre qui est balayé pour les 10 réseaux est le facteur de remplissage \cite{LiLapointe}.}
\label{graph:schéma_miroir}
\end{figure}


De la loi des réseaux à angle d'incidence normal, équation \ref{eq:loi des réseaux}, on remarque que la période des réseaux fait en sorte qu'il n'y a pas d'ordre de diffraction supérieure en transmission dans la couche épaisse SiO$_2$ en raison de l'indice de réfraction de ce matériaux étant trop petit, donc l'effet de guide d'onde ne se produit pas dans cette couche. Par contre, il peut avoir des ordre de diffraction suppérieur dans le réseaux lui même puisque l'indice de réfraction de ce millieu effectif est comprise entre celle du TiO$_2$ et de l'aire et les ondes évanescentes des ordres de diffraction ($m= -1 \text{ et } 1$) en transmission dans la couche de SiO$_2$ peuvent être es onde propagative dans les couches de TiO$_2$ du multicouche parce que l'indice de réfraction de ce matériau est suffisament élevé. Les résultats de simulation de la réponse optique fait par nos collège du Laboratoire Hubert-Currien et de "University of Eastern Finland" sont montrés aux figures \ref{graph:réponse_optique_M_pompe} et \ref{graph:réponse_optique_M_sortie}. 


\begin{figure}[h!]%Le h! assure que la figure reste proche d'ici. autre options t-top, b-bottom, p - float page. Voir par exemple: https://tex.stackexchange.com/questions/39017/how-to-influence-the-position-of-float-environments-like-figure-and-table-in-lat
\centering
\includegraphics[scale=0.775]{miroir_pompe_simulation_réseaux.jpg}
\caption{ (a) Réflexion du miroir de pompe en fonction de la longueur d'onde de la lumière incidente, avec ligne pointillée et annotation à la longueur d'onde du laser de 1030 nm (b) Déviation du déphasage TE et TM par rapport à $\pi$ autour de la longueur d'onde du laser}
\label{graph:réponse_optique_M_pompe}
\end{figure}


\begin{figure}[h!]%Le h! assure que la figure reste proche d'ici. autre options t-top, b-bottom, p - float page. Voir par exemple: https://tex.stackexchange.com/questions/39017/how-to-influence-the-position-of-float-environments-like-figure-and-table-in-lat
\centering
\includegraphics[scale=0.775]{miroir_sortie_simulation_réseaux.jpg}
\caption{ (a) Réflexion en fonction de la longueur d'onde de la lumière incidente, avec ligne pointillée et annotation à la longueur d'onde du laser de 1030 nm, et (b) déviation du déphasage TE et TM par rapport à $\pi$ du miroir de sortie.}
\label{graph:réponse_optique_M_sortie}
\end{figure}


\pagebreak
Peut-on expliquer les effets du réseau à l'aide d'un modèle plus simple qui considère le réseau seulement comme un milieu effectif qui ne diffracte pas le faisceau dans les ordres de diffraction supérieur ? Ceci peut nous indiquer l'ampleur de l'effet de couplage dans un guide d'onde dans la couche du réseau et dans les couches de TiO$_2$ par onde évanescente comparée à l'effet de milieux effectifs du réseau et possiblement guider notre intuition pour créer les propriétés optiques voulues avec des réseaux. En utilisant l'équation \ref{eq:index_refraction_millieu_effectif} pour simuler le réseau comme un milieu effectif avec les indices de réfraction des matériaux et le programme de simulation de couche mince, Optikan \rom{2}, qui simule le même empilement de couches minces que les simulations fait avec la méthode modale de Fourier, on obtient les résultats montés aux figures \ref{graph:miroir_pompe_model_millieu_effectif_indice_refrac_fangfang} et \ref{graph:miroir_sortie_model_millieu_effectif_indice_refrac_fangfang}:


\begin{figure}[h!]%Le h! assure que la figure reste proche d'ici. autre options t-top, b-bottom, p - float page. Voir par exemple: https://tex.stackexchange.com/questions/39017/how-to-influence-the-position-of-float-environments-like-figure-and-table-in-lat
\centering
\includegraphics[scale=0.45]{miroir_pompe_model_millieu_effectif_indice_refrac_fangfang.jpg}
\caption{résultat de simulation avec Optikan du miroir de pompe avec  le réseau approximé comme un milieu effectif a) polarisation TE en noir et TM en rouge, b) déphasage entre TE et TM}
\label{graph:miroir_pompe_model_millieu_effectif_indice_refrac_fangfang}
\end{figure}


\begin{figure}[h!]%Le h! assure que la figure reste proche d'ici. autre options t-top, b-bottom, p - float page. Voir par exemple: https://tex.stackexchange.com/questions/39017/how-to-influence-the-position-of-float-environments-like-figure-and-table-in-lat
\centering
\includegraphics[scale=0.45]{miroir_sortie_model_millieu_effectif_indice_refrac_fangfang.jpg}
\caption{résultat de simulation avec Optikan du miroir de sortie avec le réseau approximé comme un milieu effectif a) polarisation TE en noir et TM en rouge, b) déphasage entre TE et TM}
\label{graph:miroir_sortie_model_millieu_effectif_indice_refrac_fangfang}
\end{figure}

En comparant les simulations des réseaux \ref{graph:réponse_optique_M_pompe} avec celles utilisant un milieu effectif figure \ref{graph:miroir_pompe_model_millieu_effectif_indice_refrac_fangfang}, on constate que pour le miroir de pompe, les réflectances sont très proches de la même valeur, et que le déphasage est un peu moins proche de la même valeur. En ce qui concerne le déphasage, la différence entre les deux simulations pourrait provenir d'un modèle très idéaliste de la façon dont nous calculons l'indice de réfraction effectif des réseaux. En effet en aillant une différence d'indice $n_{\perp}-n_{\parallel} = 1.34-1.75 = 0.41$ aux lieu de ce que le modèle simple prédit $n_{\perp}-n_{\parallel}=1.30-1.77 = 0.47$ à la longueur d'onde de 1030 nm, on réussit à avoir le même déphasage pour le modèle du milieu effectif et la simulation du réseau sans changer le profil de réflectance. Les équation \ref{eq:index_refraction_millieu_effectif} pour  $n_{\perp} \text{ et } n_{\parallel}$ est un modèle très simple et donc des modèles plus complexes comme celui développé dans \cite{Lalanne:98} pourraient nous donner des prédictions des indices de réfraction plus précises.

Pour le miroir de sortie, la réflectance est grandement affectée par le remplacement du réseau par un milieu effectif et le déphasage est aussi très affecté en étant très distant du déphasage de 180$\degree$ monter au figure \ref{graph:miroir_sortie_model_millieu_effectif_indice_refrac_fangfang} comparer au simulation du réseaux, figure \ref{graph:réponse_optique_M_sortie}. Ceci est attendu puisqu'en regardant les simulations des réseaux, figure \ref{graph:réponse_optique_M_sortie}, on voit beaucoup de pics étroits autour de $\lambda=1030$ nm qui proviennent de résonance du guide d'onde donc ces effets ne sont pas négligeables à 1030 nm. Ceci est confirmer par le fait que la simulation du réseaux sans substrat, c'est à dire des doit de matériaux flotant dans l'aire, montre que l'effet de réseaux résonant est de grande importance pour le réseaux de sortie figure \ref{} \textcolor{red}{FAIRE UNE SIMULATION DU RÉSEAUX DE SORTIE TOUT SEUL}

\pagebreak

  Nous pouvons conclure de ces comparaisons que l'effet du milieu effectif est l'effet dominant de la réponse optique anisotrope du miroir de pompe, mais pas du miroir de sortie. Des couplages dans un guide d'onde sont encore nécessaires pour comprendre la réponse optique totale du miroir de sortie. Cette étude pourrait nous conduire à une conception moins complexe du miroir de sortie et de pompe qui utiliserait seulement un couche palnaire anisotrope qui nous donne la même réponse optique et aussi la possibilité d'avoir la même réponse optique par une couche mince anisotrope.


\pagebreak
\printbibliography	
\end{document}